\documentclass[12pt,a4paper]{article}

\usepackage{enumitem}
\usepackage{hyperref}
\usepackage{fancyvrb}

\newlist{vprasanja}{itemize}{1}
\setlist[vprasanja]{label=\textbf{Q:}}
\newcommand\ans{\item[\textbf{A:}]}

    
\begin{document}

    \section*{Terminološka vprašanja}
    \begin{vprasanja}
        \item Kako prevedemo ``meet"?
        \ans
        \item Ali se reče ``algebraične" ali ``algebrajske" teorije/kategorije?
        \ans
        \item Kako bi prevedli ``fragment of first order logic".
        \ans
        \item Kako bi prevedli soundness?
        \ans
        \item Kako bi rekli entailment relation?
        \ans
        \item Kako prevajamo well-powered (za kategorijo). Moj guess je ``dobro pogojena".
        \ans
        \item Kako prevajamo ``kernel pair"
        \ans
        \item Kako prevajamo ``in particular"?
        \ans V posebnem?
        \item Kako prevajamo 
    \end{vprasanja}

    \section*{\LaTeX vprašanja}
    \begin{vprasanja}
        \item Na kak način naj v \LaTeX -u označujemo da je $x$ tipa $X$ (v smislu $x:X$)?
        \ans
        \item Ali je \verb|\emph{*}| lahko začetek stavka?
        \ans
        \item Ali je lepo dati dvopičje direkt pred def okoljem?
        \ans
        \item Kaj je najboljše okolje za postavit tikzcd diagram?
        \ans \verb|\equation{}|.
        \item Kako dobiti novo vrstico (na lep način) po začetku definicije?
        \ans 
        \item Težava s šumnikom v imenu poglavja
        \ans Dodamo opcijo [unicode=true] v hyperref paket
        \url{https://tex.stackexchange.com/questions/159479/problem-with-composite-letters-in-section-labels}
        \item Kaj je s tem headheight too small problemom?
        \ans \url{https://tex.stackexchange.com/questions/327285/what-does-this-warning-mean-fancyhdr-and-headheight}
        \item Ali pri naštevanju (enumerate) uporabimo nosep opcijo?
        \ans
        \item Pri naštevanju velike začetnice?
        \ans Odvisno od stavka.
    \end{vprasanja}

    \section*{Tehnična vprašanja}
    \begin{vprasanja}
        \item Ali lahko teorijo polj formuliramo v regularni logiki? Če ne, kak je dokaz?
        \ans
        \item Pri definiciji funkcijske relacije (na strani 8), kaj je ta kanonični morfizem $R \times_X R \to X \times Y \times Y$?
        \ans
        \item Ali je za definicijo kategorije regularnih kategorij potrebno da so te regularne kategorije majhne ali je dovolj lokalno majhne oz. dobro pogojene.
        \ans
        \item Kdaj je $\lbrace \overline{z} \mid \top \rbrace^{(M)} \leq \lbrace \overline{z} \mid \varphi \rbrace^{(M)}$?
        \ans Ko je $\lbrace \overline{z} \mid \varphi \rbrace^{(M)} \cong \overline{Z}$.
        \item Ko definiramo jezik regularne teorije ali dodamo za vsak podobjekt po en relacijski simbol in en funkcijski simbol,
            kajti vsak podobjekt je tudi morizem v kategoriji.
        \ans
        \item Kaj naredimo v primeru če imamo npr.\ objekt $\lbrace \cdot \mid p \rbrace$ z formulo $\exists y . p$?
            Ali je to $\exists \emptyset . p$ in ali je to enako samo $p$?
        \ans
    \end{vprasanja}

    \section*{Misc}
    \begin{vprasanja}
        \item Ali je več ``regularnih logik"?
        \ans
        \item Kaj točno je ``spremenljivka''?
        \ans
        \item Napišimo kaj točno pomeni formula.
        \ans
        \item Kako detajlni naj bodo dokazi?
        \ans
    \end{vprasanja}

\end{document}
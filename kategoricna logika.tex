\documentclass[12pt,a4paper]{book}
\usepackage[utf8x]{inputenc}   % omogoča uporabo slovenskih črk kodiranih v formatu UTF-8
\usepackage[slovene,english]{babel}    % naloži, med drugim, slovenske delilne vzorce
\usepackage[pdftex]{graphicx}  % omogoča vlaganje slik različnih formatov
\usepackage{fancyhdr}          % poskrbi, na primer, za glave strani
\usepackage{comment}
\usepackage[pdftex, colorlinks=true,
						citecolor=black, filecolor=black, 
						linkcolor=black, urlcolor=black,
						pagebackref=false, 
						pdfproducer={LaTeX}, pdfcreator={LaTeX}, hidelinks]{hyperref}
\usepackage{color}       % dodal Solina
%\usepackage{soul}
\usepackage{amsthm}
\usepackage{amsmath}
\usepackage{amsfonts}
\usepackage{amssymb}
%
\usepackage{amsmath}
\usepackage{mathtools}
\usepackage{enumitem}
\usepackage{commath}
\usepackage{array,xparse}
\usepackage{hyperref}
\usepackage{bm} % to make greek symbols bold
%
\usepackage{ulem} % ta knjižnica redefinira komando \emph
% ta komanda jo nastavi nazaj na privzeto vrednost
\normalem
%
\usepackage{tikz-cd} 
\usetikzlibrary{babel} % this fixes problems with tikz-cd
% http://tex.stackexchange.com/questions/166772/problem-with-babel-and-tikz-using-draw
%
%
%
%%%%%%%%%%%%%%%%%%%%%%%%%%%%%%%%%%%%%%%%
%	DIPLOMA INFO
%%%%%%%%%%%%%%%%%%%%%%%%%%%%%%%%%%%%%%%%
\newcommand{\ttitle}{Kategorična logika}
\newcommand{\ttitleEn}{Naslov EN}
\newcommand{\tsubject}{\ttitle}
\newcommand{\tsubjectEn}{\ttitleEn}
\newcommand{\tauthor}{Jure Taslak}
\newcommand{\tkeywords}{Teorija kategorij, Kategorična logika, kategorije, logika, algebraične teorije}
\newcommand{\tkeywordsEn}{Category theory, Categorical logic, categories, logic, algebraic theories}
%
%
%%%%%%%%%%%%%%%%%%%%%%%%%%%%%%%%%%%%%%%%
% naslovi
%%%%%%%%%%%%%%%%%%%%%%%%%%%%%%%%%%%%%%%%  
\newcommand{\autfont}{\Large}
\newcommand{\titfont}{\LARGE\bf}
\newcommand{\clearemptydoublepage}{\newpage{\pagestyle{empty}\cleardoublepage}}
\setcounter{tocdepth}{1}
%
%
%
%%%%%%%%%%%%%%%%%%%%%%%%%%%%%%%%%%%%%
%									%
% konstrukti						%
%									%
%%%%%%%%%%%%%%%%%%%%%%%%%%%%%%%%%%%%%  
%
\theoremstyle{definition}
\newtheorem{definicija}{Definicija}[chapter]
%
\theoremstyle{plain}
\newtheorem{izrek}[definicija]{Izrek}
\newtheorem{trditev}[definicija]{Trditev}
\newtheorem{posledica}{Posledica}[definicija]
\newtheorem{lema}[definicija]{Lema}
\newenvironment{dokaz}{\emph{Dokaz.}\ }{\hspace{\fill}{$\Box$}}
%
\theoremstyle{definition}
\newtheorem{primer}{Primer}[section]
\newtheorem*{primer*}{Primer}
%
\theoremstyle{remark}
\newtheorem*{opomba}{Opomba}
%
%%%%%%%%%%%%%%%%%%%%%%%%%%%%%%%%%
%								%
%	Tikz-cd nastavitve			%
%								%
%%%%%%%%%%%%%%%%%%%%%%%%%%%%%%%%%
%
\tikzcdset{
 	diagrams={sep=large},
	labels={font=\small}
}
%
%%%%%%%%%%%%%%%%%%%%%%%%%%%%%
% Avtor in naslov			%
%%%%%%%%%%%%%%%%%%%%%%%%%%%%%
%
\author{Jure Taslak}
\title{Kategorična logika}
%
%%%%%%%%%%%%%%%%%%%%%%%%%%%%%%%%%%%%%%%%
% pdfInfo
%%%%%%%%%%%%%%%%%%%%%%%%%%%%%%%%%%%%%%%%  
\pdfinfo{%
    /Title    (\ttitle)
    /Author   (\tauthor, damjan@cvetan.si)
    /Subject  (\ttitleEn)
    /Keywords (\tkeywordsEn)
    /ModDate  (\pdfcreationdate)
    /Trapped  /False
}
%
%%%%%%%%%%%%%%%%%%%%%%%%%%%%%%%%%%%%%
%									%
% 		Misc						%
%									%
%%%%%%%%%%%%%%%%%%%%%%%%%%%%%%%%%%%%%
%
\newcommand{\eqtext}[1]{\stackrel{\mathclap{\normalfont\mbox{#1}}}{=}} % write text over =
% could use a way to make the text smaller
%
\newcommand{\cat}[1]{\textbf{#1}}
\newcommand{\homset}[2]{\mathrm{Hom(#1,#2)}}
%
\DeclareMathOperator{\Hom}{Hom}
%\DeclareMathOperator[1]{\colim}{\underset{#1}{colim}}
\DeclareMathOperator{\colim}{colim}
%\newcommand[1]{\colim}{\underset{#1}\colim_op}
\DeclareMathOperator{\Fun}{Fun}
\DeclareMathOperator{\dom}{dom}
\DeclareMathOperator{\cod}{cod}
\DeclareMathOperator{\obj}{obj}
\DeclareMathOperator{\Mod}{Mod}
%
\renewcommand{\set}[1]{\{\,#1\,\}}
%
\newcommand{\fprod}[1]{\langle #1 \rangle}
%
\newcommand{\predsnop}[1]{\cat{Set}^{\cat{#1}^{op}}}
%
\newcommand{\powerset}[1]{\mathcal{P}(#1)}
%
%\DeclareMathOperator{\eval}{eval}
%
% \newcommand{\coprod}[2]{\cat{#1} + \cat{#2}}
%
%
%%%%%%%%%%%%%%%%%%%%%%%%%%%%%%%%%%%%%%%%%%
% Two-way rule for adjunction
%TODO make this more usefull
%%%%%%%%%%%%%%%%%%%%%%%%%%%%%%%%%%%%%%%%%%
\ExplSyntaxOn
\NewDocumentEnvironment{adjunctions}{O{}}
 {
  \cs_set_eq:cN {@arraycr} \farin_arraycr:
  \keys_set:nn { farin/adjunction } { #1 }
  \begin{array}
   {
    @{ \hspace { \dim_eval:n { \l_farin_left_shift_dim + \l_farin_padding_dim } } }
    r
    @{ {\farin_strut:} \l_farin_symbol_tl {} }
    l
    @{ \hspace { \dim_eval:n { \l_farin_right_shift_dim + \l_farin_padding_dim } } }
   }
 }
 {
  \end{array}
 }
\keys_define:nn { farin/adjunction }
 {
  leftshift       .dim_set:N = \l_farin_left_shift_dim,
  leftshift       .initial:n = 0pt,
  rightshift      .dim_set:N = \l_farin_right_shift_dim,
  rightshift      .initial:n = 0pt,
  padding         .dim_set:N = \l_farin_padding_dim,
  padding         .initial:n = 6pt,
  symbol          .tl_set:N  = \l_farin_symbol_tl,
  symbol          .initial:n = \longrightarrow,
  verticalspacing .dim_set:N  = \l_farin_vertspac_dim,
  verticalspacing .initial:n = {3pt},
 }
\cs_new_protected:Npn \farin_strut:
 {
  \vrule height \dim_eval:n { \ht\strutbox + 1.2\l_farin_vertspac_dim }
         depth  \dim_eval:n { \dp\strutbox + \l_farin_vertspac_dim }
         width 0pt
 }
\makeatletter
\exp_args:NNo \cs_new:Npn \farin_arraycr:
 {
  \@arraycr\hline
 }
\makeatother
\ExplSyntaxOff
%
%
%%%%%%%%%%%%%%%%%%%%%%%%%%%%%%%%%%%%%%%%%%
%%%%%%%%%%%%%%%%%%%%%%%%%%%%%%%%%%%%%%%%%%
%
\begin{document}
\selectlanguage{slovene}
\frontmatter
\setcounter{page}{1} %
\renewcommand{\thepage}{}       % preprecimo težave s številkami strani v kazalu
\newcommand{\sn}[1]{"`#1"'}                    % dodal Solina (slovenski narekovaji)
%
%
%
%%%%%%%%%%%%%%%%%%%%%%%%%%%%%%%%%%%%%%%%
%naslovnica
 \thispagestyle{empty}%
   \begin{center}
    {\large\sc Univerza v Ljubljani\\%
      Fakulteta za matematiko in fiziko}%
    \vskip 10em%
    {\autfont \tauthor\par}%
    {\titfont \ttitle \par}%
    {\vskip 3em \textsc{MAGISTRSKO DELO\\[5mm]
    UNIVERZITETNI\\ ŠTUDIJSKI PROGRAM DRUGE STOPNJE\\ MATEMATIKA}\par}%
%
    \vfill\null%
    {\large \textsc{Mentor}: prof.\ dr.  Andrej Bauer\par}%
    {\vskip 2em \large Ljubljana, 2020 \par}%
\end{center}
% prazna stran
%\clearemptydoublepage      % dodal Solina (izjava o licencah itd. se izpiše na hrbtni strani naslovnice)
%
%%%%%%%%%%%%%%%%%%%%%%%%%%%%%%%%%%%%%%%%
%copyright stran
\thispagestyle{empty}
\vspace*{8cm}
%
\noindent
{\sc Copyright}. 
Rezultati diplomske naloge so intelektualna lastnina avtorja in Fakultete za računalništvo in informatiko Univerze v Ljubljani.
Za objavo in koriščenje rezultatov diplomske naloge je potrebno pisno privoljenje avtorja, Fakultete za računalništvo in informatiko ter mentorja.
%
\begin{center}
\mbox{}\vfill
\emph{Besedilo je oblikovano z urejevalnikom besedil \LaTeX.}
\end{center}
% prazna stran
\clearemptydoublepage
%
%
%%%%%%%%%%%%%%%%%%%%%%%%%%%%%%%%%%%%%%%%
% stran 3 med uvodnimi listi
\thispagestyle{empty}
\vspace*{4cm}
%
\noindent
Fakulteta za računalništvo in informatiko izdaja naslednjo nalogo:
\medskip
\begin{tabbing}
\hspace{32mm}\= \hspace{6cm} \= \kill
%
%
Tematika naloge:
\end{tabbing}
Delo obravnava Yonedovo lemo, ki je eden od osrednjih izrekov teorije kategorij. Predstavljene bodo tudi nekatere aplikacije Yonedove leme.
\vspace{15mm}
%
\vspace{2cm}
%
% prazna stran
\clearemptydoublepage
%
% zahvala
\thispagestyle{empty}\mbox{}\vfill\null\it%
\noindent
Zahvaljujem se mentorju prof. dr. Andreju Bauerju za usmerjanje in vso ostalo pomoč. \\
Svojim staršem, za vso podporo. \\
\rm\normalfont
%
% prazna stran
\clearemptydoublepage
%
%%%%%%%%%%%%%%%%%%%%%%%%%%%%%%%%%%%%%%%%
% kazalo
\pagestyle{empty}
\def\thepage{}% preprecimo tezave s stevilkami strani v kazalu
\tableofcontents{}
%
% prazna stran
\clearemptydoublepage
%
%
%%%%%%%%%%%%%%%%%%%%%%%%%%%%%%%%%%%%%%%%
% povzetek
\addcontentsline{toc}{chapter}{Povzetek}
\chapter*{Povzetek}
%
\noindent\textbf{Naslov:} \ttitle
\bigskip
%
\noindent\textbf{Avtor:} \tauthor
\bigskip
%
%\noindent\textbf{Povzetek:} 
\noindent
V diplomskem delu je obravnavana Yonedova lema, ki velja za enega osrednjih izrekov teorije kategorij. Uvodni del definira osnovne pojme v teoriji kategorij, ki so kasneje uporabljeni za formulacijo in dokaz leme. Skozi besedilo je predstavljen tudi kategorični način razmišljanja, ki nam omogoča specifično situacijo, ki jo srečamo v matematiki, obravnavati bistveno bolj splošno, z uporabo kategoričnih metod.
V zaključnem poglavju je predstavljena in dokazana Yonedova lema, ki nam poda način za obravnavo kategorije z obravnavo njene vložitve v kategorijo funktorjev. Predstavljenih je tudi nekaj primerov uporabe leme.
\bigskip
%
\noindent\textbf{Ključne besede:} \tkeywords.
% prazna stran
\clearemptydoublepage
%
%%%%%%%%%%%%%%%%%%%%%%%%%%%%%%%%%%%%%%%%
% abstract
\selectlanguage{english}
\addcontentsline{toc}{chapter}{Abstract}
\chapter*{Abstract}
%
\noindent\textbf{Title:} \ttitleEn
\bigskip
%
\noindent\textbf{Author:} \tauthor
\bigskip
%\noindent\textbf{Abstract:} 
%
\noindent 
The thesis discusses the Yoneda lemma, which is considered one of the central theorems in category theory. The introduction defines the basic concepts of category theory, which are later used to formulate and prove the lemma. Throughout the text there are examples of the categorical way of thinking, where we take a specific situation, that we encounter in mathematics and look at it in a more general setting. In the final chapter we describe and prove the Yoneda lemma, that presents a way of studying a category, by studying its inclusion into a category of functors. We show some use cases of the lemma.
%
\bigskip
%
\noindent\textbf{Keywords:} \tkeywordsEn.
\selectlanguage{slovene}
% prazna stran
\clearemptydoublepage
%
\mainmatter
\setcounter{page}{1}
\pagestyle{fancy}
%
\chapter{Uvod}
%
\section{Osnovne Definicije}
%
\section{Splošni izreki}
\begin{lema}
Naj bo $\cat{C}$ poljubna kategorija in $f : X \to Y$ morfizem. Denimo, da v $\cat{C}$ obstaja produkt $X \times Y$. Potem je projekcija $X \times Y \to X$ epimorfizem.
\end{lema}
\begin{dokaz}
Naj bo $p : X \times Y \to X$ projekcija iz produkta in $u = \langle id_X, f \rangle : X \to X \times Y$. Recimo, da imamo dva vzporedna morfizma $g,h : X \to Z$ za katera velja $g \circ p = h \circ p$. To prikažemo v diagramu
\begin{center}
\begin{tikzcd}
& X \ar[d, "u", dashed] \ar[dr, "id_X"] \ar[dl, "f"'] \\
Y & X\times Y \ar[l] \ar[r, "p"] & X \ar[r, shift left, "g"] \ar[r, shift right, "h"'] & Z
\end{tikzcd}
\end{center}
Sedaj velja
\begin{align*}
g \circ p &= h \circ p \\
g \circ p \circ u &= h \circ p \circ u \\
g \circ id_X &= h \circ id_X \\
g &= h
\end{align*}
\end{dokaz}
\begin{izrek} [Adjoint functor theorem]
Naj bo $\cat{C}$ lokalno majhna in polna. Za vsako kategorijo $\cat{X}$ in funktor 
$$U : \cat{X} \to \cat{C}$$
ki ohranja limite, so naslednje trditve ekvivalentne:
\begin{enumerate}
\item $U$ ima levi adjunkt
\item Za vsak objekt $X \in \cat{X}$ funktor $U$ zadošča kriteriju množice rešitev:
Obstaja množica $(C_i)_{i \in I}$ objektov v $\cat{C}$ tako, da za vsak objekt $C \in \cat{C}$ in vsak morfizem $f : X \to UC$ obstaja indeks $i \in I$ in morfizma $\phi : X \to UC_i$ in $\overline{f} : C_i \to C$, da velja
$$f = U(\overline{f}) \circ \phi$$
\[ \begin{tikzcd}
X \arrow[r, "\phi"] \arrow[rd, "f"'] & UC_i \arrow[d, "U(\overline{f})"] \\
& UC
\end{tikzcd} \]
\end{enumerate}
\end{izrek}
\begin{lema} \label{lema2}
Funktor $U : \cat{C} \to \cat{X}$ ima levi adjunkt natanko takrat, ko ima comma-kategorija $(X \downarrow U)$ začetni objekt.
\end{lema}
\begin{proof}
Naj bo $F : \cat{X} \to \cat{C}$ levi adjunkt za $U$ in $\eta : 1_{\cat{X}} \to UF$ enota adjunkcije. Potem je $(FX, \eta_X : X \to UFX)$ začetni objekt v $(X \downarrow U)$. To velja, saj če je $(C, f : X \to UC)$ objekt v $(X \downarrow U)$, potem po UMP enote obstaja natanko en $g : FX \to C$, da diagram
\[ \begin{tikzcd}
X \arrow[d, "\eta_X"'] \arrow[dr, "f"] & \\
UFX \arrow[r, "Ug"'] & UC
\end{tikzcd} \]
komutira. To je ravno enolični morfizem $(FX,\eta_X) \to (C,f)$. \\
Obratno, denimo da imamo začetni objekt v $(X \downarrow U)$, ki ga pomenljivo označimo kar z $(FX, \eta_X : X \to U(FX))$. Zaradi obstoja in enoličnosti takega objekta nam to določi funkcijo objektov 
$$F : \obj(\cat{X}) \to \obj(\cat{C})$$
Eksplicitno $F(X) = \text{prva komponenta začetnega objekta v } (X \downarrow U)$. Ta $F$ želimo razširito do funktorja, zato naj bo $f : X \to X'$ morfizem v $X$. Morfizem $F(f)$ naj bo enolični morfizem, tako da kvadrat
\[ \begin{tikzcd}
X \arrow[d, "\eta_X"'] \arrow[r, "f"] & X' \arrow[d, "\eta_{X'}"] \\
UFX \arrow[r, "UFf"', dashed] & UFX'
\end{tikzcd} \]
komutira. Jasno je, da $F$ slika domeno morfizma v domeno ter kodomeno v kodomeno. Za asociativnost denimo, da imamo morfizma 
\begin{center}
\begin{tikzcd}
X \arrow[r, "f"] & X' \arrow[r, "f'"] & X''
\end{tikzcd}
\end{center}
To nam da enolična morfizma 
\begin{center}
\begin{tikzcd}
FX \arrow[r, "Ff", dashed] & FX' \arrow[r, "Ff'", dashed] & X''
\end{tikzcd}
\end{center}
zaradi katerih diagram
\begin{center}
\begin{tikzcd}
X \ar[d, "\eta_X"'] \ar[r, "f"] & X' \ar[d, "\eta_{X'}"'] \ar[r, "f"] & X'' \ar[d, "\eta_{X''}"] \\
UFX \ar[r, "UFf"', dashed] & UFX' \ar[r, "UFf'"', dashed] & UFX''
\end{tikzcd}
\end{center}
komutira, kar določi kompozitum kot
$$F(f' \circ f) = F(f') \circ F(f).$$
Od tod tudi sledi, da je $\eta$ naravna transformacija, določena s komponentami $\eta_X$, ki zadošča UMP enote. Torej je $F$ res levi adjunkt $U$.
\end{proof}
\begin{definicija}
\mbox{}
\begin{enumerate}[label=(\roman*)]
\item Objekt $C \in \cat{C}$ je \emph{šibko začeten}, če za vsak $X \in \cat{C}$ obstaja morfizem $C \to X$ (ki ni nujno enoličen).
\item Zbirka objektov $\Phi = (C_i)_{i \in I}$ je \emph{skupno šibko začetna}, če za vsak $X \in C$ obstaja nek $C_i \in \Phi$, za katerega obstaja morfizem $C_i \to X$.
\end{enumerate}
\end{definicija}
\begin{lema}
Naj bo $\cat{C}$ kategorija za katero ima identitetni funktor $1_\cat{C} : \cat{C} \to \cat{C}$ limito. Potem ima $\cat{C}$ začetni objekt.
\end{lema}
\begin{proof}
Naj bo $l = \lim (1_\cat{C})$ z zbirko morfizmov $(\lambda_C)_{C \in \cat{C}}$. Iz definicije limite sledi, da je $l$ šibko začeten objekt. Recimo, da imamo še en morfizem $f : l \to C$. Ker je $l$ stožec sledi
$$ \lambda_C  = f \circ \lambda_l$$
Če pa vzamemo $\lambda_C$ kot morfizem diagrama $1_\cat{C}$, dobimo enačbo
$$\lambda_C = \lambda_C \circ \lambda_l$$
Iz diagrama
\begin{center}
\begin{tikzcd}
& l \ar[ddl, "1_l"', bend right] \ar[d, "u", dashed] \ar[ddr, "\lambda_C", bend left] & \\
& l \ar[dl, "\lambda_l"'] \ar[dr, "\lambda_C"]  & \\
l \ar[rr, "\lambda_C"'] & & C
\end{tikzcd}
\end{center}
dobimo enačbi $1_l = \lambda_l \circ u$ in $\lambda_C = \lambda_C \circ u$. Iz enoličnosti morfizma $u$ sledi $u = 1_l$, kar pomeni $\lambda_C = f$.
\end{proof}
\begin{lema} \label{lema1}
Naj bo $\cat{C}$ lokalno majhna, polna kategorija, ki ima skupno šibko začetno množico objektov $\Phi = (C_i)$. Potem ima $\cat{C}$ začetni objekt.
\end{lema}
\begin{proof}
Naj bo $\iota : \bm{\Phi} \to \cat{C}$ inkluzija polne podkategorije $\cat{C}$ generirane z objektov iz $\Phi$. Ker je $\bm{\Phi}$ majhna in $\cat{C}$ polna, obstaja limita
$$\ell = \lim \iota$$
Ker je množica $\Phi$ skupno šibko začetna imamo za vsak objekt $X \in \cat{C}$ morfizem $C_i \to X$ za nek $C_i \in \Phi$. Če komponiramo z ustreznim $l_i : \ell \to C_i$ dobimo morfizem
$$\lambda_X : \ell \to X$$
kar pomeni, da je $\ell$ šibko začetni objekt. Sedaj želimo pokazati, da morfizmi $\lambda_X$ določajo stožec nad $1_\cat{C}$ z vrhom $\ell$ in lastnostjo, da je $\lambda_l$ identiteta. Naj bo torej $f : X \to Y$ morfizem v $\cat{C}$. Ker je $\Phi$ šibko začetna, imamo morfizma $c_1 : C_1 \to X$ in $c_2 : C_2 \to Y$. Tvorimo povlek $P$ morfizmov $f \circ c_1$ in $c_2$
\begin{center}
\begin{tikzcd}
\ell \ar[dddrr, bend right, "l_1"'] \ar[ddrrrr, bend left, "l_2"] \ar[dr, dashed, "u"] & & & & \\
& C_3 \ar[ddr, bend right, dotted] \ar[drrr, bend left, dotted] \ar[dr, "h"] & & & \\
& & P \ar[d, "p_1"'] \ar[rr, "p_2"] & & C_2 \ar[d, "c_2"] \\
& & C_1 \ar[r, "c_1"'] & X \ar[r, "f"'] & Y
\end{tikzcd}
\end{center}
Ker je $\Phi$ šibko začetna, obstaja nek morfizem $h : C \to P$. Črtkana kompozituma $p_1 \circ h$ in $p_2 \circ h$ živita v polni podkategoriji generirani s $\Phi$, kar nam pove, da zgornja trikotnika komutirata. Iz komutativnosti ostalega dela diagrama vidimo, da je $\ell$ res vrh stožca nad $1_\cat{C}$. Iz tega sledi komutitativnost diagrama
\begin{center}
\begin{tikzcd}
& \ell \ar[dl, "\lambda_\ell"'] \ar[dr, "l_C = \lambda_C"] & \\
\ell \ar[rr, "\lambda_C"'] & & C
\end{tikzcd}
\end{center}
kar pomeni, da je $\lambda_\ell$ faktorizacija limitnega stožca skozi samega sebe, iz česar sledi $\lambda_\ell = 1_\ell$. Na dolgo, imamo komutativni diagram
\begin{center}
\begin{tikzcd}
& \ell \ar[ddl, bend right, "\lambda_\ell"'] \ar[ddr, bend left, "\lambda_C"] \ar[d, dashed, "u"] & \\
& \ell \ar[dl, "\lambda_\ell"'] \ar[dr, "\lambda_C"] & \\
\ell \ar[rr, "\lambda_C"'] & & C
\end{tikzcd}
\end{center}
iz spodnjega trikotnika imamo enačbo
$$\lambda_C = \lambda_C \circ \lambda_\ell$$
Ker pa je $\ell$ limita, dobimo enoličen $u$ za katerega velja
$$\lambda_\ell = \lambda_\ell \circ u \quad \& \quad \lambda_C = \lambda_C \circ u$$
Ker je $u$ enoličen tak, ki zadošča tema enačbama in je $1_\ell$ tudi tak, sledi $\lambda_\ell = 1_\ell$. Iz leme \ref{lema1} sledi, da ima $\cat{C}$ začetni objekt.
\end{proof}
Z uporabo teh lem lahko sedaj dokažemo izrek.
\begin{proof}
Če ima $U$ levi adjunkt, potem je množica $\set{FX}$ zadošča kriteriju množice rešitev.\\
Obratno, po lemi \ref{lema2} ima $U$ levi adjunkt natanko takrat, ko ima za vsak $X$ kategorija $(X \downarrow U)$ začetni objekt. Preveriti moramo torej
\begin{enumerate}
\item $(X \downarrow U)$ je lokalno majhna
\item $(X \downarrow U)$ je polna 
\item $(X \downarrow U)$ ima skupno šibko začetni objekt
\end{enumerate}
Ker je $\cat{C}$ lokalno majhna, je tudi $(X \downarrow U)$. Iz predpostavke sledi, da je
$$\set{(C_i, \phi : X \to UC_i) \ \vert \ i \in I}$$
skupno šibko začetni objekt. Da je $(X \downarrow U)$ polna pokažemo tako, da konstruiramo produkte in zožke, iz česar sledi obstoj limit. 
\end{proof}
%
\section{Algebrajske teorije}
%
\begin{primer}[Teorija grup]
Grupo lahko razumemo kot množico $G$ skupaj z binarno operacijo $\cdot : G \times G \to G$, ki zadošča aksiomoma:
%
\begin{enumerate}
\item $\forall x,y,z \in G . \quad (x\cdot y) \cdot z = x \cdot (y \cdot z)$
\item $\exists e \in G . \forall x \in G . \exists y \in G . \quad (e \cdot x = x \cdot e = x \wedge x \cdot y = y \cdot x = e)$
\end{enumerate}
\end{primer}
Ker pa sta enota in inverz iz teh aksiomov enolično določena, ju lahko dodamo k strukturi, kot odlikovan element $e \in G$ in preslikavo $^{-1} : G \to G$.
%
Formulacijo grupe lahko sedaj podamo samo z enačbami:
\vspace{1em}
\begin{align*}
&x \cdot (y \cdot z) = (x \cdot y) \cdot z \\
&x \cdot e = e \cdot x = x \\
&x \cdot x^{-1} = x^{-1} \cdot x = e \\
\end{align*}
Univerzalni kvantifikator $\forall x \in G$ je odveč, saj si interpretiramo vse spremenljivke, kot da tečejo po množici $G$. \\
Pri tem opisu ne rabimo eksplicitno omeniti specifične množice $G$.
%
\begin{definicija}
\emph{Jezik algebraične teorije} $\Sigma$ je sestavljen iz družine množic $\lbrace \Sigma_k \rbrace$, kjer se elementi $\Sigma_k$ imenujejo $k$-mestne osnovne operacije. Terme jezika $\Sigma$ tvorimo induktivno:
\begin{enumerate}
\item Spremenljivke $x,y,z, \ldots$
\item Če so $t_1, \ldots, t_k$ že termi in $f \in \Sigma_k$, potem je $f(t_1,\ldots, t_k)$ tudi term.
\end{enumerate}
\end{definicija}
%
\begin{definicija}
\emph{Algebraična teorija} $\mathbb{T} = (\Sigma_\mathbb{T}, A_\mathbb{T})$ je podana z jezikom $\Sigma_\mathbb{T}$ in množico $A_\mathbb{T}$, enačb med termi teorije, ki jih imenujemo \emph{aksiomi} teorije $\mathbb{T}$
\end{definicija}
%
\begin{primer}
Grupo lahko podamo kot množico $G$, skupaj s funkcijami $e : 1 = \lbrace * \rbrace \to G$, $m : G \times G \to G$ in $i : G \to G$, ki izpolnjujejo aksiome
\begin{description}
\item $m(x,m(y,z)) = m(m(x,y),z)$
\item $m(x,e) = m(e,x) = x$
\item $m(x,i(x)) = m(i(x),x) = e$
\end{description}
\end{primer}
%
Te aksiome lahko predstavimo kot komutativne diagrame
\begin{center}
\begin{tikzcd}[column sep = large]
G \times G \times G \ar[d, "\pi_0 \times m"'] \ar[r, "m \times \pi_2"] & G \times G \ar[d, "m"] \\
G \times G \ar[r, "m"] & G
\end{tikzcd}
\end{center}
\begin{center}
\begin{tikzcd}[column sep = large]
G \times 1 \ar[dr, "\pi_0"'] \ar[r, "1_G \times e"] & G \times G \ar[d, "m"] & 1 \times G \ar[l, "e \times 1_G"'] \ar[dl, "\pi_1"] \\
& G 
\end{tikzcd}
\end{center}
\begin{center}
\begin{tikzcd}[column sep = large]
G \ar[d, "!"'] \ar[r, "{\langle 1_G, i \rangle}"] & G \times G \ar[d, "m"] & G \ar[l, "{\langle i, 1_G \rangle}"'] \ar[d, "!"] \\
1 \ar[r, "e"] & G & 1 \ar[l, "e"]
\end{tikzcd}
\end{center}
%
To lahko sedaj izrazimo v vsaki kategoriji, ki ima končne produkte.
%
Grupa v kategoriji $\mathcal{C}$ s končnimi produkti je torej objekt $G$ skupaj z morfizmi
\begin{center}
\begin{tikzcd}
G \times G \ar[r, "m"] & G & G \ar[l, "i"'] \\
& 1 \ar[u, "e"] & 
\end{tikzcd}
\end{center}
za katere zgornji diagrami komutirajo.
%
Podobno posplošimo homomorfizem med grupami v $\mathcal{C}$. Morfizem $h : G \to H$ je homomorfizem grup, če komutira z interpretacijami vseh osnovnih operacij $m$, $i$ in $e$.
%
\begin{center}
\begin{tikzcd}
G^2 \ar[d, "m^G"'] \ar[r, "h^2"] & H^2 \ar[d, "m^H"] & & G \ar[d, "i^G"'] \ar[r, "h"] & H \ar[d, "i^H"] & & 1 \ar[d, "e^G"'] \ar[r] & 1 \ar[d, "e^H"] \\
G \ar[r, "h"] & H & & G \ar[r, "h"] & H & & G \ar[r, "h"] & H 
\end{tikzcd}
\end{center}
Ali z enačbami
\begin{description}
\item $m^H \circ h^2 = h \circ m^G$
\item $i^H \circ h = h \circ i^G$
\item $e^H = h \circ e^G$
\end{description}
%
\section{Model algebrajske teorije}
%
\begin{definicija}
Naj bo $\mathcal{C}$ kategorija s končnimi produkti. \emph{Interpretacija} $I$ teorije $\mathbb{T}$ je sestavljena iz
\begin{itemize}
%
\item Objekta $I \in \mathcal{C}$
%
\item Za vsako $k$-mestno osnovno operacijo $f$ morfizem $f^I : I^k \to I$
\end{itemize}
%
Interpretacijo razširimo na vse terme jezika s \emph{kontekstom}. Splošen term $b$ se interpretira skupaj s kontekstom spremenljivk $x_1, \ldots, x_n$, kjer v $b$ nastopajo le spremenljivke iz tega konteksta. To označimo z
$$x_1, \ldots, x_n \mid b$$
\end{definicija}
%
\begin{definicija}
Interpretacija terma $b^I$ je definirana reukrzivno:
\begin{enumerate}
\item Interpretacija spremenljivke $x_i$ je $i$-ta projekcija $\pi_i : I^n \to I$
%
\item Term oblike $f(t_1, \ldots, t_k)$ se interpretira kot kompozitum 
\begin{center}
\begin{tikzcd}[column sep = huge]
I^n \ar[r, "{(t_1^I, \ldots, t_k^I)}"] & I^k \ar[r, "f^I"] & I
\end{tikzcd}
\end{center}
kjer je $t_i^I : I^n \to I$ interpretacija terma $t_i$, za $i = 1, \ldots, k$ in je $f^I$ interpretacija osnovne operacije $f$.
\end{enumerate}
\end{definicija}
%
\begin{primer}
Interpretacija je odvisna od konteksta!\\
Term $f(x_1)$ se v kontekstu $x_1$ interpretira kot morfizem $f^I : I \to I$, medtem ko se v kontekstu $x_1, x_2$ interpretira kot $f^I \circ \pi_1 : I^2 \to I$.
\end{primer}
\vspace{1cm}
Če je kontekst jasen potem pišemo $[x_1, \ldots, x_n \mid b]^I = b^I$.
%
\begin{definicija}
Naj bosta $s$ in $t$ terma v kontekstu $x_1, \ldots, x_n$. 
\begin{itemize}
\item V interpretaciji $I$ je enačba $s = t$ \emph{zadoščena}, če sta morfizma $s^I$ in $t^I$ isti morfizem v $\mathcal{C}$
\end{itemize}
%
Specifično, če je $s = t$ aksiom teorije in so $x_1, \ldots, x_n$ vse spremenljivke, ki nastopanjo v $s$ in $t$, potem pravimo, da je v interpretaciji $I$ \emph{zadoščen aksiom} $s = t$, če sta $[x_1, \ldots, x_n \mid s]^I$ in $[x_1, \ldots, x_n \mid t]^I$ isti morfizem
\begin{center}
\begin{tikzcd}[column sep = huge]
I^n \ar[r, shift left=1ex, "{[x_1, \ldots, x_n \mid s]^I}"] \ar[r, shift right=1ex, "{[x_1, \ldots, x_n \mid t]^I}"'] & I
\end{tikzcd}
\end{center}
%
To označimo kot
$$I \models s = t \Longleftrightarrow s^I = t^I$$
\end{definicija}
%
\begin{definicija}
Naj bo $\mathbb{T}$ algebraična teorija. 
\begin{itemize}
%
\item \emph{Model} teorije $\mathbb{T}$ v kategoriji $\mathcal{C}$ s končnimi produkti je interpretacija $M$, ki zadošča vsem aksiomom teorije $\mathbb{T}$.
$$M \models s = t$$ za vsak $(s = t) \in A_\mathbb{T}$.
\end{itemize}
\end{definicija}
%
\begin{definicija}
\begin{itemize}
\item \emph{Homomorfizem} modelov $h : M \to N$ je morfizem v $\mathcal{C}$, ki komutira z interpretacijami osnovnih operacij,
$$h \circ f^M = f^N \circ h$$
za vsak $f \in \Sigma_T$, kar ponazorimo v diagramu
\begin{center}
\begin{tikzcd}
M^k \ar[d, "f^M"'] \ar[r, "h^k"] & N^k \ar[d, "f^N"] \\
M \ar[r, "h"] & N
\end{tikzcd}
\end{center}
\end{itemize}
\end{definicija}
%
Kategorijo modelov teorije $\mathbb{T}$ označimo z $\Mod(\mathbb{T}, \mathcal{C})$.
%
\begin{primer}
Model prazne teorije $\mathbb{T}_0$ je objekt $M \in \mathcal{C}$, morfizem med dvema modeloma pa je le morfizem v $\mathcal{C}$ brez dodatnih omejitev, torej
$$\Mod(\mathbb{T}_0, \mathcal{C}) = \mathcal{C}$$
%
Model teorije grup $\mathbb{T}_{Grp}$ v kategoriji množic $\mathbf{Set}$ je grupa v običajnem smislu
$$\Mod(\mathbb{T}_{Grp}, \mathbf{Set}) = \mathbf{Grp}$$
%
Kaj je grupa v kategoriji $\mathbf{Top}$, $\mathbf{Graph}$, $\mathbf{Grp}$?
\end{primer}
%
Alternativna predstavitev teorije grup z enoto $e$ in binarno operacijo $\odot$, ki se imenuje dvojno deljenje in enim samim aksiomom:
$$x \odot (((x \odot y ) \odot z ) \odot ( z \odot e))) \odot (e \odot e) ) = z$$
Običajne operacije teorije grup so povezane prek formul
$$x \odot y = x^{-1} \cdot y^{-1} \text{,} \quad x^{-1} = x \odot e \text{,} \quad x \cdot y = (x \odot e) \odot (y \odot e)$$
%
Želimo definicijo algebraične teorije, ki je neodvisna od predstavitve.
%
\begin{definicija}
\emph{Sintaktična kategorija} $\mathcal{C}_\mathbb{T}$ algebraične teorije $\mathbb{T}$.
%
\begin{itemize}
\item Objekti: konteksti $[x_1, \ldots, x_n]$ za $n \geq 0$
%
\item Morfizmi: morfizem $[x_1, \ldots, x_m] \to [x_1, \ldots, x_n]$ je $n$-terica $(t_1, \ldots, t_n)$ termov v kontekstu $x_1, \ldots, x_m$. Morfizma $(t_1, \ldots t_n)$, $(s_1, \ldots, s_n)$ sta enaka, če in samo če, za vsak $k = 1, \ldots, n$ v teoriji $\mathbb{T}$ velja $t_k = s_k$
$$\mathbb{T} \vdash t_k = s_k$$
\end{itemize}
\end{definicija}
%
Morfizmi so torej v resnici ekvivalenčni razredi termov v kontekstu
$$[x_1, \ldots, x_m \mid t_1, \ldots, t_n] : [x_1, \ldots, x_m] \to [x_1, \ldots, x_n]$$
%
Kompozitum morfizmov
\begin{center}
\begin{description}
\item $(t_1, \ldots, t_m) : [x_1, \ldots, x_k] \to [x_1, \ldots, x_m]$,
\item $(s_1, \ldots, s_n) : [x_1, \ldots, x_m] \to [x_1, \ldots, x_n]$
\end{description}
\end{center}
je morfizem $(r_1, \ldots, r_n)$, kjer dobimo $i$-to kompotneto tako, da v $s_i$ simultano vstavimo terme $t_1, \ldots, t_m$ namesto spremenljivk $x_1, \ldots, x_m$.
$$r_i = s_i[t_1, \ldots, t_m / x_1, \ldots, x_m]$$
%
Različne predstavitve teorije porodijo ekvivalentne sintaktične kategorije.
%
\begin{lema}
Naj bo $\mathbb{T}$ algebraična teorija in $\mathcal{C}_\mathbb{T}$ njena sintaktična kategorija. Potem ima $\mathcal{C}_\mathbb{T}$ vse končne produkte in velja
$$[x_1, \ldots, x_n] \times [x_1, \ldots, x_m] \cong [x_1, \ldots, x_{n+m}]$$
\end{lema}
%
%
\section{Modeli kot funtorji}
%
Naj bo $\mathbb{T}$ algebraična teorija in $\mathcal{C}$ kategorija s končnimi produkti.\\
Obstaja naravna ekvivalenca med modeli teorije $\mathbb{T}$ v $\mathcal{C}$ in funktorji, ki ohranjajo končne produkte iz $\mathcal{C}_\mathbb{T}$ v $\mathcal{C}$.
$$M \in \Mod(\mathbb{T}, \mathcal{C}) \leftrightsquigarrow \mathcal{M} : \mathcal{C}_\mathbb{T} \to \mathcal{C}$$
%
Vsak model je podan kot slika \emph{univerzalnega modela} $\mathcal{U}$ v $\mathcal{C}_T$, tako da velja $M \cong \mathcal{M}(\mathcal{U})$.
%
\begin{itemize}
\item Za objekt vzamemo $\mathcal{U} := [x_1]$, kontekst dolžine ena.
%
\item Interpretacijo $k$-mestne osnovne operacije $f$ definiramo kot samo sebe
$$f^\mathcal{U} := [x_1, \ldots, x_n \mid f(x_1, \ldots, x_n)] : \mathcal{U}^k = [x_1, \ldots, x_n] \to \mathcal{U} = [x_1]$$
\end{itemize}
%
Aksiomi $\mathbb{T}$ so izpolnjeni, saj za vsaka terma $t$, $s$ velja
$$\mathcal{U} \models t = s \Longleftrightarrow t^\mathcal{U} = s^\mathcal{U} \Longleftrightarrow \mathbb{T} \vdash t = s$$
Velja torej $\mathcal{U} \in \Mod(\mathbb{T}, \mathcal{C}_\mathbb{T})$.
%
Denimo sedaj, da imamo funktor $F : \mathcal{C}_\mathbb{T} \to \mathcal{C}$, ki ohranja končne produkte. \\
Potem je slika $F(\mathcal{U})$, skupaj z interpretacijami $f^{F(\mathcal{U})} := F(f^\mathcal{U})$ model v $\mathcal{C}$, saj za vsak aksiom $s = t$ velja
$$\mathbb{T} \vdash s = t \quad \Longleftrightarrow \quad s^\mathcal{U} = t^\mathcal{U}$$
%
iz česar zaradi funktorialnosti $F$ sledi
$$t^{F\mathcal{U}} = F(t^\mathcal{U}) = F(s^\mathcal{U}) = s^{F\mathcal{U}}$$
%
Vsaka naravna transformacija $\vartheta : F \to G$ med takima funktorjema določi homorfizem modelov $$h = \vartheta_\mathcal{U} : F\mathcal{U} \to G\mathcal{U}$$
%
saj za vsako osnovno operacijo $f$ zaradi naravnosti velja
$$h \circ f^{F\mathcal{U}} = f^{G\mathcal{U}} \circ h$$
kar lahko razberemo iz diagrama
\begin{center}
\begin{tikzcd}[column sep = large, row sep = large]
F\mathcal{U}  \times F\mathcal{U} \ar[d] \ar[r, "h \times h"] \ar[dd, bend right=60, "f^{F\mathcal{U}}"'] & G\mathcal{U} \times G\mathcal{U} \ar[d] \ar[dd, bend left = 60, "f^{G\mathcal{U}}"] \\
F(\mathcal{U} \times \mathcal{U}) \ar[d, "F(f)"] \ar[r, "\vartheta_{\mathcal{U} \times \mathcal{U}}"] & G(\mathcal{U} \times \mathcal{U}) \ar[d, "G(f)"] \\
F\mathcal{U} \ar[r, "h"] & G\mathcal{U}
\end{tikzcd}
\end{center}
%
\section{Polnost algebrajske teorije}
%
Evaluacija pri $\mathcal{U}$ nam torej določa funktor
$$eval_\mathcal{U} : \Hom_{FP}(\mathcal{C}_\mathbb{T}, \mathcal{C}) \to \Mod(\mathbb{T}, \mathcal{C}) \text{,} \quad F \mapsto F(\mathcal{U})$$
%
Pokazali bomo, da določa ekvivalenco kategorij, naravno v $\mathcal{C}$.\\
Za to je dovolj pokazati, da je $eval_\mathcal{U}$:
\begin{itemize}
\item surjektiven na objektih
\item zvest
\item poln
\end{itemize}
Za surjektivnost denimo, da imamo model $M \in \Mod(\mathbb{T}, \mathcal{C})$. Potem lahko definiramo funktor 
$$M^{\#} : \mathcal{C}_\mathbb{T} \to \mathcal{C}$$
ki na objektih deluje kot $M^{\#}([x_1, \ldots, x_n]) = M^n$, na morfizmih pa
$$M^{\#}(t_1, \ldots, t_n) = (t_1^M, \ldots, t_n^M)$$
%
%
Bolj natančno, je $M^{\#}$ na morfizmih definiran induktivno:
\begin{enumerate}
\item Morfizem $$x_i : [x_1, \ldots, x_n] \to [x_1]$$ se slika v $i$-to projekcijo $$\pi_i : M^n \to M$$
\item Morfizem $$f(t_1, \ldots, t_n) : [x_1, \ldots, x_n] \to [x_1]$$ se slika v kompozitum
\begin{center}
\begin{tikzcd}[column sep = 14ex]
M^m \ar[r, "{(M^{\#}t_1, \ldots, M^{\#}t_n)}"] & M^n \ar[r, "{M^{\#}f}"] & M
\end{tikzcd}
\end{center}
kjer je $M^{\#}f = f^M$ interpretacija osnovne operacije $f$ podana z modelom.
\end{enumerate}
%
Dobra definiranost $M^{\#}$ sledi iz tega, da je $M$ model, torej res zadošča vsem enačbam teorije.\\
Definirali smo ga na tak način, da velja
$$M \cong M^{\#}(\mathcal{U})$$
%
Za polnost in zvestost moramo pokazati surjektivnost in injektivnost predpisa $$\vartheta \mapsto \vartheta_\mathcal{U}$$
kjer je $\vartheta$ naravna transformacija med funktorjema $F$ in $G$, ki ohranjata končne limite.\\
Surjektivnost: $$\vartheta_{[x_1, \ldots, x_n]} := h^n$$\\
Injektivnost: $$\vartheta_\mathcal{U} = \phi_\mathcal{U} \Rightarrow (\vartheta_\mathcal{U})^n = \vartheta_{\mathcal{U}^n} = \phi_{\mathcal{U}^n} = (\phi_\mathcal{U})^n$$
\begin{definicija}
\emph{Klasifikacijska kategorija} algebraične teorije $\mathbb{T}$ je kategorija s končnimi produkti $\mathcal{C}_\mathbb{T}$, s posebnim modelom $\mathcal{U}$, imenovanim \emph{univerzalni model}, tako da velja:
\begin{enumerate}
\item Za vsak model $M$ v kategoriji s končnimi produkti $\mathcal{C}$, obstaja funktor, ki ohranja končne limite $$M^{\#} : \mathcal{C}_\mathbb{T} \to \mathcal{C}$$
in izomorfizem modelov $M \cong M^{\#}(\mathcal{U})$.
%
\item Za vsaka funktorja, ki ohranjata končne limite $F,G : \mathcal{C}_\mathbb{T} \to \mathcal{C}$ in homomorfizem modelov $h: F(\mathcal{U}) \to G(\mathcal{U})$, obstaja natanko ena naravna transformacija $\vartheta : F \to G$, da velja
$$\vartheta_\mathcal{U} = h$$
\end{enumerate}
\end{definicija}
Kot je to običajno za nekaj podano z univerzalno lastnostjo, je $\mathcal{C}_\mathbb{T}$ enolično določena, do ekvivalence kategorij natančno. \\
\vspace{1cm}
%
Naj bosta $(\mathcal{C}, U)$ in $(\mathcal{D}, V)$ obe klasifikacijski kategoriji neke teorije $\mathbb{T}$. Potem po univerzalni lastnosti dobimo funktorja
\begin{center}
\begin{tikzcd}[column sep = large]
\mathcal{C} \ar[r, bend left, "U^{\#}"]  & \mathcal{D} \ar[l, bend left, "V^{\#}"]
\end{tikzcd}
\end{center}
tako, da velja
$$U \cong U^{\#}(V) \cong U^{\#}(V^{\#}(U))$$
Pravimo, da je teorija $\mathbb{T}$ polna (semantično polna), če velja da
$$\text{Vsak model } \mathbb{T} \text{ zadošča }s=t \quad \Longrightarrow \quad \mathbb{T} \vdash s = t$$
\begin{izrek}[Gödel]
Naj bo $T$ teorija prvega reda in $\varphi$ formula v jeziku, ki opisuje $T$. Potem so naslednje trditve ekvivalentne:
\begin{enumerate}
\item Obstaja dokaz $\varphi$ iz aksiomov teorije $T$ $(T \vdash \varphi)$
\item Stavek $\varphi$ drži v vsakem modelu teorije $T$ $(T \models \varphi)$
\end{enumerate}
\end{izrek}
To velja v posebno močnem smislu za kategorično semantiko
\begin{izrek}[Polnost algebraičnih teorij]
Naj bo $\mathbb{T}$ algebraična teorija. Potem obstaja kategorija s končnimi produkti $\mathcal{C}$ in model $\mathcal{U} \in \Mod(\mathbb{T}, \mathcal{C})$ tako, da za vsako enačbo $s = t$ teorije $\mathbb{T}$ velja
$$\mathcal{U} \models s = t \quad \Longleftrightarrow \quad \mathbb{T} \vdash s = t$$
\end{izrek}
\begin{dokaz}
Kot kategorijo $\mathcal{C}$ bomo vzeli klasifikacijsko kategorijo $\mathcal{C}_\mathbb{T}$ in univerzalni model $\mathcal{U}$. Če velja $\mathbb{T} \vdash s = t$, potem iz konstrukcije $\mathcal{C}_\mathbb{T}$ sledi $s^\mathcal{U} = t^\mathcal{U}$. Obratno, če $\mathcal{U} \models s = t$, potem $s^\mathcal{U} = t^\mathcal{U}$. Ampak ponovno iz konstrukcije $\mathcal{C}_\mathbb{T}$ sledi, da mora veljati $\mathbb{T} \vdash s = t$.
\end{dokaz}
%
%
\section{Lawverjeva Teorija}
%
\section{Lawverjeva dualnost}
%
\section{Regularna logika}
Do sedaj smo videli povezavo med algebrajsko teorijo in kategorijo z neko dodatno struktuo, ki nam je omogočila interpretacijo take teorije v semantiki same kategorije. Teorija, ki smo jo obravnavali je bila precej enostavna s stališča logike, saj so bile edine logične formule, ki smo jih lahko konstruirali oblike $t_1 = t_2$ za neka terma, sestavljena induktivno iz spremenljivk in funkcijskih simbolov. 
Že iz teh osnovnih gradnikov je mogoče dobiti pomembne matematične "infrastrukture", kot je na primer teorija grup. Za opis nekih matematičnih teorij pa jasno ta stopnja kompleksnosti ne zadostuje.
Videli smo, da se teorije polj ne da opisati z algebrajsko teorijo. Moderna matematika je standardno opisana v jeziku predukatne logike, oz logike drugega (ali višjega) reda, kjer imamo poleg funkcijskih simbolov še logične veznike kot so "in", "ali", negacijo in univerzalni ter eksistenčni kvantifikator.
Naravno vprašanje je torej, ali lahko zgodbo algebrajskih teorij ponovimo z neko močnejšo logiko in če se to da, kako se to odraža v strukturi kategorijo, ki jo dobimo na ta način in ali ohranimo lepe lastnosti, ki smo jih videli v primeru algebrajskih teorij.
Kot vemo iz Gödelovega izreka, bo take polnosti, kot smo jo dobili z univerzalnim modelom za algebrajsko teorijo, pri logiki prvega reda ne moremo pričakovati.
Naredili bomo korak v to smer in našo logiko le delno razširili v tako imenovano $\emph{regularno logiko}$, kjer bomo formule gradili iz atomskih formul, logične konstante resničnosti $\top$, konjunkcij $\wedge$ in eksistenčnega kvantifikatorja $\exists$.
Videli bomo, da s to razširitvijo lahko ponovimo zgodbo iz prvega dela in nam to da tako imenovane \emph{regularne kategorije}, ki bodo imele dodatno strukturo, ki nam bo omogočala interpretacijo te logike.
Da to logiko definiramo, moramo razširiti pojem jezika.
%
\begin{definicija}
  \emph{Signatura} regularnega jezika $\Sigma$ je sestavljena iz množice \emph{osnovnih tipov} $\underline{\mathrm{sort}}_\Sigma = \set{X_1, X_2, X_3, \ldots}$, množice konstant $\underline{\mathrm{const}}_\Sigma$, množice funkcijskih simbolov $\underline{\mathrm{func}}_\Sigma$ in množice relacijskih simbolov $\underline{\mathrm{rel}}_\Sigma$. Uporabljamo oznake kot so $c : X$ za konstante tipa $X$, $f : X_1 \times \ldots \times X_n \to Y$ za funkcijske simbole in $R \rightarrowtail X_1 \times \ldots \times X_n$ za relacijske simbole.
\end{definicija}
Pogosto bomo za $X_1 \times \ldots X_n$ uporabljali tudi oznako $\overline{X}$, kjer $n$ razberemo iz konteksta.
\begin{definicija}
  Naj bo $\Sigma$ signatura regularnega jezika. Potem \emph{jezik} $\mathcal{L}(\Sigma)$ sestoji iz signature $\Sigma$, za vsak osnovni tip $X$ imamo števno mnogo spremenljivk $x:X$. Množice termov $(T)$ in formul $(F)$ definiramo na sledeče načine:
  \begin{itemize}
    \item [(T1)] Če je $x$ spremenljivka tipa $X$, potem je $x$ term tipa $X$.
    \item [(T2)] Če je $c$ konstanta tipa $X$, potem je $c$ term tipa $X$.
    \item [(T3)] Če so $t_1, \ldots t_n$ že termi tipov $X_1, \ldots, X_n$ in je $f : X_1 \times \ldots \times X_n \to Y$ funkcijski simbol, potem je $f(t_1, \ldots, t_n)$ term tipa $Y$.
    \item [(F1)] Če sta $t_1, t_2$ terma tipov $X_1$ in $X_2$, potem je $t_1 = t_2$ formula. Bolj natančno bi to zapisali kot $t_1 =_X t_2$.
    \item [(F2)] Logična konstanta $\top$ (ki predstavlja resninično izjavo) je formula.
    \item [(F3)] Če so $t_1, \ldots t_n$ termi tipov $X_1, \ldots, X_n$ in je $R \rightarrowtail X_1 \times \ldots \times X_n$ relacijski simbol, potem je $R(t_1, \ldots, t_n)$ formula.
    \item [(F4)] Če sta $\varphi$ in $\psi$ logični formuli, potem sta $\varphi \wedge \psi$ in $\exists x \varphi$ tudi logični 
  \end{itemize}
  Za logično formulo $\varphi$, množico njenih prostih spremenljiv označujemo z $\mathrm{FV}(\varphi)$. \emph{Teorija} $T$, formulirana v jeziku $\mathcal{L}(\Sigma)$, je množica \emph{sekvent} oblike 
  $$\varphi \implies \psi$$
  kjer sta $\varphi$ in $\psi$ formuli v jeziku teorije $T$.
\end{definicija}
Če je v sekventi premisa enaka $\top$ potem $\top \implies \psi$ označujemo kar kot $\psi$.
\begin{primer}
  Naj bo $\Sigma$ signatura s termi osnovnimi tipi $X,Y$ in $Z$, ki vsebuje tri funkcijske simbole $f: X \to Y$, $g : Y \to Z$ in $h : X \to Z$. V jeziku $\mathcal{L}(\Sigma)$ potem, če je $x$ spremenljivka tipa $X$, lahko v jeziku $\mathcal{L}(\Sigma)$ tvorimo formulo
  $$f(g(x)) = h(x)$$
  Ko definiramo interpretacijo teorije bomo videli, da je to ravno formula, ki pomeni, da je $h$ kompozitum $f$ in $g$.
\end{primer}
Sedaj bomo definirali pravila sklepanja za naš fragment logike prvega reda, za katera bomo kasenje pokazali, da so "sound" in polna glede na kategorično semantiko, ki jim jih bomo dali.
Podali jih bomo kot zaporedja dedukcij oblike $\varphi \vdash_F \psi$, indeksiranih po končnih množicah spremenljivk $F$. Dedukcija $\varphi \vdash \psi$ je definirana le, če vse proste spremenljivke v $\varphi$ ali $\psi$ ležijo v $F$. V prihodnje bomo privzeli, da je za izraz $\varphi \vdash \psi$ ta pogoj vedno izpolnjen (napiši to lepše).
Tu moramo biti pozorni, saj na primer izraz $x_1 = x_2 \vdash_{x_1, x_2} x_2 = x_1$ ni enak izrazu $x_1 = x_2 \vdash_{x_1, x_2, x_3} x_2 = x_1$. Razlog za to podrobnost bomo razložili kasneje, ko definiramo semantiko v kategoriji.

Sedaj definiramo pravila sklepanja, ki jih razdelimo v tri sklope: (ali je lepo dati dvopičje direkt pred def okoljem)
\begin{definicija}
  $ $
  \begin{enumerate}[label*=\arabic*.]
    \item Strukturna pravila
    \begin{enumerate}[label*=\arabic*.]
      \item $p \vdash_F p$
      \item $\dfrac{p \vdash_F q \ q \vdash_F r}{p \vdash_F r}$
      \item $\dfrac{p \vdash_F q}{p \vdash_{F \cup \set{y}} q}$
      \item $\dfrac{\varphi(y) \vdash_F \psi(y)}{\varphi(b) \vdash_{F\setminus\set{y}} \psi(b)}$
      kjer je $y : B$ spremenljivka, $b$ pa term tipa $B$ in $b$ lahko zamenjamo za $y$ v obeh izrazih.
    \end{enumerate}
    \item Logična pravila
    \begin{enumerate}[label*=\arabic*.]
      \item $p \vdash_F \top$
      \item Če $r \vdash_F p \wedge q$ potem $r \vdash_F p$ in $r \vdash_F q$; in če $r \vdash_F p$ in $r \vdash_F q$ potem $r \vdash_F p \wedge q$.
      \item Če $\exists y \psi(y) \vdash_F p$ potem $\psi(y) \vdash_{F \cup \set{y}} p$; in obratno, če $\psi \vdash_{F \cup \set{y}} p$ potem $\exists y \psi(y) \vdash_F p$.
    \end{enumerate}
    \item Pravila za enakost
    \begin{enumerate}[label*=\arabic*.]
      \item $\top \vdash_x x = x$
      \item $x_1 = x_2 \vdash_{x_1, x_2} x_2 = x_1$
      \item $x_1 = x_2 \wedge x_2 = x_3 \vdash_{x_1, x_2, x_3} x_1 = x_3$
      \item $\overline{x}^1 = \overline{x}^2 \wedge R(\overline{x}^1) \vdash_{\overline{x}^1, \overline{x}^2} \mathrm{R}(\overline{x}^2)$
      kjer je $R \rightarrowtail \overline{X}$
    \end{enumerate}
  \end{enumerate}
  Če $F = \emptyset$ potem označimo $\vdash_\emptyset$ kot $\vdash$ in $\emptyset \vdash_F \psi$ označimo z $\vdash_F \psi$.
  Če imamo podano teorijo $T$, pišemo $T, \varphi \vdash_F \psi$, če $\varphi_F^T \psi$, kjer $\varphi_F^T$ pomeni dedukcijo po zgornjih pravilih sklepanja z dodatnim aksiomom
  $$\varphi \vdash_{\mathrm{FV}(\varphi) \cup \mathrm{FV}(\psi)} \psi$$
  za vsako sekvento $\varphi \implies \psi$ v $T$. Alternativno lahko to označimo z $T \vdash_F \varphi \implies \psi$, kar nakazuje, da $\varphi$ implicira $\psi$, modulo $T$.
\end{definicija}
%
\section{Regularne kategorije}
%
\section{Interna logika regularne kategorije}
%
\section{Polnost regularne teorije}
%
\section{Zaključek}
%
\end{document}
\documentclass[12pt,a4paper]{book}
\usepackage[utf8x]{inputenc}   % omogoča uporabo slovenskih črk kodiranih v formatu UTF-8
\usepackage[slovene,english]{babel}    % naloži, med drugim, slovenske delilne vzorce
\usepackage[pdftex]{graphicx}  % omogoča vlaganje slik različnih formatov
\usepackage{fancyhdr}          % poskrbi, na primer, za glave strani
\usepackage{comment}
\usepackage[pdftex, colorlinks=true,
						citecolor=black, filecolor=black, 
						linkcolor=black, urlcolor=black,
						pagebackref=false, 
						pdfproducer={LaTeX}, pdfcreator={LaTeX}, hidelinks]{hyperref}
\usepackage{color}       % dodal Solina
%\usepackage{soul}
\usepackage{amsthm}
\usepackage{amsmath}
\usepackage{amsfonts}
\usepackage{amssymb}

\usepackage{amsmath}
\usepackage{mathtools}
\usepackage{enumitem}
\usepackage{commath}
\usepackage{array,xparse}
\usepackage{hyperref}
\usepackage{bm} % to make greek symbols bold

\usepackage{ulem} % ta knjižnica redefinira komando \emph
% ta komanda jo nastavi nazaj na privzeto vrednost
\normalem

\usepackage{tikz-cd} 
\usetikzlibrary{babel} % this fixes problems with tikz-cd
% http://tex.stackexchange.com/questions/166772/problem-with-babel-and-tikz-using-draw



%%%%%%%%%%%%%%%%%%%%%%%%%%%%%%%%%%%%%%%%
%	DIPLOMA INFO
%%%%%%%%%%%%%%%%%%%%%%%%%%%%%%%%%%%%%%%%
\newcommand{\ttitle}{Kategorična logika}
\newcommand{\ttitleEn}{Naslov EN}
\newcommand{\tsubject}{\ttitle}
\newcommand{\tsubjectEn}{\ttitleEn}
\newcommand{\tauthor}{Jure Taslak}
\newcommand{\tkeywords}{Teorija kategorij, Kategorična logika, kategorije, logika, algebraične teorije}
\newcommand{\tkeywordsEn}{Category theory, Categorical logic, categories, logic, algebraic theories}


%%%%%%%%%%%%%%%%%%%%%%%%%%%%%%%%%%%%%%%%
% naslovi
%%%%%%%%%%%%%%%%%%%%%%%%%%%%%%%%%%%%%%%%  
\newcommand{\autfont}{\Large}
\newcommand{\titfont}{\LARGE\bf}
\newcommand{\clearemptydoublepage}{\newpage{\pagestyle{empty}\cleardoublepage}}
\setcounter{tocdepth}{1}



%%%%%%%%%%%%%%%%%%%%%%%%%%%%%%%%%%%%%
%									%
% konstrukti						%
%									%
%%%%%%%%%%%%%%%%%%%%%%%%%%%%%%%%%%%%%  

\theoremstyle{definition}
\newtheorem{definicija}{Definicija}[chapter]
 
\theoremstyle{plain}
\newtheorem{izrek}[definicija]{Izrek}
\newtheorem{trditev}[definicija]{Trditev}
\newtheorem{posledica}{Posledica}[definicija]
\newtheorem{lema}[definicija]{Lema}
\newenvironment{dokaz}{\emph{Dokaz.}\ }{\hspace{\fill}{$\Box$}}

\theoremstyle{definition}
\newtheorem{primer}{Primer}[section]
\newtheorem*{primer*}{Primer}

\theoremstyle{remark}
\newtheorem*{opomba}{Opomba}

%%%%%%%%%%%%%%%%%%%%%%%%%%%%%%%%%
%								%
%	Tikz-cd nastavitve			%
%								%
%%%%%%%%%%%%%%%%%%%%%%%%%%%%%%%%%

\tikzcdset{
 	diagrams={sep=large},
	labels={font=\small}
}

%%%%%%%%%%%%%%%%%%%%%%%%%%%%%
% Avtor in naslov			%
%%%%%%%%%%%%%%%%%%%%%%%%%%%%%

\author{Jure Taslak}
\title{Kategorična logika}

%%%%%%%%%%%%%%%%%%%%%%%%%%%%%%%%%%%%%%%%
% pdfInfo
%%%%%%%%%%%%%%%%%%%%%%%%%%%%%%%%%%%%%%%%  
\pdfinfo{%
    /Title    (\ttitle)
    /Author   (\tauthor, damjan@cvetan.si)
    /Subject  (\ttitleEn)
    /Keywords (\tkeywordsEn)
    /ModDate  (\pdfcreationdate)
    /Trapped  /False
}

%%%%%%%%%%%%%%%%%%%%%%%%%%%%%%%%%%%%%
%									%
% 		Misc						%
%									%
%%%%%%%%%%%%%%%%%%%%%%%%%%%%%%%%%%%%%

\newcommand{\eqtext}[1]{\stackrel{\mathclap{\normalfont\mbox{#1}}}{=}} % write text over =
% could use a way to make the text smaller

\newcommand{\cat}[1]{\textbf{#1}}
\newcommand{\homset}[2]{\mathrm{Hom(#1,#2)}}

\DeclareMathOperator{\Hom}{Hom}
%\DeclareMathOperator[1]{\colim}{\underset{#1}{colim}}
\DeclareMathOperator{\colim}{colim}
%\newcommand[1]{\colim}{\underset{#1}\colim_op}
\DeclareMathOperator{\Fun}{Fun}
\DeclareMathOperator{\dom}{dom}
\DeclareMathOperator{\cod}{cod}
\DeclareMathOperator{\obj}{obj}

\renewcommand{\set}[1]{\{\,#1\,\}}

\newcommand{\fprod}[1]{\langle #1 \rangle}

\newcommand{\predsnop}[1]{\cat{Set}^{\cat{#1}^{op}}}

\newcommand{\powerset}[1]{\mathcal{P}(#1)}

%\DeclareMathOperator{\eval}{eval}

% \newcommand{\coprod}[2]{\cat{#1} + \cat{#2}}


%%%%%%%%%%%%%%%%%%%%%%%%%%%%%%%%%%%%%%%%%%
% Two-way rule for adjunction
%TODO make this more usefull
%%%%%%%%%%%%%%%%%%%%%%%%%%%%%%%%%%%%%%%%%%
\ExplSyntaxOn
\NewDocumentEnvironment{adjunctions}{O{}}
 {
  \cs_set_eq:cN {@arraycr} \farin_arraycr:
  \keys_set:nn { farin/adjunction } { #1 }
  \begin{array}
   {
    @{ \hspace { \dim_eval:n { \l_farin_left_shift_dim + \l_farin_padding_dim } } }
    r
    @{ {\farin_strut:} \l_farin_symbol_tl {} }
    l
    @{ \hspace { \dim_eval:n { \l_farin_right_shift_dim + \l_farin_padding_dim } } }
   }
 }
 {
  \end{array}
 }
\keys_define:nn { farin/adjunction }
 {
  leftshift       .dim_set:N = \l_farin_left_shift_dim,
  leftshift       .initial:n = 0pt,
  rightshift      .dim_set:N = \l_farin_right_shift_dim,
  rightshift      .initial:n = 0pt,
  padding         .dim_set:N = \l_farin_padding_dim,
  padding         .initial:n = 6pt,
  symbol          .tl_set:N  = \l_farin_symbol_tl,
  symbol          .initial:n = \longrightarrow,
  verticalspacing .dim_set:N  = \l_farin_vertspac_dim,
  verticalspacing .initial:n = {3pt},
 }
\cs_new_protected:Npn \farin_strut:
 {
  \vrule height \dim_eval:n { \ht\strutbox + 1.2\l_farin_vertspac_dim }
         depth  \dim_eval:n { \dp\strutbox + \l_farin_vertspac_dim }
         width 0pt
 }
\makeatletter
\exp_args:NNo \cs_new:Npn \farin_arraycr:
 {
  \@arraycr\hline
 }
\makeatother
\ExplSyntaxOff


%%%%%%%%%%%%%%%%%%%%%%%%%%%%%%%%%%%%%%%%%%
%%%%%%%%%%%%%%%%%%%%%%%%%%%%%%%%%%%%%%%%%%

\begin{document}
\selectlanguage{slovene}
\frontmatter
\setcounter{page}{1} %
\renewcommand{\thepage}{}       % preprecimo težave s številkami strani v kazalu
\newcommand{\sn}[1]{"`#1"'}                    % dodal Solina (slovenski narekovaji)



%%%%%%%%%%%%%%%%%%%%%%%%%%%%%%%%%%%%%%%%
%naslovnica
 \thispagestyle{empty}%
   \begin{center}
    {\large\sc Univerza v Ljubljani\\%
      Fakulteta za matematiko in fiziko}%
    \vskip 10em%
    {\autfont \tauthor\par}%
    {\titfont \ttitle \par}%
    {\vskip 3em \textsc{MAGISTRSKO DELO\\[5mm]
    UNIVERZITETNI\\ ŠTUDIJSKI PROGRAM DRUGE STOPNJE\\ MATEMATIKA}\par}%

    \vfill\null%
    {\large \textsc{Mentor}: prof.\ dr.  Andrej Bauer\par}%
    {\vskip 2em \large Ljubljana, 2020 \par}%
\end{center}
% prazna stran
%\clearemptydoublepage      % dodal Solina (izjava o licencah itd. se izpiše na hrbtni strani naslovnice)

%%%%%%%%%%%%%%%%%%%%%%%%%%%%%%%%%%%%%%%%
%copyright stran
\thispagestyle{empty}
\vspace*{8cm}

\noindent
{\sc Copyright}. 
Rezultati diplomske naloge so intelektualna lastnina avtorja in Fakultete za računalništvo in informatiko Univerze v Ljubljani.
Za objavo in koriščenje rezultatov diplomske naloge je potrebno pisno privoljenje avtorja, Fakultete za računalništvo in informatiko ter mentorja.

\begin{center}
\mbox{}\vfill
\emph{Besedilo je oblikovano z urejevalnikom besedil \LaTeX.}
\end{center}
% prazna stran
\clearemptydoublepage


%%%%%%%%%%%%%%%%%%%%%%%%%%%%%%%%%%%%%%%%
% stran 3 med uvodnimi listi
\thispagestyle{empty}
\vspace*{4cm}

\noindent
Fakulteta za računalništvo in informatiko izdaja naslednjo nalogo:
\medskip
\begin{tabbing}
\hspace{32mm}\= \hspace{6cm} \= \kill


Tematika naloge:
\end{tabbing}
Delo obravnava Yonedovo lemo, ki je eden od osrednjih izrekov teorije kategorij. Predstavljene bodo tudi nekatere aplikacije Yonedove leme.
\vspace{15mm}

\vspace{2cm}

% prazna stran
\clearemptydoublepage

% zahvala
\thispagestyle{empty}\mbox{}\vfill\null\it%
\noindent
Zahvaljujem se mentorju prof. dr. Andreju Bauerju za usmerjanje in vso ostalo pomoč. \\
Svojim staršem, za vso podporo. \\
\rm\normalfont

% prazna stran
\clearemptydoublepage

%%%%%%%%%%%%%%%%%%%%%%%%%%%%%%%%%%%%%%%%
% kazalo
\pagestyle{empty}
\def\thepage{}% preprecimo tezave s stevilkami strani v kazalu
\tableofcontents{}

% prazna stran
\clearemptydoublepage


%%%%%%%%%%%%%%%%%%%%%%%%%%%%%%%%%%%%%%%%
% povzetek
\addcontentsline{toc}{chapter}{Povzetek}
\chapter*{Povzetek}

\noindent\textbf{Naslov:} \ttitle
\bigskip

\noindent\textbf{Avtor:} \tauthor
\bigskip

%\noindent\textbf{Povzetek:} 
\noindent
V diplomskem delu je obravnavana Yonedova lema, ki velja za enega osrednjih izrekov teorije kategorij. Uvodni del definira osnovne pojme v teoriji kategorij, ki so kasneje uporabljeni za formulacijo in dokaz leme. Skozi besedilo je predstavljen tudi kategorični način razmišljanja, ki nam omogoča specifično situacijo, ki jo srečamo v matematiki, obravnavati bistveno bolj splošno, z uporabo kategoričnih metod.
V zaključnem poglavju je predstavljena in dokazana Yonedova lema, ki nam poda način za obravnavo kategorije z obravnavo njene vložitve v kategorijo funktorjev. Predstavljenih je tudi nekaj primerov uporabe leme.
\bigskip

\noindent\textbf{Ključne besede:} \tkeywords.
% prazna stran
\clearemptydoublepage

%%%%%%%%%%%%%%%%%%%%%%%%%%%%%%%%%%%%%%%%
% abstract
\selectlanguage{english}
\addcontentsline{toc}{chapter}{Abstract}
\chapter*{Abstract}

\noindent\textbf{Title:} \ttitleEn
\bigskip

\noindent\textbf{Author:} \tauthor
\bigskip
%\noindent\textbf{Abstract:} 

\noindent 
The thesis discusses the Yoneda lemma, which is considered one of the central theorems in category theory. The introduction defines the basic concepts of category theory, which are later used to formulate and prove the lemma. Throughout the text there are examples of the categorical way of thinking, where we take a specific situation, that we encounter in mathematics and look at it in a more general setting. In the final chapter we describe and prove the Yoneda lemma, that presents a way of studying a category, by studying its inclusion into a category of functors. We show some use cases of the lemma.

\bigskip

\noindent\textbf{Keywords:} \tkeywordsEn.
\selectlanguage{slovene}
% prazna stran
\clearemptydoublepage

\mainmatter
\setcounter{page}{1}
\pagestyle{fancy}

\chapter{Uvod}


\section{Osnovne Definicije}

\begin{lema}
Naj bo $\cat{C}$ poljubna kategorija in $f : X \to Y$ morfizem. Denimo, da v $\cat{C}$ obstaja produkt $X \times Y$. Potem je projekcija $X \times Y \to X$ epimorfizem.
\end{lema}
\begin{dokaz}
Naj bo $p : X \times Y \to X$ projekcija iz produkta in $u = \langle id_X, f \rangle : X \to X \times Y$. Recimo, da imamo dva vzporedna morfizma $g,h : X \to Z$ za katera velja $g \circ p = h \circ p$. To prikažemo v diagramu
\begin{center}
\begin{tikzcd}
& X \ar[d, "u", dashed] \ar[dr, "id_X"] \ar[dl, "f"'] \\
Y & X\times Y \ar[l] \ar[r, "p"] & X \ar[r, shift left, "g"] \ar[r, shift right, "h"'] & Z
\end{tikzcd}
\end{center}
Sedaj velja
\begin{align*}
g \circ p &= h \circ p \\
g \circ p \circ u &= h \circ p \circ u \\
g \circ id_X &= h \circ id_X \\
g &= h
\end{align*}
\end{dokaz}


\begin{izrek} [Adjoint functor theorem]
Naj bo $\cat{C}$ lokalno majhna in popolna. Za vsako kategorijo $\cat{X}$ in funktor 
$$U : \cat{X} \to \cat{C}$$
ki ohranja limite, so naslednje trditve ekvivalentne:
\begin{enumerate}
\item $U$ ima levi adjunkt
\item Za vsak objekt $X \in \cat{X}$ funktor $U$ zadošča kriteriju množice rešitev:
Obstaja množica $(C_i)_{i \in I}$ objektov v $\cat{C}$ tako, da za vsak objekt $C \in \cat{C}$ in vsak morfizem $f : X \to UC$ obstaja indeks $i \in I$ in morfizma $\phi : X \to UC_i$ in $\overline{f} : C_i \to C$, da velja
$$f = U(\overline{f}) \circ \phi$$
\[ \begin{tikzcd}
X \arrow[r, "\phi"] \arrow[rd, "f"'] & UC_i \arrow[d, "U(\overline{f})"] \\
& UC
\end{tikzcd} \]
\end{enumerate}
\end{izrek}

\begin{lema} \label{lema2}
Funktor $U : \cat{C} \to \cat{X}$ ima levi adjunkt natanko takrat, ko ima comma-kategorija $(X \downarrow U)$ začetni objekt.
\end{lema}
\begin{proof}
Naj bo $F : \cat{X} \to \cat{C}$ levi adjunkt za $U$ in $\eta : 1_{\cat{X}} \to UF$ enota adjunkcije. Potem je $(FX, \eta_X : X \to UFX)$ začetni objekt v $(X \downarrow U)$. To velja, saj če je $(C, f : X \to UC)$ objekt v $(X \downarrow U)$, potem po UMP enote obstaja natanko en $g : FX \to C$, da diagram
\[ \begin{tikzcd}
X \arrow[d, "\eta_X"'] \arrow[dr, "f"] & \\
UFX \arrow[r, "Ug"'] & UC
\end{tikzcd} \]
komutira. To je ravno enolični morfizem $(FX,\eta_X) \to (C,f)$. \\
Obratno, denimo da imamo začetni objekt v $(X \downarrow U)$, ki ga pomenljivo označimo kar z $(FX, \eta_X : X \to U(FX))$. Zaradi obstoja in enoličnosti takega objekta nam to določi funkcijo objektov 
$$F : \obj(\cat{X}) \to \obj(\cat{C})$$
Eksplicitno $F(X) = \text{prva komponenta začetnega objekta v } (X \downarrow U)$. Ta $F$ želimo razširito do funktorja, zato naj bo $f : X \to X'$ morfizem v $X$. Morfizem $F(f)$ naj bo enolični morfizem, tako da kvadrat
\[ \begin{tikzcd}
X \arrow[d, "\eta_X"'] \arrow[r, "f"] & X' \arrow[d, "\eta_{X'}"] \\
UFX \arrow[r, "UFf"', dashed] & UFX'
\end{tikzcd} \]
komutira. Jasno je, da $F$ slika domeno morfizma v domeno ter kodomeno v kodomeno. Za asociativnost denimo, da imamo morfizma 
\begin{center}
\begin{tikzcd}
X \arrow[r, "f"] & X' \arrow[r, "f'"] & X''
\end{tikzcd}
\end{center}
To nam da enolična morfizma 
\begin{center}
\begin{tikzcd}
FX \arrow[r, "Ff", dashed] & FX' \arrow[r, "Ff'", dashed] & X''
\end{tikzcd}
\end{center}
zaradi katerih diagram
\begin{center}
\begin{tikzcd}
X \ar[d, "\eta_X"'] \ar[r, "f"] & X' \ar[d, "\eta_{X'}"'] \ar[r, "f"] & X'' \ar[d, "\eta_{X''}"] \\
UFX \ar[r, "UFf"', dashed] & UFX' \ar[r, "UFf'"', dashed] & UFX''
\end{tikzcd}
\end{center}
komutira, kar določi kompozitum kot
$$F(f' \circ f) = F(f') \circ F(f).$$
Od tod tudi sledi, da je $\eta$ naravna transformacija, določena s komponentami $\eta_X$, ki zadošča UMP enote. Torej je $F$ res levi adjunkt $U$.
\end{proof}
\begin{definicija}
\mbox{}
\begin{enumerate}[label=(\roman*)]
\item Objekt $C \in \cat{C}$ je \emph{šibko začeten}, če za vsak $X \in \cat{C}$ obstaja morfizem $C \to X$ (ki ni nujno enoličen).
\item Zbirka objektov $\Phi = (C_i)_{i \in I}$ je \emph{skupno šibko začetna}, če za vsak $X \in C$ obstaja nek $C_i \in \Phi$, za katerega obstaja morfizem $C_i \to X$.
\end{enumerate}
\end{definicija}
\begin{lema}
Naj bo $\cat{C}$ kategorija za katero ima identitetni funktor $1_\cat{C} : \cat{C} \to \cat{C}$ limito. Potem ima $\cat{C}$ začetni objekt.
\end{lema}
\begin{proof}
Naj bo $l = \lim (1_\cat{C})$ z zbirko morfizmov $(\lambda_C)_{C \in \cat{C}}$. Iz definicije limite sledi, da je $l$ šibko začeten objekt. Recimo, da imamo še en morfizem $f : l \to C$. Ker je $l$ stožec sledi
$$ \lambda_C  = f \circ \lambda_l$$
Če pa vzamemo $\lambda_C$ kot morfizem diagrama $1_\cat{C}$, dobimo enačbo
$$\lambda_C = \lambda_C \circ \lambda_l$$
Iz diagrama
\begin{center}
\begin{tikzcd}
& l \ar[ddl, "1_l"', bend right] \ar[d, "u", dashed] \ar[ddr, "\lambda_C", bend left] & \\
& l \ar[dl, "\lambda_l"'] \ar[dr, "\lambda_C"]  & \\
l \ar[rr, "\lambda_C"'] & & C
\end{tikzcd}
\end{center}
dobimo enačbi $1_l = \lambda_l \circ u$ in $\lambda_C = \lambda_C \circ u$. Iz enoličnosti morfizma $u$ sledi $u = 1_l$, kar pomeni $\lambda_C = f$.
\end{proof}
\begin{lema} \label{lema1}
Naj bo $\cat{C}$ lokalno majhna, popolna kategorija, ki ima skupno šibko začetno množico objektov $\Phi = (C_i)$. Potem ima $\cat{C}$ začetni objekt.
\end{lema}
\begin{proof}
Naj bo $\iota : \bm{\Phi} \to \cat{C}$ inkluzija polne podkategorije $\cat{C}$ generirane z objektov iz $\Phi$. Ker je $\bm{\Phi}$ majhna in $\cat{C}$ popolna, obstaja limita
$$\ell = \lim \iota$$
Ker je množica $\Phi$ skupno šibko začetna imamo za vsak objekt $X \in \cat{C}$ morfizem $C_i \to X$ za nek $C_i \in \Phi$. Če komponiramo z ustreznim $l_i : \ell \to C_i$ dobimo morfizem
$$\lambda_X : \ell \to X$$
kar pomeni, da je $\ell$ šibko začetni objekt. Sedaj želimo pokazati, da morfizmi $\lambda_X$ določajo stožec nad $1_\cat{C}$ z vrhom $\ell$ in lastnostjo, da je $\lambda_l$ identiteta. Naj bo torej $f : X \to Y$ morfizem v $\cat{C}$. Ker je $\Phi$ šibko začetna, imamo morfizma $c_1 : C_1 \to X$ in $c_2 : C_2 \to Y$. Tvorimo povlek $P$ morfizmov $f \circ c_1$ in $c_2$
\begin{center}
\begin{tikzcd}
\ell \ar[dddrr, bend right, "l_1"'] \ar[ddrrrr, bend left, "l_2"] \ar[dr, dashed, "u"] & & & & \\
& C_3 \ar[ddr, bend right, dotted] \ar[drrr, bend left, dotted] \ar[dr, "h"] & & & \\
& & P \ar[d, "p_1"'] \ar[rr, "p_2"] & & C_2 \ar[d, "c_2"] \\
& & C_1 \ar[r, "c_1"'] & X \ar[r, "f"'] & Y
\end{tikzcd}
\end{center}
Ker je $\Phi$ šibko začetna, obstaja nek morfizem $h : C \to P$. Črtkana kompozituma $p_1 \circ h$ in $p_2 \circ h$ živita v polni podkategoriji generirani s $\Phi$, kar nam pove, da zgornja trikotnika komutirata. Iz komutativnosti ostalega dela diagrama vidimo, da je $\ell$ res vrh stožca nad $1_\cat{C}$. Iz tega sledi komutitativnost diagrama
\begin{center}
\begin{tikzcd}
& \ell \ar[dl, "\lambda_\ell"'] \ar[dr, "l_C = \lambda_C"] & \\
\ell \ar[rr, "\lambda_C"'] & & C
\end{tikzcd}
\end{center}
kar pomeni, da je $\lambda_\ell$ faktorizacija limitnega stožca skozi samega sebe, iz česar sledi $\lambda_\ell = 1_\ell$. Na dolgo, imamo komutativni diagram
\begin{center}
\begin{tikzcd}
& \ell \ar[ddl, bend right, "\lambda_\ell"'] \ar[ddr, bend left, "\lambda_C"] \ar[d, dashed, "u"] & \\
& \ell \ar[dl, "\lambda_\ell"'] \ar[dr, "\lambda_C"] & \\
\ell \ar[rr, "\lambda_C"'] & & C
\end{tikzcd}
\end{center}
iz spodnjega trikotnika imamo enačbo
$$\lambda_C = \lambda_C \circ \lambda_\ell$$
Ker pa je $\ell$ limita, dobimo enoličen $u$ za katerega velja
$$\lambda_\ell = \lambda_\ell \circ u \quad \& \quad \lambda_C = \lambda_C \circ u$$
Ker je $u$ enoličen tak, ki zadošča tema enačbama in je $1_\ell$ tudi tak, sledi $\lambda_\ell = 1_\ell$. Iz leme \ref{lema1} sledi, da ima $\cat{C}$ začetni objekt.
\end{proof}
Z uporabo teh lem lahko sedaj dokažemo izrek.
\begin{proof}
Če ima $U$ levi adjunkt, potem je množica $\set{FX}$ zadošča kriteriju množice rešitev.\\
Obratno, po lemi \ref{lema2} ima $U$ levi adjunkt natanko takrat, ko ima za vsak $X$ kategorija $(X \downarrow U)$ začetni objekt. Preveriti moramo torej
\begin{enumerate}
\item $(X \downarrow U)$ je lokalno majhna
\item $(X \downarrow U)$ je popolna
\item $(X \downarrow U)$ ima skupno šibko začetni objekt
\end{enumerate}
Ker je $\cat{C}$ lokalno majhna, je tudi $(X \downarrow U)$. Iz predpostavke sledi, da je
$$\set{(C_i, \phi : X \to UC_i) \ \vert \ i \in I}$$
skupno šibko začetni objekt. Da je $(X \downarrow U)$ popolna pokažemo tako, da konstruiramo produkte in zožke, iz česar sledi obstoj limit. 
\end{proof}


\end{document}
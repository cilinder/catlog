\documentclass[../kategoricna_logika.tex]{subfiles}
\begin{document}
V tem delu smo videli dva primera tako imenovane kategorične logike, kjer uporabimo
koncepte iz teorije kategorij pri preučevanju matematične logike.
Za začetek tega področja bi lahko oklicali tri članke Williama Lawvera \cite{lawvere1963functorial,lawvere1964elementary,lawvere1971quantifiers}.
V prvem poglavju smo si ogledali kategorični pristop obravnave algebrajskih teorij.
S primerno posploštivijo definicije modelov takih teorij, smo jih uspeli opisati
kot fuktorje, ki so ohranjali neko strukturo.
Za algebrajske teorije je bila ta struktura ohranjanje končnih produktov.
To je bilo mogoče prek konstrukcije posebne kategorije, ki je to teorijo zakodirala.
Tako smo lahko našli opis algebrajskih teorij v vsaki teoriji, ki tako strukturo premore.
Na primer, tako lahko dobimo opis teorije grup v kategoriji $\cat{Grp}$ grup in
homomorfizmov med njimi. Izkaže se, da so to ravno Abelove grupe.
Videli smo pa tudi, da ta pogled lahko tudi obrnemo in z definicijo Lawverjeve teorije
na vsako ustrezno kategorijo gledamo kot kategorijo, ki predstavlja neko algebrajsko
teorijo. V takih kategorijah lahko nato definiramo njihovo interno logiko, s katero
lahko transformiramo enačbe te teorije v ustrezne kategorične izjave in obratno.
V drugem delu smo to združitev logike in družine kategorij začeli z drugega konca.
Najprej smo ">našli"< zanimiv razred kategorij in opazili nekaj lastnosti tega razreda,
ki so ustrezale znanim lastnostim iz kategorije množic. Te lastnosti so bile, da lahko
v regularnih kategorijah izrazimo pojma slike morfizma in relacije. Nato smo skrbno
izbrali fragment logike prvega reda, s katerim je mogoče izraziti te dve konstrukciji.
Pokazali smo, da je ta fragment ravno tak, ki definira interno logiko regularnih kategorij.
Zato ta fragment lahko primerno poimenujemo regularna logika.
Ponovno nam je to omogočilo izraziti modele take logike kot ustrezne funktorje iz sintaktično
zgrajene kategorije, ki tako teorijo zakodira, v ustrezne kategorije v katerih je
mogoče to logiko interpretirati. To so seveda natanko regularne kategorije.

Ta dva primera sta le del obsežnejše korespondence logičnih teorij in razredov kategorij.
Tu smo videli razširitev logike z upeljavo dodatnih logičnih operacij. Druga možna smer
razširitve bi bila razširitev teorije tipov v lambda račun z enostavnimi tipi. Ustrezen
razred kategorij v tem primeru so kartezično zaprte kategorije. Več o tem si lahko preberemo
v \cite{seely1984locally}. Po drugi strani bi lahko še dodatno omejili strukturo regularnih
kategorij. Ta pot nas privede do t.\ i.\ eksaktnih kategorij in eventuelno do Abelovih kategorij,
ki služijo kot podlaga homološki teoriji. To je opisano v \cite{barr-exact-categories}.
V tej smeri je mogoče logiko posplošiti do logike drugega (in višjega) reda, kjer so
kategorije s tako interno logiko imenovane elementarni toposi. Ta teorija je razvita v
knjigi \cite{TJohnstone2002-TJOSOA-2}.

\end{document}
%%% Local Variables:
%%% mode: latex
%%% TeX-master: t
%%% End:

\documentclass[../kategoricna_logika.tex]{subfiles}
\begin{document}
V matematiki pogosto ">v naravi"< naletimo na strukture, za katere se izkaže,
da jih lahko opišemo z naborom operacij in enačb.
Slaven tak primer je odkritje strukture grupe pri iskanju splošne formule
za računanje ničel polinomov.
Tak opis strukture z operacijami in enačbami imenujemo \emph{algebrajska teorija}.
Na tak način lahko obravnavamo mnogo znanih primerov iz osnovne algebre,
kot so grupe, kolobarji, moduli itd., a tudi nekatere konstrukcije, ki niso
tako očitno algebrajske, na primer adjungirane funktorje.
Vse algebrajske teorije imajo mnogo skupnih, splošnih lastnosti.
Nekatere izmed njih bomo predstavili v sledečem poglavju.
To bomo storili z uporabo tako imenovane funktorialne semantike, katere
razvoj nam bo služil tudi kot zgled v drugem delu naloge, kjer bomo to
idejo razširili na strukture, ki jih ni mogoče opisati samo z enačbami.
\section{Kratek pregled teorije kategorij}

Še prej ponovimo nekatere pojme teorije kategorij.
S tem tudi uvedemo notacijo, ki jo bomo uporabljali v nalogi.
Tu pa ne bodo predstavljeni vsi rezultati teorije kategorij, ki jih bomo potrebovali.
Uvod v teorijo kategorij lahko najdemo na primer v
\cite{awodey2010category, riehl2017category, taslak2017}.
Večkrat se bomo naslonili tudi na koncepte in primere iz logike in algebre.
Uvod v logiko lahko najdemo na primer v \cite{prijatelj1992osnove1},
za uvod v algebro si bralec lahko pogleda \cite{bresar2018uvod}.
\begin{definicija}
  \emph{Kategorija} sestoji iz:
  \begin{itemize}
  \item razreda \emph{objektov}, ki jih bomo običajno označevali z
    velikimi tiskanimi črkami, na primer $X, Y, A, B \ldots$
  \item razreda \emph{morfizmov}, ki jih bomo običajno označevali z
    malimi tiskanimi črkami, na primer $f,g,\alpha,\beta \ldots$
    Vsak morfizem $f$ ima dva pridružena objekta:
    \[ \mathrm{dom}(f) \quad \text{in} \quad \mathrm{cod}(f), \]
    ki jima pravimo \emph{domena} in \emph{kodomena}. Če je $\mathrm{dom}(f)=X$
    in ${\mathrm{cod}(f)=Y}$, potem pišemo $f : X \to Y$.
  \item operacije \emph{kompozicija}, ki morfizmoma $f,g$,
    za katera velja $\mathrm{cod}(f) = \mathrm{dom}(g)$,
    priredi njun \emph{kompozitum}. To je morfizem, ki ga označujemo z $g \circ f$, in
    za katerega velja $\mathrm{dom}(g \circ f) = \mathrm{dom}(f)$ in
    $\mathrm{cod}(g \circ f) = \mathrm{cod}(g)$. 
  \end{itemize}
Kategorija mora zadoščati naslednjima aksiomoma:
  \begin{itemize}
  \item če so \( f: X \to Y\), \(g : Y \to Z\) in \(h : Z \to W\) morfizmi, potem
    velja \(h \circ (g \circ f) = (h \circ g) \circ f\),
  \item vsakemu objektu priredimo identitetni morfizem
    $\mathrm{id}_X : X \to X$, za katerega velja $\mathrm{id}_X \circ f = f$,
    za vsak $f : Y \to X$ in $g \circ \mathrm{id}_X = g$, za vsak $g : X \to Z$.
\end{itemize}
Kategorije običajno označujemo z velikimi tiskanimi črkami, na primer $\cat{C}, \cat{D}$.
\end{definicija}
\begin{opomba}
  Včasih kompozitum $g \circ f$ označimo kar kot $g f$, če je v kontekstu to edina
  smiselna operacija med morfizmoma.
\end{opomba}
\begin{definicija}
  V kategoriji $\cat{C}$ razred vseh morfizmov med objektoma $X$ in $Y$ označimo s $\Hom_{\cat{C}}(X,Y)$.
  Če je $\Hom_{\cat{C}}(X,Y)$ množica za vsaka $X$ in $Y$, pravimo, da je $\cat{C}$
  \emph{lokalno majhna}.
\end{definicija}
\begin{definicija}
  Naj bosta $\cat{C}$ in $\cat{D}$ kategoriji. \emph{Funktor} iz $\cat{C}$ v $\cat{D}$
  je preslikava $F$, ki vsakemu objektu $X$ iz \(\cat{C}\) priredi objekt \(F(X)\) v $\cat{D}$
  in vsakemu morfizmu $f: X \to Y$ v $\cat{C}$ priredi morfizem $F(f): F(X) \to F(Y)$ v $\cat{D}$.
  Izpolnjevati mora naslednja pogoja:
  \begin{itemize}
  \item $F(\mathrm{id}_X) = \mathrm{id}_{F(X)}$,
  \item $F(g \circ f) = F(g) \circ F(f)$ za vse $f : X \to Y$ in $g : Y \to Z$.
  \end{itemize}
\end{definicija}
\begin{definicija}
  Naj bosta $\cat{C}$ in $\cat{D}$ kategoriji in $F,G : \cat{C} \to \cat{D}$ par funktorjev
  med njima. \emph{Naravna transformacija} $\vartheta : F \to G$, iz $F$ v $G$, je
  družina morfizmov
  \[ (\vartheta_X : F(X) \to G(X))_{X \in \cat{C}}\]
  tako, da za vsak morfizem $f : X \to Y$ v $\cat{C}$ komutira diagram
  \begin{equation*}
    \begin{tikzcd}
      F(X) \ar[d, "F(f)"'] \ar[r, "\vartheta_X"] & G(X) \ar[d, "G(f)"] \\
      F(Y) \ar[r, "\vartheta_Y"'] & G(Y)
    \end{tikzcd}
  \end{equation*}
  To pomeni, da velja enačba \( G(f) \circ \vartheta_X = \vartheta_Y \circ F(f)\).
\end{definicija}
\begin{definicija}
  Naj bosta $\cat{C}$ in $\cat{D}$ kategoriji in $F : \cat{C} \to \cat{D}$, $G : \cat{D} \to \cat{C}$
  funktorja. Potem sta $F$ in $G$ \emph{adjungirana}, oziroma tvorita
  \emph{par adjungiranih funktorjev}, če za vsaka objekta $X \in \cat{C}$ in $Y \in \cat{D}$
  obstaja bijekcija
  \[ \Hom_{\cat{D}}(F(X), Y) \cong \Hom_{\cat{C}}(X, G(Y)),\]
  naravna v $X$ in $Y$.
  V tem primeru pravimo, da je $F$ \emph{levi adjunkt} $G$ in simetrično, da je $G$
  \emph{desni adjunkt} $F$. To označimo kot $F \dashv G$.
\end{definicija}

\section{Algebrajske teorije}
Začnimo s splošnim pristopom opisa algebrajskih teorij.
Te so karakterizirane aksiomatsko, s spremenljivkami, konstantami, operacijami in enačbami.
Pomembno (za naš pristop) je, da so operacije definirane za celotno strukturo,
ne pa samo za nekatere elemente te strukture.
To izključi dva pomembna primera: teorijo polj,
kjer inverz ničle ni definiran, in teorijo kategorij, kjer so kompozitumi
definirani samo za določene pare morfizmov.
Idejo bomo ilustrirali s temeljnim primerom iz algebre, teorijo grup.
\begin{primer}[Teorija grup]
  \label{primer:teorija-grup}
  Grupo lahko razumemo kot množico $G$, skupaj z dvojiško operacijo
  $\cdot : G \times G \to G$, ki zadošča aksiomoma:
  \begin{align}\label{aksiomi-grupe}
    &\forall x,y,z \in G . \quad (x\cdot y) \cdot z = x \cdot (y \cdot z) \\
    &\exists e \in G . \forall x \in G . \exists y \in G . \quad (e \cdot x = x \cdot e = x \wedge x \cdot y = y \cdot x = e)
  \end{align}
\end{primer}
\noindent
Že na prvi pogled lahko rečemo, da je struktura drugega izmed zgornjih dveh aksiomov
manj zadovoljiva od prvega, kajti vsebuje tako vgnezdene kvantifikatorje,
kot določa obstoj nekih elementov, ki pa jih je mogoče iz teh aksiomov
določiti enolično. Aksiomatizacijo lahko poenostavimo v naslednjih korakih.
Najprej k strukturi grupe dodamo
konstanto $e \in G$ in enomestno operacijo $(-)^{-1} : G \to G$ ter tako
dobimo ekvivalentno formulacijo grupe, ki jo podamo samo z enačbami:
\begin{align*}
  &x \cdot (y \cdot z) = (x \cdot y) \cdot z \\
  &x \cdot e = e \cdot x = x \\
  &x \cdot x^{-1} = x^{-1} \cdot x = e
\end{align*}
V drugem koraku se lahko znebimo univerzalnega kvantifikatorja $\forall$,
če vemo, da spremenljivke $x$, $y$ in $z$ zavzamejo vrednosti iz množice $G$.
V naslednjem koraku iz opisa odstranimo eksplicitno omembo množice $G$.
Za zadnji korak lahko konstanto $e$ opišemo kot eniško operacijo $e : 1 \to G$.
Tak način specifikacije bomo uporabili kot definicijo splošne algebrajske teorije.
%
\begin{definicija}
  \emph{Signatura} algebrajske teorije $\Sigma$ je sestavljena iz družine
  množic $\lbrace \Sigma_k \rbrace_{k \in \mathbb{N}}$, kjer se elementi $\Sigma_k$
  imenujejo $k$-mestne osnovne operacije.
  \emph{Jezik} algebrajske teorije $\mathcal{L}(\Sigma)$ je množica \emph{termov},
  ki jih tvorimo induktivno:
  \begin{enumerate}
  \item Spremenljivke $x,y,z \ldots$ so termi,
  \item Če so $t_1, \ldots, t_k$ že termi in je $f \in \Sigma_k$, potem
    je $f(t_1,\ldots, t_k)$ tudi term.
  \end{enumerate}
\end{definicija}
%
\begin{definicija}
  \emph{Algebrajska teorija}
  $\mathbb{T} = (\Sigma_\mathbb{T}, A_\mathbb{T})$ je podana s signaturo
  $\Sigma_\mathbb{T}$ in množico enačb $A_\mathbb{T}$.
  Enačbe iz $A_{\mathbb{T}}$ imenujemo \emph{aksiomi} teorije $\mathbb{T}$. Formalno so
  to pari termov jezika $\mathcal{L}(\Sigma_{\mathbb{T}})$.
\end{definicija}
\begin{primer}
  Prazna oziroma trivialna teorija $\mathbb{T}_0$ je teorija brez operacij in brez enačb.
  Ta teorija opisuje množico.
\end{primer}
\begin{primer}
Teorija z eno konstanto in brez enačb je teorija množice z odlikovanim elementom.
\end{primer}
\begin{primer}
  Teorija komutativnega kolobarja z enoto je algebrajska teorija.
  Imamo dve ničmestni operaciji (konstanti) $0$ in $1$, eno eniško operacijo
  $-$ in dve dvojiški operaciji $+$ in $\cdot$. Zanje veljajo enačbe:
\begin{align*}
  (x+y)+z &= x + (y + z) & (x \cdot y) \cdot z &= x \cdot (y \cdot z) \\
  x + 0 &= x & x \cdot 1 &= x \\
  0 + x &= x & 1\cdot x &= x \\
  x + (-x) &= 0 & (x+y)\cdot z &= x \cdot z + y\cdot z \\
  (-x) + x &= 0 & z \cdot (x+y) &= z \cdot x+ z\cdot y \\
  x + y &= y + x & x \cdot y &= y\cdot x
\end{align*}
\end{primer}
\begin{primer}
  Induktivni podatkovni tipi, kot jih poznamo v programskih jezikih, so primer algebrajske teorije.
  Na primer podatkovni tip dvojiških dreves z listi, označenimi s celimi števili,
  bi lahko definirali kot
  \begin{verbatim}
    type tree = Empty | Leaf of int | Node of tree * tree
  \end{verbatim}
  Ta definicija ustreza algebrajski teoriji, sestavljeni iz konstante $\verb|Empty|$,
  ki predstavlja prazno drevo, konstante $\verb|Leaf| \  n$, za vsak $n \in \mathbb{N}$
  in dvomestne operacije $\verb|Node|$. Zgoraj podana teorija nima enačb, in ko podamo
  podatkovni tip na tak način, imamo v mislih specifičen model, in sicer \emph{prosti}.
\end{primer}
\begin{primer}
  Pomembna struktura, ki ni primer algebrajske teorije, je teorija delno urejene množice,
  ki jo formuliramo z dvomestno operacijo $\leq$, ki ustreza aksiomom refleksivnosti,
  antisimetričnosti in tranzitivnosti.
\end{primer}
%
\begin{definicija}
  Dva terma $s, t$ teorije $\T$ sta enaka,
  če lahko v končno mnogo korakih z uporabo aksiomov $\T$ in lastnosti enakosti
  (refleksivnost, simetričnost in tranzitivnost) iz $s$ dobimo $t$ ali obratno.
  Zaporedju teh korakov pravimo \emph{izpeljava} in to, da v $\T$ obstaja izpeljava
  iz $s$ v $t$ ozačimo s
  \[ T \vdash s = t.\]
\end{definicija}
%
\section{Model algebrajske teorije}
Na primeru grup si poglejmo, kaj je \emph{model} algebrajske teorije.
Kot smo povedali zgoraj, v klasični algebri grupo razumemo kot množico $G$,
skupaj s konstanto $e \in G$ in preslikavama $i : G \to G$ in $m : G \times G \to G$.
Grupa mora ustrezati aksiomom:
\begin{align*}
   m(x,m(y,z)) &= m(m(x,y),z) \\
   m(x,e) &= m(e,x) = x \\
   m(x,i(x)) &= m(i(x),x) = e.
\end{align*}
%
To idejo klasičnega modela lahko posplošimo tako, da jo ">prevedemo"< v
kategorični jezik, kar nam bo omogočilo, da govorimo o modelih v drugih
kategorijah kot $\mathbf{Set}$. To storimo na naslednji način:
grupa je objekt $G \in \mathbf{Set}$, skupaj s tremi morfizmi:
\[ e : 1 \to G, \qquad m : G \times G \to G, \qquad i : G \to G. \]
Aksiome grupe lahko predstavimo s komutativnimi diagrami
\begin{equation}\label{diagram:asociativnost}
  \begin{tikzcd}[column sep = large]
    G \times G \times G \ar[d, "\pi_0 \times m"'] \ar[r, "m \times \pi_2"] & G \times G \ar[d, "m"] \\
    G \times G \ar[r, "m"'] & G
  \end{tikzcd}
\end{equation}
\begin{equation}\label{diagram:enota}
  \begin{tikzcd}[column sep = large]
    G \times 1 \ar[dr, "\pi_0"'] \ar[r, "1_G \times e"] & G \times G \ar[d, "m"] & 1 \times G \ar[l, "e \times 1_G"'] \ar[dl, "\pi_1"] \\
    & G
  \end{tikzcd}
\end{equation}
\begin{equation}\label{diagram:inverz}
  \begin{tikzcd}[column sep = large]
    G \ar[d, "!"'] \ar[r, "{\langle 1_G, i \rangle}"] & G \times G \ar[d, "m"] & G \ar[l, "{\langle i, 1_G \rangle}"'] \ar[d, "!"] \\
    1 \ar[r, "e"'] & G & 1 \ar[l, "e"]
  \end{tikzcd}
\end{equation}
%
Za kategorno formulacijo smo potrebovali le končne produkte v $\cat{Set}$.
Torej je isti opis smiselen v vsaki kategoriji s končnimi produkti. Torej lahko definiramo:
\emph{grupa} v kategoriji s končnimi produkti $\mathbf{C}$ je objekt
$G$, skupaj s tremi morfizmi:
\begin{center}
  \begin{tikzcd}
    G \times G \ar[r, "m"] & G & G \ar[l, "i"'] \\
    & 1 \ar[u, "e"] &
  \end{tikzcd}
\end{center}
za katere diagrami \eqref{diagram:asociativnost}, \eqref{diagram:enota} in \eqref{diagram:inverz}
komutirajo.

Podobno posplošimo homomorfizem med grupami v $\cat{C}$: Morfizem
$h : G \to H$ je homomorfizem grup, če komutira z osnovnimi operacijami,
kar lahko izrazimo s komutirajočimi diagrami
%
\begin{center}
  \begin{tikzcd}
    G \times G \ar[d, "m^G"'] \ar[r, "h^2"] & H^2 \ar[d, "m^H"] & G \ar[d, "i^G"'] \ar[r, "h"] & H \ar[d, "i^H"] & 1 \ar[d, "e^G"'] \ar[r] & 1 \ar[d, "e^H"] \\
    G \ar[r, "h"'] & H & G \ar[r, "h"'] & H & G \ar[r, "h"'] & H
  \end{tikzcd}
\end{center}
ali z enačbami
\begin{align*}
m^H \circ h^2 &= h \circ m^G \\
i^H \circ h &= h \circ i^G \\
e^H &= h \circ e^G.
\end{align*}
Skupaj s kompozitumi in identitetnimi homomorfizmi, tako dobimo
kategorijo grup v $\mathbf{C}$, ki jo označimo z $\mathbf{Group}(\mathbf{C})$.

V nadaljevanju bomo bolj pazljivo obravnavali spremenljivke.
Za dane spremenljivke $x_1, \ldots, x_n$ pravimo, da je $t$ \emph{term v kontekstu} $x_1, \ldots, x_n$,
če velja, da v $x_1, \ldots, x_n$ nastopajo vse spremenljivke, ki smo jih uporabili za
konsturkcijo terma $t$. Term $t$ v kontekstu $x_1, \ldots, x_n$ označimo z
$$x_1, \ldots, x_n \mid t.$$
Aksiomi teorije $\mathbb{T}$ so formalno gledano v bistvu pari termov v skupnem kontekstu.
Če zapišemo enačbo $s = t$ brez konteksta, se privzame kontekst, ki vsebuje natanko vse spremenljivke
iz enačbe.
\begin{definicija}
  Interpretacija $I$ signature $\Sigma$ v kategoriji $\cat{C}$ s končnimi produkti, je podana z:
  \begin{itemize}
  \item objektom $I \in \cat{C}$,
  \item morfizmom $f^I : I^k \to I$, za vsako $k$-mestno operacijo $f$.
  \end{itemize}
%
  Interpretacijo razširimo na vse terme jezika $\mathcal{L}(\Sigma)$.
  Interpretacija terma v kontekstu $[x_1, \ldots, x_n \mid t]^{I}$ je definirana rekurzivno:
  \begin{enumerate}
  \item Interpretacija spremenljivke $x_i$ je $i$-ta projekcija
    $\pi_i : I^n \to I.$
%
  \item Term oblike $f(t_1, \ldots, t_k)$ se interpretira kot
    kompozitum
    \begin{equation*}
      \begin{tikzcd}[column sep = 5em]
        I^n \ar[r, "{\langle t_1^I, \ldots, t_k^I\rangle}"] & I^k \ar[r, "f^I"] & I,
      \end{tikzcd}
    \end{equation*}
    kjer je $t_i^I : I^n \to I$ interpretacija terma $t_i$ za
    $i = 1, \ldots, k$ in je $f^I$ interpretacija osnovne operacije
    $f$.
  \end{enumerate}
\end{definicija}
%
\begin{opomba}
Interpretacija je odvisna od konteksta! Na primer, 
  term $f(x_1)$ se v kontekstu~$x_1$ interpretira kot morfizem
  $f^I : I \to I$, medtem ko se v kontekstu $x_1, x_2$ interpretira
  kot $f^I \circ \pi_1 : I^2 \to I$. Ko napišemo $t^I$ se moramo
  torej zavedati, da je to le okrajšava za bolj natančen zapis
  $[x_1, \ldots, x_n \mid t]^I$.
\end{opomba}
%
\begin{definicija}
  Naj bosta $s$ in $t$ terma v kontekstu $x_1, \ldots, x_n$.  V
  interpretaciji $I$ je enačba $s = t$ \emph{zadoščena}, če sta
  morfizma $s^I$ in $t^I$ isti morfizem v~$\cat{C}$.  V posebnem, če
  je $s = t$ aksiom teorije, potem pravimo, da je v
  interpretaciji $I$ \emph{zadoščen aksiom} $s = t$, če sta
  $[x_1, \ldots, x_n \mid s]^I$ in $[x_1, \ldots, x_n \mid t]^I$ isti
  morfizem v $\cat{C}$:
  \begin{equation*}
    \begin{tikzcd}[column sep = 7em]
      I^n \ar[r, shift left=1ex, "{[x_1, \ldots, x_n \mid s]^I}"]
      \ar[r, shift right=1ex, "{[x_1, \ldots, x_n \mid t]^I}"'] & I
    \end{tikzcd}
  \end{equation*}
  To označimo kot
  \[I \models x_1, \ldots, x_n \mid s = t.\] 
\end{definicija}
%
\begin{definicija}
Interpretacija teorije $(\Sigma_{\T}, \mathbb{A}_{\T})$ je interpretacija njene signature $\Sigma_{\T}$.
\end{definicija}
\begin{definicija}
Naj bo $\mathbb{T}$ algebrajska teorija.
\emph{Model} teorijje $\mathbb{T}$ v kategoriji~$\cat{C}$ s
končnimi produkti je interpretacija $M$, za katero velja
$$M \models s = t,$$
za vsak aksiom $s = t$ teorije $\T$.
\end{definicija}
%
\begin{definicija}
  Naj bosta $M, N$ interpretaciji signature $\Sigma_{\T}$.
 \emph{Homomorfizem interpretacij} $h : M \to N$ je morfizem v
 $\cat{C}$, ki komutira z interpretacijami osnovnih operacij,
$$h \circ f^M = f^N \circ h,$$
za vsak $f \in \Sigma_T$, kar ponazorimo s komutativnim diagramom
\begin{equation*}
  \begin{tikzcd}
    M^k \ar[d, "f^M"'] \ar[r, "h^k"] & N^k \ar[d, "f^N"] \\
    M \ar[r, "h"] & N
  \end{tikzcd}
\end{equation*}
\end{definicija}
\noindent
Ker je identitetni morfizem vedno homomorfizem in so kompozicije homomorfizmov spet homomorfizmi,
ki asociatovnost podedujejo iz $\cat{C}$, modeli in homomorfizmi tvorijo kategorijo $\Mod(\T,\cat{C})$.
%
\begin{primer}
  Model prazne teorije $\mathbb{T}_0$ je samo objekt $M \in \cat{C}$,
  homomorfizem med dvema modeloma pa je le morfizem v $\cat{C}$, brez
  dodatnih omejitev, torej
$$\mathbf{Mod}(\mathbb{T}_0, \cat{C}) = \cat{C}.$$
%
Model teorije grup $\mathbb{T}_{\mathrm{Grp}}$, v kategoriji množic
$\mathbf{Set}$, je grupa v običajnem smislu. Homomorfizem med dvema modeloma
teorije $\mathrm{Grp}$ je homomorfizem grup. Torej je
$$\mathbf{Mod}(\mathbb{T}_{\mathrm{Grp}}, \mathbf{Set}) = \mathbf{Grp}.$$
\end{primer}
% 
\begin{primer}
  Model teorije grup v funktorski kategoriji $\mathbf{Set}^{\mathbf{C}}$ je
  natanko funktor iz $\mathbf{C}$, v kategorijo grup:
  \[ \mathbf{Grp}(\mathbf{Set}^{\mathbf{C}}) \cong \mathbf{Hom}(\mathbf{C}, \mathbf{Grp}). \]
  Ta izomorfizem bomo pokazali s pomočjo evaluacijskega funktorja
  \[ \mathrm{eval}_C : \mathbf{Set}^{\mathbf{C}} \to \mathbf{Set}, \]
  ki obstaja za vsak objekt $C \in \cat{C}$ in je definiran kot
  $\mathrm{eval}_{C}(F) := F(C)$, za funktor $F : \mathbf{C} \to \mathbf{Set}$.
  Ker produkte računamo ">po komponentah"<, evaluacija ohranja produkte.
  Za vsak morfizem~${h : C \to D}$ v $\mathbf{C}$ dobimo naravno
  transformacijo $h : \mathrm{eval}_{C} \to \mathrm{eval}_{D}$.
  Torej za vsako grupo $G$ v $\mathbf{Set}^{\mathbf{C}}$ dobimo grupo
  $\mathrm{eval}_{C}(G)$, za vsak $C \in \mathbf{C}$ in homomorfizem grup
  $h_{G} : C(G) \to D(G)$, za vsak morfizem $h : C \to D$,
  kar definira funktor $G : \mathbf{C} \to \mathbf{Grp}$.
  Obratno, vsak funktor $H : \mathbf{C} \to \mathbf{Grp}$, ki ohranja produkte,
  dobimo kot grupo v $\mathbf{Set}^{\mathbf{C}}$, kjer so komponente
  naravnih transformacij grupnih operacij ravno operacije tiste grupe.
  Aksiomom grupe zadoščajo, ker jih zadoščajo za vsako komponento posebej.
  Torej je res vsak funktor $\cat{C} \to \cat{Grp}$ grupa v $\cat{Grp}(\cat{Set}^{\cat{C}})$.
\end{primer}
%
\section{Teorije kot kategorije}
\label{sec:teorije-kot-kategorije}
Predstavitev algebrajske teorije, kot smo jo podali zgoraj za primer
teorije grup, je v svojem bistvu sintaktične narave, kar pa ima
svoje pomankljivosti.
Želeli bi najti pogled, ki bi nam omogočal govoriti o tem,
kaj grupa \emph{je}, brez tega, da bi se morali odločati,
na kak način bomo grupo \emph{zapisali}.
To nam bo omočila kategorija z določeno \emph{univerzalno lastnostjo},
ki jo bo natanko določala, do ekvivalence natančno.
To nas bo privedlo do reformulacije klasičnih konceptov
sintakse in semantike na način, ki bolj ustreza strukturalističnemu pogledu na matematiko,
kot ga zaobjema teorija kategorij.

Kaj je mišljeno z izbiro predstavitve, si poglejmo na primeru grup.
Formulacija, ki smo jo podali zgoraj, ni edina možna.
Obstaja alternativna predstavitev teorije grup,
z enoto $e$ in dvojiško operacijo $\odot$,
imenovano dvojno deljenje, in enim samim aksiomom \cite{mccune1993single}:
$$x \odot (((x \odot y ) \odot z ) \odot ( z \odot e))) \odot (e \odot e) ) = z.$$
Običajne operacije teorije grup lahko dobimo s formulami:
$$x \odot y = x^{-1} \cdot y^{-1} \text{,} \quad x^{-1} = x \odot e \text{,} \quad x \cdot y = (x \odot e) \odot (y \odot e)$$
%
Obstajajo različni razlogi za uporabo ene ali druge predstavitve
teorije grup, mogoče ima ena predstavitev boljše računske lastnosti,
a se da z drugo na bolj jasen način definirati nek koncept.
Želimo pa si definicijo, ki bi hkrati zaobjela vse možne predstavitve grupe.
Zato bi se radi izognili specifični izbiri konstant, operacij in aksiomov.

Kot prvi korak do take sintaktično neodvisne specifikacije
bi lahko kot osnovne vzeli vse operacije, ki jih lahko zgradimo iz
enote, množenja in inverza in vse
veljavne enačbe v teoriji grup kot aksiome. Lahko gremo pa še korak dlje
in zberemo vse operacije v kategorijo in tako pozabimo, katere so bile
osnovne in katere izpeljane ter katere enakosti so bile podane kot aksiomi.
%
\begin{definicija}
  Naj bo $\mathbb{T}$ algebrajska teorija.
  \emph{Sintaktična kategorija} $\cat{C}_\mathbb{T}$ je kategorija,
  v kateri so
  \begin{itemize}
  \item objekti: konteksti $[x_1, \ldots, x_n]$ za $n \geq 0$,
%
  \item morfizmi: morfizem $[x_1, \ldots, x_m] \to [x_1, \ldots, x_n]$
    je $n$-terica $\langle t_1, \ldots, t_n \rangle$ termov v kontekstu
    $x_1, \ldots, x_m$. Morfizma $\langle t_1 \ldots t_n \rangle$ in
    $\langle s_1, \ldots, s_n \rangle$ sta enaka, če in samo če,
    v teoriji $\mathbb{T}$, za vsak $k = 1, \ldots, n$ velja
    $$\mathbb{T} \vdash t_k = s_k.$$
\end{itemize}
%
Morfizmi so torej v resnici ekvivalenčni razredi termov v kontekstu
$$[x_1, \ldots, x_m \mid t_1, \ldots, t_n] : [x_1, \ldots, x_m] \to [x_1, \ldots, x_n].$$
%
Kompozitum morfizmov
  \begin{align*}
    \langle t_1, \ldots, t_m \rangle : [x_1, \ldots, x_k] &\to [x_1, \ldots, x_m], \\
    \langle s_1, \ldots, s_n \rangle : [x_1, \ldots, x_m] &\to [x_1, \ldots, x_n],
  \end{align*}
je morfizem $\langle r_1, \ldots, r_n \rangle$, kjer dobimo $i$-to komponento tako,
da v $s_i$ hkrati vstavimo terme $t_1, \ldots, t_m$ namesto
spremenljivk $x_1, \ldots, x_m$.
$$r_i = s_i[t_1, \ldots, t_m / x_1, \ldots, x_m].$$
%
\end{definicija}
Kategorijo $\mathbf{C}_{\mathbb{T}}$ si lahko predstavljamo kot
kategorijo Lindenbaum-Tarskega, ki je kategorična različica znane konstrukcije v algebri.
Vsebuje vse algebraične
informacije kot originalna teorija, le na sintaktično neodvisen način.
Različne predstavitve teorije porodijo ekvivalentne sintaktične
kategorije. V nadaljevanju bomo videli, da ima $\mathbf{C}_{\mathbb{T}}$ še eno,
še bolj pomembno lastnost, s katero predstavlja teorijo $\mathbb{T}$.
%
\begin{lema}
  Naj bo $\mathbb{T}$ algebrajska teorija in $\cat{C}_\mathbb{T}$
  njena sintaktična kategorija. Potem ima $\cat{C}_\mathbb{T}$ vse
  končne produkte in velja
$$[x_1, \ldots, x_n] \times [x_1, \ldots, x_m] \cong [x_1, \ldots, x_{n+m}].$$
\end{lema}
\begin{dokaz}
  Projekcijo na prvo komponento $\pi_1: [x_1, \ldots, x_{n+m}] \to [x_1, \ldots, x_n]$
  definiramo kot $n$-terico termov, kjer je $i$-ta komponenta enaka $i$-ti spremenljivki
  \[ x_1, \ldots, x_{n+m} \mid x_i\]
  za $i = 1, \ldots, n$. Drugo projekcijo $\pi_2$ definiramo podobno, le da je $j$-ta
  komponenta enaka spremenljivki $x_{n+j}$ za $j = 1, \ldots, m$. Recimo, da imamo morfizma
  \[\langle t_1, \ldots t_n \rangle: [x_1, \ldots, x_k] \to [x_1, \ldots, x_n]\]
  in
  \[\langle s_1, \ldots s_n \rangle: [x_1, \ldots, x_k] \to [x_1, \ldots, x_m],\]
  za nek objekt $[x_1, \ldots, x_k] \in \cat{C}_{\T}$.
  Potem dobimo morfizem
  \[\langle t_1, \ldots, t_n, s_1, \ldots s_m \rangle: [x_1, \ldots, x_k] \to [x_1, \ldots, x_{n+m}]\]
  za katerega velja
  \[\pi_1 \circ \langle t_1, \ldots, t_n, s_1, \ldots s_m \rangle = \langle t_1, \ldots t_n \rangle\]
  in
  \[\pi_2 \circ \langle t_1, \ldots, t_n, s_1, \ldots s_m \rangle = \langle s_1, \ldots s_m \rangle.\]
  To je samo substitucija za $i$-to komponento.
  Za enoličnost predpostavimo, da obstaja še en morfizem
  $\langle r_1, \ldots, r_{n+m} \rangle :  [x_1, \ldots, x_k] \to [x_1, \ldots, x_{n+m}]$,
  za katerega pri kompozitumu s $\pi_1$ dobimo $\langle t_1, \ldots t_n \rangle$,
  s $\pi_2$ pa $\langle s_1, \ldots s_m \rangle$. To pa pomeni, da velja
  \[ \T \vdash r_i = t_i \quad \text{in} \quad  \T \vdash r_{j+n} = s_j,\]
  za  $i = 1, \ldots, n$ in $j = 1, \ldots, m$.
\end{dokaz}
%
\begin{definicija}\label{def:sintaktični_model}
\emph{Sintaktični model} $U$ definiramo kot:
\begin{itemize}
\item Za objekt vzamemo $U := [x_1]$, kontekst dolžine ena.
%
\item Interpretacijo $k$-mestne osnovne operacije $f$ definiramo kot
  samo sebe
$$f^U := [x_1, \ldots, x_n \mid f(x_1, \ldots, x_n)] : U^k \to U$$
\end{itemize}
%
Interpretacijo z indukcijo razširimo na vse terme.
Aksiomi $\mathbb{T}$ so izpolnjeni, saj za vsaka terma $t$, $s$ velja
$$U \models t = s \Longleftrightarrow t^U = s^U \Longleftrightarrow \mathbb{T} \vdash t = s.$$
Velja torej
$U \in \mathbf{Mod}(\mathbb{T}, \mathbf{C}_\mathbb{T})$.
\end{definicija}
%
\section{Modeli kot funktorji}
%
Videli bomo, da ima $\mathbf{C}_{\mathbb{T}}$ to posebno lastnost, da modeli
teorije $\mathbb{T}$ natanko ustrezajo določenemu razredu funktorjem iz $\mathbf{C}_{\mathbb{T}}$.

\begin{izrek}\label{sec:modeli-kot-funtorji}
Naj bo $\mathbb{T}$ algebrajska teorija in $\mathbf{C}_{\mathbb{T}}$ njena
sintaktična kategorija. Potem za vsako kategorijo $\mathbf{C}$, ki ima končne
produkte, obstaja naravna ekvivalenca med modeli teorije $\mathbb{T}$ v
$\mathbf{C}$ in funktorji, ki ohranjajo končne produkte, iz
$\mathbf{C}_{\mathbb{T}}$ v $\mathbf{C}$.
$$M \in \mathbf{Mod}(\mathbb{T}, \mathbf{C})\quad \leftrightsquigarrow \quad \cat{M} : \mathbf{C}_\mathbb{T} \to \mathbf{C}$$
Vsak model je podan kot funktorialna slika sintaktičnega modela
$U$ z do izomorfizma natančno enoličnim funktorjem
$M : \mathbf{C}_{\mathbb{T}} \to \mathbf{C}$, ki ohranja produkte tako, da velja
$M \cong \mathcal{M}(\mathcal{U})$.
\end{izrek}
\begin{opomba}
  Tu imamo v mislih funktorje $F$, ki \emph{strogo} ohranjajo produkte, kar pomeni,
  da $F(A \times B) = F(A) \times F(B)$.
\end{opomba}
V nadaljevanju podrazdelka podamo vse sestavine za dokaz zgornjega izreka.

%
\begin{trditev}
  Naj bo $U$ sintaktični model iz definicije \ref{def:sintaktični_model}. Potem evaluacija pri $U$
  določa funktor
  \[ \mathrm{eval}_U : \mathbf{Hom}_{\mathrm{FP}}(\mathbf{C}_{\mathbb{T}}, \mathbf{C})
    \to \mathbf{Mod}(\mathbb{T}, \mathbf{C}), \]
  iz kategorije funktorjev $\mathbf{C}_{\mathbb{T}} \to \mathbf{C}$,
  ki ohranjajo končne produkte in naravnih transformacij med njimi,
  v kategorijo modelov teorije $\mathbb{T}$ v $\mathbf{C}$.
\end{trditev}
\begin{dokaz}
Denimo, da imamo funktor $F : \mathbf{C}_{\mathbb{T}} \to \mathbf{C}$,
ki ohranja končne produkte.
Skupaj z interpretacijami $f^{F(U)} := F(f^U)$, za $k$-mestne osnovne operacije $f$,
je slika $F(U)$ model v $\cat{C}$.
To sledi iz tega, da za vsak aksiom $s = t$ velja
$$\mathbb{T} \vdash s = t  \iff  s^U = t^U,$$
iz česar zaradi funktorialnosti $F$ sledi
$$t^{FU} = F(t^U) = F(s^U) = s^{FU}.$$
%
Vsaka naravna transformacija $\vartheta : F \to G$, med takima
funktorjema, določi homomorfizem modelov
$$\vartheta_U : FU \to GU,$$
%
saj za vsako $k$-mestno osnovno operacijo $f$, zaradi naravnosti velja
$$\vartheta_U \circ f^{FU} = f^{GU} \circ (\vartheta_U)^k.$$
To lahko razberemo iz diagrama
\[
  \begin{tikzcd}[column sep = large, row sep = large]
    (FU)^k \ar[d] \ar[r, "(\vartheta_U)^k"] \ar[dd, bend right=60, "f^{FU}"'] &
    (GU)^k \ar[d] \ar[dd, bend left = 60, "f^{GU}"] \\
    F(U^k) \ar[d, "F(f^U)"] \ar[r, "\vartheta_{U^k}"] &
    G(U^k) \ar[d, "G(f^U)"'] \\
    FU \ar[r, "\vartheta_U"'] & GU
  \end{tikzcd}
\]
kjer sta v zgornjem kvadratu navpična morfizma kar identiteti,
ker $F$ ohranja produkte.
Spodnji kvadrat komutira zaradi naravnosti $\vartheta$.
Torej evaluacija pri $U$ res določa funktor.
\end{dokaz}
\begin{trditev}\label{trditev:modeli-so-funktorji}
  Funktor $\mathrm{eval}_U$, opisan zgoraj, določa ekvivalenco kategorij,
  ki je naravna v $\mathbf{C}$.
\end{trditev}
\begin{dokaz}
Za to je dovolj pokazati, da je $eval_U$ surjektiven na objektih, zvest in poln.
Za surjektivnost denimo, da imamo model
$M \in \mathbf{Mod}(\mathbb{T}, \mathbf{C})$. Potem lahko definiramo funktor
$$M^{\#} : \mathbf{C}_\mathbb{T} \to \mathbf{C},$$
ki na objektih deluje kot $M^{\#}([x_1, \ldots, x_n]) = M^n$, na
morfizmih pa
$$M^{\#}(t_1, \ldots, t_n) = \langle t_1^M, \ldots, t_n^M \rangle.$$
%
Bolj natančno, je $M^{\#}$ na morfizmih definiran induktivno:
\begin{enumerate}
\item Morfizem $$x_i : [x_1, \ldots, x_n] \to [x_1]$$ se slika v
  $i$-to projekcijo $$\pi_i : M^n \to M.$$
\item Morfizem $$f(t_1, \ldots, t_n) : [x_1, \ldots, x_n] \to [x_1]$$
  se slika v kompozitum
\begin{equation*}
    \begin{tikzcd}[column sep = 8em]
      M^m \ar[r, "{\langle M^{\#}t_1, \ldots, M^{\#}t_n \rangle}"] & M^n \ar[r,
      "{M^{\#}f}"] & M,
    \end{tikzcd}
  \end{equation*}
  kjer je $M^{\#}f = f^M$ interpretacija osnovne operacije $f$,
  podana z modelom.
\end{enumerate}
%
Dobra definiranost $M^{\#}$ sledi iz tega, da je $M$ model, torej res zadošča vsem enačbam teorije.
Definirali smo ga na tak način, da velja
$$M = M^{\#}(U).$$
%
Za polnost in zvestost moramo pokazati surjektivnost ter injektivnost
predpisa $$\vartheta \mapsto \vartheta_U,$$
kjer je $\vartheta$ naravna transformacija med funktorjema $F$ in $G$, ki ohranjata končne limite.
Za surjektivnost denimo, da je $h : F(U) \to G(U)$ homomorfizem modelov. Tedaj je $h = \vartheta_U$,
kjer je $\vartheta$ naravna transformacija definirana z družino $\vartheta_{[x_1, \ldots, x_n]} := h^n$.
Da je to res naravna transformacija sledi iz dejstva, da je $h$ homomorfizem modelov,
torej komutira z interpretacijami osnovnih operacij.
Za injektivnost denimo, da velja $\vartheta_U = \phi_U $.
Potem zaradi ohranjanja končnih produktov dobimo
\[ \vartheta_{[x_1, \ldots, x_n]} = (\vartheta_{[x_1]})^k = (\phi_{[x_1]})^k = \phi_{[x_1,\ldots, x_n]}. \]

Pokazati moramo še, da je ta ekvivalenca naravna v $\mathbf{C}$.
To pomeni naslednje, denimo da je $M$ model $\mathbb{T}$ v neki kategoriji
$\mathbf{C}$ s končnimi produkti. Vsak funktor $F : \mathbf{C} \to \mathbf{D}$,
ki ohranja končne produkte, v drugo kategorijo s končnimi produkti,
pošlje $M$ v model $F(M)$ v $\mathbf{D}$. Interpretacijo osnovne operacije~$f$
v $\mathbf{D}$ dobimo s predpisom $f^{F(M)} = F(f^M)$, ki jo razširimo na
vse terme preko enakosti $F(M)^k = F(M^k)$.
Ker enačbe podamo s komutativnimi diagrami, $F$ pošlje model v model in
homomorfizem modelov v homomorfizem modelov. To pomeni, da $F$ inducira
funktor
\[ \mathbf{Mod}(\mathbb{T}, F) : \mathbf{Mod}(\mathbb{T}, \mathbf{C}) \to \mathbf{Mod}(\mathbb{T}, \mathbf{D}). \]
Da je $\mathrm{eval}_U$ res naraven, mora diagram
\begin{equation}
\begin{tikzcd}
  \mathbf{Hom}_{\mathrm{FP}}(\mathbf{C}_{\mathbb{T}}, \mathbf{C}) \ar[r, "{\mathrm{eval}_U}"]
  \ar[d, "{\mathbf{Hom}_{\mathrm{FP}}(\cat{C}^{\mathbb{T}}, F)}"']& \mathbf{Mod}(\mathbb{T}, \mathbf{C})
  \ar[d, "{\mathbf{Mod}(\mathbb{T}, F)}"] \\
  \mathbf{Hom}_{\mathrm{FP}}(\mathbf{C}_{\mathbb{T}}, \mathbf{D}) \ar[r, "{\mathrm{eval}_{U}}"']   & \mathbf{Mod}(\mathbb{T}, \mathbf{C})
\end{tikzcd}
\end{equation}
komutirati. To lahko vidimo, saj za vsak funktor $M : \mathbf{C}_{\mathbb{T}} \to \mathbf{C}$,
ki ohranja produkte velja
\begin{align*}
  (\mathrm{eval}_U \circ \mathbf{Hom}_{\mathrm{FP}}(\mathbf{C}_{\mathbb{T}}, F))(M) &= (\mathbf{Hom}_{\mathrm{FP}}(\mathbf{C}_{\mathbb{T}}, F)(M))(U) \\
                                                                                  &= (F \circ M)(U) \\
                                                                                  &= F(M(U)) \\
                                                                                  &= F(\mathrm{eval}_U(M)) \\
                                                                                  &\cong \mathbf{Mod}(\mathbb{T}, F)(\mathrm{eval}_U(M)) \\
  &= \mathbf{Mod}(\mathbb{T}, F) \circ \mathrm{eval}_U(M).
\end{align*}
\end{dokaz}
S tem smo dokazali izrek \ref{sec:modeli-kot-funtorji}.
Ekvivalenca kategorij
\[ \mathbf{Hom}_{\mathrm{FP}}(\mathbf{C}_{\mathbb{T}}, \mathbf{C}) \simeq \mathbf{Mod}(\mathbb{T}, \mathbf{C}) \]
dejansko določi kategorijo $\mathbf{C}_{\mathbb{T}}$ in sintaktični model $U$ enolično,
do ekvivalence kategorij in izomorfizma modela natančno.
Da dobimo $U$ lahko v zgornjo ekvivalenco za $\mathbf{C}$ vstavimo $\mathbf{C}_{\mathbb{T}}$
in identitetni funktor $1_{\mathbf{C}_{\mathbb{T}}}$ na levi in na desni dobimo
model $U$. To lahko formuliramo kot univerzalno lastnost:
\begin{definicija}
  \emph{Klasifikacijska kategorija} algebrajske teorije $\mathbb{T}$
  je kategorija s končnimi produkti $\mathbf{C}_\mathbb{T}$, s
  posebnim modelom $\mathcal{U}$, imenovanim \emph{univerzalni model},
  tako da velja:
  \begin{enumerate}
  \item Za vsak model $M$, v kategoriji s končnimi produkti
    $\mathbf{C}$, obstaja funktor, ki ohranja končne limite
    $$M^{\#} : \mathbf{C}_\mathbb{T} \to \mathbf{C}$$ in izomorfizem
    modelov $M \cong M^{\#}(\mathcal{U})$.
%
  \item Za vsaka funktorja, ki ohranjata končne limite
    $F,G : \mathbf{C}_\mathbb{T} \to \mathbf{C}$ in homomorfizem
    modelov $h: F(\mathcal{U}) \to G(\mathcal{U})$, obstaja natanko
    ena naravna transformacija $\vartheta : F \to G$, da velja
$$\vartheta_\mathcal{U} = h.$$
\end{enumerate}
\end{definicija}
Pokazali smo naslednji izrek
\begin{izrek}
\label{sec:klasifikacijska-kategorija-alg-teorije}
Za vsako algebrajsko teorijo $\mathbb{T}$ je sintaktična kategorija $\mathbf{C}_{\mathbb{T}}$
njena klasifikacijska kategorija.
\end{izrek}
Kot običajno univerzalna lastnost določa kategorijo $\cat{C}_{\mathbb{T}}$ do ekvivalence kategorij natančno.
%
Naj bosta $(\mathbf{C}, U)$ in $(\cat{D}, V)$ obe klasifikacijski
kategoriji neke teorije $\mathbb{T}$. Potem po univerzalni lastnosti
dobimo funktorja
\begin{center}
  \begin{tikzcd}[column sep = large]
    \mathbf{C} \ar[r, bend left, "U^{\#}"] & \cat{D} \ar[l, bend
    left, "V^{\#}"]
  \end{tikzcd}
\end{center}
tako, da velja
$$U \cong U^{\#}(V) \cong U^{\#}(V^{\#}(U))$$
in obratno
\[ V \cong V^{\#}(U) \cong V^{\#}(U^{\#}(V)).\]

\begin{primer}
  Poglejmo si, kaj nam ta konstrukcija pove v primeru teorije grup
  $\mathbb{G} = \mathbf{C}_{\mathbb{T}_{\mathrm{Grp}}}$. Spomnimo se,
  da je $\mathbb{G}$ sestavljena iz kontekstov $[x_1, \ldots x_n]$ in
  termov zgrajenih iz spremenljivk in osnovnih grupnih operacij.
  Funktor, ki ohranja končne produkte, $M : \mathbb{G} \to \mathbf{Set}$
  je določen do izomorfizma natančno s tem, kam pošlje kontekst $[x_1]$
  in terme, ki predstavljajo osnovne operacije. Če definiramo
\begin{align*}
  G &= M([x_1]), & e &= M(\cdot \mid e), \\
  i &= M(x_1 \mid x_1^{-1}), & m &= M(x_1,x_2 \mid x_1 \cdot x_2),
\end{align*}
potem je $(G,e,i,m)$ navadna grupa z enoto $e$, inverzom $i$ in
množenjem $m$. Množica $G$ z operacijami res zadošča aksiomom
grup, zaradi funktorialnosti~$M$. Obratno, če je $(G,e,i,m)$ grupa,
potem dobimo funktor $M_{G} : \mathbb{G} \to \mathbf{Set}$,
ki ohranja končne produkte tako, da definiramo
\begin{align*}
  M_{G}([x_1, \ldots, x_n]) &= G^n & M_{G}(\cdot \mid e) &= e \\
  M_{G}(x_1 \mid x_1^{-1}) &= i & M_{G}(x_1, x_2 \mid x_1 \cdot x_2) &= m.
\end{align*}

Recimo, da sta $(G,e_{G}, i_{G}, m_{G})$ in $(H, e_H, i_H, m_H)$ grupi
in naj bo ${\phi : M_{G} \to M_H}$ naravna transformacija.
Potem je $\phi$ določena s tem, kam slika kontekst $[x_1]$,
kajti po naravnosti za $1 \leq k \leq n$, naslednji diagram
\begin{equation*}
\begin{tikzcd}\label{primer:grupe-kot-funktorji}
  G^n \ar[d, "G\pi_k = \pi_k"'] \ar[r, "\phi_{[x_1, \ldots, x_n]}"] & H^n \ar[d, "H\pi_k = \pi_k"] \\
  G \ar[r, "\phi_{[x_1]}"'] & H
\end{tikzcd}
\end{equation*}
komutira. Če označimo $\phi' = \phi_{[x_1]}$, potem velja
$\phi_{[x_1, \ldots, x_n]} = \phi' \times \ldots \times \phi'$.
Spet lahko vidimo, da zaradi naravnosti naslednji diagram
\begin{equation*}
\begin{tikzcd}\label{sec:modeli-kot-funtorji-3}
  G \times G \ar[d, "m_{G}"'] \ar[r, "\phi' \times \phi'"] & H \times H \ar[d, "m_H"] \\
  G \ar[r, "\phi'"'] & H
\end{tikzcd}
\end{equation*}
komutira. Podobni diagrami pokažejo, da $\phi'$ ohranja enoto in
inverze, kar pomeni da je $\phi' : G \to H$ res homomorfizem grup.
Obratno, homomorfizem grup ${\varphi' : G \to H}$ inducira naravno
transformacijo $\varphi : G \to H$, katere komponenta pri
$[x_1, \ldots, x_n]$ je enaka $\varphi' \times \ldots \times \varphi' : G^n \to H^n$.
Demonstrirali smo ekvivalenco kategorij
\[ \mathbf{Mod}_{\mathbf{Set}}(\mathbb{G}) \simeq \mathbf{Grp}. \]
\end{primer}
\section{Polnost algebrajske teorije}
V matematični logiki pravimo, da je logičen sistem \emph{veljaven},
če vsaka formula, ki jo lahko dokažemo v tem sistemu, resnična
glede na semantiko tega sistema. Logične formule algebrajske teorije
$\mathbb{T}$ imajo zelo preprosto strukturo, vse so namreč enačbe
oblike $s = t$ med termi te teorije.
Taka teorija je veljavna v danem modelu, ko le-ta zadošča vsem enačbam teorije.
Obratna lastnost logičnega sistema se imenuje \emph{polnost}.
Pravimo, da je teorija $\mathbb{T}$ \emph{(semantično) polna}, če velja:
$$\forall M \in \Mod(\mathbb{T}).M \models s = t \implies \mathbb{T} \vdash s = t.$$
Za teorije prvega reda velja naslednji slaven izrek o polnosti, ki
ga lahko ljubitelji nemščine najdemo v \cite{godel-completeness},
ostali pa v \cite{prijatelj1992osnove2}.
\begin{izrek}[Gödel]
  Naj bo $T$ teorija klasične logike prvega reda in $\varphi$ formula v jeziku teorije $T$.
  Potem so za modele v kategoriji množic naslednje trditve ekvivalentne:
  \begin{enumerate}
  \item Obstaja dokaz $\varphi$ iz aksiomov teorije $T$.
  \item Stavek $\varphi$ drži v vsakem modelu teorije $T$.
  \end{enumerate}
\end{izrek}
To velja v posebno močnem smislu tudi za kategorično semantiko.
\begin{izrek}[Polnost algebrajskih teorij]
  Naj bo $\mathbb{T}$ algebrajska teorija. Potem obstaja kategorija s
  končnimi produkti $\mathbf{C}$ in model
  $\mathcal{U} \in \mathbf{Mod}(\mathbb{T}, \mathbf{C})$ tako, da za vsako
  enačbo $s = t$ teorije $\mathbb{T}$ velja
$$\mathcal{U} \models s = t \quad \Longleftrightarrow \quad \mathbb{T} \vdash s = t.$$
\end{izrek}
\begin{dokaz}
  Kot kategorijo $\mathbf{C}$ vzamemo klasifikacijsko kategorijo
  $\mathbf{C}_\mathbb{T}$ in univerzalni model $\mathcal{U}$. Če
  velja $\mathbb{T} \vdash s = t$, potem iz konstrukcije
  $\mathbf{C}_\mathbb{T}$ sledi $s^\mathcal{U} =
  t^\mathcal{U}$. Vsak model $M$ v kategoriji $\mathbf{C}$
  s končnimi produkti ima klasifikacijski funktor
  \[  M^{\#} : \mathbf{C}_{\mathbb{T}} \to \mathbf{C}, \]
  ki ohranja interpretacijo termov $s$ in $t$, kajti velja
  \[ M^{\#}(s^\mathcal{U}) = s^{M^{\#}(\mathcal{U})} = s^M. \]
  Podobno za $t$.
  To pomeni, da iz $s^{\mathcal{U}} = t^{\mathcal{U}}$ sledi $s^M = t^M$.
  Obratno, če ${\mathcal{U} \models s = t}$, potem velja $s^{\mathcal{U}} = t^{\mathcal{U}}$.
  Iz konstrukcije $\mathbf{C}_{\mathbb{T}}$ potem sledi ${\mathbb{T} \vdash s = t}$.
\end{dokaz}
\begin{definicija}
  Pravimo, da je model teorije $\mathbb{T}$ \emph{logično generičen},
  kadar v njem veljajo natanko tiste izjave, ki jih teorija $\mathbb{T}$ dokaže.
\end{definicija}
Po prejšnjem izreku je torej univerzalni model logično generičen.
V klasičnem smislu v kategoriji $\mathbf{Set}$ le redko obstaja
logično generičen model. Za polnost moramo gledati veljavnost
v vseh modelih teorije hkrati. Za algebrajske teorije v tem posplošenem
smislu model velja torej močnejša lastnost.
Za modele v kategoriji $\mathbf{Set}^{\mathbf{C}^{\mathrm{op}}}$
imamo posebej lep generičen model.
\begin{trditev}
  Naj bo $\mathbb{T}$ algebrajska teorija. Potem je Yonedova vložitev
  $y(U)$ univerzalnega modela generičen model za $\mathbb{T}$.
\end{trditev}
\begin{dokaz}
  Yonedova vložitev ohranja limite, torej v posebnem končne produkte.
  Sledi, da nam da model $\tilde{U} = \mathsf{y}(U)$ teorije $\mathbb{T}$.
  Ker je $\mathsf{y}$ funktor, so v~$\tilde{U}$ zadoščene vse enačbe,
  ki veljajo v $U$, ker pa je $\mathsf{y}$ zvest funktor, je vsaka enačba,
  ki velja v $\tilde{U}$, morala veljati že v $U$.
  Ker je $U$ logično generičen, mora biti tudi $\tilde{U}$.
\end{dokaz}
\begin{primer}
  Poglejmo si, kaj nam to pove v primeru teorije grup. Univerzalen model
  grupe je grupa, ki zadošča vsem enačbam, ki veljajo za vse grupe
  in nobenim drugim. Spomnimo se, da je univerzalni model $U$
  predstavljen s kontekstom $[x_1]$ v $\mathbb{G}$.
  Yonedova vložitev nam da
  \[ \tilde{U} = \mathsf{y}[x_1] = \mathbb{G}(-, [x_1]). \]
  To je množica, ki jo parametrizirajo objekti $\mathbb{G}$.
  Za vsak $n \in \mathbb{}N$ dobimo množico $\mathbb{G}([x_1, \ldots, x_x], [x_1])$,
  vseh termov, ki jih lahko konstruiramo iz $n$ spremenljivk,
  modulo enačbe teorije grup.
  To pa je ravno prosta grupa z $n$ generatorji. Enota, inverz in množenje
  v $\tilde{U}$ so definirani za vsak $n$ kot te operacije na
  prosti grupi z $n$ generatorji.

  Univerzalna grupa je torej družina vseh prostih grup na $n$ generatorjih,
  kjer je $n \in \mathbb{N}$ parameter.
\end{primer}
%
\section{Funktorialna semantika}
\label{sec:funkt-semant}
%
Stopimo sedaj korak nazaj in si poglejmo naš pristop do te točke.
V ta namen si poglejmo, kako se primerja s klasičnim pristopom logike.
Klasični pristop sestavljajo naslednji štirje deli:
\begin{enumerate}
\item \emph{Teorija tipov}, na kateri temelji jezik logike, ki jo obravnavamo.
  V našem primeru gre za enostavno teorijo tipov z enim samim tipom in termi,
  ki so sestavljeni iz spremenljivk in osnovnih operacij.
\item \emph{Logika} s katero obravnavamo našo teorijo. Tu gre za vprašanje,
  kaj so logične operacije, ki jih imamo na voljo. V primeru algebrajskih teorij
  gre za zelo enostavno logiko, ki vključuje le enakosti med termi jezika.
\item \emph{Teorija} je sestavljena iz osnovnih tipov, termov in aksiomov.
  Tipi in termi so podani s teorijo tipov, aksiomi pa z logiko našega sistema.
\item \emph{Interpretacija} algebrajske teorije je definirana kot množica $I$,
  skupaj z zbirko $n$-mestnih operacij predstavljenih s preslikavami $I^n \to I$.
  Interpretaciji včasih rečemo tudi \emph{struktura}. Videli smo, da je
  koncept interpretacije algebrajske teorije smiselno definirati za vsako
  kategorijo s končnimi produkti. V taki kategoriji je predstavljena z
  objektom in zbirko morfizmov, ki predstavljajo osnovne operacije, ki jih
  z indukcijo razširimo na vse terme teorije. Interpretacija je \emph{model}
  teorije, če izpolnjuje vse aksiome. To pomeni, da so
  morfizmi, ki predstavljajo terme, ki nastopajo v aksiomih, enaki.  
\end{enumerate}
V tem poglavju smo razvili alternativen pristop, ki ga lahko poimenujemo
\emph{funktorialna semantika}, kjer so ključne naslednje ideje.
\begin{enumerate}
\item \emph{Teorije kot kategorije}. Iz teorije smo konstruirali posebno
  kategorijo, ki odraža praktično enake informacije kot teorija sama,
  a to stori na način, ki je tako rekoč sintaktično neodvisen, namreč
  ne glede na izbiro predstavitve porodi ekvivalentno kategorijo.
  Za algebrajske teorije na teoriji tipov z enim samim tipom dobimo
  kategorije s končnimi produkti.
\item \emph{Modeli kot funktorji}. Videli smo, da model teorije lahko izrazimo
  kot funktor iz kategorije, ki predstavlja teorijo, v kategorijo z ustrezno
  strukturo, da lahko interpretiramo logiko te teorije. V našem primeru
  je bila ta struktura obstoj končnih produktov. 
\item \emph{Homomorfizmi kot naravne transformacije}. Če definiramo model
  kot funktor, ki ohranja ustrezno strukturo, nas to neposredno privede
  do koncepta homomorfizma med modeli.
  To je naravna transformacija med funktorji, ki
  predstavljajo te modele. Videli smo, da ohranjanje algebrajske strukture
  ustreza ravno pogoju naravnosti.
\item \emph{Univerzalni model}. Ker interpretacije teorije nismo omejili le na kategorijo množic,
  ima naš pristop univerzalne modele.
\item \emph{Veljavnost in polnost}. S konstrukcijo klasifikacijske kategorije
  smo pokazali veljavnost in polnost algebrajskih teorij, glede na
  kategorično semantiko.
\end{enumerate}
% 
%
%
\section{Lawverjeva dualnost}
%
V logiki obstaja fascinantna in daleč razsežna dualnost oblike
\[ \mathrm{Sintaksa} \simeq \mathrm{Semantika}^{\mathrm{op}}, \]
ki jo je odkril F.~W.~Lawvere v svoji disertaciji \cite{lawvere1963functorial}.

To dualnost je mogoče zelo jasno opisati za primer algebrajskih
teorij. Naj bo \(\mathbf{C}_{\mathbb{T}}\) klasifikacijska kategorija
za algebrajsko teorijo \(\mathbb{T}\). Videli bomo, da je \(\mathbf{C}_{\mathbb{T}}\)
dualna neki podkategoriji \(\mathbb{M}\) modelov \(\mathbb{T}\) (v \(\mathbf{Set}\)).
Specifično, obstaja \emph{polna} podkategorija 
\(\mathbb{\mathbb{M}} \hookrightarrow \mathbf{Mod}(\mathbb{T})\) in
ekvivalenca kategorij
\[ \mathbf{C}_{\mathbb{T}} \simeq \mathbb{M}^{\mathrm{op}} \] tako, da
je \emph{sintaktična} kategorija \(\mathbf{C}_{\mathbb{T}}\) dualna
podkategoriji \emph{semantične} kategorije
\(\mathbf{Mod}(\mathbb{T})\). Torej imamo v kategoriji modelov
\(\mathbb{T}\) ">skrito"< neodvisno predstavitev sintakse teorije
\(\mathbb{T}\). Imamo očiten pozabljivi funktor
$\mathbf{Mod}(\mathbb{T}) \to \mathbf{Set}$, ki pošlje model~$M$ v
množico $|M|$ njegovih elementov. Ta funktor
ima levi adjunkt, ki mu pravimo \emph{prosti} funktor in ga označimo z
\begin{equation}
  \label{eq:prosti-funktor}
F : \mathbf{Set} \to \mathbf{Mod}(\mathbb{T}, \mathbf{Set}).
\end{equation}
Iz adjunkcije dobimo izomorfizem
\[ \mathbf{Hom}_{\mathbf{Mod}_{\mathbb{T}}}(F(X), M) \cong
  \mathbf{Hom}_{\mathbf{Set}}(X, |M|),\] za vsako množico $X$ in model $M$.
\begin{definicija}
  \emph{Prosti model} teorije $\mathbb{T}$ je model oblike $F(X)$,
  za prosti funktor $F$ iz \eqref{eq:prosti-funktor} in množico $X$.
  Če je množica $X$ končna, modelu $F(X)$ pravimo \emph{končno generiran}.
\end{definicija}
\begin{izrek}
  Naj bo $\mathbb{T}$ algebrajska teorija in naj bo $\mathbb{M}$
  polna podkategorija modelov na objektih $F(n)$ za $n \in \mathbb{N}$.
  Potem $\mathbb{M}^{\mathrm{op}}$ klasificira modele teorije $\mathbb{T}$.
  To pomeni, da imamo za vsako kategorijo s končnimi produkti $\mathbf{C}$ ekvivalenco kategorij
  \[ \mathbf{Hom}_{\mathrm{FP}}(\mathbb{M}^{\mathrm{op}}, \mathbf{C}) \simeq \mathbf{Mod}(\mathbb{T},\mathbf{C}), \]
  naravno v $\mathbf{C}$.
\end{izrek}
\begin{opomba}
Tu interpretiramo naravno število $n$ kot končno množico po standardni von Neumannovi konstrukciji.
\end{opomba}
\begin{opomba}
  Modelom teorije $\mathbb{T}$ v $\cat{Set}$ pravimo tudi $\mathbb{T}$-algebre.
\end{opomba}
\begin{dokaz}
   Najprej opazimo, da ima $\mathbb{M}$ vse končne
  koprodukte
  \begin{align*}
    F(n + m) &\cong F(n) \times F(m), \\
    1 &\cong F(0),
  \end{align*}
  saj ima $\mathbf{Set}$ vse koprodukte in levi adjunkti ohranjajo kolimite.
  To pa pomeni, da ima $\mathbb{M}^{\mathrm{op}}$ vse končne produkte. Naj bo
  \[ U = F(1) \]
  tako, da je vsak objekt v $\mathbb{M}^{\mathrm{op}}$ potenca $U$,
  \[ F(n) \cong U^n.\]
  Pokazali bomo, da je $U$ model $\mathbb{T}$.
  Za vsako osnovno $n$-mestno operacijo $f$ teorije $\mathbb{T}$ dobimo element modela
  $F(n)$, ki je sestavljen iz $f$ in generatorjev $x_1, \ldots, x_n$ (tu uporabimo
  iste oznake za generatorje in spremenljivke), označimo ga z
  \[ f(x_1, \ldots, x_n) \in F(n).\]
  Na primer, v prosti grupi na dveh generatorjih imamo element $x_1 \cdot x_2 \in F(2)$.
  Iz lastnosti prostega funktorja $F$, za vsak element $f(x_1, \ldots, x_n) \in F(n)$
  dobimo enoličen homomorfizem modelov
  \[ \overline{f(x_1, \ldots, x_n)} : F(1) \to F(n) \qquad \text{v $\mathbb{M}$}\]
  to je tisti homomorfizem, ki pošlje generator $F(1)$ v $f(x_1, \ldots, x_n) \in F(n)$.
  Če to dualiziramo, dobimo interpretacijo $n$-mestne operacije
  \[ [f] : U^n \to U \qquad \text{v $\mathbb{M}^{\mathrm{op}}$}.\]
  Z indukcijo to sedaj lahko razširimo na vse terme v kontekstu $x_1, \ldots, x_n \mid t$:
  \[ [x_{1}, \ldots, x_n \mid t] : U^n \to U.\]
  Sledi, da je vsaka izpeljiva enačba $s = t$ teorije $\mathbb{T}$ zadoščena v $U$
  saj, če velja $\mathbb{T} \vdash s = t$, potem morata ta dva terma sovpadati
  že v prosti $\mathbb{T}$-algebri $F(n)$.

  Sedaj opazimo, da obstaja kanoničen model $V$ teorije $\mathbb{T}$
  v funktorski kategoriji $\mathbf{Set}^{\mathbf{Mod}(\mathbb{T})}$, namreč
  pozabljivi funktor $V : \mathbf{Mod}(\mathbb{T}) \to \mathbf{Set}$.
  To je res model, saj za vsako $n$-mestno osnovno operacijo $f$ dobimo
  naravno transformacijo $[f]^V : V^n \to V$, kajti homomorfizmi v $\mathbf{Mod}(\mathbb{T})$
  komutirajo z operacijami, ki interpretirajo $f$.
  Enačbam teorije $\mathbb{T}$ zadošča, ker jim zadoščajo vsi modeli v $\Mod(\mathbb{T})$.
  Kot funktor je $V$ predstavljiv s prostim modelom $U = F(1)$, v smislu (kovariantne)
  Yonedove vložitve
  \[ \mathsf{y} : \mathbf{Mod}(\mathbb{T})^{\mathrm{op}} \to \mathbf{Set}^{\mathbf{Mod}(\mathbb{T})}.\]
  Za vsak $A \in \mathbf{Mod}(\mathbb{T})$ namreč velja:
  \begin{align*}
    V(A) &\cong \Hom_{\Mod(\mathbb{T})}(F(1), A) = \Hom_{\Mod(\mathbb{T})^{\mathrm{op}}}(A,U) \\
    V^n(A) &\cong \Hom_{\Mod(\mathbb{T})}(F(n), A) \\
    &\cong \Hom_{\Mod(\mathbb{T})}(F(1) \times \ldots \times F(1), A) = \Hom_{\Mod(\mathbb{T})^{\mathrm{op}}}(A,U^n).
  \end{align*}
  Torej $V \cong \mathsf{y}(U)$. Vsaka interpretacija $[f]^V : V^n \to V$ je oblike
  \[ \Hom_{\Mod(\T)}([f]^U, -) : \Hom_{\Mod(\T)}(F(n), -) \to \Hom_{\Mod(\T)}(F(1), -),\]
  za interpretacijo $[f]^U: U^n \to U$, kajti za vsako $\mathbb{T}$-algebro $A$ in element
  \[a = (a_1, \ldots ,a_n) \in \Hom_{\Mod(\T)}(F(n), A)\]
  velja
  \[ ([f]^V)_A(a) = f(a_1, \ldots, a_n) = a \circ \overline{f(x_1, \ldots, x_n)},\]
  kjer je $\overline{f(x_1, \ldots, x_n)} : F(1) \to F(n)$ enolični morfizem, ki nam
  $f(x_1, \ldots x_n)$ poda kot sliko generatorja $F(1)$.

  Na primer, za dva elementa $g_1, g_2$ grupe $G$ imamo preslikavo
  \[\overline{(g_1, g_2)} : F(2) \to G, \]
  ki pošlje generatorja v ta elementa
  $\overline{(g_1, g_2)}(x_1) = g_1$ in $\overline{(g_1,g_2)}(x_2) = g_2$.
  Produkt ${x_1 \cdot x_2 \in F(2)}$ enolično določa homomorfizem
  $\overline{x_1 \cdot x_2}: F(1) \to F(2)$. Kompozitum nam da očitni homomorfizem
  $\overline{g_1 \cdot g_2} : F(1) \to G$,
  \[ \overline{(g_1, g_2)} \circ \overline{x \cdot y} = \overline{g_1 \cdot g_2}, \]
  kar lahko prikažemo v diagramu:
  \begin{equation*}
    \begin{tikzcd}
      F(1) \ar[dr, "{\overline{g_1 \cdot g_2}}"'] \ar[r, "{\overline{x \cdot y}}"] &
      F(x,y) \ar[d, "\overline{(g_1,g_2)}"] \\
      & G      
    \end{tikzcd}
  \end{equation*}
  Na tak način je vsaka operacija $[f]^A : A^n \to A$, na $\T$-algebri $A$, inducirana s
  predkompozicijo z ">univerzalno"< operacijo $[f]^U : U^n \to U$, kar prikažemo kot
  \begin{equation}\label{diag:univerzalna-operacija}
    \begin{tikzcd}
      A^n \ar[d, "\cong"'] \ar[r, "{[f]^{A}}"] & A \ar[d, "\cong"] \\
      \Hom(F(n), A) \ar[r, "{([f]^U)^{*}}"'] & \Hom(F(1), A) \\
      F(n) & F(1) \ar[l, "{\overline{f(x_1, \ldots, x_n)}}"'] \\
      U^n \ar[r, "{[f]^U}"] & U
    \end{tikzcd}
  \end{equation}
  Skupaj z operacijami $[f]^U$ je $U$ torej res model teorije $\mathbb{T}$.
  Sestoji iz prostih $\T$-algeber $F(n)$ in
  preslikav $F(1) \to F(n)$, ki so določene z elementi $f(x_1, \ldots, x_n) \in F(n)$,
  kjer so $x_1, \ldots , x_n \in F(n)$ generatorji. Sedaj bomo v treh korakih pokazali,
  da je $U$ res univerzalna $\T$ algebra.

  Prvi korak: Naj bo $A$ poljubna $\T$-algebra v $\mathbf{Set}$. Potem obstaja funktor,
  ki ohranja končne produkte
  \[ A^{\#}: \mathbb{M}^{\mathrm{op}} \to \mathbf{Set}\]
  definiran kot $A^{\#}(U) \cong A$,
  \[ A^{\#}(-) = \Hom_{\Mod(\T)}(-,A).\]
  Dobimo ga tako, da zožimo predstavljivi funktor
  \[ \Hom_{\Mod(\T)}(-,A) : \Mod(\T)^{\mathrm{op}} \to \mathbf{Set}\]
  po polni vložitvi
  \[ \mathbb{M} = \Mod_{\mathrm{fg}}(\T) \to \Mod(\T)\]
  končno generiranih prostih $\T$-algeber. Eksplicitno, za vsak objekt
  $U^n \in \mathbb{M}^{\mathrm{op}}$:
  \[ A^{\#}(U^n) = \Hom_{\Mod(\T)}(F(n), A) \cong V(A)^{n}, \]
  kjer je $V(A)$ množica modela $A$. Ta funktor
  \[ A^{\#} : \mathbb{M}^{\mathrm{op}} \hookrightarrow \Mod(\T)^{\mathrm{op}} \to \mathbf{Set}\]
  jasno ohranja produkte, in velja
  \[ A^{\#}(U) = \Hom_{\Mod(\T)}(F(1), A) \cong V(A).\]
  Dodatno iz diagrama \eqref{diag:univerzalna-operacija} sledi, da za vsako osnovno operacijo $f$,
  do izomorfizma natančno velja
  \[ [f]^A = ([f]^U)^{*} = \Hom([f]^U,A) = A^{\#}([f]^U).\]
  Torej, kot $\mathbb{T}$-algebri $A^{\#}(U) \cong A$.
  Vsak homomorfizem $h : F(U) \to G(U)$ $\T$-algeber
  je induciran s funktorjema, ki ohranjata produkte
  ${F,G : \mathbb{M}^{\mathrm{op}} \to \mathbf{Set}}$
  in je oblike $h = \vartheta_U$, za enolično določeno naravno transformacijo $\vartheta : F \to G$.

  Drugi korak: Naj bo $\mathbb{C}$ poljubna (lokalno majhna) kategorija in $\mathcal{A}$ naj bo
  $\T$-algebra. Ker je
  \[ \Mod(\T, \mathbf{Set}^{\mathbb{C}}) \cong
    \Mod(\T, \mathbf{Set})^{\mathbb{C}} = \Mod(\T)^{\mathbb{C}},\]
  je vsak $\mathcal{A}(X)$ $\T$-algebra v $\mathbf{Set}$,
  ki ima po prvem koraku klasifikacijski funktor
  \[ \mathcal{A}(X)^{\#} : \mathbb{M}^{\mathrm{op}} \to \mathbf{Set}.\]
  Tej funktorji skupaj določajo en funktor
  $\mathcal{A}^{\#} : \mathbb{M}^{\mathrm{op}} \to \mathbf{Set}^{\mathbb{C}}$,
  ki je definiran na $U \in \mathbb{M}^{\mathrm{op}}$ kot
  \[ (\mathcal{A}^{\#}(U))(X) = \mathcal{A}(X)^{\#}(U) \cong \mathcal{A}(X)\]
  in na $U^n$ kot $(\mathcal{A}^{\#}(U^{n}))(X) = \mathcal{A}(X)^{\#}(U^n) \cong \mathcal{A}(X)^n$.
  Funktor ${\mathcal{A}^{\#}(U) : \mathbb{C} \to \mathbf{Set}}$ deluje na morfizmih $g : X \to Y$ v
  $\mathbb{C}$, kot je prikazano v naslednjem diagramu
  \begin{equation*}
    \begin{tikzcd}[column sep=large]
      \mathcal{A}^{\#}(U)(X) \ar[d, "\cong"'] \ar[r, "{\mathcal{A}^{\#}(U)(g)}"] &
      \mathcal{A}^{\#}(U)(Y) \ar[d, "\cong"] \\
      \mathcal{A}(X) \ar[r, "{\mathcal{A}(g)}"] & \mathcal{A}(Y)
    \end{tikzcd}
  \end{equation*}
  Primer za $U^n$ je povsem analogen. Delovanje
  $\mathcal{A} : \mathbb{M}^{\mathrm{op}} \to \mathbf{Set}$ na morfizmih $U^n \to U^m$
  v $\mathbb{M}^{\mathrm{op}}$ je definirano prav tako po komponentah
  \[ (\mathcal{A}^{\#}(U^n))(X) \cong \mathcal{A}(X)^{\#}(U^n) \to
    \mathcal{A}(X)^{\#}(U^m) \cong \mathcal{A}^{\#}(U^m)(X),\]
  za vsak $X \in \mathbb{C}$.

  Tretji korak: Za splošni primer naj bo $\mathbf{C}$ poljubna (lokalno majhna) kategorija
  s končnimi produkti in $A$ naj bo model $\mathbb{T}$ v $\mathbf{C}$.
  Sedaj po drugem koraku dobimo klasifikacijski funktor
  \[ \mathbb{M}^{\mathrm{op}} \to \mathbf{Set}^{\mathbf{C}^{\mathrm{op}}}\]
  in trdimo, da se $\mathcal{A}^{\#}$ faktorizira skozi Yonedovo vložitev
  \begin{equation*}
    \begin{tikzcd}
      & \mathbf{Set}^{\mathbf{C}^{\mathrm{op}}} \\
      \mathbb{M}^{\mathrm{op}} \ar[ur, "\mathcal{A}^{\#}"] \ar[r, dashed, "A^{\#}"'] &
      \mathbf{C} \ar[u, hook, "\mathsf{y}"']
    \end{tikzcd}
  \end{equation*}
  Vemo, da so vsi objekti v $\mathbb{M}^{\mathrm{op}}$ oblike $U^n$, torej so njihove slike
  \[ \mathcal{A}(U^n) \cong \mathcal{A}(U)^n \cong \mathsf{y}(A)^n \cong \mathsf{y}(A^n)\]
  vse predstavljive. Ker je $\mathsf{y}$ poln in zvest, zgornja trditev drži, in
  dobljeni funktor $A^{\#} : M^{\mathrm{op}} \to \mathbf{Set}$ ohranja končne produkte, saj jih
  že $\mathcal{A}^{\#}$ in $\mathsf{y}$ jih ustvari. Jasno je
  \[ A^{\#}(U) \cong A.\]
  Naravnost ekvivalence
  \[ \Hom_{\mathrm{FP}}(\mathbb{M}^{\mathrm{op}},\mathbf{C}) \simeq \Mod(\mathbb{T}, \mathbf{C})\]
  v $\mathbf{C}$ sledi iz dejstva, da je inducirana z evaluacijo
  funktorja, ki ohranja končne produkte $F : \mathbb{M}^{\mathrm{op}} \to \mathbf{C}$,
  pri univerzalnem modelu $U$ v $\mathbb{M}^{\mathrm{op}}$.
\end{dokaz}
Ker je klasifikacijska kategorija enolično določena z univerzalno lastnostjo,
do ekvivalence natančno, nam zgornji izrek direktno poda naslednji opis sintaktične
konstrukcije $\mathbf{C}_{\mathbb{T}}$.
\begin{posledica}\label{posledica:logična-dualnost}
Za vsako algebrajsko teorijo \(\mathbb{T}\) imamo ekvivalenco
\[ \mathbf{C}_{\mathbb{T}} \simeq \mathbf{Mod}_{\mathrm{fg}}(\mathbb{T})^{\mathrm{op}} \]
med sintaktično kategorijo \(\mathbf{C}_{\mathbb{T}}\) in dualom
\(\Mod_{\mathrm{fg}}(\mathbb{T})\), končno generiranih,
prostih modelov.
\end{posledica}
V kategoriji modelov lahko torej najdemo neodvisno sintaktično
predstavitev naše teorije.
Poglejmo si kaj nam to pove v posebnem primeru prazne teorije \(\mathbb{T}_0\),
brez operacij in brez konstant. Model te teorije je množica~\(X\),
brez dodatne strukture. Torej je \(\mathbb{T}_0\) teorija enakosti
med objekti.
Vse $\mathbb{T}\textsubscript{0}$-algebre so proste, in končno generirane med njimi
so ravno končne množice, torej je
\[ \Mod_{\mathrm{fg}}(\mathbb{T}_0) \simeq \mathbf{Set} _{\mathrm{fin}} \]
kategorija končnih množic. Dualnost nam potem pove, da
imamo ekvivalenco
\[ \mathbf{Hom}_{\mathrm{FP}}(\mathbf{Set}_{\mathrm{fin}}^{\mathrm{op}}, \mathbf{C}) \simeq
   \mathbf{Mod}(\mathbb{T}_0, \mathbf{C}) \simeq \mathbf{C}. \]
Kar pove preprosto, da je \(\mathbf{Set}_{\mathrm{fin}}^{\mathrm{op}}\)
prosta kategorija s končnimi produkti generirana z enim objektom.

Zgornja posledica potem pravi, da je teorija enakosti \(\mathbb{T}_0\) dualna
kategoriji končnih množic. Termi te teorije so enostavno urejeni seznami
spremenljivk~\((x_1, \ldots, x_n)\) in enačbe, ki veljajo so tiste,
ki veljajo točno za terme, na primer:
\[(x_2, x_5) = (x_2, x_5).\]
Posledica potem pravi, da je to ravno kategorija končnih množic, če
urejene sezname \((x_1, \ldots, x_n) : X \times \ldots \times X\) beremo
kot urejene \emph{kosezname} \[[x_1, \ldots, x_n] : 1 + \ldots + 1 \to 1.\]
%
%
\section{Lawverjeve teorije}
\label{sec:org3518651}
Če natančno pogledamo zgornjo obravnavo dualnosti za algebrajske teorije, nismo uporabili nič,
kar se direktno nanaša na njihovo primarno sintaktično konstrukcijo, npr.\ specifikacijo
kaj so osnovne operacije, kaj so aksiomi in kaj izpeljane enakosti.
To pomeni, da lahko to konstrukcijo posplošimo na ">abstraktne"< algebrajske teorije,
da lahko povemo kaj je algebrajska teorija na bolj sintaktično neodvisen način,
analogno kot pri pojmih vektorskega prostora, ki jih lahko
opišemo neodvisno od izbire baze.
\begin{definicija}
\emph{Lawverjeva algebrajska teorija} \(\mathbb{A}\) je majhna kategorija
v kateri objekti tvorijo zaporedje \(A^0, A^1, A^2, \ldots\) in obstajajo
vsi končni produkti, ki so oblike \(A^m \times A^n = A^{m + n}\), za vse \(m,n \in \mathbb{N}\).
Tukaj je mišljeno, da je \(A^0 = 1\) končni objekt v \(\mathbb{A}\).
Vsak objekt je torej končni produkt objekta \(A\) samim s sabo.

\emph{Model} Lawverjeve teorije \(\mathbb{A}\) v kategoriji \(\cat{C}\),
ki ima končne produkte, je funktor \(M : \mathbb{A} \to \cat{C}\),
ki ohranja končne produkte. \emph{Homomorfizem modelov} je naravna
transformacija \(\vartheta : M \to M'\).
\end{definicija}
%
\begin{opomba}
Kategorija \(\mathbb{A}\) ima lahko poljubne morfizme med objekti \(A^n\).
\end{opomba}
\begin{opomba}
Za objekte \(\mathbb{A}\) bi lahko vzeli kar naravna števila \(0, 1, 2, \ldots\),
a je notacija \(A^n\) bolj opisna.
\end{opomba}
%
Iz vsake Lawverjeve teorije lahko dobimo algebrajsko teorijo na naslednji način.
Za osnovne \(k\)-mestne operacije vzamemo vse morfizme \(A^k \to A\). 
Za tako zgrajeno teorijo dobimo kanonično interpretacijo v \(\mathbb{A}\) termov
zgrajenih iz spremenljivk in morfizmov \(A^k \to A\), kjer vsak morfizem interpretiramo
kar kot samega sebe, spremenljivke pa kot projekcije iz produktov (te vedno obstajajo,
ker ima \(\mathbb{A}\) končne produkte). Enačbo \(u = v\) vzamemo kot aksiom te teorije,
če sta kanonični interpretaciji termov \(u\) in \(v\) isti morfizem v \(\mathbb{A}\).

Modeli te teorije in homomorfizmi med njimi na očiten način ustrezajo naši novi
funktorialni semantiki.
\begin{primer}
Algebrajska teorija \(\mathcal{C}^{\infty}\) gladkih preslikav je kategorija
v kateri so objekti $n$-dimenzionalni evklidski prostori \(1, \mathbb{R}, \mathbb{R}^2, \ldots\),
in morfizmi \(\mathcal{C}^{\infty}\) preslikave med njimi.

Model te teorije v \(\cat{Set}\) je funktor \(A : \mathcal{C}^{\infty} \to \cat{Set}\),
ki ohranja končne produkte. Do naravnega izomorfizma natančno ga lahko opišemo tako, da
podamo množico \(A\) in za vsako gladko funkcijo \(f : \mathbb{R}^n \to \mathbb{R}\)
podamo funkcijo
\[Af : A^n \to A.\]
Za \(Af\) mora veljati, da za vse gladke preslikave
\(g_i : \mathbb{R}^m \to \mathbb{R}, i = 1, \ldots, n\) in vse \(a_1, \ldots, a_m \in A\),
velja
\begin{multline}
   Af \left( (Ag_1)\langle a_1, \ldots, a^m \rangle, \ldots, (Ag_n)\langle a_1, \ldots, a_m \rangle \right) = \\
   A(f \circ \langle g_1, \ldots g_n \rangle)\langle a_1, \ldots, a^m \rangle.
\end{multline}   
Ker sta seštevanje in množenje gladki preslikavi, to v posebnem pomeni,
da je \(A\) komutativen kolobar z enoto.
Takim strukturam pravimo tudi $\mathcal{C}\textsuperscript{\(\infty\)}$-kolobarji
ali gladke algebre. Torej so modeli teorije gladkih preslikav v $\cat{Set}$ ravno gladke algebre.
\end{primer}
%
\begin{primer}
V kategoriji \(\cat{C}\) s končnimi produkti vsak objekt \(A\) določi polno podkategorijo
končnih potenc \(1, A, A^2, A^3, \ldots\) in morfizmov med njimi. Tej kategoriji
pravimo teorija objekta \(A\).
\end{primer}
\subsection{Dualnost}
\label{sec:orgef98ac4}
Da lahko razširimo teorijo dualnosti na našo novo definicijo teorije, potrebujemo
koncept \emph{prostega modela}. Za to recimo, da je \(\mathbb{A}\) Lawverjeva teorija,
z objekti \(1, A, A^2, \ldots\). Označimo kategorijo modelov (v kategoriji $\cat{Set}$) z
\[ \mathbf{Mod}(\mathbb{A}) = \mathbf{Hom}_{FP}(\mathbb{A}, \mathbf{Set}). \]
Definirajmo pozabljivi funktor
\begin{align*}
U : \mathbf{Mod}(\mathbb{A}) &\to \mathbf{Set} \\
(M : \mathbb{A} \to \mathbf{Set}) &\mapsto M(A)
\end{align*}
To označujemo tudi kot \(|M| = U(M) = M(A)\).

Sedaj definirajmo (končni) prosti funktor \(F : \mathbf{Set}_{ \mathrm{fin}} \to \mathbf{Mod}(\mathbb{A})\), 
kot:
\begin{align*}
F(0) &= \mathbf{Hom}_{\mathbb{A}}(1,-) \\
F(1) &= \mathbf{Hom}_{\mathbb{A}}(A,-) \\
&\vdots \\
F(n) &= \mathbf{Hom}_{\mathbb{A}}(A^n, -)
\end{align*}
Da smo res dobili adjungirani par prosti--pozabljivi funktor, moramo dokazati, da
imamo izomorfizem
\[ \mathbf{Hom}_{\mathbf{Mod}_{\mathbb{A}}}(F(n), M) \cong \mathbf{Hom}_{\mathbf{Set}}(n, |M|), \]
ki mora biti naraven v obeh argumentih. Desna stran je 
\[\mathbf{Hom}_{\mathbf{Set}}(n, |M|) \cong |M|^n. \]
Za levo stran uporabimo Yonedovo lemo in dobimo
\begin{align*}
\mathbf{Hom}_{\mathbf{Hom}(\mathbb{A})}(F(n),M) &\cong  \mathbf{Hom}_{\mathbf{Hom}(\mathbb{A})}(\mathbf{Hom}_{\mathbb{A}}(A^n, -), M) \\
&\cong M(A^n) \qquad \text{(Yoneda)} \\
&\cong M(A)^n \qquad \text{(ohranja produkte)} \\
&\cong |M|^n.
\end{align*}

Ker so končno generirani, so prosti modeli torej ravno končno predstavljivi, potem
kot ">obratno"< semantiko
\(\mathbf{Mod}_{\mathrm{fg}}(\mathbb{A}) \hookrightarrow \mathbf{Mod}(\mathbb{A})\)
teorije \(\mathbb{A}\), v smislu posledice \ref{posledica:logična-dualnost},
dobimo polno podkategorijo kategorije
\(\mathbf{Hom}_{\mathrm{FP}}(\mathbb{A}, \mathbf{Set})\) kot sliko Yonedove vložitve
\begin{equation*}
\begin{tikzcd}
\mathbf{Mod}_{\mathrm{fg}}(\mathbb{A}) \ar[r, hook] & \mathbf{Mod}(\mathbb{A}) = \mathbf{Hom}_{\mathrm{FG}}(\mathbb{A}, \mathbf{Set}) \ar[r, hook] & \mathbf{Set}^{\mathbb{A}} \\
\mathbb{A}^{\mathrm{op}} \ar[u, "\cong"] \ar[urr, "\mathsf{y}"']
\end{tikzcd}
\end{equation*}
V abstraktnem primeru algebrajske teorije lahko torej logično dualnost
\[ \mathbb{A} \cong \mathbf{Mod}_{\mathrm{fg}}(\mathbb{A})^{\mathrm{op}} \]
opišemo s tem, da (kontravariantna) Yonedova vložitev
\[ \mathbb{A}^{\mathrm{op}} \hookrightarrow \mathbf{Set}^{\mathbb{A}} \]
predstavi \(\mathbb{A}\) kot dual polne podkategorije funktorjev, ki ohranjajo produkte.
%
\subsection{Karakterizacija}
\label{subsec:karakterizacija-lawverjevih-teorij}
%
%
Videli smo nekaj primerov struktur, ki jih lahko opišemo kot modele
Lawverjeve teorije. Večino standardnih primerov iz algebre je mogoče
tako opisati. To nam da način obravnave takih struktur,
ki je neodvisen od izbire njihove predstavitve.
Ta je analogen obravnavi linearnih preslikav brez izbire baze vektorskega
prostora. Biti pa moramo malce pazljivi, niso namreč
vsi primeri struktur, ki se jih obravnava na področju algebre, primeri
algebrajskih teorij v zgornjem smislu. Radi bi način kako ločiti te
strukture, ki so na prvi pogled algebrajske od tistih, ki so dejansko
modeli neke Lawverjeve teorije. Za primere struktur v kategoriji $\cat{Set}$,
torej množic z dodatno strukturo obstaja naslednji izrek, ki te
primere klasificira. Izreka tukaj ne bomo dokazovali, lahko ga najdemo v
izviznem članku \cite{birkhoff_1935} ali v bolj moderni obravnavi v \cite{cohn1981universal},
kjer sta relevantna Izrek $9.5$ in Izrek $11.9$.
\begin{izrek}[Birkhoffov HSP izrek]\label{izrek:Birkhoff-HSP}
  Naj bo $\mathcal{L}$ jezik na teoriji z enim samim tipom, generiran z množico
  osnovnih operacij in naj bo $\cat{C}$ razred interpretacij jezika $\mathcal{L}$,
  za katerega obstaja pozabljivi funktor $V : \cat{C} \to \cat{Set}$, ki nam za
  vsako strukturo poda njeno množico. Potem je $\cat{C}$ razred modelov
  algebrajske teorije, če in samo če:
  \begin{enumerate}
  \item Razred $\cat{C}$ je zaprt za homomorfne slike,
  \item Razred $\cat{C}$ je zaprt za podstrukture, 
  \item Razred $\cat{C}$ je zaprt za produkte.
  \end{enumerate}
\end{izrek}
%
\begin{primer}
  Z uporabo zgornjega izreka lahko na primer pokažemo, da teorija polj ni algebrajska.
  Čeprav smo videli, da standardni opis polj ni algebrajski, saj definicija inverza
  ni univerzalno kvantificirana, ker izključuje ničlo, bi mogoče lahko obstajala
  alternativna predstavitev, ki pa bi bila algebrajska. Kot pa vemo iz elementarne
  algebre, direktni produkt polj ni nikoli polje. Po Birkhoffovem izreku polja
  torej niso razred modelov nobene algebrajske teorije.
\end{primer}
\end{document}

%%% Local Variables:
%%% mode: latex
%%% TeX-master: t
%%% End:

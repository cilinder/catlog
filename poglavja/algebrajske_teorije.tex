\documentclass[../kategoricna_logika.tex]{subfiles}
\begin{document}
V matematiki pogosto ``v naravi'' naletimo na strukture, za katere se izkaže,
da jih lahko opišemo z naborom operacij in enačb.
Slaven tak primer je odkritje strukture grupe pri iskanju splošne formule
za računanje ničel polinomov.
Tak opis strukture z operacijami in enačbami imenujemo \emph{algebraična teorija}.
Na tak način lahko tretiramo mnogo znanih primerov iz osnovne algebre,
kot so grupe, kolobarji, modulu, itd., a tudi nekatere konstrukcije, ki niso
tako očitno algebraične, na primer adjungirane funktorje.
Vse algebraične teorije imajo mnogo skupnih precej splošnih lastnosti.
Nekaj izmed njih bomo predstavili v sledečem poglavju.
To bomo storili z uporabo tako imenovane funktorialne semantike, katere
razvoj nam bo služil tudi kot zgled v drugem delu naloge, kjer bomo to
idejo razširili.

Začeli bomo s splošnim pristopom opisa algebraičnih teorij.
Te so karakterizirane z aksiomatskim opisom, ki sestoji izključno iz
spremenljivk, konstant, operacij in enačb. Pomembno (za naš pristop) je,
da so operacije definirane za celotno strukturo, ne pa samo za nekatere
elemente strukture. To izključi dva pomembna primera: teorijo polj,
kjer inverz ničle ni definiran, in teorijo kategorij, kjer so kompozitumi
definirani samo za določene pare morfizmov.
Idejo bomo ilustrirali s temeljnim primerom iz algebre, teorijo grup.
\begin{primer}[Teorija grup]
  \label{primer:teorija-grup}
  Grupo lahko razumemo kot množico $G$, skupaj z binarno operacijo
  $\cdot : G \times G \to G$, ki zadošča aksiomoma:
  \begin{align}\label{aksiomi-grupe}
    &\forall x,y,z \in G . \quad (x\cdot y) \cdot z = x \cdot (y \cdot z) \\
    &\exists e \in G . \forall x \in G . \exists y \in G . \quad (e \cdot x = x \cdot e = x \wedge x \cdot y = y \cdot x = e)
  \end{align}
\end{primer}
Že na prvi pogled lahko rečemo, da je struktura drugega izmed zgornjih dveh aksiomov
manj zadovoljiva od prvega, kajti vsebuje tako vgnezdene kvantifikatorje,
kot določa obstoj nekih elementov, ki pa jih je mogoče iz teh aksiomov
določiti enolično. To pomeni, da lahko k strukturi grupe dodamo
konstanto $e \in G$ in eno-mestno operacijo $(-)^{-1} : G \to G$ in
dobimo ekvivalentno formulacijo grupe, ki jo lahko podamo samo z enačbami:
\begin{align*}
  &x \cdot (y \cdot z) = (x \cdot y) \cdot z \\
  &x \cdot e = e \cdot x = x \\
  &x \cdot x^{-1} = x^{-1} \cdot x = e
\end{align*}
V tem opisu univerzalnega kvantifikatorja $\forall$ več ne potrebujemo,
saj si interpretiramo vse spremenljivke, kot da tečejo po vseh elementih množice $G$.
Nepotrebno pa je tudi eksplicitno omenjati množico $G$.
Za zadnji korak lahko konstanto $e$ opišemo kot nič-mestno operacijo $e : 1 \to G$.
Tak način specifikacije bomo uporabili kot definicijo splošne algebrajske teorije.
%
\begin{definicija}
  \emph{Signatura} algebraične teorije $\Sigma$ je sestavljena iz družine
  množic $\lbrace \Sigma_k \rbrace$, kjer se elementi $\Sigma_k$
  imenujejo $k$-mestne osnovne operacije.
  \emph{Jezik} algebraične teorije $\mathcal{L}(\Sigma)$ je množica termov,
  ki jih tvorimo induktivno:
  \begin{enumerate}
  \item Spremenljivke $x,y,z, \ldots$ so termi,
  \item Če so $t_1, \ldots, t_k$ že termi in $f \in \Sigma_k$, potem
    je $f(t_1,\ldots, t_k)$ tudi term.
  \end{enumerate}
\end{definicija}
%
\begin{definicija}
  \emph{Algebraična teorija}
  $\mathbb{T} = (\mathcal{L}(\Sigma_\mathbb{T}), A_\mathbb{T})$ je podana z jezikom
  $\mathcal{L}(\Sigma_\mathbb{T})$ in množico $A_\mathbb{T}$, enačb med termi
  teorije, ki jih imenujemo \emph{aksiomi} teorije $\mathbb{T}$. Formalno so
  to pari termov jezika $\mathcal{L}(\Sigma_{\mathbb{T}})$.
\end{definicija}
\begin{primer}
  Prazna oziroma trivialna teorija $\mathbb{T}_0$ je teorija brez operacij in brez enačb.
  Ta teorija opisuje množico.
\end{primer}
\begin{primer}
Teorija z eno konstanto in brez enačb je teorija množice z odlikovanim elementom.
\end{primer}
\begin{primer}
  Teorija komutativnega kolobarja z enoto je algebrajska teorija.
  Imamo dve nič-mestni operaciji (konstanti) $0$ in $1$, eno unitarno operacijo
  $-$ in dve binarni operaciji $+$ in $\cdot$. Zanje veljajo enačbe:
  
\begin{align*}
  (x+y)+z &= x + (y + z) & (x \cdot y) \cdot z &= x \cdot (y \cdot z) \\
  x + 0 &= x & x \cdot 1 &= x \\
  0 + x &= x & 1\cdot x &= x \\
  x + (-x) &= 0 & (x+y)\cdot z &= x \cdot z + y\cdot z \\
  (-x) + x &= 0 & z \cdot (x+y) &= z \cdot x+ z\cdot y \\
  x + y &= y + x & x \cdot y &= y\cdot x
\end{align*}
\end{primer}
\begin{primer}
  Iz računalništva so induktivni podatkovni tipi primer algebrajske teorije.
  Na primer, podatkovni tip binarnih dreves z listi označenimi s celimi števili
  bi lahko definirali kot
  \begin{verbatim}
    type tree = Empty | Leaf of int | Node of tree * tree
  \end{verbatim}
  Ta definicija bi ustrezala algebrajski teoriji sestavljeni iz konstante $\verb|Empty|$,
  ki bi prestavljala prazno drevo, konstante $\verb|Leaf| \  n$ za vsak $n \in \mathbb{N}$
  in binarne operacije $\verb|Node|$. Ta teorija nima enačb. 
\end{primer}
\begin{primer}
  Pomembna struktura, ki ni primer algebrajske teorije, je teorija delno urejene množice,
  ki jo formuliramo z binarno operacijo $\leq$, ki ustreza aksiomom refleksivnosti,
  anti-simetričnosti in tranzitivnosti.
\end{primer}
%
%
\section{Model algebrajske teorije}
\markright{Modeli}
Poglejmo si sedaj na primeru grup, kaj je \emph{model} algebrajske teorije.
Kot smo povedali zgoraj, v klasični algebri grupo razumemo kot množico $G$,
konstanto $e \in G$ in dve operaciji, ki jih bomo zaradi ilustrativnosti sedaj
označili z $i : G \to G$ in $m : G \times G \to G$. Ustrezati mora aksiomom:
\begin{align*}
   m(x,m(y,z)) &= m(m(x,y),z) \\
   m(x,e) &= m(e,x) = x \\
   m(x,i(x)) &= m(i(x),x) = e.
\end{align*}
%
To idejo lahko posplošimo tako, da jo ``prevedemo'' v
kategorični jezik, kar nam bo omogočilo, da govorimo o modelih v drugih
kategorijah kot $\mathbf{Set}$. To storimo na naslednji način:
Grupa je objekt $G \in \mathbf{Set}$ skupaj s tremi morfizmi:
\[ e : 1 \to G, \qquad m : G \times G \to G, \qquad i : G \to G. \]
Aksiome grupe lahko predstavimo s komutativnimi diagrami
\begin{center}
  \begin{tikzcd}[column sep = large]
    G \times G \times G \ar[d, "\pi_0 \times m"'] \ar[r, "m \times \pi_2"] & G \times G \ar[d, "m"] \\
    G \times G \ar[r, "m"'] & G
  \end{tikzcd}
\end{center}
\begin{center}
  \begin{tikzcd}[column sep = large]
    G \times 1 \ar[dr, "\pi_0"'] \ar[r, "1_G \times e"] & G \times G \ar[d, "m"] & 1 \times G \ar[l, "e \times 1_G"'] \ar[dl, "\pi_1"] \\
    & G
  \end{tikzcd}
\end{center}
\begin{center}
  \begin{tikzcd}[column sep = large]
    G \ar[d, "!"'] \ar[r, "{\langle 1_G, i \rangle}"] & G \times G \ar[d, "m"] & G \ar[l, "{\langle i, 1_G \rangle}"'] \ar[d, "!"] \\
    1 \ar[r, "e"'] & G & 1 \ar[l, "e"]
  \end{tikzcd}
\end{center}
%
V tej formulaciji je bila edina lastnost kategorije $\mathbf{Set}$, ki smo jo uporabili to,
da ima končne produkte. Torej je tak opis smiselen v vsaki kategoriji $\mathbf{C}$,
ki ima končne produkte. Torej lahko definiramo:
\emph{Grupa} v kategoriji s končnimi produkti $\mathbf{C}$ je objekt
$G$, skupaj z morfizmi
\begin{center}
  \begin{tikzcd}
    G \times G \ar[r, "m"] & G & G \ar[l, "i"'] \\
    & 1 \ar[u, "e"] &
  \end{tikzcd}
\end{center}
za katere zgornji diagrami komutirajo.
%
Podobno posplošimo homomorfizem med grupami v $\mathcal{C}$. Morfizem
$h : G \to H$ je homomorfizem grup, če komutira z interpretacijami
vseh osnovnih operacij $m$, $i$ in $e$.
To lahko izrazimo, z diagrami
%
\begin{center}
  \begin{tikzcd}
    G^2 \ar[d, "m^G"'] \ar[r, "h^2"] & H^2 \ar[d, "m^H"] & & G \ar[d, "i^G"'] \ar[r, "h"] & H \ar[d, "i^H"] & & 1 \ar[d, "e^G"'] \ar[r] & 1 \ar[d, "e^H"] \\
    G \ar[r, "h"'] & H & & G \ar[r, "h"'] & H & & G \ar[r, "h"'] & H
  \end{tikzcd}
\end{center}
Ali z enačbami
\begin{align*}
m^H \circ h^2 &= h \circ m^G \\
i^H \circ h &= h \circ i^G \\
e^H &= h \circ e^G.
\end{align*}
Skupaj s kompozitumi in identitetnimi homomorfizmi, tako dobimo
kategorijo grup v $\mathbf{C}$, ki jo označimo z $\mathbf{Group}(\mathbf{C})$.
To konstrukcijo sedaj posplošimo na poljubno algebrajsko teorijo.
\begin{definicija}
  Naj bo $\mathbf{C}$ kategorija s končnimi
  produkti. \emph{Interpretacija} $I$ teorije $\mathbb{T}$ je
  sestavljena iz
  \begin{itemize}
  \item Objekta $I \in \mathcal{C}$,
  \item Morfizma $f^I : I^k \to I$, za vsako $k$-mestno operacijo $f$.
  \end{itemize}
%
  Interpretacijo razširimo na vse terme jezika s \emph{kontekstom}.
  Splošen term $t$ se interpretira skupaj s
  kontekstom spremenljivk $x_1, \ldots, x_n$, kjer v $t$ nastopajo le
  spremenljivke iz tega konteksta. To označimo z
$$x_1, \ldots, x_n \mid t.$$
%
Interpretacija terma $t^I$ je definirana reukrzivno:
  \begin{enumerate}
  \item Interpretacija spremenljivke $x_i$ je $i$-ta projekcija
    $\pi_i : I^n \to I.$
%
  \item Term oblike $f(t_1, \ldots, t_k)$ se interpretira kot
    kompozitum
    \begin{equation*}
      \begin{tikzcd}[column sep = 5em]
        I^n \ar[r, "{\langle t_1^I, \ldots, t_k^I\rangle}"] & I^k \ar[r, "f^I"] & I,
      \end{tikzcd}
    \end{equation*}
    kjer je $t_i^I : I^n \to I$ interpretacija terma $t_i$ za
    $i = 1, \ldots, k$ in je $f^I$ interpretacija osnovne operacije
    $f$.
  \end{enumerate}
\end{definicija}
%
\begin{opomba}
Interpretacija je odvisna od konteksta! Ko bo potrebno, bomo to označevali s
$t^I = [x_1, \ldots, x_n \mid t]^I$. Na primer, 
  term $f(x_1)$ se v kontekstu~$x_1$ interpretira kot morfizem
  $f^I : I \to I$, medtem ko se v kontekstu $x_1, x_2$ interpretira
  kot $f^I \circ \pi_1 : I^2 \to I$.
\end{opomba}
%
\begin{definicija}
  Naj bosta $s$ in $t$ terma v kontekstu $x_1, \ldots, x_n$.
 V interpretaciji $I$ je enačba $s = t$ \emph{zadoščena}, če
    sta morfizma $s^I$ in $t^I$ isti morfizem v $\mathcal{C}$.
%
V posebnem, če je $s = t$ aksiom teorije in so $x_1, \ldots, x_n$
  vse spremenljivke, ki nastopanjo v $s$ in $t$, potem pravimo, da je
  v interpretaciji $I$ \emph{zadoščen aksiom} $s = t$, če sta
  $[x_1, \ldots, x_n \mid s]^I$ in $[x_1, \ldots, x_n \mid t]^I$ isti
  morfizem
  \begin{equation*}
    \begin{tikzcd}[column sep = 7em]
      I^n \ar[r, shift left=1ex, "{[x_1, \ldots, x_n \mid s]^I}"]
      \ar[r, shift right=1ex, "{[x_1, \ldots, x_n \mid t]^I}"'] & I
    \end{tikzcd}
  \end{equation*}
%
  To označimo kot
$$I \models s = t \iff s^I = t^I.$$
\end{definicija}
%
\begin{definicija}
Naj bo $\mathbb{T}$ algebraična teorija.
\emph{Model} teorije $\mathbb{T}$ v kategoriji $\mathcal{C}$ s
končnimi produkti je interpretacija $M$, ki zadošča vsem aksiomom
teorije $\mathbb{T}$,
$$M \models s = t,$$
za vsak $(s = t) \in A_\mathbb{T}$.
\end{definicija}
%
\begin{definicija}
 \emph{Homomorfizem modelov} $h : M \to N$ je morfizem v
 $\mathcal{C}$, ki komutira z interpretacijami osnovnih operacij,
$$h \circ f^M = f^N \circ h,$$
za vsak $f \in \Sigma_T$, kar ponazorimo v diagramu
\begin{equation*}
  \begin{tikzcd}
    M^k \ar[d, "f^M"'] \ar[r, "h^k"] & N^k \ar[d, "f^N"] \\
    M \ar[r, "h"] & N
  \end{tikzcd}
\end{equation*}
\end{definicija}
%
Kategorijo modelov teorije $\mathbb{T}$ v $\mathbf{C}$ označimo z
$\mathbf{Mod}(\mathbb{T}, \mathcal{C})$.
%
\begin{primer}
  Model prazne teorije $\mathbb{T}_0$ je objekt $M \in \mathcal{C}$,
  morfizem med dvema modeloma pa je le morfizem v $\mathcal{C}$ brez
  dodatnih omejitev, torej
$$\mathbf{Mod}(\mathbb{T}_0, \mathcal{C}) = \mathcal{C}.$$
%
Model teorije grup $\mathbb{T}_{\mathrm{Grp}}$, v kategoriji množic
$\mathbf{Set}$ je grupa v običajnem smislu. Homorfizem med dvemi modeli
teorije $\mathrm{Grp}$ je homomorfizem grup. Torej je
$$\mathbf{Mod}(\mathbb{T}_{\mathrm{Grp}}, \mathbf{Set}) = \mathbf{Grp}.$$
\end{primer}
% 
\begin{primer}
  Model teorije grup v funktorski kategoriji $\mathbf{Set}^{\mathbf{C}}$ je
  natanko funktor iz $\mathbf{C}$ v kategorijo grup,
  \[ \mathbf{Grp}(\mathbf{Set}^{\mathbf{C}}) \cong \mathbf{Hom}(\mathbf{C}, \mathbf{Grp}). \]
  To velja, kajti za vsak objekt $C \in \mathbf{C}$ obstaja evaluacijski
  funktor
  \[ \mathrm{eval}_{\mathbf{C}} : \mathbf{Set}^{\mathbf{C}} \to \mathbf{Set}, \]
  ki je definiran kot $\mathrm{eval}_{\mathbf{C}}(F) = F(C)$, za $F : \mathbf{C} \to \mathbf{Set}$.
  Ker produkte računamo ``po točkah'', evaluacija ohranja produkte.
  Za vsak morfizem~${h : C \to D}$ v $\mathbf{C}$ dobimo očitno naravno
  transformacijo $h : \mathrm{eval}_{\mathbf{C}} \to \mathrm{eval}_{\mathbf{D}}$.
  Torej za vsako grupo $G$ v $\mathbf{Set}^{\mathbf{C}}$ dobimo grupo
  $\mathrm{eval}_C(G)$, za vsak $C \in \mathbf{C}$ in homorfizem $h_{G} : C(G) \to D(G)$
  grup za vsak morfizem $h : C \to D$, kar definira funktor $G : \mathbf{C} \to \mathbf{Grp}$.
  Obratno, vsak funktor $H : \mathbf{C} \to \mathbf{Grp}$, ki ohranja produkte,
  dobimo kot grupo v $\mathbf{Set}^{\mathbf{C}}$, kjer so komponente
  naravnih transformacij grupnih operacij ravno operacije tiste grupe.
  Aksiomom grupe zadoščajo, ker jih zadoščajo pri vsaki komponenti posebej.
\end{primer}
%
\section{Teorije kot kategorije}
\label{sec:teorije-kot-kategorije}
Predstavitev algebrajske teorije kot smo jo podali zgoraj za primer
teorije grup je v svojem bistvu sintaktične narave. To pa ima,
kot definicija matematičnega koncepta svoje pomanjkljivosti.
Želeli bi najti pogled, ki bi nam omogočal govoriti o tem,
kaj grupa \emph{je}, brez tega, da bi se morali odločati,
na kak način bomo grupo \emph{zapisali}.
Ta pogled nam do dala kategorija z določeno \emph{univerzalno lastnostjo},
ki jo bo natanko določala, do ekvivalence natančno.
To nas bo privedlo do reformulacije klasičnih konceptov
sintakse in semantike na način, ki bolj ustreza sodobni matematični praksi.

Vrnimo se na primer grupe. Formulacija, ki smo jo podali zgoraj,
ni edina možna. Obstaja alternativna predstavitev teorije grup,
z enoto $e$ in binarno operacijo $\odot$,
ki se imenuje dvojno deljenje in enim samim aksiomom:
$$x \odot (((x \odot y ) \odot z ) \odot ( z \odot e))) \odot (e \odot e) ) = z.$$
Običajne operacije teorije grup so povezane prek formul
$$x \odot y = x^{-1} \cdot y^{-1} \text{,} \quad x^{-1} = x \odot e \text{,} \quad x \cdot y = (x \odot e) \odot (y \odot e)$$
%
Obstajajo različni razlogi za uporabo ene ali druge predstavitve
teorije grup, a ta izbira naj ne odraža tega kaj grupa v svojem bistvu je.
Želimo se izogniti specifični izbiri konstant, operacij in aksiomov.
To je podobna situacija, kot v linearni algebri, kjer želimo podati
teorijo brez specificiranja baze vektorskega prostora o katerem govorimo,
kljub temu, da lahko tako bazo vedno določimo. Za računanje v vektorskem
prostoru je izbira baze brez dvoma ključnega pomena, a ni to tisti
aspekt, ki določa naravo vektorskega prostora.

Kot prvi korak, bi lahko vzeli vse operacije, ki jih lahko zgradimo iz
enote, množenja in inverza kot osnovne operacije in vse
veljavne enačbe v teoriji grup kot aksiome. Lahko pa gremo še korak dlje
in zberemo vse operacije v kategorijo in tako pozabimo katere so bile
osnovne in katere izpeljane ter katere enakosti so bile podane kot aksiomi.
%
\begin{definicija}
  Naj bo $\mathbb{T}$ algebrajska teorija.
  \emph{Sintaktična kategorija} $\mathcal{C}_\mathbb{T}$ je kategorija,
  v kateri so
  \begin{itemize}
  \item Objekti: konteksti $[x_1, \ldots, x_n]$ za $n \geq 0$.
%
  \item Morfizmi: morfizem $[x_1, \ldots, x_m] \to [x_1, \ldots, x_n]$
    je $n$-terica $\langle t_1, \ldots, t_n \rangle$ termov v kontekstu
    $x_1, \ldots, x_m$. Morfizma $\langle t_1, \ldots t_n \rangle$,
    $\langle s_1, \ldots, s_n \rangle$ sta enaka, če in samo če, za vsak
    $k = 1, \ldots, n$ v teoriji $\mathbb{T}$ velja
    $$\mathbb{T} \vdash t_k = s_k.$$
\end{itemize}
%
Morfizmi so torej v resnici ekvivalenčni razredi termov v kontekstu
$$[x_1, \ldots, x_m \mid t_1, \ldots, t_n] : [x_1, \ldots, x_m] \to [x_1, \ldots, x_n]$$
%
Kompozitum morfizmov
  \begin{align*}
    \langle t_1, \ldots, t_m \rangle : [x_1, \ldots, x_k] &\to [x_1, \ldots, x_m], \\
    \langle s_1, \ldots, s_n \rangle : [x_1, \ldots, x_m] &\to [x_1, \ldots, x_n],
  \end{align*}
je morfizem $\langle r_1, \ldots, r_n \rangle$, kjer dobimo $i$-to kompotneto tako,
da v $s_i$ simultano vstavimo terme $t_1, \ldots, t_m$ namesto
spremenljivk $x_1, \ldots, x_m$.
$$r_i = s_i[t_1, \ldots, t_m / x_1, \ldots, x_m].$$
%
\end{definicija}
Sintaktično kategorijo $\mathbf{C}_{\mathbb{T}}$ si lahko predstavljamo kot
``Lindenbaum--Tarski'' kategorijo teorije $\mathbb{T}$. Vsebuje vse algebraične
informacije kot originalna teorija $\mathbb{T}$ le na sintaktično invarianten način.
Različne predstavitve teorije porodijo ekvivalentne sintaktične
kategorije. V nadaljevanju bomo videli, da ima $\mathbf{C}_{\mathbb{T}}$ še eno,
še bolj pomembno lastnost, s katero predstavlja teorijo $\mathbb{T}$.
%
\begin{lema}
  Naj bo $\mathbb{T}$ algebraična teorija in $\mathcal{C}_\mathbb{T}$
  njena sintaktična kategorija. Potem ima $\mathcal{C}_\mathbb{T}$ vse
  končne produkte in velja
$$[x_1, \ldots, x_n] \times [x_1, \ldots, x_m] \cong [x_1, \ldots, x_{n+m}].$$
\end{lema}
\begin{dokaz}
sorry.
\end{dokaz}
%
%
\section{Modeli kot funtorji}
%
Videli bomo, da ima $\mathbf{C}_{\mathbb{T}}$ to posebno lastnost, da modeli
teorije $\mathbb{T}$ natanko ustrezajo funktorjem iz $\mathbf{C}_{\mathbb{T}}$.
\begin{izrek}
\label{sec:modeli-kot-funtorji}
Naj bo $\mathbb{T}$ algebrajska teorija in $\mathbf{C}_{\mathbb{T}}$ njena
sintaktična kategorija. Potem za vsako kategorijo $\mathbf{C}$, ki ima končne
produkte, obstaja naravna ekvivalenca med modeli teorije $\mathbb{T}$ v
$\mathbf{C}$ in funktorji, ki ohranjajo končne produkte, iz
$\mathbf{C}_{\mathbb{T}}$ v $\mathbf{C}$.
$$M \in \mathbf{Mod}(\mathbb{T}, \mathbf{C})\quad \leftrightsquigarrow \quad \mathcal{M} : \mathbf{C}_\mathbb{T} \to \mathbf{C}$$
Vsak model je podan kot funktorialna slika \emph{univerzalnega modela}
$\mathcal{U}$ v $\mathbf{C}_T$, tako da velja
$M \cong \mathcal{M}(\mathcal{U})$, v bistvu enoličnega funktorja
$M : \mathbf{C}_{\mathbb{T}} \to \mathbf{C}$, ki ohranja produkte.
\end{izrek}
\begin{opomba}
  Univerzalni model $\mathcal{U}$ po tej ekvivalenci ustreza identitetnemu
  funktorju $1_{\mathbf{C}_{\mathbb{T}}} : \mathbf{C}_{\mathbb{T}} \to \mathbf{C}_{\mathbb{T}}$.
  Dejstvo, da obstajajo taki univerzalni modeli je prednost posplošitve
  koncepta modela na druge kategorije kot $\mathbf{Set}$.
\end{opomba}
%
\begin{definicija}
\emph{Sintaktični model} $U$ definiramo kot:
\begin{itemize}
\item Za objekt vzamemo $U := [x_1]$, kontekst dolžine ena.
%
\item Interpretacijo $k$-mestne osnovne operacije $f$ definiramo kot
  samo sebe
$$f^U := [x_1, \ldots, x_n \mid f(x_1, \ldots, x_n)] : U^k = [x_1, \ldots, x_n] \to U = [x_1]$$
\end{itemize}
%
Aksiomi $\mathbb{T}$ so izpolnjeni, saj za vsaka terma $t$, $s$ velja
$$U \models t = s \Longleftrightarrow t^U = s^U \Longleftrightarrow \mathbb{T} \vdash t = s.$$
Velja torej
$U \in \mathbf{Mod}(\mathbb{T}, \mathbf{C}_\mathbb{T})$.
\end{definicija}
%
Sintaktični model $U$ je univerzalen na naslednji način:
vsak model $M$ v kategoriji $\mathbf{C}$ s končnimi produkti je
slika $U$-ja pod enolično določenim funktorjem $\mathbf{C}_{\mathbb{T}} \to \mathbf{C}$.
Opišimo to konstrukcijo bolj natančno.

Denimo, da imamo funktor $F : \mathbf{C}_{\mathbb{T}} \to \mathbf{C}$,
ki ohranja končne produkte.
Potem je slika $F(U)$, skupaj z interpretacijami
$f^{F(U)} := F(f^U)$ model v $\mathbf{C}$, saj za
vsak aksiom $s = t$ velja
$$\mathbb{T} \vdash s = t  \iff  s^U = t^U,$$
%
iz česar zaradi funktorialnosti $F$ sledi
$$t^{FU} = F(t^U) = F(s^U) = s^{FU}.$$
%
Vsaka naravna transformacija $\vartheta : F \to G$ med takima
funktorjema določi homorfizem modelov
$$h = \vartheta_\mathcal{U} : F\mathcal{U} \to G\mathcal{U}$$
%
saj za vsako osnovno operacijo $f$ zaradi naravnosti velja
$$h \circ f^{F\mathcal{U}} = f^{G\mathcal{U}} \circ h,$$
kar lahko razberemo iz diagrama
\begin{center}
  \begin{tikzcd}[column sep = large, row sep = large]
    F\mathcal{U}  \times F\mathcal{U} \ar[d] \ar[r, "h \times h"] \ar[dd, bend right=60, "f^{F\mathcal{U}}"'] & G\mathcal{U} \times G\mathcal{U} \ar[d] \ar[dd, bend left = 60, "f^{G\mathcal{U}}"] \\
    F(\mathcal{U} \times \mathcal{U}) \ar[d, "F(f)"] \ar[r, "\vartheta_{\mathcal{U} \times \mathcal{U}}"] & G(\mathcal{U} \times \mathcal{U}) \ar[d, "G(f)"] \\
    F\mathcal{U} \ar[r, "h"'] & G\mathcal{U}
  \end{tikzcd}
\end{center}
kjer zgornji kvadrat komutira ker $F$ ohranja produkte, spodnji
pa zaradi naravnosti $\vartheta$. Torej evaluacija pri $U$ določa funktor
\[ \mathrm{eval}_U : \mathbf{Hom}_{\mathrm{FP}}(\mathbf{C}_{\mathbb{T}}, \mathbf{C}) \to \mathbf{Mod}(\mathbb{T}, \mathbf{C}), \]
iz kategorije funktorjev $\mathbf{C}_{\mathbb{T}} \to \mathbf{C}$,
ki ohranjajo končne produkte in naravnih transformacij med njimi,
v kategorijo modelov teorije $\mathbb{T}$ v $\mathbf{C}$.
\begin{trditev}\label{trditev:modeli-so-funktorji}
  Funktor $\mathrm{eval}_U$ opisan zgoraj, določa ekvivalenco kategorij,
  ki je naravna v $\mathbf{C}$.
\end{trditev}
\begin{dokaz}
Za to je dovolj pokazati, da je $eval_U$:
surjektiven na objektih, zvest in poln.
Za surjektivnost denimo, da imamo model
$M \in \mathbf{Mod}(\mathbb{T}, \mathbf{C})$. Potem lahko definiramo funktor
$$M^{\#} : \mathbf{C}_\mathbb{T} \to \mathbf{C}$$
ki na objektih deluje kot $M^{\#}([x_1, \ldots, x_n]) = M^n$, na
morfizmih pa
$$M^{\#}(t_1, \ldots, t_n) = \langle t_1^M, \ldots, t_n^M \rangle$$
%
Bolj natančno, je $M^{\#}$ na morfizmih definiran induktivno:
\begin{enumerate}
\item Morfizem $$x_i : [x_1, \ldots, x_n] \to [x_1]$$ se slika v
  $i$-to projekcijo $$\pi_i : M^n \to M.$$
\item Morfizem $$f(t_1, \ldots, t_n) : [x_1, \ldots, x_n] \to [x_1]$$
  se slika v kompozitum
\begin{equation*}
    \begin{tikzcd}[column sep = 8em]
      M^m \ar[r, "{\langle M^{\#}t_1, \ldots, M^{\#}t_n \rangle}"] & M^n \ar[r,
      "{M^{\#}f}"] & M,
    \end{tikzcd}
  \end{equation*}
  kjer je $M^{\#}f = f^M$ interpretacija osnovne operacije $f$ podana
  z modelom.
\end{enumerate}
%
Dobra definiranost $M^{\#}$ sledi iz tega, da je $M$ model, torej res zadošča vsem enačbam teorije.
Definirali smo ga na tak način, da velja
$$M \cong M^{\#}(U).$$
%
Za polnost in zvestost moramo pokazati surjektivnost ter injektivnost
predpisa $$\vartheta \mapsto \vartheta_U,$$
kjer je $\vartheta$ naravna transformacija med funktorjema $F$ in $G$, ki ohranjata končne limite.
Za surjektivnost definiramo:
$$\vartheta_{[x_1, \ldots, x_n]} := h^n.$$
Za injektivnost denimo da velja $\vartheta_U = \phi_U $.
Potem zaradi ohranjanja končnih produktov dobimo
\[ (\vartheta_U)^n = \vartheta_{U^n} = \phi_{U^n} = (\phi_U)^n. \]

Pokazati moramo še, da je ta ekvivalenca naravna v $\mathbf{C}$.
To pomeni naslednje, denimo da je $M$ model $\mathbb{T}$ v neki kategoriji
$\mathbf{C}$ s končnimi produkti. Vsak funktor $F : \mathbf{C} \to \mathbf{D}$,
ki ohranja končne produkte, v drugo kategorijo s končnimi produkti,
pošlje $M$ v model $F(M)$ v $\mathbf{D}$. Interpretacijo osnovne operacije $f$
v $\mathbf{D}$ dobimo s predpisom $f^{F(M)} = F(f^M)$, ki jo razširimo na
vse terme preko kanoničnih izomorfizmov $F(M) \times F(M) \cong F(M \times M)$.
Ker enačbe podamo z komutativnimi diagrami, $F$ pošlje model v model in
homomorfizem modelov v homomorfizem modelov. To pomeni, da $F$ inducira
funktor
\[ \mathbf{Mod}(\mathbb{T}, F) : \mathbf{Mod}(\mathbb{T}, \mathbf{C}) \to \mathbf{Mod}(\mathbb{T}, \mathbf{D}). \]
Da je $\mathrm{eval}_U$ res naraven, mora diagram
\begin{diagram}
  \mathbf{Hom}_{\mathrm{FP}}(\mathbf{C}_{\mathbb{T}}, \mathbf{C}) \ar[r, "{\mathrm{eval}_U}"]
  \ar[d, "{\mathbf{Hom}_{\mathrm{FP}}(\cat{C}^{\mathbb{T}}, F)}"']& \mathbf{Mod}(\mathbb{T}, \mathbf{C})
  \ar[d, "{\mathbf{Mod}(\mathbb{T}, F)}"] \\
  \mathbf{Hom}_{\mathrm{FP}}(\mathbf{C}_{\mathbb{T}}, \mathbf{D}) \ar[r, "{\mathrm{eval}_{U}}"']   & \mathbf{Mod}(\mathbb{T}, \mathbf{C})
\end{diagram}
komutirati. To lahko vidimo, saj za vsak funktor $M : \mathbf{C}_{\mathbb{T}} \to \mathbf{C}$,
ki ohranja produkte velja
\begin{align*}
  \mathrm{eval}_U \circ \mathbf{Hom}_{\mathrm{FP}}(\mathbf{C}_{\mathbb{T}}, F)(M) &= (\mathbf{Hom}_{\mathrm{FP}}(\mathbf{C}_{\mathbb{T}}, F)(M))(U) \\
                                                                                  &= (F \circ M)(U) \\
                                                                                  &= F(M(U)) \\
                                                                                  &= F(\mathrm{eval}_U(M)) \\
                                                                                  &\cong \mathbf{Mod}(\mathbb{T}, F)(\mathrm{eval}_U(M)) \\
  &= \mathbf{Mod}(\mathbb{T}, F) \circ \mathrm{eval}_U(M).
\end{align*}
\end{dokaz}
Ekvivalenca kategorij
\[ \mathbf{Hom}_{\mathrm{FP}}(\mathbf{C}_{\mathbb{T}}, \mathbf{C}) \simeq \mathbf{Mod}(\mathbb{T}, \mathbf{C}) \]
dejansko določi kategorijo $\mathbf{C}_{\mathbb{T}}$ in univerzalni model $U$ enolično,
do ekvivalence kategorij in izomorfizma modela natančno.
Da dobimo $U$ lahko v zgornjo ekvivalenco za $\mathbf{C}$ vstavimo $\mathbf{C}_{\mathbb{T}}$
in indentitetni funktor $1_{\mathbf{C}_{\mathbb{T}}}$ na levi in na desni dobimo
model $U$. To lahko formuliramo kot univerzalno lastnost:
\begin{definicija}
  \emph{Klasifikacijska kategorija} algebraične teorije $\mathbb{T}$
  je kategorija s končnimi produkti $\mathbf{C}_\mathbb{T}$, s
  posebnim modelom $\mathcal{U}$, imenovanim \emph{univerzalni model},
  tako da velja:
  \begin{enumerate}
  \item Za vsak model $M$, v kategoriji s končnimi produkti
    $\mathbf{C}$, obstaja funktor, ki ohranja končne limite
    $$M^{\#} : \mathbf{C}_\mathbb{T} \to \mathbf{C}$$ in izomorfizem
    modelov $M \cong M^{\#}(\mathcal{U})$.
%
  \item Za vsaka funktorja, ki ohranjata končne limite
    $F,G : \mathbf{C}_\mathbb{T} \to \mathbf{C}$ in homomorfizem
    modelov $h: F(\mathcal{U}) \to G(\mathcal{U})$, obstaja natanko
    ena naravna transformacija $\vartheta : F \to G$, da velja
$$\vartheta_\mathcal{U} = h.$$
\end{enumerate}
\end{definicija}
Kot je to običajno za nekaj podano z univerzalno lastnostjo,
je $\mathbf{C}_\mathbb{T}$ enolično določena, do ekvivalence kategorij natančno.
%
Naj bosta $(\mathbf{C}, U)$ in $(\mathcal{D}, V)$ obe klasifikacijski
kategoriji neke teorije $\mathbb{T}$. Potem po univerzalni lastnosti
dobimo funktorja
\begin{center}
  \begin{tikzcd}[column sep = large]
    \mathbf{C} \ar[r, bend left, "U^{\#}"] & \mathcal{D} \ar[l, bend
    left, "V^{\#}"]
  \end{tikzcd}
\end{center}
tako, da velja
$$U \cong U^{\#}(V) \cong U^{\#}(V^{\#}(U)).$$
Pokazali smo naslednji izrek
\begin{izrek}
\label{sec:klasifikacijska-kategorija-alg-teorije}
Za vsako algebrajsko teorijo $\mathbb{T}$ je sintaktična kategorija $\mathbf{C}_{\mathbb{T}}$
njena klasifikacijska kategorija.
\end{izrek}
\begin{primer}
  Poglejmo si, kaj nam ta konstrukcija pove v primeru teorije grup
  $\mathbb{G} = \mathbf{C}_{\mathbb{T}_{\mathrm{Grp}}}$. Spomnimo se,
  da je $\mathbb{G}$ sestanljena iz kontekstov $[x_1, \ldots x_n]$ in
  termov zgajenih iz spremenljivk in osnovnih grupnih operacij.
  Funktor $M : \mathbb{G} \to \mathbf{Set}$, ki ohranja produkte
  je določen do izomorfizma natančno s tem, kam pošlje kontekst $[x_1]$
  in terme, ki predstavljajo osnovne operacije. Če definiramo
\begin{align*}
  G &= M([x_1]), & e &= M(\cdot \mid e), \\
  i &= M(x_1 \mid x_1^{-1}), & m &= M(x_1,x_2 \mid x_1 \cdot x_2),
\end{align*}
potem je $(G,e,i,m)$ navadna grupa z enoto $e$, inverzom $i$ in
množenjem $m$. Množica $G$ z operacijami res zadošča aksiomom
grup, zaradi funktorialnosti $M$. Obratno, če je $(G,e,i,m)$ grupa,
potem dobimo funktor $M_{G} : \mathbb{G} \to \mathbf{Set}$,
ki ohranja končne produkte tako, da definiramo
\begin{align*}
  M_{G}([x_1, \ldots, x_n]) &= G^n & M_{G}(\cdot \mid e) &= e \\
  M_{G}(x_1 \mid x_1^{-1}) &= i & M_{G}(x_1, x_2 \mid x_1 \cdot x_2) &= m.
\end{align*}

Recimo, da sta $(G,e_{G}, i_{G}, m_{G}$ in $(H, e_H, i_H, m_H)$ grupi
in naj bo $\phi : M_{G} \to M_H$ naravna transformacija med pripadajočima
funktorjema. Potem je $\phi$ določena s tem, kam slika kontekst $[x_1]$,
kajti po naravnosti za $1 \leq k \leq n$, naslednji diagram
\begin{equation*}
\begin{tikzcd}\label{primer:grupe-kot-funktorji}
  G^n \ar[d, "G\pi_k = \pi_k"'] \ar[r, "\phi_{[x_1, \ldots, x_n]}"] & H^n \ar[d, "H\pi_k = \pi_k"] \\
  G \ar[r, "\phi_{[x_1]}"'] & H
\end{tikzcd}
\end{equation*}
komutira. Če označimo $\phi' = \phi_{[x_1]}$, potem velja
$\phi_{[x_1, \ldots, x_n]} = \phi' \times \ldots \times \phi'$.
Spet lahko vidimo, da zaradi naravnosti naslednji diagram
\begin{equation*}
\begin{tikzcd}\label{sec:modeli-kot-funtorji-3}
  G \times G \ar[d, "m_{G}"'] \ar[r, "\phi' \times \phi'"] & H \times H \ar[d, "m_H"] \\
  G \ar[r, "\phi'"'] & H
\end{tikzcd}
\end{equation*}
komutira. Podobni diagrami pokažejo, da $\phi'$ ohranja enoto in
inverze, torej je $\phi' : G \to H$ res homorfizem grup.
Obratno, homomorfizem grup $\varphi' : G \to H$ inducira naravno
transformacijo $\varphi : G \to H$, katere komponenta pri
$[x_1, \ldots, x_n]$ je enaka $\varphi' \times \ldots \times \varphi' : G^n \to H^n$.
Demonstrirali smo ekvivalenco kategorij
\[ \mathbf{Mod}_{\mathbf{Set}}(\mathbb{G}) \simeq \mathbf{Grp}. \]
\end{primer}
\section{Polnost algebrajske teorije}
V matematični logiki pravimo, da je logičen sistem \emph{validen},
če vsaka formula, ki jo lahko dokažemo v tem sistemu, resnična
glede na semantiko tega sistema. Logične formule algebrajske teorije
$\mathbb{T}$ imajo zelo preprosto strukturo, vse so namreč enačbe
oblike $s = t$ med termi te teorije. To, da je ta teorija veljavna
pomeni, da vsak model zadošča vsem enačbam, ki jih lahko izpeljemo
iz aksiomov. Obratna lastnost logičnega sistema se imenuje \emph{polnost}.
Pravimo, da je teorija $\mathbb{T}$ polna (semantično polna), če velja
da
$$\text{Vsak model } \mathbb{T} \text{ zadošča }s=t \quad \Longrightarrow \quad \mathbb{T} \vdash s = t,$$
kjer $\vdash$ predstavlja izpeljavo enačbe iz aksiomov teorije.
Za teorije prvega reda velja naslednji slaven izrek o polnosti
\begin{izrek}[Gödel]
  Naj bo $T$ teorija prvega reda in $\varphi$ formula v jeziku, ki
  opisuje $T$. Potem so naslednje trditve ekvivalentne:
  \begin{enumerate}
  \item Obstaja dokaz $\varphi$ iz aksiomov teorije $T$
    $(T \vdash \varphi)$
  \item Stavek $\varphi$ drži v vsakem modelu teorije $T$
    $(T \models \varphi)$
  \end{enumerate}
\end{izrek}
To velja v posebno močnem smislu za kategorično semantiko
\begin{izrek}[Polnost algebraičnih teorij]
  Naj bo $\mathbb{T}$ algebraična teorija. Potem obstaja kategorija s
  končnimi produkti $\mathbf{C}$ in model
  $\mathcal{U} \in \mathbf{Mod}(\mathbb{T}, \mathbf{C})$ tako, da za vsako
  enačbo $s = t$ teorije $\mathbb{T}$ velja
$$\mathcal{U} \models s = t \quad \Longleftrightarrow \quad \mathbb{T} \vdash s = t$$
\end{izrek}
\begin{dokaz}
  Kot kategorijo $\mathbf{C}$ vzamemo klasifikacijsko kategorijo
  $\mathbf{C}_\mathbb{T}$ in univerzalni model $\mathcal{U}$. Če
  velja $\mathbb{T} \vdash s = t$, potem iz konstrukcije
  $\mathbf{C}_\mathbb{T}$ sledi $s^\mathcal{U} =
  t^\mathcal{U}$. Vsak model $M$ v kategoriji $\mathbf{C}$,
  s končnimi produkti ima klasifikacijski funktor $M^{\#} : \mathbf{C}_{\mathbb{T}} \to \mathbf{C}$,
  ki ohranja interpretacijo termov $s$ in $t$, kajti velja
  \[ M^{\#}(s^\mathcal{U}) = s^{M^{\#}(\mathcal{U})} = s^M. \]
  Podobno za $t$.
  To pomeni, da iz $s^{\mathcal{U}} = t^{\mathcal{U}}$ sledi $s^M = t^M$.
  Obratno, če ${\mathcal{U} \models s = t}$, potem velja $s^{\mathcal{U}} = t^{\mathcal{U}}$.
  Iz konstrukcije $\mathbf{C}_{\mathbb{T}}$ potem sledi ${\mathbb{T} \vdash s = t}$.
\end{dokaz}
\begin{definicija}
  Modelu za katerega velja lastnost iz zgornjega izreka, da zadošča
  natanko tistim enačbam, ki jih lahko izpeljemo iz aksiomov teorije,
  pravimo \emph{logično generičen}.
\end{definicija}
Po prejšnjem izreku je torej univerzalni model logično generičen.
V klasičnem smislu models v kategoriji $\mathbf{Set}$, le redko obstaja
logično generičen model. Za polnost moramo gledati veljavnost
v vseh modelih teorije hkrati. Za algebrajske teorije v tem posplošenem
smislu model velja torej močnejša lastnost.
Za modele v kategoriji $\mathbf{Set}^{\mathbf{C}^{\mathrm{op}}}$
imamo posebej lep generičen model.
\begin{trditev}
  Naj bo $\mathbb{T}$ algebrajska teorija. Potem je Yonedova vložitev
  \[ y : \mathbf{C}_{\mathbb{T}} \to \mathbf{Fun}(\mathbf{C}_{\mathbb{T}}^{\mathrm{op}}, \mathbf{Set}) \]
  generičen model $\mathbb{T}$.
\end{trditev}
\begin{dokaz}
  Yonedova vložitev ohranja limite, torej v posebnem končne produkte.
  Sledi, da nam da model $\tilde{U} = y(U)$ teorije $\mathbb{T}$.
  Ker je $y$ funktor, so v $\tilde{U}$ zadoščene vse enačbe,
  ki veljajo v $U$, ker pa je $y$ zvest funktor, je vsaka enačba,
  ki velja v $\tilde{U}$, morala veljati že v $U$.
  Ker je $U$ logično generičen, mora biti tudi $\tilde{U}$.
\end{dokaz}
\begin{primer}
  Poglejmo si, kaj nam to pove v primeru teorije grup. Univerzalen model
  grupe je grupa, ki zadošča vsem enačbam, ki veljajo za vse grupe
  in nobenim drugim. Spomnimo se, da je univerzalni model $U$
  predstavljen s kontekstom $[x_1]$ v $\mathbb{G}$.
  Yonedova vložitev nam da
  \[ \tilde{U} = y[x_1] = \mathbb{G}(-, [x_1]). \]
  To je množica, ki jo parametrizirajo objekti $\mathbb{G}$.
  Za vsak $n \in \mathbb{}N$ dobimo množico $\mathbb{G}([x_1, \ldots, x_x], [x_1])$,
  vseh termov, ki jih lahko konstruiramo iz $n$ spremenljivk,
  modulo enačbe teorije grup.
  To pa je ravno prosta grupa z $n$ generatorji. Enota, inverz in množenje
  v $\tilde{U}$ so definirani za vsak $n$ kot te operacije na
  prosti grupi z $n$ generatorji.

  Univerzalna grupa je torej prosta grupa na $n$ generatorjih,
  kjer je $n \in \mathbb{N}$ parameter.
\end{primer}
%
\section{Funktorialna semantika}
\label{sec:funkt-semant}
%
%
%
%
\section{Lawverjeva dualnost}
%
%
%
%
\section{Lawverjeva Teorija}
\end{document}

%%% Local Variables:
%%% mode: latex
%%% TeX-master: t
%%% End:

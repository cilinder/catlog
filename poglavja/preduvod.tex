\documentclass[../kategoricna_logika.tex]{subfiles}
\begin{document}
\selectlanguage{slovene}
\setcounter{page}{1} %
\renewcommand{\thepage}{}       % preprecimo težave s številkami strani v kazalu
\newcommand{\sn}[1]{"`#1"'}                    % dodal Solina (slovenski narekovaji)
%
%
%
%%%%%%%%%%%%%%%%%%%%%%%%%%%%%%%%%%%%%%%%
%naslovnica
 \thispagestyle{empty}%
   \begin{center}
    {\large\sc Univerza v Ljubljani\\%
      Fakulteta za matematiko in fiziko}%
    \vskip 10em%
    {\autfont \tauthor\par}%
    {\titfont \ttitle \par}%
    {\vskip 3em \textsc{MAGISTRSKO DELO\\[5mm]
    UNIVERZITETNI\\ ŠTUDIJSKI PROGRAM DRUGE STOPNJE\\ MATEMATIKA}\par}%
%
    \vfill\null%
    {\large \textsc{Mentor}: prof.\ dr.  Andrej Bauer\par}%
    {\vskip 2em \large Ljubljana, 2020 \par}%
\end{center}
% prazna stran
%\clearemptydoublepage      % dodal Solina (izjava o licencah itd. se izpiše na hrbtni strani naslovnice)
%
%%%%%%%%%%%%%%%%%%%%%%%%%%%%%%%%%%%%%%%%
%copyright stran
\thispagestyle{empty}
\vspace*{8cm}
%
\noindent
{\sc Copyright}. 
Rezultati diplomske naloge so intelektualna lastnina avtorja in Fakultete za računalništvo in informatiko Univerze v Ljubljani.
Za objavo in koriščenje rezultatov diplomske naloge je potrebno pisno privoljenje avtorja, Fakultete za računalništvo in informatiko ter mentorja.
%
\begin{center}
\mbox{}\vfill
\emph{Besedilo je oblikovano z urejevalnikom besedil \LaTeX.}
\end{center}
% prazna stran
\clearemptydoublepage
%
%
%%%%%%%%%%%%%%%%%%%%%%%%%%%%%%%%%%%%%%%%
% stran 3 med uvodnimi listi
\thispagestyle{empty}
\vspace*{4cm}
%
\noindent
Fakulteta za računalništvo in informatiko izdaja naslednjo nalogo:
\medskip
\begin{tabbing}
\hspace{32mm}\= \hspace{6cm} \= \kill
%
%
Tematika naloge:
\end{tabbing}
Delo obravnava Yonedovo lemo, ki je eden od osrednjih izrekov teorije kategorij. Predstavljene bodo tudi nekatere aplikacije Yonedove leme.
\vspace{15mm}
%
\vspace{2cm}
%
% prazna stran
\clearemptydoublepage
%
% zahvala
\thispagestyle{empty}\mbox{}\vfill\null\it%
\noindent
Zahvaljujem se mentorju prof. dr. Andreju Bauerju za usmerjanje in vso ostalo pomoč. \\
Svojim staršem, za vso podporo. \\
\rm\normalfont
%
% prazna stran
\clearemptydoublepage
%
%%%%%%%%%%%%%%%%%%%%%%%%%%%%%%%%%%%%%%%%
% kazalo
\pagestyle{empty}
\def\thepage{}% preprecimo tezave s stevilkami strani v kazalu
\tableofcontents{}
%
% prazna stran
\clearemptydoublepage
%
%
%%%%%%%%%%%%%%%%%%%%%%%%%%%%%%%%%%%%%%%%
% povzetek
\addcontentsline{toc}{chapter}{Povzetek}
\chapter*{Povzetek}
%
\noindent\textbf{Naslov:} \ttitle
\bigskip
%
\noindent\textbf{Avtor:} \tauthor
\bigskip
%
%\noindent\textbf{Povzetek:} 
\noindent
V diplomskem delu je obravnavana Yonedova lema, ki velja za enega osrednjih izrekov teorije kategorij. Uvodni del definira osnovne pojme v teoriji kategorij, ki so kasneje uporabljeni za formulacijo in dokaz leme. Skozi besedilo je predstavljen tudi kategorični način razmišljanja, ki nam omogoča specifično situacijo, ki jo srečamo v matematiki, obravnavati bistveno bolj splošno, z uporabo kategoričnih metod.
V zaključnem poglavju je predstavljena in dokazana Yonedova lema, ki nam poda način za obravnavo kategorije z obravnavo njene vložitve v kategorijo funktorjev. Predstavljenih je tudi nekaj primerov uporabe leme.
\bigskip
%
\noindent\textbf{Ključne besede:} \tkeywords.
% prazna stran
\clearemptydoublepage
%
%%%%%%%%%%%%%%%%%%%%%%%%%%%%%%%%%%%%%%%%
% abstract
\selectlanguage{english}
\addcontentsline{toc}{chapter}{Abstract}
\chapter*{Abstract}
%
\noindent\textbf{Title:} \ttitleEn
\bigskip
%
\noindent\textbf{Author:} \tauthor
\bigskip
%\noindent\textbf{Abstract:} 
%
\noindent 
The thesis discusses the Yoneda lemma, which is considered one of the central theorems in category theory. The introduction defines the basic concepts of category theory, which are later used to formulate and prove the lemma. Throughout the text there are examples of the categorical way of thinking, where we take a specific situation, that we encounter in mathematics and look at it in a more general setting. In the final chapter we describe and prove the Yoneda lemma, that presents a way of studying a category, by studying its inclusion into a category of functors. We show some use cases of the lemma.
%
\bigskip
%
\noindent\textbf{Keywords:} \tkeywordsEn.
\selectlanguage{slovene}
% prazna stran
\clearemptydoublepage
\end{document}
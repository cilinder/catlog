\documentclass[../kategoricna_logika.tex]{subfiles}
\begin{document}
\selectlanguage{slovene}
\setcounter{page}{1} %
\renewcommand{\thepage}{}       % preprecimo težave s številkami strani v kazalu
\newcommand{\sn}[1]{"`#1"'}                    % dodal Solina (slovenski narekovaji)
%
%
%
%%%%%%%%%%%%%%%%%%%%%%%%%%%%%%%%%%%%%%%%
%naslovnica
 \thispagestyle{empty}%
   \begin{center}
    {\large\sc Univerza v Ljubljani\\%
      Fakulteta za matematiko in fiziko}%
    \vskip 10em%
    {\autfont \tauthor\par}%
    {\titfont \ttitle \par}%
    {\vskip 3em \textsc{MAGISTRSKO DELO\\[5mm]
    UNIVERZITETNI\\ ŠTUDIJSKI PROGRAM DRUGE STOPNJE\\ MATEMATIKA}\par}%
%
    \vfill\null%
    {\large \textsc{Mentor}: prof.\ dr.\ Andrej Bauer\par}%
    {\vskip 2em \large Ljubljana, 2020 \par}%
\end{center}
% prazna stran
%\clearemptydoublepage      % dodal Solina (izjava o licencah itd. se izpiše na hrbtni strani naslovnice)
%
%%%%%%%%%%%%%%%%%%%%%%%%%%%%%%%%%%%%%%%%
%copyright stran
\thispagestyle{empty}
\vspace*{8cm}
%
\noindent
{\sc Copyright}. 
Rezultati magistrske naloge so intelektualna lastnina avtorja in Fakultete za matematiko in fiziko Univerze v Ljubljani.
Za objavo in koriščenje rezultatov magistrske naloge je potrebno pisno privoljenje avtorja, Fakultete za matematiko in fiziko ter mentorja.
%
\begin{center}
\mbox{}\vfill
\emph{Besedilo je oblikovano z urejevalnikom besedil \LaTeX.}
\end{center}
% prazna stran
\clearemptydoublepage
%
%
%%%%%%%%%%%%%%%%%%%%%%%%%%%%%%%%%%%%%%%%
% stran 3 med uvodnimi listi
\thispagestyle{empty}
\vspace*{4cm}
%
\noindent
Fakulteta za matematiko in fiziko izdaja naslednjo nalogo:

Algebrajska in regularna kategorna logika

\medskip
\begin{tabbing}
\hspace{32mm}\= \hspace{6cm} \= \kill
%
%
Tematika naloge:
\end{tabbing}
V delu obravnavajte algebrajsko in regularno logiko in njune modele v teoriji kategorij,
vključno z Lawverovo funktorialno semantiko.
Predstavite in dokažite osnovne izreke o veljavnoti in polnosti.


Viri:

- (Butz)

- (Zapiski iz kategorne logike: Awodey \& Bauer)
\vspace{15mm}
%
\vspace{2cm}
%
% prazna stran
\clearemptydoublepage
%
% zahvala
\thispagestyle{empty}\mbox{}\vfill\null\it%
\vfill
{\Large \bf Zahvala}
\vspace{1cm}\\
Zahvaljujem se mentorju prof.\ dr.\ Andreju Bauerju za usmerjanje in vso ostalo pomoč. \\
Svojim staršem, za vso podporo. \\
\rm\normalfont
%
% prazna stran
\clearemptydoublepage
%


%%%%%%%%%%%%%%%%%%%%%%%%%%%%%%%%%%%%%%%%
% kazalo
\pagestyle{empty}
\def\thepage{}% preprecimo tezave s stevilkami strani v kazalu
\tableofcontents{}
%
% prazna stran
\clearemptydoublepage
%
%
%%%%%%%%%%%%%%%%%%%%%%%%%%%%%%%%%%%%%%%%
\end{document}
\documentclass[../kategoricna_logika.tex]{subfiles}
\begin{document}
\selectlanguage{slovene}
\setcounter{page}{1} %
\renewcommand{\thepage}{}       % preprecimo težave s številkami strani v kazalu
\newcommand{\sn}[1]{"`#1"'}                    % dodal Solina (slovenski narekovaji)
\newcommand{\program}{Matematika} % ime studijskega programa
\newcommand{\imeavtorja}{Jure Taslak} % ime avtorja
\newcommand{\imementorja}{prof.~dr.~Andrej Bauer} % akademski naziv in ime mentorja, uporabi poln naziv, prof.~dr.~, doc.~dr., ali izr.~prof.~dr.
\newcommand{\imesomentorja}{} % akademski naziv in ime somentorja, če ga imate
\newcommand{\naslovdela}{Algebrajska in regularna kategorna logika}
\newcommand{\letnica}{2021} % letnica magistriranja
\newcommand{\opis}{Delo obravnava integracijo po ω-kompleksih, njene lastnosti in posplošitve
na Levy-jeve topološke prostore.}  % Opis dela v eni povedi. Ne sme vsebovati matematičnih simbolov v $ $.
\newcommand{\kljucnebesede}{integracija\sep kompleks} % ključne besede, ločene z \sep, da se PDF metapodatki prav procesirajo
\newcommand{\keywords}{integration\sep complex} % ključne besede v angleščini
\newcommand{\organization}{Univerza v Ljubljani, Fakulteta za matematiko in fiziko} % fakulteta
\newcommand{\sep}{, }  % separator med ključnimi besedami v besedilu
%
%
%
%%%%%%%%%%%%%%%%%%%%%%%%%%%%%%%%%%%%%%%%
%naslovnica
\pagenumbering{roman} % začnemo z rimskimi številkami
\thispagestyle{empty} % ampak na prvi strani ni številke

\noindent{\large
UNIVERZA V LJUBLJANI\\[1mm]
FAKULTETA ZA MATEMATIKO IN FIZIKO\\[5mm]
\program\ -- 2.~stopnja}
% ustrezno dopolni za IŠRM
\vfill

\begin{center}
  \large
  \imeavtorja\\[3mm]
  \Large
  \textbf{\MakeUppercase{\naslovdela}}\\[10mm]
  \large
  Magistrsko delo \\[1cm]
  Mentor: \imementorja \\[2mm] % ustrezno popravi spol
%   Somentor: \imesomentorja   % dodaj, če potrebno
\end{center}
\vfill

\noindent{\large Ljubljana, \letnica}

\cleardoublepage

% prazna stran
%\clearemptydoublepage      % dodal Solina (izjava o licencah itd. se izpiše na hrbtni strani naslovnice)
%
%%%%%%%%%%%%%%%%%%%%%%%%%%%%%%%%%%%%%%%%
% zahvala
\thispagestyle{empty}\mbox{}\vfill\null\it%
\vfill
{\Large \bf Zahvala}
\vspace{1cm}\\
Zahvaljujem se mentorju prof.\ dr.\ Andreju Bauerju za usmerjanje in vso ostalo pomoč. \\
Svojim staršem za vso podporo. \\
Janji za pomoč pri odpravljanju slovničnih napak.
\rm\normalfont
%
% prazna stran
\clearemptydoublepage
%%%%%%%%%%%%%%%%%%%%%%%%%%%%%%%%%%%%%%%%
% kazalo
\pagestyle{empty}
\def\thepage{}% preprecimo tezave s stevilkami strani v kazalu
\tableofcontents{}
%
% prazna stran
\clearemptydoublepage
%
%
%%%%%%%%%%%%%%%%%%%%%%%%%%%%%%%%%%%%%%%%
\section*{Program dela}
\addcontentsline{toc}{section}{Program dela} % dodajmo v kazalo
V delu obravnavajte algebrajsko in regularno logiko in njune modele v teoriji kategorij,
vključno z Lawverovo funktorialno semantiko.
Predstavite in dokažite osnovne izreke o veljavnoti in polnosti.

\section*{Osnovna literatura}

\begin{itemize}
  \plancite{butz1998regular}
  \plancite{algebraic-logic}
\end{itemize}

\vspace{2cm}
\hspace*{\fill} Podpis mentorja: \phantom{prostor za podpis}

% \vspace{2cm}
% \hspace*{\fill} Podpis somentorja: \phantom{prostor za podpis}
%%%%%%%%%%%%%%%%%%%%%%%%%%%%%%%%%%%%%%%%%%%%%%%%%%%%%%%%%%%%%%%%%%%%%%
% Povzetek
\cleardoublepage
\pdfbookmark[1]{Povzetek}{abstract}

\begin{center}
\textbf{\naslovdela} \\[3mm]
\textsc{Povzetek} \\[2mm]
\end{center}
V nalogi je razvita funktorialna semantika za algebrajsko in regularno kategorno logiko.
V prvem delu je najprej na kratko predstavljena teorija kategorij, nato se uvede
pojem algebrajske teorije, ki je poseben primer logične teorije prvega reda, v kateri
nastopajo samo enačbe in operacije. Razširi se klasična interpretacija modela teorije
na vse kategorije, v katerih je mogoče tako teorijo izraziti.
Za vsako algebrajsko teorijo lahko definiramo posebno sintaktično kategorijo,
ki to teorijo predstavlja. Izkaže se, da lahko vsak model algebrajske teorije
enolično identificiramo s funktorjem, ki ohranja strukturo sintaktične kategorije.
To je izraženo v obliki ekvivalence kategorij. S pomočjo te ekvivalence je raziskana
dualnost med sintakso in semantiko algebrajske teorije.
Drugi del se začne z opisom razreda kategorij imenovanih regularne in motivacijo
za njihovo vpeljavo v obliki primerov in lepih lastnosti s katerimi se ponašajo.
Nato se razvije razširitev enostavne algebrajske logike iz prvega dela na tako imenovano
regularno logiko, v kateri poleg enačb in operacij nastopajo še relacijski simboli,
resničnostna konstanta, konjunkcija in kvantifikator obstoja. To naredi logiko
bolj bogato in v njej je mogoče izraziti koncepte kot je slika morfizma.
Analogno kot v prvem delu se za regularno teorijo definira njeno sintaktično kategorijo,
s pomočjo katere se pokaže ekvivalenco med modeli regularne logike in funktorji, ki
ohranjajo regularno strukturo.
\vfill
\begin{center}
\textbf{\ttitleEn} \\[3mm] % prevod slovenskega naslova dela
\textsc{Abstract}\\[2mm]
\end{center}
The thesis develops functorial semantics for algebraic and regular logic.
The first part starts by briefly presenting category theory, then the concept of
an algebraic theory is introduced as a special case of a first order logic theory,
in which you only have equations and operations. The classical notion of a model
is expanded to categories in which such a theory can be expressed.
For each algebraic theory we may define a special syntactic category, which represents
it. It turns out that you can uniquely identify each model of such a theory with a
functor that preserves the structure of the syntactic category. This is expressed
in the form of an equivalence of categories. With the help of this equivalence a
duality between syntax and semantics is explored.
The second part begins with the description of a class of categories called regular
categories and the motivation for their definition in terms of examples and nice
properties that these categories posses. An extension of the simple algebraic logic is
then developed into the so called regular logic which besides equations and
operations includes relation symbols, the truth constant, conjunction and the
existential quantifier. This makes the logic more rich and makes it possible
to express concepts like the image of a morphism. Analogous with the first part
we define the syntactic category of a regular theory, with the help of which
you can show an equivalence between models of a regular theory and functors that
preserve regular structure.
%%%%%%%%%%%%%%%%%%%%%%%%%%%%%%%%%%%%%%%%%%%%%%%%%%%%%%%%%%%%%%%%%%%%%%%%
\vfill\noindent
\textbf{Math.~Subj.~Class.~(2010):} oznake kot 74B05, 65N99, na voljo so na naslovu
\url{http://www.ams.org/msc/msc2010.html} \\[1mm]
\textbf{Ključne besede:} \tkeywords \\[1mm]
\textbf{Keywords:} \tkeywordsEn

\cleardoublepage

\thispagestyle{empty}
\vspace*{8cm}
%
\noindent
% {\sc Copyright}. 
% Rezultati magistrske naloge so intelektualna lastnina avtorja in Fakultete za matematiko in fiziko Univerze v Ljubljani.
% Za objavo in koriščenje rezultatov magistrske naloge je potrebno pisno privoljenje avtorja, Fakultete za matematiko in fiziko ter mentorja.
%
\begin{center}
\mbox{}\vfill
\emph{Besedilo je oblikovano z urejevalnikom besedil \LaTeX.}
\end{center}
% prazna stran
\clearemptydoublepage
%
%
%%%%%%%%%%%%%%%%%%%%%%%%%%%%%%%%%%%%%%%%
% stran 3 med uvodnimi listi
\thispagestyle{empty}
\vspace*{4cm}
%
\noindent
Fakulteta za matematiko in fiziko izdaja naslednjo nalogo:

Algebrajska in regularna kategorna logika

\medskip
\begin{tabbing}
\hspace{32mm}\= \hspace{6cm} \= \kill
%
%
Tematika naloge:
\end{tabbing}
V delu obravnavajte algebrajsko in regularno logiko in njune modele v teoriji kategorij,
vključno z Lawverovo funktorialno semantiko.
Predstavite in dokažite osnovne izreke o veljavnoti in polnosti.


Viri:

- (Butz)

- (Zapiski iz kategorne logike: Awodey \& Bauer)
\vspace{15mm}
%
\vspace{2cm}
%
% prazna stran
\clearemptydoublepage
%

%



\end{document}
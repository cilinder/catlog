Do sedaj smo videli povezavo med algebrajsko teorijo in kategorijo z neko dodatno struktuo, ki nam je omogočila interpretacijo take teorije v semantiki same kategorije. Teorija, ki smo jo obravnavali je bila precej enostavna s stališča logike, saj so bile edine logične formule, ki smo jih lahko konstruirali oblike $t_1 = t_2$ za neka terma, sestavljena induktivno iz spremenljivk in funkcijskih simbolov. 
Že iz teh osnovnih gradnikov je mogoče dobiti pomembne matematične "infrastrukture", kot je na primer teorija grup. Za opis nekih matematičnih teorij pa jasno ta stopnja kompleksnosti ne zadostuje.
Videli smo, da se teorije polj ne da opisati z algebrajsko teorijo. Moderna matematika je standardno opisana v jeziku predukatne logike, oz logike drugega (ali višjega) reda, kjer imamo poleg funkcijskih simbolov še logične veznike kot so "in", "ali", negacijo in univerzalni ter eksistenčni kvantifikator.
Naravno vprašanje je torej, ali lahko zgodbo algebrajskih teorij ponovimo z neko močnejšo logiko in če se to da, kako se to odraža v strukturi kategorijo, ki jo dobimo na ta način in ali ohranimo lepe lastnosti, ki smo jih videli v primeru algebrajskih teorij.
Kot vemo iz Gödelovega izreka o polnosti logike prvega reda, za modele v kategoriji množic to velja. Zanima nas, ali je to definicijo modela mogoče razširiti na podoben način kot za algebrajske teorije.
Naredili bomo korak v to smer in našo logiko le delno razširili v tako imenovano $\emph{regularno logiko}$, kjer bomo formule gradili iz atomskih formul, logične konstante resničnosti $\top$, konjunkcij $\wedge$ in eksistenčnega kvantifikatorja $\exists$.
Videli bomo, da s to razširitvijo lahko ponovimo zgodbo iz prvega dela in nam to da tako imenovane \emph{regularne kategorije}, ki bodo imele dodatno strukturo, ki nam bo omogočala interpretacijo te logike.
Da to logiko definiramo, moramo najprej razširiti pojem jezika.
%
\begin{definicija}
  \emph{Signatura} regularnega jezika $\Sigma$ je sestavljena iz množice \emph{osnovnih tipov} $\underline{\mathrm{sort}}_\Sigma = \set{X_1, X_2, X_3, \ldots}$, množice konstant $\underline{\mathrm{const}}_\Sigma$, množice funkcijskih simbolov $\underline{\mathrm{func}}_\Sigma$ in množice relacijskih simbolov $\underline{\mathrm{rel}}_\Sigma$. Uporabljamo oznake kot so $c : X$ za konstante tipa $X$, $f : X_1 \times \ldots \times X_n \to Y$ za funkcijske simbole in $R \rightarrowtail X_1 \times \ldots \times X_n$ za relacijske simbole.
\end{definicija}
Pogosto bomo za $X_1 \times \ldots X_n$ uporabljali tudi oznako $\overline{X}$, kjer $n$ razberemo iz konteksta.
\begin{definicija}
  Naj bo $\Sigma$ signatura regularnega jezika. Potem \emph{jezik} $\mathcal{L}(\Sigma)$ sestoji iz signature $\Sigma$, za vsak osnovni tip $X$ imamo števno mnogo spremenljivk $x:X$. Množice termov $(T)$ in formul $(F)$ definiramo na sledeče načine:
  \begin{itemize}
    \item [(T1)] Če je $x$ spremenljivka tipa $X$, potem je $x$ term tipa $X$.
    \item [(T2)] Če je $c$ konstanta tipa $X$, potem je $c$ term tipa $X$.
    \item [(T3)] Če so $t_1, \ldots t_n$ že termi tipov $X_1, \ldots, X_n$ in je $f : X_1 \times \ldots \times X_n \to Y$ funkcijski simbol, potem je $f(t_1, \ldots, t_n)$ term tipa $Y$.
    \item [(F1)] Če sta $t_1, t_2$ terma tipov $X_1$ in $X_2$, potem je $t_1 = t_2$ formula. Bolj natančno bi to zapisali kot $t_1 =_X t_2$.
    \item [(F2)] Logična konstanta $\top$ (ki predstavlja resninično izjavo) je formula.
    \item [(F3)] Če so $t_1, \ldots t_n$ termi tipov $X_1, \ldots, X_n$ in je $R \rightarrowtail X_1 \times \ldots \times X_n$ relacijski simbol, potem je $R(t_1, \ldots, t_n)$ formula.
    \item [(F4)] Če sta $\varphi$ in $\psi$ logični formuli, potem sta $\varphi \wedge \psi$ in $\exists x \varphi$ tudi logični 
  \end{itemize}
  Za logično formulo $\varphi$, množico njenih prostih spremenljiv označujemo z $\mathrm{FV}(\varphi)$. \emph{Teorija} $T$, formulirana v jeziku $\mathcal{L}(\Sigma)$, je množica \emph{sekvent} oblike 
  $$\varphi \implies \psi$$
  kjer sta $\varphi$ in $\psi$ formuli v jeziku teorije $T$.
\end{definicija}
Če je v sekventi premisa enaka $\top$ potem $\top \implies \psi$ označujemo kar kot $\psi$.
\begin{primer}
  Naj bo $\Sigma$ signatura s termi osnovnimi tipi $X,Y$ in $Z$, ki vsebuje tri funkcijske simbole $f: X \to Y$, $g : Y \to Z$ in $h : X \to Z$. V jeziku $\mathcal{L}(\Sigma)$ potem, če je $x$ spremenljivka tipa $X$, lahko v jeziku $\mathcal{L}(\Sigma)$ tvorimo formulo
  $$f(g(x)) = h(x)$$
  Ko definiramo interpretacijo teorije bomo videli, da je to ravno formula, ki pomeni, da je $h$ kompozitum $f$ in $g$.
\end{primer}
Sedaj bomo definirali pravila sklepanja za naš fragment logike prvega reda, za katera bomo kasenje pokazali, da so "sound" in polna glede na kategorično semantiko, ki jim jih bomo dali.
Podali jih bomo kot zaporedja dedukcij oblike $\varphi \vdash_F \psi$, indeksiranih po končnih množicah spremenljivk $F$. Dedukcija $\varphi \vdash \psi$ je definirana le, če vse proste spremenljivke v $\varphi$ ali $\psi$ ležijo v $F$. V prihodnje bomo privzeli, da je za izraz $\varphi \vdash \psi$ ta pogoj vedno izpolnjen (napiši to lepše).
Tu moramo biti pozorni, saj na primer izraz $x_1 = x_2 \vdash_{x_1, x_2} x_2 = x_1$ ni enak izrazu $x_1 = x_2 \vdash_{x_1, x_2, x_3} x_2 = x_1$. Razlog za to podrobnost bomo razložili kasneje, ko definiramo semantiko v kategoriji.

Sedaj definiramo pravila sklepanja, ki jih razdelimo v tri sklope: (ali je lepo dati dvopičje direkt pred def okoljem)
\begin{definicija}
  $ $
  \begin{enumerate}[label*=\arabic*.]
    \item Strukturna pravila
    \begin{enumerate}[label*=\arabic*.]
      \item $p \vdash_F p$
      \item $\dfrac{p \vdash_F q \ q \vdash_F r}{p \vdash_F r}$
      \item $\dfrac{p \vdash_F q}{p \vdash_{F \cup \set{y}} q}$
      \item $\dfrac{\varphi(y) \vdash_F \psi(y)}{\varphi(b) \vdash_{F\setminus\set{y}} \psi(b)}$
      kjer je $y : B$ spremenljivka, $b$ pa term tipa $B$ in $b$ lahko zamenjamo za $y$ v obeh izrazih.
    \end{enumerate}
    \item Logična pravila
    \begin{enumerate}[label*=\arabic*.]
      \item $p \vdash_F \top$
      \item Če $r \vdash_F p \wedge q$ potem $r \vdash_F p$ in $r \vdash_F q$; in če $r \vdash_F p$ in $r \vdash_F q$ potem $r \vdash_F p \wedge q$.
      \item Če doda$\exists y \psi(y) \vdash_F p$ potem $\psi(y) \vdash_{F \cup \set{y}} p$; in obratno, če $\psi \vdash_{F \cup \set{y}} p$ potem $\exists y \psi(y) \vdash_F p$.
    \end{enumerate}
    \item Pravila za enakost
    \begin{enumerate}[label*=\arabic*.]
      \item $\top \vdash_x x = x$
      \item $x_1 = x_2 \vdash_{x_1, x_2} x_2 = x_1$
      \item $x_1 = x_2 \wedge x_2 = x_3 \vdash_{x_1, x_2, x_3} x_1 = x_3$
      \item $\overline{x}^1 = \overline{x}^2 \wedge f(\overline{x}^1) = f(\overline{x}^2)$
      \item $\overline{x}^1 = \overline{x}^2 \wedge R(\overline{x}^1) \vdash_{\overline{x}^1, \overline{x}^2} \mathrm{R}(\overline{x}^2)$
      kjer je $R \rightarrowtail \overline{X}$
    \end{enumerate}
  \end{enumerate}
  Če $F = \emptyset$ potem označimo $\vdash_\emptyset$ kot $\vdash$ in $\emptyset \vdash_F \psi$ označimo z $\vdash_F \psi$.
  Če imamo podano teorijo $T$, pišemo $T, \varphi \vdash_F \psi$, če $\varphi_F^T \psi$, kjer $\varphi_F^T$ pomeni dedukcijo po zgornjih pravilih sklepanja z dodatnim aksiomom
  $$\varphi \vdash_{\mathrm{FV}(\varphi) \cup \mathrm{FV}(\psi)} \psi$$
  za vsako sekvento $\varphi \implies \psi$ v $T$. Alternativno lahko to označimo z $T \vdash_F \varphi \implies \psi$, kar nakazuje, da $\varphi$ implicira $\psi$, modulo $T$.
\end{definicija}
\begin{primer}[izpeljava]
  
\end{primer}
%
\section{Regularne kategorije}
Regularna logika ima jasno v primerjavi z algebrajskimi teorijami dodatno strukturo, ki se bo odražala v strukturi kategorije v kateri jo bomo lahko interpretirali.
Za to nalogo bo pomemben koncept podobjekta v kategoriji.
\begin{definicija}
 Naj bo $\cat{C}$ kategorija in $X \in \cat{C}$ objekt.
 \emph{Podobjekt} objekta $X$ je ekvivalenčni razred monomorfizmov $\alpha : A \rightarrowtail X$, kjer sta dva monomorfizma $\alpha$ in $\beta : B \rightarrowtail X$ ekvivalenta, če sta $A$ in $B$ izomorfna nad $X$.
 (Z drugimi besedami, če sta $A$ in $B$ isti objekt v $\cat{C}/X$). To pomeni, da obstaja tak izomorfizem $i : A \xrightarrow{\cong} B$, da diagram
 \begin{equation*}
   \begin{tikzcd}[column sep=small]
     A \ar[rr, "i"']{a}{\cong} \ar[dr, "\alpha"'] & & B \ar[dl, "\beta"] \\
     & X &
   \end{tikzcd}
 \end{equation*}
 komutira. Pogosto morfizma $\alpha$ ne navedemo eksplicitno ampak pišemo samo $A \rightarrowtail X$. Razred (ki ni nujno množica) podoobjektov $X$ označujemo z $\Sub(X)$.
\end{definicija}
%
\section{Interna logika regularne kategorije}
%
\section{Polnost regularne teorije}
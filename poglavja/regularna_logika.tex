\documentclass[../kategoricna_logika.tex]{subfiles}
\begin{document}
V prvem delu smo videli kako lahko klasično obravnavo algebrajske
teorije posplošimo v okvir teorije kategorij, kjer je opis teorije z
operacijami in enačbami porodil interpretacijo v poljubni kategoriji s
končnimi produkti. To nam je omogočilo tudi interpretacijo končnih
produktov v kategoriji kot predstavitev neke algebrajske
teorije. Kategorije, ki jih srečujemo v matematiki pa imajo pogosto
bolj bogato strukturo, kot zgolj končne produkte.
\section{Regularne kategorije}
En tak razred
kategorij, ki se pojavljajo ">v naravi"< so tako imenovane
\emph{regularne kategorije}.  To so na nek način kategorije, v katerih
je mogoče govoriti o funkcijskih relacijah, kot jih poznamo iz logike in teorije
množic.  Izkaže se, da je v regularnih kategorijah prav tako mogoče
najti interpretacijo neke logike, kakor smo to storili v kategorijah s
končnimi produkti, kjer smo lahko interpretirali logiko termov
sestavljenih iz spremenljivk in operacij ter enačb med njimi.  Pri
regularnih kategorijah gre za fragment logike prvega reda, ki ga
imenujemo regularna logika.  Kaj so regularne kategorije, kje jih
najdemo v matematiki, kako v njih najdemo relacije in kaj je ta
fragment, ki ga je mogoče v njih interpretirati, si bomo ogledali v
naslednjem razdelku.

Za to bomo najprej potrebovali nekaj terminologije.
\begin{definicija}
  Naj bo $\cat{C}$ kategorija in $X \in \cat{C}$ objekt.
  \emph{Podobjekt} objekta~$X$ je ekvivalenčni razred monomorfizmov s
  kodomeno $X$, kjer sta dva monomorfizma
  $\alpha: A \rightarrowtail X$ in $\beta : B \rightarrowtail X$
  ekvivalenta, če sta $A$ in $B$ izomorfna nad~$X$ (z drugimi
  besedami, če sta $A$ in $B$ izomorfna v rezini $\cat{C}/X$).  To pomeni,
  da obstaja tak izomorfizem $i : A \xrightarrow{\cong} B$, da diagram
  \begin{equation*}
    \begin{tikzcd}[column sep=small]
      A \ar[rr, "i"']{a}{\cong} \ar[dr, "\alpha"'] & & B \ar[dl, "\beta"] \\
      & X &
    \end{tikzcd}
  \end{equation*}
  komutira. Pogosto pišemo samo~${A \rightarrowtail X}$ in morfizma
  $\alpha$ ne navajamo eksplicitno.  Razred vseh podobjektov objekta
  $X$ označimo s $\Sub(X)$ in ga opremimo z relacijo delne urejenosti,
  kjer je $A \leq B$, če lahko $\alpha : A \rightarrowtail X$
  faktoriziramo skozi podobjekt $\beta : B \rightarrowtail X$.
\end{definicija}
Ta delno urejeni razred ima največji element, in sicer
$\mathrm{id}_X$.
\begin{definicija}
  Pravimo da je kategorija $\cat{C}$ \emph{dobro potencirana}, če je
  $\Sub(X)$ množica za vsak objekt $X \in \cat{C}$.
\end{definicija}
Od sedaj naprej bomo privzeli, da je $\cat{C}$ dobro potencirana.
Oglejmo si interakcijo med množico $\Sub(X)$ in nekaterimi
kategoričnimi konstrukti.

Recimo, da $\cat{C}$ ima vse povleke.  Potem lahko za vsak
$X \in \cat{C}$ iz~$\Sub(X)$ tvorimo konjunkcije.  Namreč recimo, da
imamo $\alpha : A \rightarrowtail X$ in
${\beta : B \rightarrowtail X}$.  Potem njuno konjunkcijo predstavlja
kompozitum $A \times_X B \rightarrowtail X$, ki ga dobimo iz diagrama
povleka:
\begin{equation*}
  \begin{tikzcd}
    A \times_X B \ar[r, tail] \ar[d, tail] & B \ar[d, tail, "\beta"] \\
    A \ar[r, tail, "\alpha"] & X
  \end{tikzcd}
\end{equation*}
Opazimo lahko tudi, da sta obe projekciji v povleku monomorfizma.  To
pa pomeni, da je $A \times_X B$ največja spodnja meja za $A$ in $B$.
\begin{lema}
Naj bo~$f : X \to Y$ morfizem v $\cat{C}$.  Potem dobimo s povlekom
inducirano preslikavo
$$f^{-1} : \Sub(Y) \to \Sub(X),$$
ki pošlje podobjekt $\beta : B \rightarrowtail Y$ v
$f^{-1}(B) \rightarrowtail X$, kar lahko prikažemo v diagramu
\begin{equation*}
  \begin{tikzcd}
    f^{-1}B \ar[r, tail] \ar[d, tail] & B \ar[d, tail, "\beta"] \\
    X \ar[r, "f"] & Y
  \end{tikzcd}
\end{equation*}
  Potem $f^{-1}$ ohranja konjunkcije.
\end{lema}
\begin{dokaz}
  To lahko razberemo iz diagrama
  \begin{equation*}
    \begin{tikzcd}[column sep=small, row sep=normal]
      & f^{-1}(B \land C) \ar[dl] \ar[dd, tail] \ar[rr, tail] & & f^{-1}C \ar[dl] \ar[dd, tail] \\
      B \land C \ar[dd, tail] \ar[rr, crossing over, tail] & & C  & \\
      & f^{-1}B \ar[dl] \ar[rr, tail] & & X \ar[dl, "f"] \\
      B \ar[rr, tail] & & Y \ar[from=uu, crossing over, tail] &
    \end{tikzcd}
  \end{equation*}
  v katerem sta spodnje in desno lice povleka podobjektov $B$ in $C$ po $f$.
  Sprednje lice je povlek po definiciji konjunkcije $B$ in $C$.
  Levo lice dobimo kot povlek s skladanjem s spodnjim licem in
  zgornje lice s skladanjem z desnim licem.
  Zadnje lice je nato povlek, ker so vsa ostala lica povleki.
\end{dokaz}

Za morfizma $f : X \to Y$ in $g : Y \to Z$ velja $f^{-1}(g^{-1}(C)) = (g \circ f)^{-1}(C)$
za vsak $C$ podobjekt $Z$.
To pomeni, da za dobro potencirano kategorijo $\cat{C}$ dobimo funktor
$$\Sub(\_) : \cat{C}^{\mathrm{op}} \to \cat{\wedge-\mathbf{SLat}}$$
v kategorijo $\land$-polmrež z enoto, ki ga imenujemo \emph{podobjektni
  funktor}.
\begin{definicija}
  Recimo, da je $f : X \to Y$ morfizem v $\cat{C}$. Potem paru
  morfizmov $(p_1, p_2)$ v povleku
  \begin{equation*}
    \begin{tikzcd}
      X \times_Y X \ar[d, "p_1"'] \ar[r, "p_2"] & X \ar[d, "f"] \\
      X \ar[r, "f"'] & Y
    \end{tikzcd}
  \end{equation*}
  pravimo \emph{par jedra} morfizma $f$, ali \emph{par jedra} $f$,
  oziroma, če je $f$ jasen iz konteksta, kar \emph{jedrni par}.
\end{definicija}
\begin{definicija}
  Epimorfizmu $f$ pravimo \emph{regularen}, če je kozožek.
\end{definicija}
Sedaj imamo pripravljeno vse potrebno, da definiramo pojem regularne
kategorije.
\begin{definicija}
  Kategorija $\cat{C}$ je \emph{regularna}, če ima vse končne limite,
  vsak jedrni par ima kozožek in povleki ohranjajo
  regularne epimorfizme.
\end{definicija}

\begin{primer}
  Primeri regularnih kategorij:
  \begin{itemize}
  \item Kategorija $\cat{Set}$ množic in funkcij je
    regularna. Kategorija $\mathbf{Set}$ je kartezično zaprta,
    ima pa tudi vse končne limite in kolimite.  Regularni epimorfizmi
    so ravno surjektivne funkcije, za katere se ni težko prepričati,
    da jih povleki ohranjajo.
  \item Kategorija $\mathbf{Grp}$ grup in homomorfizmov med njimi.
    Končni produkti so tu direktni produkti grup, slika morfizma je ravno slika
    homomorfizma in epimorfizmi v $\mathbf{Grp}$ so vsi regularni.
    Povleki grup ohranjajo epimorfizme, ker se povleki računajo tako kot v $\cat{Set}$
    in so epimorfizmi ravno surjektivni homomorfizmi.
  \item Če je $\mathbb{T}$ algebrajska teorija, potem je kategorija
    modelov te teorije (v $\mathbf{Set}$), regularna.
    To je ravno ena smer Birkhoffovega izreka
    \ref{izrek:Birkhoff-HSP}.
  \item Bolj splošno, če je $\mathbb{T}$ algebrajska teorija in
    $\cat{C}$ regularna, potem je $\cat{Mod}(\mathbb{T}, \cat{C})$
    regularna. Dokaz lahko najdemo v \cite{barr-exact-categories}.
  \item Vsaka Abelova kategorija je regularna. Abelove kategorije
    so opisane v \cite{freyd1964abelian}. Dokaz, da so regularne
    lahko najdemo v \cite{borceux1994handbook}.
  \item Če je $\cat{C}$ regularna in je $\cat{D}$ poljubna kategorija,
    potem je funktorska kategorija $\cat{C}^{\cat{D}}$ regularna \cite{borceux1994handbook}.
  \item Če je $\cat{C}$ regularna in $X \in \cat{C}$, potem je
    $\cat{C}/X$ regularna \cite{borceux1994handbook}.
  \item Kategorija $\cat{Top}^{\mathrm{op}}$ je regularna. Tu je
    ključno dejstvo to, da so regularni monomorfizmi, ki so dual
    regularnih epimorfizmov, ravno vložitve podprostorov (glede na
    topologijo podprostora) in potiski so spet vložitve.
  \end{itemize}
\end{primer}
\begin{primer}[Primeri neregularnih kategorij]
  Kategorije $\cat{Cat}, \cat{Pos}\ \text{in}\ \cat{Top}$ niso
  regularne. Denimo, da smo v kategoriji $\cat{Pos}$.
  Naj bodo: $A$ delno urejena množica $\set{a, b} \times (0 \leq 1)$,
  z urejenostjo $(a,0) \leq (a,1)$ in $(b,0) \leq (b,1)$.
  $B$ naj bo enaka $(0 \leq 1 \leq 2)$ in $C$ naj bo podobjekt $B$
  z očitno vložitvijo ${\dot{\imath}} : (0 \leq 2) \to (0 \leq 1 \leq 2)$.
  Obstaja regularen epimorfizem ${p : A \to B}$, ki ga
  dobimo z identifikacijo $(a,1)$ z $(b, 0)$. Povlek morfizma $p$ po
  $\dot{\imath}$ nam da inkluzijo $\set{0, 2} \to (0 \leq 2)$, ki je
  epimorfizem, a ni regularen, kar pomeni, da povleki v $\cat{Pos}$ ne
  ohranjajo regularnih epimorfizmov.

  Če delno urejene množice interpretiramo kot kategorije, isti primer
  deluje za $\cat{Cat}$ in, ker je kategorija končnih delno urejenih
  množic ekvivalentna kategoriji končnih topoloških prostorov, isti
  primer deluje tudi za kategorijo $\cat{Top}$.
\end{primer}
Poglejmo si nekaj lastnosti regularnih kategorij.
\begin{lema}\label{lema:lastnosti-regularnih-epimorfizmov}
  Naj bo $\cat{C}$ regularna kategorija. Potem
  \begin{enumerate}[label=(\roman*), nosep]
  \item Vsak regularen epimorfizem je kozožek svojega jedrnega para.
  \item Morfizem, ki je hkrati regularen epimorfizem in monomorfizem
    je izo\-morfizem.
  \item Kompozitum dveh regularnih epimorfizmov je regularen
    epimorfizem.
  \item Če sta morfizma $f : X \to Y$ in $g : Y \to Z$ taka, da sta
    $g \circ f$ in $f$ regularna epimorfizma, potem je tudi $g$
    regularen epimorfizem.
  \end{enumerate}
\end{lema}
\begin{dokaz}
  \begin{enumerate}[label=(\roman*)]
  \item % točka (i)
    Naj bo $f : X \to Y$ kozožek morfizmov $g,h : Z \to X$ in
    $(p_1, p_2)$ par jedra $f$.  Naj bo $t : X \to T$ tak, da velja
    $t \circ g = t \circ h$. Poglejmo si diagram
    \begin{equation*}
      \begin{tikzcd}[row sep=normal, column sep=normal]
        Z \ar[ddr, bend right, "h"'] \ar[dr, dashed, "u"] \ar[drr, bend left, "g"] & & & \\
        & X \times_Y X \ar[d, "p_1"'] \ar[r, "p_2"] & X \ar[d, "f"] \ar[ddr, bend left, "t"] & \\
        & X \ar[r, "f"'] \ar[drr, bend right, "t"'] & Y \ar[dr, dashed, "v"] & \\
        & & & T
      \end{tikzcd}
    \end{equation*}
    kjer je morfizem $u$ podan z univerzalno lastnostjo povleka in
    morfizem~$v$ dobimo iz univerzalne lastnosti kozožka $g$ in $h$.
    Torej je $f$ res kozožek morfizmov $p_1$ in $p_2$. Ker je $f$ epimorfizem
    je morfizem $v$ enolično določen.

  \item % (ii)
    Naj bo $f : X \to Y$ regularen epimorfizem, ki je tudi
    monomorfizem.  Ker je $f$ kozožek svojega para jedra $(p_1, p_2)$,
    velja $f \circ p_1 = f \circ p_2$, iz česar sledi $p_1 = p_2$,
    kajti $f$ je monomorfizem.  Ker je $f$ kozožek, dobimo enoličen
    morfizem $g : Y \to X$, da je $g \circ f = \mathrm{id}_X$.  To pa
    pomeni, da je~$f$ hkrati sekcija in epimorfizem, torej je
    izomorfizem.

  \item % (iii)
    Naj bosta $f : X \to Y$ in $g : Y \to Z$ regularna epimorfizma.
    Pokazati želimo, da je $g \circ f$ kozožek svojega jedrnega para
    $(q_1, q_2)$, ki je v diagramu
    \begin{equation*}
      \begin{tikzcd}[column sep=normal]
        X \times_Z X \ar[d] \ar[r] \ar[dr, dashed, two heads, "e"]  & Y \times_Z X \ar[d ] \ar[r] & X \ar[d, two heads, "f"] \\
        X \times_Z Y \ar[d] \ar[r] & Y \times_Z Y \ar[d, "\pi_1"'] \ar[r, "\pi_2"] & Y \ar[d, two heads, "g"] \\
        X \ar[r, two heads, "f"'] & Y \ar[r, two heads, "g"'] & Z
      \end{tikzcd}
    \end{equation*}
    predstavljen kot kompozitum levega in zgornjega roba.  Ker povleki
    v regularni kategoriji ohranjajo regularne epimorfizme, je
    kanonični morfizem $e$, ki je v diagramu prikazan črtkano,
    epimorfizem.  Iz univerzalne lastnosti povleka dobimo še enolični
    morfizem $v : X \times_Y X \to X \times_Z X$
    \begin{equation*}
      \begin{tikzcd}[column sep=normal, row sep=normal]
        X \times_Y X \ar[ddr, bend right, "p_1"'] \ar[drr, bend left, "p_2"] \ar[dr, dashed, "v"] & & \\
        & X \times_Z X \ar[d, "q_1"'] \ar[r, "q_2"] & X \ar[d] \\
        & X \ar[r] & Z
      \end{tikzcd}
    \end{equation*}
    Nazadnje iz diagrama
    \begin{equation*}
      \begin{tikzcd}
        X \times_Y X \ar[dr, shift left, "p_1"] \ar[dr, shift right,
        "p_2"'] \ar[r, "v"] & X \times_Z X \ar[d, shift left, "q_1"]
        \ar[d, shift right, "q_2"'] \ar[r, "e"] &
        Y \times_Z Y \ar[d, shift left, "\pi_1"] \ar[d, shift right, "\pi_2"'] &  \\
        & X \ar[r, "f"] \ar[drr, "t"'] & Y \ar[r, "g"] \ar[dr, dashed, "h"] & Z \ar[d, dashed, "r"] \\
        & & & T
      \end{tikzcd}
    \end{equation*}
    razberemo, da je $(q_1, q_2)$ res kozožek morfizma $g \circ f$.

  \item % (iv)
    Naj bosta $f : X \to Y$ in $g : Y \to Z$ taka, da sta $g \circ f$
    in $f$ oba regularna epimorfizma.  Podobno kot pri prejšnji točki,
    naj bo $(\pi_1, \pi_2)$ par jedra~$g$ in $(q_1, q_2)$ par jedra
    $g \circ f$.  Če imamo $t : Y \to T$ tak, da je
    $t \pi_1 = t \pi_2$, potem morfizem $tf$ zoži $q_1$ in $q_2$ in
    dobimo morfizem $h : Z \to T$, da velja $t f = h g f$. Ker je $f$
    epimorfizem, res dobimo $t = h g$.
  \end{enumerate}
\end{dokaz}
Mnogo lepih lastnosti, ki smo
jih vajeni iz surjektivnih funkcij v kategoriji $\cat{Set}$, imajo
tudi regularni epimorfizmi.
Naslednja lema na pove, da analogno kot lahko v kategoriji $\cat{Set}$
vsako preslikavo razstavimo na kompozitum surjektivne in
injektivne preslikave, podobno storimo v regularnih kategorijah.
\begin{lema}
  Naj bo $\cat{C}$ regularna kategorija in $f: X \to Y$ morfizem
  v~$\cat{C}$.  Potem lahko $f$ razcepimo na kompozitum regularnega
  epimorfizma in monomorfizma.  Velja še, da za vsak komutativni
  diagram
  \begin{equation*}
    \begin{tikzcd}
      X \ar[d, "e"'] \ar[r, "f"] & Y \ar[d, "m"] \\
      X' \ar[r, "f'"] & Y'
    \end{tikzcd}
  \end{equation*}
  v katerem je $e$ regularen epimorfizem in $m$ monomorfizem, obstaja
  enoličen diagonalni morfizem $d : X' \to Y$, tako da oba trikotnika
  komutirata.  V posebnem, je zgornji razcep enoličen do izomorfizma
  natančno.
\end{lema}
\begin{comment}
  \begin{opomba}
    Pokažimo, da zgornja trditev res pomeni, da so slike morfizmov
    enolične.
  \end{opomba}
\end{comment}
\begin{dokaz}
  Naj bo $f : X \to Y$ morfizem.  Označimo njegov par jedra s
  $(p_1, p_2)$ in naj bo $q : X \to Q$ kozožek $(p_1, p_2)$.  To nam da
  diagram
  \begin{equation*}
    \begin{tikzcd}
      X \times_Y X \ar[r, shift left, "p_1"] \ar[r, shift right,
      "p_2"'] &
      X \ar[dr, "f"'] \ar[r, two heads, "q"] & Q \ar[d, dashed, "m"] \\
      & & Y
    \end{tikzcd}
  \end{equation*}
  Dokazati moramo še, da je $m$ monomorfizem. V ta namen recimo, da
  imamo dva morfizma $t_1, t_2 : T \to Q$, da velja $m t_1 = m
  t_2$. Povlek po $m$ nam da par morfizmov
  $(\pi_1, \pi_2) : Q \times_Y Q \to Y$ in morfizem
  $t : T \to Q \times_Y Q$, da velja $t_1 = \pi_1 t$ in
  $t_2 = \pi_2 t$.  Podobno kot pri dokazu prejšnje leme dobimo
  epimorfizem $e : X \times_Y X \to Q \times_Y Q$, ki nam da enačbi
  $p_1 e = q p_1$ in $\pi_2 e = q p_2$.  Ampak, ker je $q$ kozožek
  $p_1$ in $p_2$ dobimo $\pi_1 e = \pi_2 e$, iz česar sledi
  $\pi_1 = \pi_2$, kajti $e$ je epimorfizem.  Potem pa je
  $t_1 = \pi_1 t = \pi_2 t = t_2$, kar pomeni, da je~$m$ res
  monomorfizem.

  Da pokažemo drugi del trditve, denimo, da je $(p_1, p_2)$ par jedra
  $e$.  Tedaj imamo enakosti
  $m f p_1 = m f p_2 = f' e p_1 = f' e p_2$.  Ker je $m$ monomorfizem,
  sledi $f p_1 = f p_2$ in po univerzalni lastnosti kozožka dobimo
  enoličen morfizem~$d : X' \to Y$, da velja $f = d e$. Ker diagram
  iz leme komutira, dobimo
  \[f' e = m f = m d e, \] kar pomeni $f' = m d$, saj je $e$
  epimorfizem.
\end{dokaz}

\begin{definicija}
  V faktorizaciji $f = m \circ e$ na monomorfizem $m$ in epimorfizem $e$
  pravimo podobjektu, ki ga predstavlja $m$, \emph{slika} morfizma $f$
  in ga označimo z $\mathrm{Im}(f)$.
  Včasih rečemo kar objektu $E$ slika $f$.
  Slika morfizma je določena le do izomorfizma natančno, a določa
  natanko en podobjekt objekta $Y$, ki ga označimo z $\exists_f(X)$.
  Oznake $\mathrm{Im}(f)$, $\exists_f(X)$ in $E$ uporabljamo
  izmenljivo, če je pomen jasen iz konteksta.

  Za podobjekt $A \overset{\alpha}{\rightarrowtail} X$ definiramo
  sliko kot
$$\exists_f A := \mathrm{Im}(f \circ \alpha),$$
kar nam da dobro definirano preslikavo
$\exists_f : \Sub(X) \to \Sub(Y)$.
\end{definicija}
\begin{comment}
  Če se malo poigramo s temi slikami, dobimo diagram
  \begin{equation*}
    \begin{tikzcd}
      A \times_Y A \ar[d, shift left , "a_1"] \ar[d, shift right,
      "a_2"'] \ar[r, dashed, "u"] &
      X \times_Y X \ar[d, shift left, "p_1"] \ar[d, shift right, "p_2"'] & \\
      A \ar[d, two heads, "a"] \ar[r, tail, "\alpha"] & X \ar[dr, "f"]
      \ar[r, two heads, "q"] &
      \exists_f X \ar[d, tail, "m"] \\
      \exists_f A \ar[rr, tail] \ar[urr, dashed, "\dot{\imath}"] & & Y
    \end{tikzcd}
  \end{equation*}
  iz katerega lahko zaradi enoličnosti slik (do izomorfizma natančno)
  razberemo, da je $\dot{\imath}$ monomorfizem.
\end{comment}
\begin{lema}
  Naj bo $f : X \to Y$ morfizem v regularni kategoriji $\cat{C}$.
  Tedaj velja:
  \begin{enumerate}[label=(\roman*)]
  \item Preslikava $\exists_f$ je monotona in levo adjungirana povleku
    $f^{-1}$, imamo torej par adjungiranih funktorjev
    $$\exists_f : \Sub(X) \rightleftarrows \Sub(Y) : f^{-1}.$$
  \item Za vsak morfizem $g : Y \to Z$ velja
    $$\exists_g \circ \exists_f = \exists_{g \circ f} : \Sub(X) \to
    \Sub(Z)$$
  \end{enumerate}
\end{lema}
\begin{dokaz}
  Da pokažemo monotonost recimo, da imamo dva podobjekta $A' \leq A$ v
  $\Sub(X)$.  Najprej faktoriziramo
  $A \xhookrightarrow{\alpha} X \xrightarrow{f} Y$ in potem
  $A' \to \exists_f A$.  Iz diagrama
  \begin{equation*}
    \begin{tikzcd}
      A' \ar[d, two heads] \ar[r, tail] \ar[rr, bend left, "\alpha'"]
      &
      A \ar[d, two heads] \ar[r, tail, "\alpha"] & X \ar[d, "f"] \\
      Z \ar[r, tail] & \exists_f A \ar[r, tail] & Y
    \end{tikzcd}
  \end{equation*}
  lahko razberemo, da je $A' \twoheadrightarrow Z \hookrightarrow Y$
  zaradi enoličnosti to faktorizacija $f \circ \alpha'$ in velja
  $Z \cong \exists_f A' \leq \exists_f A$.

  Da pokažemo, da sta to adjungirana funktorja moramo pokazati
  $$\Sub(Y)(\exists_f A, B) \cong \Sub(X)(A, f^{-1}B),$$
  kjer sta $A \leq X$ in $B \leq Y$. Če v diagramu
  \begin{equation*}
    \begin{tikzcd}
      A \ar[ddr, bend right, tail, "\alpha"'] \ar[dr, dashed, "g"]
      \ar[rr, "\rho"] & &
      \exists_f A \ar[d, dashed, "h"] \\
      & f^{-1}B \ar[d] \ar[r, "\pi"] & B \ar[d, tail, "\beta"] \\
      & X \ar[r, "f"] & Y
    \end{tikzcd}
  \end{equation*}
  predpostavimo, da je $A \leq f^{-1}B$, potem lahko
  razcepimo morfizem~$A \to B$ in dobimo monomorfizem
  $\exists_f A \hookrightarrow B$, po enoličnosti slik.  Obratno, če
  predpostavimo $\exists_fA \leq B$, potem dobimo enolični
  morfizem $g$ iz univerzalne lastnosti povleka.

  Drugi del trditve sledi iz tega, da to velja za povleke in je
  $\exists_f$ adjungiran funktor.
\end{dokaz}
%

\noindent
Kot posledico tega lahko vidimo, da $f^{-1}$ ohranja konjunkcije v
$\Sub(Y)$, kajti desni adjunkti ohranjajo limite.
\begin{lema}[Frobeniusova lema]\label{lema:frobeniusova-lema}
  Naj bo $\cat{C}$ regularna kategorija in ${f : X \to Y}$ morfizem v
  $\cat{C}$.  Naj bosta $A \overset{\alpha}{\rightarrowtail} X$ in
  $B \overset{\alpha}{\rightarrowtail} Y$ dva podobjekta.  Potem velja
  $$\exists_f(A \wedge f^{-1}B) = \exists_f(A) \wedge B.$$
\end{lema}
\begin{dokaz}
  V diagramu
  \begin{equation*}
    \begin{tikzcd}
      X \ar[rr, bend left, "f"] \ar[r, two heads] & \exists_f X \ar[r, tail] & Y \\
      A \ar[u, tail, "\alpha"] \ar[r, two heads] & \exists_f A \ar[u, tail] \ar[ur] & \\
      A \land f^{-1}B \ar[u, tail] \ar[r, dashed, two heads] &
      \exists_f A \land B \ar[r, tail] \ar[u, tail] & B \ar[uu, tail,
      "\beta"']
    \end{tikzcd}
  \end{equation*}
  je spodnji levi kvadrat povlek, torej je črtkani morfizem regularen
  epimorfizem, saj je povlek regularnega epimorfizma.  To pa pomeni,
  da je kompozitum
  \[A \land f^{-1}B \twoheadrightarrow \exists_f A \land B
    \hookrightarrow B\] kanonična faktorizacija morfizma
  $A \land f^{-1}B \to Y$, kar nam da želeno enakost.
\end{dokaz}
Še ena izmed lepih lastnosti funkcij med množicami je, da so
določene s svojimi grafi.  Izkaže se, da so regularne kategorije ravno
pravo okolje za tako konstrukcijo.
\begin{definicija}\label{definicija:graf-morfizma}
  Naj bo $f : X \to Y$ morfizem v regularni kategoriji
  $\cat{C}$. Potem je \emph{graf} tega morfizma podobjekt
  $$\graf(f) \rightarrowtail X \times Y,$$
  ki ga definiramo kot sliko morfizma $\fprod{\mathrm{id}_X, f}$.
\end{definicija}
Opazimo lahko, da je kanonični morfizem $X \to \graf(f)$ izomorfizem,
kajti po definiciji je regularen epimorfizem in kot sekcija je
monomorfizem, torej je izomorfizem.
\begin{lema}\label{lema:graf-kot-zozek}
  Monomorfizem $\graf(f) \hookrightarrow X \times Y$ lahko izrazimo
  kot zožek morfizmov $f\circ \pi_1, \pi_2 : X \times Y \to Y$.
\end{lema}
\begin{dokaz}
  Ker je $\pi_1 \circ \fprod{\mathrm{id}_X, f} = \mathrm{id}_X$, je
  $\fprod{\mathrm{id}_X, f}$ sekcija, torej monomorfizem. To pomeni,
  da je kompozitum
  \[X \xrightarrow{\mathrm{id}_X} X
    \xrightarrow{\fprod{\mathrm{id}_X,f}} X \times Y\] že
  faktorizacija $\fprod{\mathrm{id}_X, f}$. Da ta morfizem res zoži
  $f \pi_1$ in $\pi_2$ velja, ker
  \[ f \circ \pi_1 \circ \fprod{\mathrm{id}_X, f} = f \circ
    \mathrm{id}_X = f\] in $\pi_2 \circ \fprod{\mathrm{id}_X, f} = f$.
  Denimo sedaj, da obstaja morfizem $t : T \to X \times Y$, za
  katerega velja $f \circ \pi_1 \circ t = \pi_2 \circ t$. Po
  univerzalni lastnosti produkta lahko~$t$ zapišemo kot
  $\fprod{t_1, t_2}$, kjer
  \[t_1 = \pi_1 \circ t : T \to X \quad \text{in} \quad t_2 = \pi_2
    \circ t : T \to Y. \] Dobili smo torej morfizem od $T$ do $X$, ki
  faktorizira~$t$. Ker je $\fprod{\mathrm{id}_X, f}$ monomorfizem, je
  to enolični morfizem, za katerega ta diagram komutira. (mogoče bi
  bilo lepše, če samo narišemo diagramček)
\end{dokaz}
\begin{lema}
  Naj bodo oznake kot v zgornji definiciji in naj bo
  $A \overset{\alpha}{\rightarrowtail} X$ podobjekt. Če s $\pi_1$ in
  $\pi_2$ označimo projekciji iz produkta, potem velja
  $$\exists_f(A) = \exists_{\pi_2}(\pi_1^{-1}(A) \wedge \graf(f)).$$
\end{lema}
\begin{dokaz}
  V ta namen konstruiramo diagram
  \begin{equation*}
    \begin{tikzcd}
      & & Y & \\
      X \ar[urr, "f"] \ar[r, two heads] & \operatorname{graph}(f)
      \ar[r, tail] &
      X \times Y \ar[u, "\pi_2"] \ar[r, "\pi_1"] & X \\
      A \ar[u, tail, "\alpha"] \ar[r, two heads] & \pi_1^{-1}A \land
      \operatorname{graph}(f) \ar[u, tail] \ar[r, tail] & \pi_1^{-1}A
      \ar[u, tail] \ar[r] & A \ar[u, tail, "\alpha"]
    \end{tikzcd}
  \end{equation*}
  kjer so kvadrati povleki.  Levi spodnji kvadrat dobimo kot povlek
  zunanjega trivialnega povleka $\mathrm{id}_X$ in $\alpha$.  Tako
  dobimo, da je morfizem
  $$A \twoheadrightarrow \pi_1^{-1}A \land \operatorname{graph}(f)$$
  regularen epimorfizem.  Torej nam da sliko morfizma
  $f \circ \alpha$, kar pokaže iskano enakost.
\end{dokaz}
Sedaj bomo opisali, kako lahko dobimo nazaj morfizem iz njegovega
grafa.
\begin{definicija}
  Na podobjekt $R \rightarrowtail X \times Y$ lahko gledamo kot na
  relacijo med ">elementi"< $X$ in $Y$. Relaciji $R$ pravimo
  \begin{itemize}
  \item \emph{totalna}, če je $\exists_{\pi_1}R = X$ (to intuitivno
    pomeni, da je množica vseh tistih $x$, za katere obstaja nek $y$
    za katerega velja $xRy$, enaka $X$).
  \item \emph{funkcijska}, če kanonični morfizem
    $R \times_X R \to X \times Y \times Y$, lahko faktoriziramo skozi
    inkluzijo
    $\mathrm{id}_X \times \Delta_Y : X \times Y \to X \times Y \times
    Y$ (tu si lahko predstavljamo, da ker $R \times_X R$ predstavlja
    trojice elementov $(x, y_1, y_2)$, za katere velja $xRy_1$ in
    $xRy_2$, nam ta faktorizacija omogoča, da iz tega izpeljemo
    $y_1 = y_2$).
  \end{itemize}
\end{definicija}
\begin{opomba}
  Kanonični morfizem $R \times_X R \to X \times Y \times Y$ lahko
  dobimo tako, da v diagramu
  \begin{equation*}
    \begin{tikzcd}[row sep=normal, column sep=small]
      R \times_X R \ar[d] \ar[r] & Y \times R \ar[d] \ar[r] & R \ar[d] \\
      R \times Y \ar[d] \ar[r] & (X \times Y) \times_X (X \times Y)
      \ar[d] \ar[r] &
      X \times Y \ar[d] \\
      R \ar[r] & X \times Y \ar[r] & X
    \end{tikzcd}
  \end{equation*}
  uporabimo izomorfizem
  $(X \times Y) \times_X (X \times Y) \cong X \times Y \times Y$, ki
  ga dobimo iz diagrama povleka
  \begin{equation*}
    \begin{tikzcd}[row sep=normal, column sep=normal]
      X \times Y  \times Y\ar[d] \ar[r] & X \times Y \ar[d, "\pi_1"] \ar[r] & Y \ar[d] \\
      X \times Y \ar[r, "\pi_1"] & X \ar[r] & 1
    \end{tikzcd}
  \end{equation*}
\end{opomba}
\begin{lema}\label{lema:funkcijska-relacija-ima-graf}
  Naj bo $\cat{C}$ regularna kategorija.
  \begin{enumerate}[label=(\roman*)]
  \item Graf morfizma $f : X \to Y$ je totalna in funkcijska relacija
    na $X \times Y$.
  \item Za vsako totalno in funkcijsko relacijo
    $R \rightarrowtail X \times Y$ obstaja natanko en morfizem
    $f : X \to Y$, za katerega je $R = \graf(f)$.
  \end{enumerate}
\end{lema}
\begin{dokaz}
  \begin{enumerate}[label=(\roman*)]
  \item Ker je $X \to \graf(f)$ izomorfizem, lahko faktoriziramo
    kompozitum
    $\graf(f) \hookrightarrow X \times Y \xrightarrow{\pi_1} X$ kot
    \begin{equation*}
      \begin{tikzcd}
        \operatorname{graph}(f) \ar[d, two heads] \ar[r, tail]  & X \times Y \ar[d, "\pi_1"] \\
        X \ar[r, tail, "\mathrm{id}_X"] & X
      \end{tikzcd}
    \end{equation*}
    Iz enoličnosti faktorizacije sledi
    $X \cong \exists_{\pi_1}\graf(f)$, torej je $\graf(f)$ totalna.

    Da je tudi funkcijska relacija bomo razbrali iz diagrama povlekov:
    \begin{equation*}
      \begin{tikzcd}[row sep=9ex]
        \graf(f) \times_X \graf(f) \ar[d] \ar[r] \ar[dr, dashed, "e"] &
        Y \times \graf(f) \ar[d] \ar[r] & \graf(f) \ar[d]\\
        \graf(f) \times Y \ar[d] \ar[r] & X \times Y \times Y \ar[d, "\pi_{1,2}"'] \ar[r, "\pi_{1,3}"'] &
        X \times Y \ar[d] \ar[l, bend right, " \mathrm{id}_X \times \Delta"'] \\
        \graf(f) \ar[r] & X \times Y \ar[u, bend right, " \mathrm{id}_X \times \Delta"'] \ar[r] & X
      \end{tikzcd}
    \end{equation*}
    Opazimo lahko, da sta obe poti $\graf(f) \times \graf(f) \to X \times Y$ enaki, saj ju lahko
    podaljšamo do $X$ s projekcijo $\pi_1$ in do $Y$ s projekcijo $\pi_2$, kar nam da identiteto.
    Če te dve poti komponiramo z morfizmom $\mathrm{id}_X \times \Delta$ dobimo ravno enolični morfizem
    $\graf(f) \times \graf(f) \to X \times Y \times Y$ podan z univerzalno lastnostjo tega povelka.
    To velja, ker $\mathrm{id}_X \times \Delta$ lahko podaljšamo s $\pi_{1,2}$ na eni strani in
    $\pi_{1,3}$ na drugi in v obeh primerih dobimo identiteto. Torej lahko $e$ faktoriziramo
    skozi $\mathrm{id}_X \times \Delta$.

  \item Ker je $R$ totalna, je
    $R \xhookrightarrow{r} X \times Y \xrightarrow{\pi_1} X$ regularen
    epimorfizem, torej je kozožek jedrnega para $(p_1, p_2)$, kar
    lahko predstavimo v diagramu
    \begin{equation*}
      \begin{tikzcd}[column sep=normal]
        R \times_X R \ar[ddd, bend right=30, "p_1"'] \ar[dd, tail]
        \ar[dr, dashed] \ar[rr, tail]
        \ar[rrr, bend left=20, "p_2"] & &  Y \times R \ar[dd, hook] \ar[r] & R \ar[dd, tail, "r"] & \\
        & X \times Y \ar[dr, dashed, "\mathrm{id}_X \times \Delta_Y"'] & & & \\
        R \times Y \ar[d] \ar[rr, tail] & & X \times Y \times Y \ar[d,
        "\pi_{1,2}"']
        \ar[r, "\pi_{1,3}"] & X\times Y \ar[d, "\pi_1"'] \ar[ddr, "\pi_2"] & \\
        R \ar[rr, tail, "r"] & & X \times Y \ar[r, "\pi_1"] \ar[drr, "\pi_2"'] & X & \\
        & & & & Y
      \end{tikzcd}
    \end{equation*}
    v katerem je predstavljena tudi faktorizacija
    $$R \times_X R \to X \times Y \xrightarrow{\mathrm{id}_X \times \Delta_Y} X \times Y \times Y,$$
    ki obstaja, ker je $R$ funkcijska relacija.  Obe poti od
    $X \times Y$ do $Y$ sta samo projekciji na drugo koordinato, torej
    velja $\pi_2 r p_1 = \pi_2 r p_2$.  Ker pa je $X$ kozožek svojega
    para jedra, obstaja enoličen morfizem $f : X \to Y$, da diagram
    komutira.

    Pokazati moramo še, da je $\graf(f)$ res izomorfen $R$. Za to
    pogledamo diagram
    \begin{equation*}
      \begin{tikzcd}
        & & & & Y \\
        R \ar[r, tail, "r"'] \ar[rr, two heads, bend left] & X \times
        Y \ar[r, "\pi_1"'] & X \ar[r, two heads] \ar[rr, bend left,
        pos=0.3, "\fprod{\mathrm{id}_X, f}"] \ar[urr, crossing over,
        pos=0.75, "f"] \ar[drr, "\mathrm{id}_X"'] &
        \operatorname{graph}(f) \ar[r, tail] & X \times Y \ar[d, "\pi_1"] \ar[u, "\pi_2"'] \\
        & & & & X
      \end{tikzcd}
    \end{equation*}
    v katerem je desni del samo definicija grafa morfizma $f$.  Levi
    kompozitum $R \to X$ je regularen epimorfizem.  Torej dobimo
    enolični morfizem
    $R \twoheadrightarrow \graf(f) \hookrightarrow X \times Y$, za
    katerega nam kompozitum s $\pi_1$ da $\mathrm{id}_X \pi_1 r$ in
    kompozitum s $\pi_1$ da
    $f \circ \pi_1 \circ r = \pi_2 \circ \fprod{\mathrm{id}_X, f} \circ \pi_1
    \circ r = \pi_2 \circ  r$.  To pa je ravno monomorfizem
    $r : R \hookrightarrow X \times Y$, katerega slika je kar $R$.
    Torej je $R \cong \graf(f)$.\qedhere
  \end{enumerate}
\end{dokaz}

Naslednja pomembna lastnost regularnih kategorij je interakcija med
slikami in povleki.
\begin{lema}\label{lema:zamenjava-povleka-in-slike}
  Naj bo
  \begin{equation*}
    \begin{tikzcd}
      Z \times_Y X \ar[d, "f'"'] \ar[r, "g'"] & X \ar[d, "f"] \\
      Z \ar[r, "g"'] & Y
    \end{tikzcd}
  \end{equation*}
  kvadrat povleka. Potem je
  $$\exists_{g'}f^{'-1} = f^{-1}\exists_g : \Sub(Z) \to \Sub(X).$$
\end{lema}
\begin{dokaz}
  Naj bo $A \xhookrightarrow{\alpha} Z$ podobjekt.  Situacijo lahko
  prikažemo v diagramu
  \begin{equation*}
    \begin{tikzcd}[column sep=small]
      A \times_Y X \ar[dd] \ar[dr, tail] \ar[rr, two heads] & &
      f^{-1}\exists_g A \ar[dd] \ar[dr, tail] & \\
      & Z \times_Y X  \ar[rr, crossing over] & & X \ar[dd, "f"] \\
      A \ar[dr, tail] \ar[rr, two heads] & & \exists_g A \ar[dr, tail] & \\
      & Z \ar[rr, "g"'] \ar[from=uu, pos=0.2, crossing over, "f'"] & &
      Y
    \end{tikzcd}
  \end{equation*}
  kjer je sprednja stranica kvadrat povleka.  Leva stran je povlek $A$
  po $f'$, spodnja stran je faktorizacija morfizma $g$, desna stran je
  povlek po $f$ in zadnja stran je tudi kvadrat povleka.  Ker je
  $A \times_Y X = f'^{-1}A$, je
  \[A \times_Y X \twoheadrightarrow f^{-1}\exists_g A \hookrightarrow
    X\] ravno faktorizacija kompozituma
  $f'^{-1}A \hookrightarrow Z \times_Y X \to X$, kar nam da želeno
  enakost.
\end{dokaz}
\begin{definicija}
  Naj bosta $\cat{C}$ in $\cat{D}$ regularni kategoriji.  Funktorju
  ${F : \cat{C} \to \cat{D}}$ pravimo \emph{regularen}, če ohranja
  končne limite in kozožke jedrnih parov.
\end{definicija}
\begin{definicija}
  $\cat{RegCat}$ je kategorija v kateri so objekti majhne regularne
  kategorije in morfizmi regularni funktorji med njimi.
\end{definicija}
\begin{lema}
  Regularen funktor med regularnima kategorijama ohranja povleke in
  slike.
\end{lema}
\begin{dokaz}
Ker je v kategoriji s povleki morfizem $f$ monomorfizem natanko
takrat, ko je diagram
\begin{equation*}
  \begin{tikzcd}
    \bullet \ar[d, "\mathrm{id}"'] \ar[r, "\mathrm{id}"] & \bullet \ar[d, "f"] \\
    \bullet \ar[r, "f"] & \bullet
  \end{tikzcd}
\end{equation*}
povlek, to pomeni, da vsak regularen funktor $F : \cat{C} \to \cat{D}$
inducira, za vsak objekt $X \in \cat{C}$, preslikavo
$$F_X : \Sub_\cat{C}(X) \to \Sub_\cat{D}(F(X)),$$
ki ohranja končne konjunkcije in največji element. Torej je v posebnem
tudi monotona.  Funktor $F$ pa ohranja tudi slike morfizmov, kar
pomeni, če imamo podobjekt
$A \xhookrightarrow{\alpha} X \xrightarrow{f} Y$ potem je
$$F(\exists_f A) = \exists_{F(f)}(F(A)),$$
kajti $\exists_f A$ je definiran kot kozožek para jedra
$f \circ \alpha : A \to Y$.
\end{dokaz}
%
%
\section{Regularna logika}
V prvem poglavju smo videli povezavo med algebrajsko teorijo in
kategorijami s končnimi produkti.  Teorija, ki smo jo obravnavali je
bila precej enostavna s stališča logike, saj so bile edine logične
formule, ki smo jih lahko konstruirali oblike $t_1 = t_2$ za neka
terma, sestavljena induktivno iz spremenljivk in funkcijskih simbolov.
Že iz teh osnovnih gradnikov je mogoče dobiti veliko pomembne
matematične ">infrastrukture"<, kot je na primer teorija grup.  Za
opis vseh matematičnih teorij pa jasno ta stopnja kompleksnosti ne
zadostuje.  Videli smo, da se teorije polj ne da opisati z algebrajsko
teorijo.  Moderna matematika je standardno opisana v jeziku predikatnega
računa, oziroma logike prvega (ali višjega) reda, kjer imamo poleg
funkcijskih simbolov še logične veznike kot so ">in"<, ">ali"<,
negacijo in univerzalni ter eksistenčni kvantifikator.  Naravno
vprašanje je torej, ali lahko zgodbo algebrajskih teorij ponovimo z
neko močnejšo logiko. Naredili bomo korak v to
smer in našo logiko le delno razširili v tako imenovano
$\emph{regularno logiko}$, kjer bomo formule gradili iz atomskih
formul, logične konstante resničnosti $\top$, konjunkcij $\wedge$ in
eksistenčnega kvantifikatorja $\exists$.
\begin{definicija}
  \emph{Regularna signatura} $\Sigma$ je sestavljena iz
  \begin{itemize}
  \item množice \emph{sort}
    $\underline{\mathrm{sort}}_\Sigma = \set{X_1, X_2, X_3, \ldots}$,
  \item množice funkcijskih simbolov
    $\underline{\mathrm{func}}_\Sigma$ z elementi oblike
    \[ f : (X_1, \ldots, X_n) \to Y,\]
    kjer so $X_1,\ldots, X_n,Y$ sorte.
  \item množice relacijskih simbolov
    $\underline{\mathrm{rel}}_\Sigma$ z elementi oblike
    \[ R : (X_1,\ldots, X_n),\]
    kjer so $X_1,\ldots, X_n$ sorte.
  \end{itemize}
\end{definicija}
\begin{opomba}
  Poseben primer funkcijskih simbolov so \emph{konstante}, ki imajo
  prazno domeno in so oblike $c : () \to X$. 
\end{opomba}
Pogosto bomo za $(X_1, \ldots, X_n)$ uporabljali tudi oznako
$\bar{X}$, kjer $n$ razberemo iz konteksta.
\begin{definicija}
  Naj bo $\Sigma$ regularna signatura.  Potem \emph{jezik}
  $\mathcal{L}(\Sigma)$ sestoji iz signature $\Sigma$, za vsako
  sorto $X$ števno mnogo spremenljivk $x_1,x_2,x_3,\ldots : X$.
  Množice termov in formul definiramo induktivno:
  \begin{itemize}
  \item [(T1)] Če je $x$ spremenljivka sorte $X$, potem je $x$ term
    sorte $X$.
  \item [(T2)] Če je $c$ konstanta sorte $X$, potem je $c$ term sorte
    $X$.
  \item [(T3)] Če so $t_1, \ldots t_n$ že termi sorte
    $X_1, \ldots, X_n$ in je $f : (X_1, \ldots, X_n) \to Y$
    funkcijski simbol, potem je $f(t_1, \ldots, t_n)$ term sorte $Y$.
  \item [(F1)] Če sta $t_1, t_2$ terma sorte $X$, potem je $t_1 = t_2$
    formula. Bolj natančno bi to zapisali kot $t_1 =_X t_2$.
  \item [(F2)] Logična konstanta $\top$, ki predstavlja resnično
    izjavo, je formula.
  \item [(F3)] Če so $t_1, \ldots t_n$ termi sort $X_1, \ldots, X_n$
    in je $R : (X_1, \ldots, X_n)$ relacijski
    simbol, potem je $R(t_1, \ldots, t_n)$ formula.
  \item [(F4)] Če sta $\varphi$ in $\psi$ logični formuli, potem sta
    $\varphi \wedge \psi$ in $\exists (x:X) \varphi$ tudi logični formuli.
  \end{itemize}
  Za logično formulo $\varphi$, označujemo množico njenih prostih
  spremenljiv s $\mathrm{FV}(\varphi)$.  \emph{Teorija} $\T$,
  formulirana v jeziku $\mathcal{L}(\Sigma)$, je množica
  \emph{sekvent}, ki so sestavljene iz \emph{premise} in
  \emph{zaključka}, oblike
  \[ \Gamma \mid \varphi \implies \psi,\]
  kjer je $\Gamma$ kontekst spremenljivk in sta
  $\varphi$ in $\psi$ formuli v jeziku teorije $\T$.
  V kontekstu $\Gamma$ morajo biti vse proste spremenljivke,
  ki nastopajo v $\varphi$ in $\psi$.
\end{definicija}
Če je v sekventi premisa enaka $\top$, potem sekvento
$\Gamma \mid \top \implies \psi$ označujemo kar kot $\Gamma \mid \psi$, in pravimo da $\psi$
\emph{velja} v kontekstu $\Gamma$.
\begin{primer}\label{primer:kompozitum}
  Naj bo $\Sigma$ signatura s tremi sortami $X,Y$ in $Z$, ki
  vsebuje tri funkcijske simbole $f: X \to Y$, $g : Y \to Z$ in
  $h : X \to Z$.  Potem, če je $x$ spremenljivka sorte $X$, lahko v
  jeziku $\mathcal{L}(\Sigma)$ tvorimo formulo
  $$x:X \mid f(g(x)) = h(x).$$
  Ko definiramo interpretacijo teorije, bomo videli, da je to
  ravno formula, ki pravi, da je $h$ kompozitum $f$ in $g$.
\end{primer}
\begin{primer}[Delno urejene grupe]\label{primer:delno-urejene-grupe}
  Naj bo $\Sigma$ signatura z eno sorto $X$. Za delno
  urejeno grupo potrebujemo operacije grupe:
  \begin{itemize}
  \item konstanto $e : X$, ki predstavlja enoto grupe,
  \item operacijo množenja $m : X \times X \to X$,
  \item operacijo inverza $i : X \to X$,
  \end{itemize}
  ki zadoščajo aksiomom:
  \begin{itemize}
  \item[(G1)] $m(x,(m(y,z))) = m(m(x,y), z)$,
  \item[(G2)] $m(x,e) = m(e,x) = x$,
  \item[(G3)] $m(x,i(x)) = m(i(x), x) = e$.
  \end{itemize}
  Poleg tega imamo še relacijo $\leq$ na $X \times X$, za katero
  veljajo aksiomi delne urejenosti:
  \begin{itemize}
  \item[(U1)] $x \leq x$ (refleksivnost)
  \item[(U2)] $x \leq y \wedge y \leq x \implies x = y$
    (anti-simetričnost)
  \item[(U3)] $x \leq y \wedge y \leq z \implies x \leq z$
    (tranzitivnost)
  \end{itemize}
  Za definicijo delno urejene grupe potrebujemo še aksiom, ki mu
  pravimo {\emph{invarianca za translacijo}} in pravi:
  \begin{itemize}
  \item[(TI)]
    $x \leq y \implies m(x, g) \leq m(y, g) \wedge m(g,x) \leq
    m(g,y)$.
  \end{itemize}
\end{primer}

Sedaj bomo definirali pravila sklepanja za naš fragment logike prvega
reda, za katera bomo kasneje pokazali, da so veljavna in polna glede
na kategorično semantiko, ki jim jih bomo dali.  Podali jih bomo kot
zaporedja izpeljav oblike $\Gamma \mid \varphi \vdash_F \psi$,
indeksiranih po končnih množicah spremenljivk $\Gamma$.
Tu moramo biti pozorni, saj na primer izraz
\[x_1,x_2:X \mid x_1 = x_2 \vdash x_2 = x_1\] ni enak izrazu
\[x_1,x_2,x_3:X \mid x_1 = x_2 \vdash x_2 = x_1.\] Razlog za to
podrobnost bomo razložili kasneje, ko definiramo semantiko v
kategoriji. Če je kontekst jasen, ga občasno tudi ne pišemo.

Sedaj definiramo pravila sklepanja v regularni logiki, ki jih
razdelimo v tri sklope.  Podali jih bomo kot izpeljave, zapisane v
obliki ulomka, kjer števec predstavlja premise, imenovalec pa sklep
izpeljave.  Beremo jih kot: če veljajo vse premise nekega pravila,
lahko izpeljemo njegov sklep.  Če neka izpeljava velja v obe smeri
torej, če lahko iz premise $A$ izpeljemo sklep $B$ in iz premise $B$
izpeljemo sklep~$A$, potem to podamo kot dvosmerno izpeljavo, ki jo
označimo z dvojno črto ulomka.

\begin{definicija}
  \hfill
  \begin{enumerate}[label*=(\arabic*]
  \item Strukturna pravila
    \begin{enumerate}[label*=.\arabic*)]
    \item\label{pravilo:refl}
      \begin{prooftree}
        \AxiomC{} \UnaryInfC{$\Gamma \mid \varphi \vdash \varphi$}
      \end{prooftree}
    \item\label{pravilo:tranz}
      \begin{prooftree}
        \AxiomC{$\Gamma \mid \varphi \vdash \psi$} \AxiomC{$\Gamma \mid \psi \vdash \rho$}
        \BinaryInfC{$\Gamma \mid \varphi \vdash r$}
      \end{prooftree}
    \item\label{pravilo:slepa-spr}
      \begin{prooftree}
        \AxiomC{$\Gamma \mid \varphi \vdash \psi$}
        \UnaryInfC{$\Gamma, y:Y \mid \varphi \vdash \psi$}
      \end{prooftree}
    \item\label{pravilo:subst}
      \begin{prooftree}
        \AxiomC{$\Gamma, y:B \mid \varphi \vdash \psi$}
        \AxiomC{$\Gamma \mid b : B$}
        \BinaryInfC{$\Gamma \mid \varphi[b/y] \vdash \psi[b/y]$}
      \end{prooftree}
      kjer je $y : B$ spremenljivka, $b$ pa term sorte $B$ in $b$ lahko
      zamenjamo za $y$ v obeh izrazih.
    \end{enumerate}
  \item Logična pravila
    \begin{enumerate}[label*=.\arabic*)]
    \item\label{pravilo:resnica}
      \begin{prooftree}
        \AxiomC{} \UnaryInfC{$\Gamma \mid \varphi \vdash \top$}
      \end{prooftree}
    \item\label{pravilo:konj}
      \begin{prooftree}
        \AxiomC{$\Gamma \mid \rho \vdash \varphi$} \AxiomC{$\Gamma \mid \rho \vdash \psi$} \doubleLine
        \BinaryInfC{$\Gamma \mid \rho \vdash \varphi \wedge \psi$}
      \end{prooftree}
    \item\label{pravilo:eksist}
      \begin{prooftree}
        \AxiomC{$\Gamma \mid \exists (y:Y) \psi \vdash \varphi$} \doubleLine
        \UnaryInfC{$\Gamma, y:Y \mid \psi \vdash \varphi$}
      \end{prooftree}
    \end{enumerate}
  \item Pravila za enakost
    \begin{enumerate}[label*=.\arabic*)]
    \item\label{pravilo:enakost-refl} $x:X \mid \top \vdash x = x$
    \item\label{pravilo:enakost-sim}
      $x_1,x_2:X \mid x_1 = x_2 \vdash x_2 = x_1$
    \item\label{pravilo:enakost-tranz}
      $x_1,x_2,x_3:X \mid x_1 = x_2 \wedge x_2 = x_3 \vdash x_1 = x_3$
    \item\label{pravilo:enakost-subst-fun}
      $\bar{x},\bar{y}:\bar{X} \mid \bar{x} = \bar{y} \vdash
      f(\bar{x}) = f(\bar{y})$
    \item\label{pravilo:enakost-subst-rel}
      $\bar{x},\bar{y} : \bar{X} \mid \bar{x} = \bar{y} \wedge R(\bar{x}) \vdash
      \mathrm{R}(\bar{y})$, kjer je
      $R : \bar{X}$
    \end{enumerate}
  \end{enumerate}
  Če je $\Gamma = \emptyset$, potem $\Gamma \mid \varphi \vdash \psi$
  označimo kot $\varphi \vdash \psi$.
  Izpeljavo $\Gamma \mid \emptyset \vdash \psi$ označimo kar kot
  $\Gamma \mid \psi$.
  Če imamo podano teorijo $\T$, potem izpeljavo v
  tej teoriji označujemo z
  \[ \Gamma \mid \varphi \vdash_{\mathbb{T}} \psi, \]
  kar pomeni izpeljava po
  zgornjih pravilih sklepanja, z dodatnim aksiomom
  \[\mathrm{FV}(\varphi) \cup \mathrm{FV}(\psi) \mid \varphi \vdash \psi,\]
  za vsako sekvento $\varphi \Rightarrow \psi$ v $\T$.
  Izpeljave gradimo v obliki dreves,
  kjer so listi drevesa naše predpostavke in koren sklep izpeljave.
\end{definicija}
\begin{primer}\label{primer:vpeljava-eksist-kvantifikatorja}
  Poglejmo si kako bi izpeljali pravilo za vpeljavo eksistenčnega
  kvantifikatorja, ki pravi
  \begin{prooftree}
    \AxiomC{$\Gamma,t:T \mid \rho \vdash \psi(t)$}
    \UnaryInfC{$\Gamma \mid \rho \vdash \exists (x:X) \psi(x)$}
  \end{prooftree}
  Podali bi ga v obliki drevesa
  \begin{prooftree}
    \AxiomC{$\Gamma,t:T \mid \rho \vdash \psi(t)$} \AxiomC{}
    \RightLabel{\scriptsize(1.1)}
    \UnaryInfC{\(\Gamma \mid \exists (x:X) \psi(x) \vdash \exists (x:X) \psi(x)\)}
    \RightLabel{\scriptsize(2.3)}
    \UnaryInfC{$\Gamma, x:X \mid \psi(x) \vdash \exists (x:X) \psi(x)$}
    \AxiomC{$\Gamma \mid t:T$}
    \RightLabel{\scriptsize(1.4)}
    \BinaryInfC{$\Gamma \mid \psi(t) \vdash \exists (x:X) \psi(x)$}
    \RightLabel{\scriptsize(1.2)}
    \BinaryInfC{$\Gamma \mid \rho \vdash \exists (x:X) \psi(x)$}
  \end{prooftree}
\end{primer}
Brez dokaza podamo še naslednji dve izpeljavi. V prvi lahko prepoznamo
Frobeniusovo lemo \ref{lema:frobeniusova-lema}.
\begin{lema}\label{lema:uporabne-izpeljave}
  V regularni logiki velja
  \begin{enumerate}
  \item
    $\vdash_{\bar{z}} \exists \bar{x}(p(\bar{x}) \land q) \iff \exists
    \bar{x} p(\bar{x}) \land q$, pri pogoju, da $\bar{x}$ ne nastopa
    prosto v $q$.
  \item
    $\vdash_{\bar{x}, \bar{x}'} p(\bar{x}) \land \bar{x} = \bar{x}'
    \implies p(\bar{x}').$
  \end{enumerate}
\end{lema}
%
%
%
\section{Model regularnega jezika}
Recimo, da imamo jezik \(\mathcal{L}(\Sigma)\). Podobno kot smo to
storili za algebrajsko teorijo, bi radi definirali model tega jezika v
regularni kategoriji.  Za to najprej potrebujemo interpretacijo.
\begin{opomba}
  Za definicijo interpretacije bomo potrebovali končne produkte in
  zožke, ki so v splošnem določeni le do izomorfizma natančno.
  Privzeli bomo, da lahko vedno \emph{izberemo} nek objekt, ki
  produkt, oziroma zožek, predstavlja. To pa pomeni, da moramo
  privzeti neko verzijo aksioma izbire.
\end{opomba}
\begin{definicija}
  Naj bo $\mathcal{L}(\Sigma)$ jezik. Potem \emph{interpretacija} $M$
  v kategoriji $\cat{C}$ sestoji iz:
  \begin{itemize}
  \item Objekta $X^{(M)}$ v $\cat{C}$, za vsako sorto
    $X \in \underline{\mathrm{sort}}_\Sigma$
  \item Morfizma $c^{(M)} : 1 \to X^{(M)}$, za vsako konstanto
    $c : () \to X^{(M)}$. Tu je $1$ končni objekt
    v $\cat{C}$.
  \item Morfizma
    $f^{(M)} : X_1^{(M)} \times \cdots \times X_n^{(M)} \to Y^{(M)}$
    za vsak funkcijski simbol
    \[f : (X_1, \ldots, X_n) \to Y \in
      \underline{\mathrm{func}}_\Sigma.\]
  \item Podobjekta
    $R^{(M)} \rightarrowtail X_1^{(M)} \times \cdots \times X_n^{(M)}$
    za vsak relacijski simbol
    \[R : (X_1, \ldots, X_n) \in
      \underline{\mathrm{rel}}_\Sigma.\]
  \end{itemize}
  Produkt $X_1^{(M)} \times \cdots \times X_n^{(M)}$ označimo z
  $\bar{X}^{(M)}$.  Interpretacijo $M$ bomo razširili na vse terme in
  formule jezika.  Za term $t$ sorte $Y$, s prostimi spremenljivkami
  med $\bar{z} : \bar{Z}$ bomo predpisali morfizem
  $t(\bar{z})^{(M)} : \bar{Z}^{(M)} \to Y^{(M)}$, za formulo
  $\varphi$, s prostimi spremenljivkami med $\bar{z} : \bar{Z}$, pa
  predpišemo podobjekt $\set{\bar{z} \mid \varphi}^{(M)}$ objekta
  $\bar{Z}^{(M)}$ po naslednjih pravilih:
  \begin{itemize}
  \item[(T1)] Če je $x$ spremenljivka sorte $X$, potem
    $x(\bar{z})^{(M)}$ definiramo kot kompo\-zi\-tum
    $\bar{Z}^{(M)} \xrightarrow{\pi} X^{(M)}
    \xrightarrow{\mathrm{id}_X} X^{(M)}$.  Tu je $\mathrm{id}_X$
    tisti, ki interpretira $x$, medtem, ko je projekcija $\pi$
    potrebna zaradi ">slepih"< spremenljivk, ki nastopajo v $\bar{z}$
    (opomnimo, da po predpostavki proste spremenljivke v termu $x$
    (torej tudi spremenljivka $x$ sama) nastopajo v $\bar{z}$).

  \item[(T2)] Če je $c:X$ konstanta, potem $c(\bar{z})^{(M)}$
    interpretiramo kot kompozitum
    \[\bar{Z} \xrightarrow{!} 1 \xrightarrow{c^{(M)}} X^{(M)}.\]

  \item[(T3)] Naj bo $f : (X_1, \ldots, X_n) \to Y$
    funkcijski simbol in $t_i$ term sorte $X_i$ za $i = 1, \ldots, n$.
    Po indukciji imamo interpretacije
    $t_i(\bar{z})^{(M)} : \bar{Z}^{(M)} \to X_i^{(M)}$.  Potem term
    $f(t_1, \ldots, t_n)$ interpretiramo kot kompozitum
    \[ f^{(M)}(t_1(\bar{z})^{(M)}, \ldots, t_n(\bar{z})^{(M)}) :
      \bar{Z}^{(M)} \xrightarrow{\fprod{t_1^{(M)}, \ldots, t_n^{(M)}}}
      \bar{X}^{(M)} \xrightarrow{f^{(M)}} Y^{(M)}.
    \]
  \item[(F1)] Formuli $t_1 = t_2$ priredimo podobjekt
    $\set{\bar{z} \mid t_1 = t_2}^{(M)}$, ki ga definiramo kot zožek
    morfizmov
    \begin{tikzcd}[column sep=huge]\bar{Z}^{(M)} \ar[r, shift left,
      "t_1(\bar{z})^{(M)}"] \ar[r, shift right, "t_2(\bar{z})^{(M)}"']
      & X^{(M)}\end{tikzcd}.

  \item[(F2)] Za relacijski simbol $R : \bar{X}$
    interpretiramo $\set{\bar{z} \mid R(t_1, \ldots, t_n)}^{(M)}$ kot
    podobjekt $\bar{Z}^{(M)}$, definiran diagramom povleka
    \begin{equation*}
      \begin{tikzcd}
        \set{\bar{z} \mid R(t_1, \ldots, t_n)}^{(M)} \ar[d] \ar[r,
        hook] &
        \bar{Z}^{(M)} \ar[d, "\fprod{t_1^{(M)}, \ldots ,t_n^{(M)}}"] \\
        R^{(M)} \ar[r, hook] & \bar{X}^{(M)}
      \end{tikzcd}
    \end{equation*} 

  \item[(F3)] $\set{\bar{z} \mid \top}^{(M)}$ je enak $\bar{Z}^{(M)}$.

  \item[(F4)]
    $\set{\bar{z} \mid \varphi \wedge \psi}^{(M)} = \set{\bar{z} \mid
      \varphi}^{(M)} \wedge \set{\bar{z} \mid \psi}^{(M)}$.

  \item[(F5)]
    $\set{\bar{z} \mid \exists x \varphi}^{(M)} =
    \exists_{\pi}\set{(x, \bar{z}) \mid \varphi}^{(M)}$, kjer je
    $\pi$ projekcija\\
    $X^{(M)} \times \bar{Z}^{(M)} \xrightarrow{\pi} \bar{Z}^{(M)}$.

  \end{itemize}
\end{definicija}
\begin{definicija}
  Interpretaciji $M$ pravimo \emph{model} za sekvento
  $\varphi \implies \psi$, kar označimo kot
  $$M \models \varphi \implies \psi,$$
  če velja
  $\set{\bar{x} \mid \varphi}^{(M)} \leq \set{\bar{x} \mid
    \psi}^{(M)}$, kot podobjekta $\bar{X}^{(M)}$, kjer je $\bar{x}$
  množica spremenljivk, ki nastopajo prosto v $\varphi$ ali $\psi$.
  Interpretacija $M$ je model teorije~$\T$, če je model vsake sekvente
  v $\T$.  To označimo z $M \models T$.
\end{definicija}
\begin{opomba}
  Da je $M$ model $\varphi \Rightarrow \psi$, je v $\cat{Set}$ to
  ekvivalentno temu, da je $M$ model (v klasičnem smislu) formule
  $\forall \bar{x} (\varphi \rightarrow \psi)$. To je intuicija, ki jo
  je dobro imeti v mislih.
\end{opomba}
\begin{primer}
  Naj bo $\mathcal{L}(\Sigma)$ jezik iz primera
  \ref{primer:kompozitum} s tremi sortami $X,Y,Z$ in funkcijskimi simboli
  $f : X \to Y$, $g : Y \to Z$ in $h : X \to Z$.  Potem za
  interpretacijo $M$ v regularni kategoriji $\cat{C}$ velja, da je $M$
  model formule kompozituma $f(g(x)) = h(x)$, oziroma da
  $$M \models \top \implies f(g(x)) = h(x),$$ 
  natanko tedaj, ko je $h^{(M)} = g^{(M)} \circ f^{(M)}$.  Poglejmo si
  ta primer bolj detajlno.  Edina prosta spremenljivka, ki nastopa v
  tej formuli je $x$. Gledamo torej interpretacijo
  $\set{x \,\middle|\, f(g(x)) = h(x)}^{(M)}$, ki je realizirana kot
  zožek
  \begin{equation*}
    \begin{tikzcd}[column sep=normal]
      \set{x \,\middle|\, f(g(x)) = h(x)}^{(M)} \ar[r, hook, "e"] &
      X^{(M)} \ar[r, "x^{(M)}"] & X^{(M)} \ar[r, "f^{(M)}"] \ar[rr,
      bend right, "h^{(M)}"] & Y^{(M)} \ar[r, "g^{(M)}"] & Z^{(M)}
    \end{tikzcd}
  \end{equation*}
  Po definiciji ta formula velja, ko je
  $\set{x \,\middle|\, \top}^{(M)} \leq \set{x \,\middle|\, f(g(x)) =
    h(x)}^{(M)}$ kot podobjekta $X^{(M)}$.  Ker pa je
  $\set{x \,\middle|\, \top}^{(M)} = X^{(M)}$, to velja natanko
  takrat, ko sta $f^{(M)} \circ g^{(M)}$ in $h^{(M)}$ isti morfizem v
  $\cat{C}$.
\end{primer}
Z modelom teorije v rokah se lahko vprašamo, ali le-ta lepo sodeluje s
pravili sklepanja, ki smo jih definirali za regularno logiko.  Bolj
natančno, ali so tako--definirana pravila sklepanja veljavna glede na
modele te teorije.  Za to bomo najprej potrebovali dve tehnični lemi o
">slepih spremenljivkah"< in substituciji.
\begin{lema}\label{lema:slepe-spremenljivke}
  Naj bo $\varphi$ formula s prostimi spremenljivkami izmed
  $\bar{z} = (z_1, \ldots, z_n)$.  Naj bo
  $\pi : \bar{Z}^{(M)} \times W^{(M)} \to \interp{\bar{Z}}$
  projekcija. Potem je
  $$\set{(\bar{z}, w) \,\middle|\, \varphi}^{(M)} = \pi^{-1} \set{\bar{z} \,\middle|\, \varphi}^{(M)}.$$
\end{lema}
\begin{dokaz}
  Z indukcijo po strukturi $\varphi$.  Najprej pokažimo, da za terme
  $t$ sorte~$Y$ velja
  $$t(\bar{z}, w)^{(M)} = t(\bar{z})^{(M)} \circ \pi : \interp{\bar{Z}} \times W^{(M)} \to \interp{Y}.$$
  \begin{itemize}
  \item Če je $t = y$ spremenljivka, potem je $x(\bar{z},w)^{(M)}$
    definirana kot projekcija
    $\interp{\bar{Z}} \times \interp{W} \to Y^{(M)}$, kar lahko
    izrazimo kot kompozitum dveh projekcij
      $$\interp{\bar{Z}} \times \interp{W} \to \bar{Z}^{(M)} \xrightarrow{\pi} Y^{(M)}.$$

    \item Za konstanto $c$ dobimo komutativni diagram
      \begin{equation*}
        \begin{tikzcd}[column sep=tiny]
          \bar{Z}^{(M)} \times \interp{W} \ar[ddr, bend right] \ar[dr]
          \ar[rr, "\pi"] & &
          \bar{Z}^{(M)} \ar[dl] \ar[dl] \ar[ddl, bend left] \\
          & 1 \ar[d, "c^{(M)}"'] & \\
          & Y^{(M)} &
        \end{tikzcd}
      \end{equation*}
    
    \item Za funkcijski simbol
      $f : \interp{X_1} \times \ldots \interp{X_n} \to \interp{Y}$ in
      terme $t_i$ sort~$X_i$, po indukcijski predpostavki, za vsak
      $i = 1, \ldots, n$ velja
      \[t_i(\bar{z},w)^{(M)} = t_i(\bar{z})^{(M)} \circ \pi :
        \interp{\bar{Z}} \to \interp{X_i}.\] Torej dobimo
      \begin{align*}
        \fprod{t_1(\bar{z}, w)^{(M)}, \ldots, t_n(\bar{z}, w)^{(M)}} &= \fprod{t_1(\bar{z})^{(M)} \circ \pi, \ldots, t_n(\bar{z})^{(M)} \circ \pi} \\
                                                                     &= \fprod{t_1(\bar{z})^{(M)}, \ldots, t_n(\bar{z})^{(M)}} \circ \pi.
      \end{align*}
      To nam, skupaj z definicijo interpretacije funkcijskega simbola,
      da iskano enakost.

    \end{itemize}
    Sedaj se lotimo trditve leme, po strukturi formule $\varphi$:
    \begin{itemize}
    \item Če je $\varphi \equiv \top$, je
      $\set{(\bar{z}, w) \,\middle|\, \top}^{(M)} = \interp{\bar{Z}}
      \times \interp{W}$ in zaključek trivialno sledi.
      
    \item Če je $\varphi \equiv (t_1 = t_2)$, za terma $t_1, t_2$, potem
      zaključek sledi, ker zožki komutirajo s produkti.
      
    \item Če je $\varphi \equiv R(\bar{t})$, za relacijski simbol
      $R : \bar{X}$, želimo pokazati, da je v diagramu
      \begin{equation*}
        \begin{tikzcd}[column sep=normal]
          \set{(\bar{z},w) \,\middle|\, R(\bar{t})}^{(M)} \ar[d]
          \ar[r, tail] &
          \interp{\bar{Z}} \times \interp{W} \ar[d, "\pi"] \\
          \set{\bar{z} \,\middle|\, R(\bar{t})}^{(M)} \ar[d] \ar[r,
          tail] &
          \interp{\bar{Z}} \ar[d, "\fprod{\interp{t_1}, \ldots, \interp{t_n}}"] \\
          R \ar[r, tail] & \interp{\bar{X}}
        \end{tikzcd}
      \end{equation*}
      zgornji kvadrat povlek. Po definiciji $\{ (\bar{z},w) \mid R(\bar{t}) \}^{(M)}$
      in $\{\bar{z} \mid R(\bar{t})\}^{(M)}$ sta zunanji in spodnji kvadrat povleka.
      Iz splošne teorije kategorij nato sledi, da je tudi zgornji kvadrat povlek.
    \item Za formulo $\varphi \land \psi$ želimo pokazati, da imamo povlek
      \begin{equation*}
        \begin{tikzcd}
          \setb{(\bar{z}, w)}{\varphi \land \psi}^{(M)} \ar[d] \ar[r] & \bar{Z}^{(M)} \times \interp{W} \ar[d, "\pi"] \\
          \setb{\bar{z}}{\varphi \land \psi}^{(M)} \ar[r] & \interp{\bar{Z}}
  \end{tikzcd}
\end{equation*}
To razberemo iz diagrama
      \begin{equation*}
        \begin{tikzcd}[row sep=normal, column sep=0.2em]
          \setb{(\bar{z}, w)}{\varphi \land \psi}^{(M)} \ar[dd]
          \ar[dr] \ar[rr] & &
          \setb{(\bar{z}, w)}{\psi}^{(M)} \ar[dd] \ar[dr] \\
          & \setb{\bar{z}}{\varphi \land \psi}^{(M)} \ar[rr, crossing
          over] & &
          \setb{\bar{z}}{\psi}^{(M)} \ar[dd] \\
          \setb{(\bar{z}, w)}{\varphi}^{(M)} \ar[dr] \ar[rr] & &
          \bar{Z}^{(M)} \times \interp{W} \ar[dr, "\pi"] \\
          & \setb{\bar{z}}{\varphi}^{(M)} \ar[rr] \ar[from=uu,
          crossing over] && \interp{\bar{Z}}
        \end{tikzcd}
      \end{equation*}
      kjer sta sprednje in zadnje lice povleka po definiciji interpretacije konjunkcije, spodnje in desno
      lice sta povleka po indukcijski predpostavki

    \item Za formulo $\varphi \equiv \exists x \psi$ bomo zaradi
      jasnosti poimenovali naslednje projekcije
      \begin{equation*}
        \begin{tikzcd}[column sep=small]
          \interp{X} \times \interp{W} \times \interp{\bar{Z}} \ar[d,
          "\pi"'] \ar[r, "q"] &
          \interp{W} \times \interp{\bar{Z}} \ar[d, "\tilde{\pi}"] \\
          \interp{X} \times \interp{\bar{Z}} \ar[r, "p"'] &
          \interp{\bar{Z}}
        \end{tikzcd}
      \end{equation*}
      To sestavimo v diagram povlekov {\scriptsize
        \begin{equation*}
          \begin{tikzcd}[column sep=tiny]
            & \setb{(x,w,\bar{z})}{\psi}^{(M)} \ar[dl, equal] \ar[d] \ar[r] & \exists_q \setb{(w,\bar{z})}{\psi}^{(M)} \ar[d] \ar[dr, equal] & \\
            \pi^{-1}\setb{(x,\bar{z})}{\psi}^{(M)} \ar[d] \ar[r] & X^{(M)} \times W^{(M)} \times \bar{Z}^{(M)} \ar[d, "\pi"'] \ar[r, "q"] & W^{(M)} \times \bar{Z}^{(M)} \ar[d, "\tilde{\pi}"] & \tilde{\pi}^{-1} \setb{\bar{z}}{\exists x \psi}^{(M)} \ar[l] \ar[d] \\
            \setb{(x,\bar{z})}{\psi}^{(M)} \ar[dr, equal] \ar[r] & X^{(M)} \times \bar{Z}^{(M)} \ar[r, "p"'] & \bar{Z}^{(M)}  & \setb{\bar{z}}{\exists x \psi}^{(M)} \ar[l] \ar[dl, equal] \\
            & \setb{(x,\bar{z})}{\psi}^{(M)} \ar[u] \ar[r] & \exists_p
            \setb{(x,\bar{z})}{\psi}^{(M)} \ar[u] &
          \end{tikzcd}
        \end{equation*}
      } od koder lahko potem preberemo
      \begin{align*}
        \setb{(w, \bar{z})}{\exists x \psi}^{(M)} &= \exists_q \setb{(x,w,\bar{z})}{\psi}^{(M)} &\text{(po definiciji)} \\
                                                  &= \exists_q \left( \pi^{-1} \setb{(x, \bar{z})}{\psi}^{(M)}\right) &\text{(po indukciji)} \\
                                                  &= \tilde{\pi}^{-1} \left( \exists_p \setb{(x,\bar{z})}{\psi}^{(M)} \right) &\text{(po lemi \ref{lema:zamenjava-povleka-in-slike})} \\
                                                  &= \tilde{\pi}^{-1} \setb{\bar{z}}{\exists x \psi}^{(M)} &\text{(po definiciji)}
      \end{align*}
    \end{itemize}
  \end{dokaz}
  Kot posledico te leme, lahko vedno pišemo $t^{(M)}$ namesto
  $t(\bar{z})^{(M)}$, saj so tej morfizmi enolično določeni, če
  poznamo morfizem za primer $\bar{z} = \operatorname{FV}(\varphi)$.
  Sledi rezultat, ki nam pove kako se obnaša interpretacija pri
  substituciji spremenljivk. Pred tem pa še tehnična lema.
  \begin{lema}\label{lema:zožek-dodatnega-faktorja}
    Naj bo $\cat{C}$ kategorija s končnimi limitami in $e : E \to X$
    zožek morfizmov $p_1, p_2 : X \to Y$. Potem za poljuben objekt $Z$
    dobimo diagram
    \begin{equation*}
      \begin{tikzcd}
        E' \ar[d] \ar[r, "e'"] & X \times Z \ar[d, "\pi_1"] \ar[r, shift left, "p_1 \times \mathrm{id}_Z"] \ar[r, shift right, "p_2 \times \mathrm{id}_Z"'] & Y \times Z \ar[d] \\
        E \ar[r, "e"] & X \ar[r, shift left, "p_1"] \ar[r, shift right, "p_2"'] & Y
      \end{tikzcd}
    \end{equation*}
    v katerem je $\pi_1$ projekcija in je levi kvadrat povlek.
  \end{lema}
  \begin{dokaz}
    Naj bo $E' \xrightarrow{e'} X \times Z$ zožek morfizmov
    $p_1 \times \mathrm{id}_Z$ in $p_2 \times \mathrm{id}_Z$. Potem
    lahko zapišemo $e' = \fprod{e'_1, e'_2}$, kjer sta $e_1'$ in
    $e_2'$ enolična morfizma, za katera komutira diagram
    \begin{equation*}
      \begin{tikzcd}[row sep = normal]
        & X \\
        E' \ar[ur, "e_1'"] \ar[r, "e'"] \ar[dr, "e_2'"'] & X \times Z
        \ar[u, "\pi_1"'] \ar[d, "\pi_2"] \\
        & Z
      \end{tikzcd}
    \end{equation*}
    Če označimo projekcijo iz $Y \times Z$ na $Y$ kot
    $\rho_1$ potem, ker je $e'$ zožek $p_1 \times \mathrm{id}_Z$ in
    $p_2 \times \mathrm{id}_Z$, velja
    \[ \rho_1 \circ( p_1 \times \mathrm{id}_Z) \circ e' = \rho_1 \circ
     ( p_2 \times \mathrm{id}_Z) \circ e'.\] To pa pomeni
    $p_1 \circ \pi_1 \circ e' = p_2 \circ \pi_1 \circ e'$, iz česar
    sledi $p_1 \circ e_1' = p_2 \circ e_1'$. Tako dobimo enoličen
    morfizem $u : E' \to E$, za katerega velja $e_1' = e \circ
    u$. Povlek~$e$ po $\pi_1$ prikažemo v diagramu
    \begin{equation*}
      \begin{tikzcd}
        \pi_1^{-1}E \ar[d, "q"'] \ar[r, "\pi_1^{-1}e"] & X \times Z \ar[d, "\pi_1"] \\
        E \ar[r, tail, "e"'] & X
      \end{tikzcd}
    \end{equation*}
    od koder lahko preberemo
    \[ p_1 \circ \pi_1 \circ \pi_1^{-1}e = p_1 \circ e \circ q = p_2
      \circ e \circ q = p_2 \circ \pi_1 \circ \pi_1^{-1}e.\]
    Od tod sledi $(p_1 \times \mathrm{id}_Z) \circ \pi_1^{-1}e = (p_2 \times \mathrm{id}_Z) \circ \pi_1^{-1}e$.
    Iz univerzalne lastnosti $E'$ nato dobimo
    enoličen morfizem $\theta : \pi_1^{-1}E \to E'$, da velja
    $\pi_1^{-1}e = e' \circ \theta$.  Morfizem v drugo smer dobimo z
    univerzalno lastnostjo povleka $\pi_1^{-1}E$, ker imamo morfizma
    $u : E' \to E$ in $e' : E' \to X \times Z$, za katera velja
    $e \circ u = e_1' = \pi_1 \circ e'$. To nam da enoličen morfizem
    $\eta : E' \to \pi_1^{-1}E$, za katerega ustrezen diagram
    komutira.  Sledi, da sta $E'$ in $\pi_1^{-1}E$ izomorfna.
  \end{dokaz}
  \begin{lema}\label{lema:substitucija}
    Naj bo 
    
    Naj bo $\psi(y_1, \ldots, y_n)$ formula s prostimi spremenljivkami 
    $y_1, \ldots, y_n$ sort $Y_1, \ldots, Y_n$ in $t_1, \ldots, t_n$ termi s prostimi spremenljivkami
    $x_1, \ldots, x_m$ sort $X_1, \ldots, X_m$, kjer se sorti $y_i$ in $t_i$ ujemata in po substituciji
    $t_i$ za $y_i$ nobena prosta spremenljivka v $t_i$ ne postane vezana v $\psi(y_i/t_i)$.
    Potem je
    $$\set{\bar{x} \,\middle|\, \psi(\bar{y}/\bar{t})}^{(M)} =
  \langle \bar{t}^{(M)}, \mathrm{id}_{\interp{\bar{Z}}}\rangle^{-1} \set{(\bar{y},
    \bar{x}) \,\middle|\, \psi}^{(M)}.$$
\end{lema}
\begin{dokaz}
  Z indukcijo na strukturo $\psi$.  Najprej opazimo, da podobno kot
  pri prejšnji lemi z indukcijo lahko dokažemo, da za vse terme $t$
  sorte $X$ s prostimi spremenljivkami izmed $(y, \bar{z})$ velja
  $$t(b/y, \bar{z})^{(M)} = t^{(M)} \circ \fprod{b^{(M)}, \mathrm{id}_{\bar{Z}}^{(M)}} : \bar{Z}^{(M)} \to Y^{(M)} \times \bar{Z}^{(M)}$$
  \begin{itemize}
  \item Če je $t \equiv x$ spremenljivka, potem to sledi iz
    komutiranja projekcij v diagramu
    \begin{equation*}
      \begin{tikzcd}
        \bar{Z}^{(M)} \ar[d, "\fprod{b^{(M)}, \mathrm{id}_{\bar{Z}}^{(M)}}"'] \ar[dr, "\pi'"] & \\
        Y^{(M)} \times \bar{Z}^{(M)} \ar[r, "\pi"] & X^{(M)}
      \end{tikzcd}
    \end{equation*}
    
  \item Za konstanto ${c : X}$, to lahko razberemo iz diagrama
    \begin{equation*}
      \begin{tikzcd}
        \bar{Z}^{(M)} \ar[d, "\fprod{b^{(M)}, \mathrm{id}_{\bar{Z}}^{(M)}}"'] \ar[dr] & & \\
        Y^{(M)} \times \bar{Z}^{(M)} \ar[r] & 1 \ar[r, "c^{(M)}"] &
        X^{(M)}
      \end{tikzcd}
    \end{equation*}

  \item Za funkcijski simbol $f : \bar{W} \to X$ in terme $t_i$ sort
    $W_i$ to sledi iz definicije interpretacije in indukcijske
    predpostavke.  (ta diagram tukaj bi bil malo nabit z oznakami in
    pomoje nebi dodal nič k razumevanju?)
  \end{itemize}

  S tem dejstvom v rokavu, se obrnimo na indukcijo po strukturi
  $\psi$.  Če je $\psi \equiv (t_1 = t_2)$, zaključek sledi iz leme~\ref{lema:zožek-dodatnega-faktorja}.
  Primer $\psi \equiv \top$ je
  trivialen.  Za primer $\psi \equiv R(\bar{t})$ to sledi iz lastnosti
  zaporednih povlekov (ta primer bi mogoče lahko dejansko razpisali?).
  Primer $\psi \equiv \psi_1 \land \psi_2$ sledi po indukciji.  Edini
  netrivialni primer je $\psi \equiv \exists x \varphi$.  V
  tem primeru po predpostavki velja $x \notin \operatorname{FV}(b)$.
  Pokazati želimo, da imamo povlek
  \begin{equation*}
    \begin{tikzcd}
      \setb{\bar{z}}{\exists x \varphi(b)} \ar[d] \ar[r] & \interp{\bar{Z}} \ar[d, "\fprod{b^{(M)}, \mathrm{id}}"] \\
      \setb{(y,\bar{z})}{\psi} \ar[r] & Y^{(M)} \times \bar{Z}^{(M)}
    \end{tikzcd}
  \end{equation*}
  Za dokaz bomo uporabili naslednji diagram, kjer bomo zaradi jasnosti
  privzeli, da je $\bar{z}$ prazen seznam.  V nasprotnem primeru je
  potrebno dodati in poimenovati vse dodatne projekcije, ideja
  argumenta pa ostane enaka. Diagram je sledeč
  \begin{equation*}
    \begin{tikzcd}[column sep=tiny]
      \setb{x}{\varphi (x, b)}^{(M)} \ar[dd] \ar[dr, tail] \ar[rr, two heads] & & ( b^{(M)} )^{-1} \setb{y}{\exists x \varphi}^{(M)} \ar[dd] \ar[dr, tail] & \\
      & X^{(M)}  \ar[rr, crossing over] & & 1 \ar[dd, "b^{(M)}"] \\
      \setb{(x,y)}{\varphi}^{(M)} \ar[dr, tail] \ar[rr, two heads] & & \setb{y}{\exists x \varphi}^{(M)} \ar[dr, tail] & \\
      & X^{(M)} \times Y^{(M)} \ar[from=uu, pos=0.3, crossing over,
      "\fprod{\mathrm{id}, b^{(M)}}"] \ar[rr, "\pi"'] & & Y^{(M)}
    \end{tikzcd}
  \end{equation*}
  Tu je sprednja stran povlek. Leva stran je povlek po
  indukcijski predpostavki.  Spodnja stran je faktorizacija na
  regularen epimorfizem in monomorfizem, po definiciji kvantifikatorja
  obstoja.  Desna stran je potem samo povlek po morfizmu $b^{(M)}$.
  Od tod sledi, da je zadnja stran faktorizacija po morfizmu $b^{(M)}$, kar
  pomeni, da je kompozitum
  $$\setb{x}{\varphi (x, b)}^{(M)} \twoheadrightarrow ( b^{(M)} )^{-1} \setb{y}{\exists x \varphi}^{(M)} \rightarrowtail 1$$
  faktorizacija morfizma $\setb{x}{\varphi (x, b)}^{(M)} \to 1$
  Iskana enakost
  $$\setb{\cdot}{\exists x \varphi(x,b)}^{(M)} = ( b^{(M)} )^{-1} \setb{y}{\exists x \varphi}^{(M)}$$
  nato sledi iz enoličnosti slik.
\end{dokaz}
\subsection{Veljavnost}
Pripravili smo si vse potrebno, da lahko govorimo o veljavnosti pravil
sklepanja, ki smo jih definirali za regularno logiko, glede na
interpretacije v regularnih kategorijah.
\begin{izrek}[Veljavnost]
  Naj bo $\T$ regularna teorija formulirana v
  jeziku~$\mathcal{L}(\Sigma)$ in naj bo $M$ model te teorije v
  regularni kategoriji $\cat{C}$.  Če v teoriji $\T$ obstaja izpeljava
  $\bar{x}:\bar{X} \mid \varphi \vdash_{\T} \psi$, potem je
  $M \models \varphi \implies \psi$, oziroma je
  \[\set{\bar{x} \,\middle|\, \varphi}^{(M)} \leq \set{\bar{x}
      \,\middle|\, \psi}^{(M)},\] kot podobjekta $\interp{X}$.
\end{izrek}
\begin{dokaz}
  Z indukcijo po izpeljavi $\bar{x}:\bar{X} \mid \varphi \vdash_{\T} \psi$
  \begin{itemize}
  \item Če je $\varphi \implies \psi$ aksiom $\T$, to sledi iz
    definicije modela.
  \item Za pravilo \ref{pravilo:refl} to velja trivialno.
  \item Za pravilo \ref{pravilo:tranz} to velja, ker je
    $\Sub(\bar{X}^{(M)})$ delno urejena množica, torej je v posebnem
    relacija podobjekt tranzitivna.
  \item Za pravilo \ref{pravilo:slepa-spr} to velja po lemi
    \ref{lema:slepe-spremenljivke}.
  \item Za pravilo \ref{pravilo:subst} to velja po lemi
    \ref{lema:substitucija}.
  \item Za pravilo \ref{pravilo:resnica} to velja, ker
    $\{\bar{x} \mid \varphi\}^{(M)} \leq \bar{X}^{(M)} = \{\bar{x} \mid
    \top\}^{(M)}$, po definiciji.
  \item Za pravilo \ref{pravilo:konj} to velja, ker smo interpretacijo
    konjunkcije definirali kot
    $\{ \bar{x} \mid \varphi \land \psi\}^{(M)} = \{ \bar{x} \mid \varphi\}^{(M)}
    \land \{ \bar{x} \mid \psi\}^{(M)}$ in je $\land$ v $\Sub(X)$ infimum.
  \item Podobno za \ref{pravilo:eksist} to velja, ker po definiciji
    \[ \{ \bar{x} \mid \exists (y:Y) \psi(y)\}^{(M)} =
      \exists_{\pi}\{(\bar{y}, \bar{x}) \mid \psi(y)\}^{(M)} \leq
      \{\bar{x} \mid \varphi\}^{(M)}\] če in samo če,
    $\{(\bar{y},\bar{x}) \mid \psi(y)\}^{(M)} \leq
    Y^{(M)} \times \{\bar{x} \mid \varphi\}^{(M)} =
    \{(y,\bar{x}) \mid \varphi\}^{(M)}$, kjer je
    $\pi$ projekcija $Y^{(M)} \times \bar{X}^{(M)} \to \bar{X}^{(M)}$.
    Ponovno uporabimo lemo \ref{lema:slepe-spremenljivke}.
  \item Za pravilo \ref{pravilo:enakost-refl} to velja, ker je
    $\{x \mid x = x\}$ definiran kot zožek dveh identitet
    $X^{(M)} \to X^{(M)}$, ki je kar enak $X^{(M)}$.
  \item Za pravilo \ref{pravilo:enakost-sim} to velja, ker je zožek
    $\pi_1, \pi_2 : X^{(M)} \times X^{(M)} \to X^{(M)}$ enak zožku
    $\pi_2, \pi_1$.
  \item Pri pravilu \ref{pravilo:enakost-tranz} je interpretacija
    $x_1 = x_2 \land x_2 = x_3$ enaka povleku
    \begin{equation*}
      \begin{tikzcd}
        P \ar[d, "p"'] \ar[r, "q"] & E_{2,3} \ar[d, "i_{2,3}"] \\
        E_{1,2} \ar[r, "i_{1,2}"'] & X^{(M)} \times X^{(M)} \times
        X^{(M)} \ar[d, shift left, "\pi_3"] \ar[d, shift right,
        "\pi_2"'] \ar[r, shift left, "\pi_1"]
        \ar[r, shift right, "\pi_2"'] & X^{(M)} \\
        & X^{(M)}
      \end{tikzcd}
    \end{equation*}
    Ker je
    \[ \pi_1 \circ i_{1,2} \circ p = \pi_2 \circ i_{1,2} \circ p =
      \pi_2 \circ i_{2,3} \circ q = \pi_3i_{2,3}q,\]
    lahko morfizem
    $P \to X^{(M)} \times X^{(M)} \times X^{(M)}$ faktoriziramo skozi
    zožek projekcij $\pi_1$ in $\pi_3$, ki je ravno interpretacija
    $\{x_1, x_2, x_3 \mid x_1 = x_3\}$.
  \item Za pravilo \ref{pravilo:enakost-subst-fun} lahko iz diagrama
    \begin{equation*}
      \begin{tikzcd}
        \{\bar{x}_1, \bar{x}_2 \mid f(\bar{x}_1) =
        f(\bar{x}_2)\}^{(M)} \ar[r, hook] & \bar{X}^{(M)} \times
        \bar{X}^{(M)} \ar[r, shift left, "f \pi_1"] \ar[r, shift
        right, "f \pi_2"']
        \ar[d, shift left, "\pi_1"] \ar[d, shift right, "\pi_2"'] & Y \\
        \bar{X}^{(M)} \ar[u, dashed] \ar[ur, hook] \ar[r, "
        \mathrm{id}"'] & \bar{X}^{(M)} \ar[ur, bend right, "f"']
      \end{tikzcd}
    \end{equation*}
    razberemo, da diagonala
    $\bar{X}^{(M)} \to \bar{X}^{(M)} \times \bar{X}^{(M)}$ zoži
    $f \pi_1$ in $f \pi_2$.
  \item Nazadnje, pri pravilu \ref{pravilo:enakost-subst-rel}, za
    interpretacijo $\bar{x}_1 = \bar{x}_2 \land R(\bar{x}_1)$ dobimo povlek
    \begin{equation*}
      \begin{tikzcd}[column sep=5em]
        P \ar[d, hook, "p"'] \ar[r, hook, "q"] & \bar{X}^{(M)} \ar[d, hook, "\Delta"] \\
        R^{(M)} \times \bar{X}^{(M)} \ar[d] \ar[r, hook, "r^{(M)}
        \times \mathrm{id}"] &
        \bar{X}^{(M)} \times \bar{X}^{(M)} \ar[d, "\pi_1"] \\
        R^{(M)} \ar[r, hook, "r^{(M)}"'] & \bar{X}^{(M)}
      \end{tikzcd}
    \end{equation*}
    Sledi, da $P \to \bar{X}^{(M)} \times \bar{X}^{(M)}$ lahko
    faktoriziramo skozi
    \[R^{(M)} \hookrightarrow \bar{X}^{(M)} \hookrightarrow
      \bar{X}^{(M)} \times \bar{X}^{(M)},\] v posebnem skozi
    $\bar{X}^{(M)} \times R^{(M)}$, kar je ravno interpretacija
    $R(\bar{x}_2)$.\qedhere
  \end{itemize}
\end{dokaz}
%
\section{Interna logika regularne kategorije}
Naj bo $\cat{C}$ regularna kategorija.  Priredili ji bomo signaturo
$\Sigma_\cat{C}$ in regularen jezik na naslednji način: Signatura
$\Sigma_\cat{C}$ ima za sorte objekte kategorije~$\cat{C}$.
Fiksiramo končni objekt in za vsak končen seznam objektov fiksiramo
objekt, ki predstavlja njihov produkt. Potem
\begin{itemize}
\item Za vsak morfizem $c : 1 \to X$ v jezik dodamo konstanto $c:X$.
\item Za vsak morfizem $f : X_1 \times \cdots \times X_n \to Y$ v jezik dodamo funkcijski
  simbol $f_{\bar{X},Y} : (X_1, \ldots, X_n) \to Y$, kjer si zapomnemo tudi seznam objektov
  s katerimi smo tvorili produkt, kajti iz produkta sicer ne moremo nazaj dobiti objektov.
\item Za vsak podobjekt $R \hookrightarrow X_1 \times \cdots \times X_n$ v jezik dodamo
  relacijski simbol $R_{\bar{X}} : (X_1, \ldots, X_n)$.
\end{itemize}
Jezik $\mathcal{L}(\Sigma_\cat{C})$ ima sedaj kanonično interpretacijo
$I^{\cat{C}}$ v $\cat{C}$.  Teorijo $\T_\cat{C}$ definiramo kot
teorijo te interpretacije, torej množico vseh sekvent
$\Gamma \mid \varphi \vdash \psi$ v jeziku~$\mathcal{L}(\Sigma_\cat{C})$, ki so
resnične pod interpretacijo $I^\cat{C}$. Z zlorabo notacije bomo
namesto $I^\cat{C}$ pisali kar $\cat{C}$, hkrati pa tudi ne bomo
ločevali med npr.\ funkcijskim simbolom $f$ v tem jeziku in njegovo
interpretacijo $f^{(I^\cat{C})}$ v $\cat{C}$.  Sedaj lahko s pomočjo
internega jezika regularne teorije opišemo kategorične pojme.
\begin{lema}\label{lema:morfizmi-v-interni-logiki}
  Naj bo $\cat{C}$ regularna kategorija. Potem
  \begin{enumerate}[label=(\roman*)]
  \item Naj bodo $X \xrightarrow{f} Y \xrightarrow{g} Z$ in
    $X \xrightarrow{h} Z$ morfizmi v $\cat{C}$.  Potem je $h$
    kompozitum $f$ in $g$ natanko takrat, ko velja
    ${\cat{C} \models (x:X \mid  h(x) = g(f(x)))}$.
  \item Morfizem $m : X \to Y$ je monomorfizem natanko takrat, ko
    velja
    \[\cat{C} \models (x_1:X, x_2:X \mid m(x_1) = m(x_2) \implies x_1 = x_2). \]
  \item Morfizem $f : X \to Y$ je regularen epimorfizem natanko, ko
    velja
    \[ \cat{C} \models (y:Y \mid \exists (x:X) f(x) = y). \]
  \end{enumerate}
\end{lema}
\begin{dokaz}
  \begin{enumerate}[label=(\roman*)]
  \item Smo videli v primeru~\ref{primer:kompozitum}.

  \item Po definiciji interne logike regularne
    kategorije dobimo diagram
    \begin{equation*}
      \begin{tikzcd}
        \setb{x_1,x_2}{x_1 = x_2}^{(\cat{C})} \ar[r, tail, "e"] & X \times X \ar[r, shift left, "x_1"] \ar[r, shift right, "x_2"'] & X \ar[r, tail, "m"] & Y \\
        \setb{x_1,x_2}{m(x_1) = m(x_2)}^{(\cat{C})} \ar[ur, tail,
        "f"'] \ar[u, dashed, "u"]
      \end{tikzcd}
    \end{equation*}
    v katerem je $e$ zožek morfizmov $x_1$ in $x_2$ in $f$ zožek
    morfizmov $m x_1$ in $m x_2$.  Ker je $m$ monomorfizem iz
    $m x_1 f = m x_2 f$ sledi, da obstaja enoličen morfizem $u$ tako,
    da velja $f = e \circ u$. Torej je res
      $$\setb{x_1,x_2}{m(x_1) = m(x_2)}^{(\cat{C})} \leq \setb{x_1,x_2}{x_1 = x_2}^{(\cat{C})}$$
      kot podobjekta $X \times X$, kar po definiciji interne
      kategorije pomeni ravno
      \[\cat{C} \models x_1:X, x_2:X \mid  m(x_1) = m(x_2) \implies x_1 = x_2.\]
      Obratno privzamemo, da obstaja tak $u$, za katerega velja
      $f = e \circ u$ in želimo pokazati, da je $m$ monomorfizem.  V
      ta namen, denimo, da imamo morfizma $t_1, t_2 : T \to X$, da je
      $m t_1 = m t_2$.  Potem lahko morfizem $\fprod{t_1, t_2}$
      faktoriziramo preko morfizma $f$ in dobimo diagram
      \begin{equation*}
        \begin{tikzcd}
          \setb{x_1,x_2}{x_1 = x_2}^{(\cat{C})} \ar[r, tail, "e"] & X \times X \ar[r, shift left, "x_1"] \ar[r, shift right, "x_2"'] & X \ar[r, tail, "m"] & Y \\
          \setb{x_1,x_2}{m(x_1) = m(x_2)}^{(\cat{C})} \ar[ur, tail,
          "f"'] \ar[u, "u"] & T \ar[u, "\fprod{t_1, t_2}"'] \ar[l,
          dashed, "v"]
        \end{tikzcd}
      \end{equation*}
      tako, da velja
      $\fprod{t_1, t_2} = f \circ v = e \circ u \circ v$. Kar pa
      pomeni
      $$t_1 = x_1 \circ \fprod{t_1, t_2} = x_1 \circ e \circ u \circ v = x_2 \circ e \circ u \circ v = x_2 \circ \fprod{t_1, t_2} = t_2,$$
      torej je $m$ res monomorfizem.

    \item V interni logiki $\cat{C}$ lahko $y:Y \mid \exists (x:X) f(x) = y$
      izrazimo v diagramu
      \begin{equation*}
        \begin{tikzcd}
          \setb{x,y}{f(x) = y}^{(\cat{C})} \ar[r, tail] \ar[d, two heads] & X \times Y \ar[r, shift left, "f(x)"] \ar[r, shift right, "y"'] \ar[d, "\pi_2"] & Y \\
          \setb{y}{\exists (x:X) f(x) = y}^{(\cat{C})} \ar[r, tail, "m"] &
          Y
        \end{tikzcd}
      \end{equation*}
      Poleg tega, pa po lemi \ref{lema:graf-kot-zozek} vemo, da lahko
      zožek $f \circ \pi_1$ in $\pi_2$ izrazimo kot $\graf(f)$.  Z
      uporabo leme \ref{lema:lastnosti-regularnih-epimorfizmov} lahko
      nato iz diagrama
      \begin{equation*}
        \begin{tikzcd}
          X \ar[drr, "f"'] \ar[r, two heads] & \operatorname{graph}(f) \ar[dr, "g"] \ar[r, tail] & X \times Y \ar[d, "\pi_2"] \\
          & & Y
        \end{tikzcd}
      \end{equation*}
      razberemo, da je $f$ regularen epimorfizem natanko takrat, ko je
      $g$ regularen epimorfizem, kar velja natanko takrat, ko je $m$
      izomorfizem, oziroma, ko
      $\cat{C} \models y:Y \mid \top \implies \exists (x:X) f(x) = y$.\qedhere
    \end{enumerate}
  \end{dokaz}
  Naslednja lema karakterizira končne limite v regularni kategoriji, v
  internem jeziku regularne kategorije.
  \begin{lema}\label{lema:limite-v-interni-logiki}
    Naj bo $\cat{C}$ regularna kategorija. Potem
    \begin{enumerate}[label=(\roman*)]
    \item Objekt $X$ v $\cat{C}$ je končen natanko takrat, ko
      $$\cat{C} \models x_1:X,x_2:X \mid x_1 = x_2 \quad \text{in} \quad
      {\cat{C} \models \exists (x:X) (x=x)}.$$

    \item Morfizma $f : Z \to X$ in $g : Z \to Y$ tvorita $Z$ kot
      produkt $X$ in $Y$ natanko takrat, ko velja
      \[\cat{C} \models (z_1:Z,z_2:Z \mid f(z_1) = f(z_2) \wedge
        g(z_1) = g(z_2) \implies z_1 = z_2)\] in
      \[\cat{C} \models (x:X,y:Y \mid \exists (z:Z)(f(z) = x \wedge
        g(z) = y)).\]

    \item Diagram \begin{tikzcd}[column sep=normal] Z \ar[r, "e"] & X
        \ar[r, shift left, "f"] \ar[r, shift right, "g"'] &
        Y \end{tikzcd} je zožek, natanko tedaj, ko velja
      \begin{align*}
        C \models& x:X \mid f(e(x)) = g(e(x)) \\
        C \models& x:X, y:Y \mid e(x) = e(y) \implies x = y \\
        C \models& x:X \mid f(x) = g(x) \implies \exists (z:Z)(e(z) = x).
      \end{align*}
    \end{enumerate}
  \end{lema}
  \begin{dokaz}
    \begin{enumerate}[label=(\roman*)]
    \item Iz diagrama
      \begin{equation*}
        \begin{tikzcd}[row sep=normal]
          & X & \\
          X \ar[ur] \ar[r, "\Delta"] \ar[dr] & X \times X \ar[u, "\pi_1"'] \ar[d, "\pi_2"] \ar[r, "!"] & 1 \\
          & X &
        \end{tikzcd}
      \end{equation*}
      lahko razberemo, da za objekt $X$ velja
      $\cat{C} \models x_1 = x_2$ natanko takrat, kot je diagonalni
      morfizem $\Delta : X \to X \times X$ izomorfizem, kar je
      ekvivalentno temu, da je $X \to 1$ monomorfizem.  Če upoštevamo
      še, da je $\cat{C} \models \exists x (x = x)$ natanko takrat, ko
      je $X \to 1$ regularen epimorfizem, smo pokazali prvi del.
      
    \item Pokažimo, da je $\fprod{f,g}$ izomorfizem, kar pomeni, da
      $Z$ predstavlja produkt $X$ in $Y$. Za to uporabimo dejstvo, da
      je pogoj,
      \begin{equation}
        \label{eq:a}
        \mathbf{C} \models z_1:Z,z_2:Z \mid f(z_{1}) = f(z_{2}) \land g(z_{1}) = g(z_{2}), 
      \end{equation}
      ekvivalenten temu, da je \( \fprod{f,g}\) monomorfizem.  Drugi
      pogoj
      \begin{equation}
        \label{eq:b}
        \mathbf{C} \models x:X,y:Y \mid  \exists (z:Z)(f(z) = x \land g(z) = y),
      \end{equation}
      pa temu, da je \( \fprod{f,g}\)
      regularen epimorfizem.  Iz leme~\ref{lema:lastnosti-regularnih-epimorfizmov}
      nato lahko
      sklepamo, da gre za izomorfizem.  Najprej opazimo, da lahko
      interpretacijo formule
      \[ f(z_{1}) = f(z_{2}) \land g(z_{1}) = g(z_{2})\] izrazimo kot
      zožek
      \begin{equation*}
        \begin{tikzcd}
          E \ar[r, hook, "e"] & Z \times Z \ar[r, shift left,
          "\pi_{1}"] \ar[r, shift right, "\pi_{2}"'] & Z \ar[r, ,
          "\fprod{f, g}"] & X \times Y.
        \end{tikzcd}
      \end{equation*}
      Potem kakor v lemi~\ref{lema:morfizmi-v-interni-logiki}, lahko faktoriziramo $E$ skozi
      diagonalo $\Delta : Z \to Z \times Z$, ki predstavlja
      interpretacijo $z_{1} = z_{2}$, če in samo če, je $\fprod{f, g}$
      monomorfizem.  Da pokažemo drugi del, najprej opazimo, da lahko
      interpretacijo formule iz $\eqref{eq:b}$ predstavimo kot
      \begin{equation}\label{diag:injektivnost-produkta}
        \begin{tikzcd}
          \set{x,z,y \mid f(z) = x \land g(z) = y} \ar[d] \ar[r] &
          X \times \mathrm{graph}(g) \ar[d] \\
          \mathrm{graph}(f) \times Y \ar[r] & X \times Z \times Y
        \end{tikzcd}
      \end{equation}
      Sedaj zgradimo naslednji diagram povlekov, kjer desni kvadrat
      dobimo s povlekom $Z \to X \times Y$ po morfizmu $X \to 1$. Nato
      naredimo povlek dobljenega morfizma
      \[X \times Z \to X \times Z \times Y\] z morfizmom
      $Z \times Y \to X \times Z \times Y$, nakar lahko iz
      \eqref{diag:injektivnost-produkta} izpeljemo, da je
      interpretacija formule iz pogoja $\eqref{eq:a}$ ravno slika morfizma
      \[\fprod{f, \mathrm{id}_{Z}, g} : Z \to X \times Z \times Y\]
      \begin{equation*}
        \begin{tikzcd}[column sep=tiny]
          Z \ar[d, two heads] \ar[r, two heads] & \mathrm{graph}(f)
          \ar[d, two heads] \ar[r, hook] &
          X \times Z \ar[d, two heads] \ar[r] & Z \ar[d, two heads] \\
          \mathrm{graph}(g) \ar[d, hook] \ar[r, two heads] &
          \set{x,z,y \mid f(z) = x \land g(z) = y} \ar[d, hook] \ar[r,
          hook] & X \times \mathrm{graph}(g) \ar[d, hook] \ar[r] &
          \mathrm{graph}(g) \ar[d, hook] \\
          Z \times Y \ar[r, two heads] & \mathrm{graph}(f) \times Y
          \ar[r, hook] & X \times Z \times Y \ar[r] & Z \times Y
        \end{tikzcd}
      \end{equation*}
      Iz leme \ref{lema:morfizmi-v-interni-logiki} sedaj lahko
      sklepamo, da je $\fprod{f,g}$ regularen epimorfizem natanko
      takrat, ko
      $\mathbf{C} \models x:X,y:Y \mid \exists (z:Z) (f(z) = x \land g(z) = y)$.
      
    \item Zožek morfizmov $f$ in $g$ je enak
      $\set{x \mid f(x) = g(x)}$.  Ker velja
      ${f \circ e = g \circ e}$, imamo faktorizacijo
      \begin{equation*}
        \begin{tikzcd}
          \exists_{e}Z \ar[r, hook] & \set{x \mid f(x) = g(x)} \ar[r,
          hook] &
          X \ar[r, shift left, "f"] \ar[r, shift right, "g"'] & Y \\
          Z \ar[u, two heads] \ar[ur, dashed] \ar[urr, "e"']
        \end{tikzcd}
      \end{equation*}
      Potem je $e$ monomorfizem, če in samo če, je
      $Z \to \exists_{e}Z$ izomorfizem in
      \[\mathbf{C} \models x:X \mid f(x) = g(x) \implies (\exists (z:Z) e(z) =
        x),\] če in samo če, je monomorfizem
      $\exists_{e}Z \hookrightarrow \set{x \mid f(x) = g(x)}$ tudi
      regularen epimorfizem.
    \end{enumerate}
  \end{dokaz}%
  Prejšnje leme pokažejo, da lahko vse lastnosti, ki določajo
  regularno kategorijo izrazimo v interni logiki regularne kategorije.
  Ker so regularni funktorji ravno tisti, ki ohranjajo strukturo regularne
  kategorije, je jasno tudi zanje mogoče predpisati tako karakterizacijo.
  To bomo storili v lemi~\ref{lema:regularen-funktor-v-interni-logiki}.
%
  \section{Generični model in polnost regularnih teorij}
  Modeli teorije $\T$ tvorijo kategorijo na naslednji način
  \begin{definicija}
    $\Mod(\T, \cat{C})$ je kategorija, katere objekti so modeli
    teorije $\T$ v regularni kategoriji $\cat{C}$. Morfizmi v
    $\Mod(\T, \cat{C})$ so družine morfizmov
    $$\set{h_X : \interp{X} \to \interp[N]{X}}_{X \in \underline{\mathrm{sort}}_\Sigma}$$
    za modela $M$ in $N$, ki komutirajo z interpretacijami osnovnih operacij in relacij jezika
    $\mathcal{L}(\Sigma)$.
    To pomeni, da za vsak funkcijski simbol $f : \bar{X} \to Y$ komutira diagram
    \begin{equation*}
  \begin{tikzcd}
    \interp{\bar{X}} \ar[d, "h_{X_1} \times \cdots \times h_{X_n}"'] \ar[r, "\interp{f}"] & \interp{Y} \ar[d, "h_Y"]\\
    \interp{\bar{X}} \ar[r, "{\interp[N]{f}}"'] & \interp[N]{Y}
  \end{tikzcd}
\end{equation*}
in za vsak relacijski simbol $R : (\bar{X})$ lahko kompozitum
$$(h_{X_1} \times \cdots \times h_{X_n}) \circ \interp{i} : \interp{R} \to \interp{\bar{X}} \to \interp[N]{\bar{X}}$$
faktoriziramo skozi inkluzijo $\interp[N]{R} \hookrightarrow \interp[N]{X}$.
\end{definicija}
To z indukcijo razširimo na vse terme in formule. Po indukciji
velja, da za vsak term $t(\bar{z})$ sorte $Y$, diagram
\begin{equation*}
  \begin{tikzcd}
    \interp{\bar{Z}} \ar[d, "h_{\bar{Z}}"'] \ar[r, "{t(\bar{z})^{(M)}}"] & \interp{Y} \ar[d, "h_Y"] \\
    \interp[N]{\bar{Z}} \ar[r, "{\interp[N]{t(\bar{z})}}"'] &
    \interp[N]{Y}
  \end{tikzcd}
\end{equation*}
komutira (tu je $h_{\bar{Z}}$ mišljen kot
$h_{Z_1 \times \cdots \times Z_n}$).  Podobno z indukcijo pokažemo, da
za vsako formulo $\varphi(\bar{z})$ kompozitum
$$\interp{\set{\bar{z} \,\middle|\, \varphi}} \hookrightarrow \interp{\bar{Z}} \xrightarrow{h_{\bar{Z}}} \interp[N]{\bar{Z}}$$
lahko faktoriziramo skozi
$\interp[N]{\set{\bar{z} \,\middle|\, \varphi}}$.
\begin{definicija}
  Naj bo $F : \cat{C} \to \cat{D}$ regularen funktor med regularnima
  kategorijama.  Če je $M$ model teorije $\T$ s signaturo $\Sigma$ v $\cat{C}$, potem definiramo interpretacijo
  $F(M)$ jezika, $\mathcal{L}(\Sigma)$ v $\cat{D}$ kot:
  \begin{itemize}
  \item Za sorto $X$, iz $\underline{\mathrm{sort}}_\Sigma$,
    definiramo $\interp[F(M)]{X} = F(\interp{X})$.
  \item Za konstanto $c:X$ definiramo
    $\interp[F(M)]{c} = F(\interp{x})$.
  \item Za funkcijski simbol $f : \bar{X} \to Y$ definiramo
    $\interp[F(M)]{f} = F(\interp{f})$.
  \item Za relacijski simbol $R : \bar{X}$ definiramo
    $\interp[F(M)]{R} = F(\interp{R}) \rightarrowtail \bar{X}$.
  \end{itemize}
\end{definicija}
\begin{lema}\label{lema:regularen-funktor-slika-model-v-model}
  Regularen funktor $F : \cat{C} \to \cat{D}$ inducira funktor
  $$\Mod(\T, \_)=F_T : \Mod(\T,\cat{C}) \to \Mod(\T, \cat{D}),$$
  ki slika model $M$ v model $F(M)$ in morfizem $h : M \to N$ v
  morfizem $F(h) : F(M) \to F(N)$.
\end{lema}
\begin{dokaz}
  Najprej z indukcijo pokažemo, da za vse terme $t$ sorte $X$, katerih
  proste spremenljivke so vsebovanje v $\bar{z} : \bar{Z}$ velja
  \[ t^{(F(M))} = F(t^{(M)}) : \bar{Z}^{(F(M))} \to X^{(F(M))}.\] Nato
  z indukcijo po strukturi formul pokažemo, da za vse regularne
  formule~$\varphi$ velja
  \[ \{ \bar{x} \mid \varphi\}^{(F(M))} = F(\{\bar{x} \mid \varphi
    \}^{(M)}).\] Ker za vsako sekvento $\varphi \Rightarrow \psi$ v
  $\T$ velja
  $\{\bar{x} \mid \varphi\}^{(M)} \leq \{\bar{x} \mid \psi \}^{(M)}$
  in ker~$F$ ohranja urejenost podobjektov sledi, da je $F(M)$ model
  teorije $\T$. Vsak morfizem $h : M \to N$ v $\Mod(\T, \cat{C})$
  inducira morfizem
  \[ F(h) = \{ F(h_X) : X^{(F(M))} \to X^{(F(N))}\}_{X \in
      \mathrm{sort}_{\Sigma}}\] med modeloma $F(M)$ in $F(N)$.
\end{dokaz}
Sedaj lahko izrazimo regularne funktorje z interno
logiko regularne kategorije.
\begin{lema}\label{lema:regularen-funktor-v-interni-logiki}
  Če je $F : \cat{C} \to \cat{D}$ funktor med regularnima kategorijama
  (ki ni nujno regularen), potem dobimo interpretacijo~$\mathfrak{F}$
  funkcijskega dela signature~$\Sigma_\cat{C}$ na sledeč način:
  \begin{itemize}
  \item $\interp[\mathfrak{F}]{X} = F(X)$, za
    $X \in \underline{\mathrm{sort}}_{\Sigma}$.
  \item
    $\interp[\mathfrak{F}]{f} : \interp[\mathfrak{F}]{X} \to
    \interp[\mathfrak{F}]{Y} = F(f : X \to Y)$, za morfizem
    $f: X \to Y$ v $\cat{C}$.
  \end{itemize}
  Funktor $F$ je regularen natanko takrat, ko velja
  $\mathfrak{F} \models \T_\cat{C}$.
\end{lema}
\begin{dokaz}
  Če je $F$ regularen, potem po lemi
  \ref{lema:regularen-funktor-slika-model-v-model} velja, da je
  $\mathfrak{F} \models \mathbb{T}_{\mathbf{C}}$.  Obratno želimo
  pokazati, da $F$ ohranja strukturo regularne kategorije. Ker je $F$
  funktor, jasno ohranja vse enakosti interne logike $\cat{C}$. Ker pa
  po predpostavki v interpretaciji $\mathfrak{F}$ veljajo vse sekvente
  iz interne logike $\cat{C}$, bo $F$ ohranjal tudi ostalo regularno
  strukturo. Vse lastnosti, ki kategorijo naredijo regularno, je
  namreč mogoče izraziti v interni logiki. Na primer recimo, da je $X$
  končen objekt v $\mathbf{C}$. Potem iz leme
  \ref{lema:limite-v-interni-logiki} sledi
  \[ \mathbf{C} \models x_1:X,x_2:X \mid  x_{1} = x_{2}\quad \text{in} \quad \mathbf{C}
    \models \exists (x:X) (x = x).\]
  Po predpostavki torej velja
  $\mathfrak{F} \models x_1:X, x_2:X \mid x_{1} = x_{2}$
  in~${\mathfrak{F} \models \exists (x:X) (x = x)}$.  To pa pomeni, da je
  $X^{(\mathfrak{F})}$ končen objekt v $\mathbf{D}$.  Podoben
  argument, z uporabo karakterizacije limit v interni logiki, pokaže
  še, da $F$ ohranja produkte in zožke.  Z uporabo leme
  \ref{lema:morfizmi-v-interni-logiki} lahko pokažemo še, da $F$
  ohranja tudi faktorizacijo morfizmov, v posebnem tiste regularne
  epimorfizme, ki nastanejo kot kozožki jedrnega para (jedrni par, ki
  je povlek, se tudi ohranja).
\end{dokaz}
Po drugi strani, če je $M$ fiksen model teorije $\T$ v regularni
kategoriji $\cat{E}$, dobimo za vsako regularno kategorijo $\cat{D}$
funktor
\begin{equation}\label{equation:def-funktor-iz-regcat-v-mod}
  \mathfrak{M}_{M,\cat{D}} = (\_)_\T(M): \cat{RegCat}(\cat{E}, \cat{D}) \to \Mod(\T, \cat{D}),
\end{equation}
ki pošlje funktor $G : \cat{E} \to \cat{D}$ v model $G(M)$ v
$\cat{D}$.  Naravno transformacijo $\alpha : G \to H$ pošlje v
morfizem modelov
$$\set{\alpha_{\interp{X}}: G(\interp{X}) \to H(\interp{X})}_{X \in \underline{\mathrm{sort}}_{\Sigma}}$$
Dodatno velja, da če imamo regularen funktor
$F : \cat{D} \to \cat{C}$, potem diagram funktorjev
\begin{equation}\label{diag:naravnost-M}
  \begin{tikzcd}
    \cat{RegCat}(\cat{E}, \cat{D}) \ar[d, "F \circ (\_)"'] \ar[r, "\mathfrak{M}_{M,\cat{D}}"] & \Mod(\T, \cat{D}) \ar[d, "F_{\T}"] \\
    \cat{RegCat}(\cat{E}, \cat{C}) \ar[r, "\mathfrak{M}_{M,\cat{C}}"']
    & \Mod(\T, \cat{C})
  \end{tikzcd}
\end{equation}
komutira.


V primeru algebrajskih teorij smo videli, da za vsako algebrajsko
teorijo~$\mathbb{T}$ obstaja poseben model $\mathcal{U}$, ki smo ga
imenovali generični model teorije $\mathbb{T}$, za katerega velja, da
je dokazljivost v teoriji ekvivalentna veljavnosti v $\mathcal{U}$.
Dobili smo tudi ekvivalenco med kategorijama modelov algebrajske
teorije in kategorijo funktorjev, ki ohranjajo končne produkte
$$\Hom_{\mathrm{FP}}(\cat{C}_{\mathbb{T}}, \cat{C}) \simeq \Mod(\mathbb{T}, \cat{C}).$$
Izkaže se, da je podobno konstrukcijo mogoče ponoviti za regularne
teorije.  Za teorijo $\T$ bomo konstruirali regularno kategorijo
$\mathcal{R}(\T)$, ki bo inducirala ekvivalenco kategorij
$$\Mod(\T, \cat{C}) \simeq \cat{RegCat}(\mathcal{R}(\T), \cat{C}),$$
naravno v $\cat{C}$. Analogno kot pri algebrajskih kategorijah bo
$\mathcal{R}(\T)$ vsebovala \emph{logično generičen} model teorije
$\T$, za katerega se bosta pojma veljavnosti in dokazljivosti ujemala.
\begin{definicija}
  Naj bo $\T$ regularna teorija.
  Sintaktična kategorija $\mathcal{R}(\T)$ je:
  \begin{itemize}
  \item \emph{objekti} so ekvivalenčni razredi
    formul v kontekstu, se pravi $\Gamma \mid p$, pri čemer formuli
    $\Gamma \mid p$ in $\Gamma \mid q$ predstavljata isti objekt, kadar
    $\T$ izpelje $\Gamma \mid p \implies q$ in $\Gamma \mid q \implies p$. 
      Ekvivalenčne razrede označimo z $\set{\Gamma \,\middle|\, p}$
    \item
      \emph{Morfizem}
      $$\{ \bar{x} : \bar{X}, \bar{y} : \bar{Y} \mid \gamma(\bar{x},\bar{y})\} : \{\bar{x}:\bar{X} \mid p(\bar{x})\}
      \to \{\bar{y} : \bar{Y} \mid q(\bar{y})\}$$
      je funkcijska relacija $\gamma$ v $\T$, torej imamo izpeljave:
      \begin{align*}
        &\bar{x}:\bar{X},\bar{y}:\bar{Y} \mid \gamma(\bar{x},\bar{y})
          \vdash p(\bar{x}) \wedge q(\bar{y}) \\
        &\bar{x}:\bar{X} \mid (\bar{x}) \vdash \exists \bar{y} \gamma(\bar{x},\bar{y}) \\
        &\bar{x}:\bar{X},\bar{y}_1 : \bar{Y},\bar{y}_2:\bar{Y} \mid
          \gamma(\bar{x},\bar{y}_1) \wedge \gamma(\bar{x},\bar{y}_2) \vdash \bar{y}_1 = \bar{y}_2
      \end{align*}
      
    \item Kompozitum morfizmov
      \begin{equation*}
        \begin{tikzcd}[column sep=normal]
          \lbrace \bar{x} \mid p\rbrace \ar[r, "\lbrace \gamma
          \rbrace"] & \lbrace \bar{y} \mid q \rbrace \ar[r, "\lbrace
          \chi \rbrace"] & \lbrace \bar{z} \mid r \rbrace
        \end{tikzcd}
      \end{equation*}           
      je podan s formulo
      \[ \exists (\bar{x}:\bar{X}) (\gamma(\bar{y},\bar{x}) \wedge
        \chi(\bar{x},\bar{z})).\]
    \end{itemize}
  \end{definicija}
  \begin{lema}
    Kompozitum morfizmov v $\mathcal{R}(\T)$ je dobro definiran.
  \end{lema}
  \begin{dokaz}
    Pri dokazu bomo zaradi preglednosti predpostavili, da so formule
    odvisne le od ene proste spremenljivke. Na idejo dokaza to ne
    vpliva.

    Naj bosta $\set{\varphi} : \set{x \mid p} \to \set{y \mid q}$ in
    $\set{\psi} : \set{y \mid q} \to \set{z \mid r}$ morfizma v
    $\mathcal{R}(\mathbb{T})$.  Potem je
    $\set{\psi} \circ \set{\varphi}$ predstavljen s formulo
    \[ \exists (y:Y)(\varphi(x,y) \land \psi(y,z)). \]
    Pokazati moramo, da določa morfizem v
    $\mathcal{R}(\mathbb{T})$. Za totalnost računamo:
    \begin{align*}
      x:X \mid p(x) &\vdash \exists y \varphi(x,y) \\
                    &\vdash \exists y \exists z (\varphi(x,y) \land \psi(y,z)) \\
                    &\vdash \exists z \exists y(\varphi(x,y) \land \psi(y,z)),
    \end{align*}
    kjer smo za drugi korak uporabili
    \[x:X,y:Y \mid \varphi(x,y) \vdash q(y) \quad \text{in}\quad y:Y
      \mid q(y) \vdash \exists (z:Z) \psi(y,z).\] Za funkcionalnost
    ponovno računamo:
    \begin{align*}
      x:X,z_1,z_2:Z \mid&
                          \exists (y:Y)(\varphi(x,y) \land \psi(y,z_{1})) \land \exists (y':Y) (\varphi(x,y') \land \psi(y',z_{2})) \\
                        &\vdash \exists (y,y':Y) (\varphi(x,y) \land \psi(y,z_{1}) \land \varphi(x,y') \land \psi(y',z_{2})) \\
                        &\vdash y = y' \qquad \text{(ker $\varphi(x,y) \land \varphi(x,y')$)} \\
                        &\vdash z_{1} = z_{2} \qquad \text{(ker $\psi(y,z_{1}) \land \psi(y',z_{2}) \land y=y'$)}.
    \end{align*}
    Ta formula torej res določa dobro definiran morfizem v
    $\mathcal{R}(\mathbb{T})$.
  \end{dokaz}
  \begin{opomba}
    Iz konstrukcije je jasno, da dobimo majhno kategorijo.
  \end{opomba}
  \begin{lema}\label{lema:limite-v-sintaktični-kategoriji}
    Kategorija $\mathcal{R}(\T)$ ima vse končne limite
    \begin{enumerate}[label=(\roman*)]
    \item Objekt $\set{\cdot \,\middle|\, \top}$ je končni objekt v
      $\mathcal{R}(\T)$.
    \item Produkt objektov $\set{\bar{x}\,\middle|\, p}$ in
      $\set{\bar{y}\,\middle|\, q}$ je podan z objektom
      $\set{(\bar{x},\bar{y})\,\middle|\, p \wedge q}$.  Projekcija na
      $\set{\bar{x}\,\middle|\, p}$ je podana z
      $$\set{(\bar{x}\bar{y},\bar{x}')\,\middle|\, p(\bar{x}) \wedge q(\bar{y}) \wedge \bar{x} = \bar{x}'}.$$
      Projekcija na $\set{\bar{y}\,\middle|\, q}$ pa z ekvivalenčnim
      razredom
      $$\set{(\bar{x}\bar{y},\bar{y}')\,\middle|\, p(\bar{x}) \wedge q(\bar{y}) \wedge \bar{y} = \bar{y}'}.$$
    \item Zožek morfizmov
      $\set{\gamma}, \set{\gamma'} : \set{\bar{x}\,\middle|\, p} \to
      \set{\bar{y}\,\middle|\, q}$ je podan z objektom
      $E = \set{\bar{x}\,\middle|\, \epsilon(\bar{x})}$, kjer je
      $$\epsilon(\bar{x}) \equiv \exists \bar{y}\left(\gamma(\bar{x},\bar{y}) \wedge \gamma'(\bar{x},\bar{y})\right)$$
      in morfizmom
      $\set{(\bar{x},\bar{x}')\,\middle|\, \epsilon(\bar{x}) \wedge
        \bar{x} = \bar{x}'}$.
    \item Povlek morfizmov
      $\set{\varphi} : \set{\bar{x}\,\middle|\, p} \to
      \set{\bar{z}\,\middle|\, r}$ in
      $\set{\gamma} : \set{\bar{y}\,\middle|\, q} \to
      \set{\bar{z}\,\middle|\, r}$ je podan z objektom
      $$\set{(\bar{x},\bar{y})\,\middle|\, \exists \bar{z} \left(\varphi(\bar{x},\bar{z}) \wedge \gamma(\bar{y},\bar{z})\right)}$$
      in kanoničnima projekcijama.
    \end{enumerate}
  \end{lema}
  \begin{opomba}
    Opazimo, da so v $\mathcal{R}(\T)$ vse limite podane kot operacije, ne
    samo enolično do izomorfizma.
  \end{opomba}
  \begin{dokaz}
    Uporabljali bomo lastnosti relacije $\vdash_{\T}$.  Ponovno bomo
    predpostavljali, da so formule odvisne od ene same proste
    spremenljivke.
    \begin{enumerate}[label=(\roman*)]
    \item Za poljuben objekt $\set{\Gamma \mid p}$ dobimo morfizem v
      $\set{\cdot \mid \top}$.  Recimo, da imamo dva morfizma
      $\set{x \mid \gamma}, \set{x \mid \gamma'}$ iz $\set{x \mid p}$
      v $\set{\cdot \mid \top}$.  Potem iz definicije morfizma v
      $\mathcal{R}(\mathbb{T})$ sledi
      \[x:X \mid \gamma \vdash p(x)\quad \text{in}\quad x:X \mid
        p(x) \vdash \exists \emptyset \gamma'(x).\] Zadnja
      izpeljava je ekvivalentna
      $x:X \mid p(x) \vdash \gamma'(x)$.  Iz tranzitivnosti
      izpeljav sledi $x:X \mid \gamma(x) \vdash \gamma'(x)$.
      Podobno dobimo še izpeljavo $\gamma'(x) \vdash(x) $, torej
      sta morfizma, ki ju ti dve formuli definirata, enaka.
    
    \item Naj bosta $\set{x \mid p}$ in $\set{y \mid q}$ objekta.
      Označimo s $\set{\pi_1}$ in $\set{\pi_2}$ morfizma podana s
      formulama
      \[ \pi_1(x,y,x') \equiv p(x) \land q(y) \land x=x', \qquad
        \pi_2(x,y,y') \equiv p(x) \land q(y) \land y=y'.
      \]
      Najprej moramo preveriti, da sta to res morfizma v
      $\mathcal{R}(\mathbb{T})$.  Iz definicije~$\pi_1$ in leme
      \ref{lema:uporabne-izpeljave} sledi
      $\pi_1(x,y,x') \vdash_{x,y,x'} p(x')$, kar implicira
    $$\pi_1(x,y,x') \vdash_{x,y,x'} p(x) \land q(y) \land p(x').$$
    Da je to totalna relacija:
    $p(x) \land q(y) \vdash_{x,y} \exists x' \pi_1(x,y,x')$ sledi iz
    primera \ref{primer:vpeljava-eksist-kvantifikatorja} in leme
    \ref{lema:uporabne-izpeljave}.  Funkcijskost
    \[\T \vdash_{x,y,x_1',x_2'} (\pi_1(x,y,x_1') \land
      \pi_1(x,y,x_2')) \implies x_1' = x_2',\] dobimo z uporabo pravil
    \ref{pravilo:konj}, \ref{pravilo:enakost-sim} in
    \ref{pravilo:enakost-tranz}.

    Denimo sedaj, da imamo objekt $\set{z \mid r}$ in morfizma
    ${\set{\varphi} : \set{z \mid r} \to \set{x \mid p}}$ in
    $\set{\gamma} : \set{z \mid r} \to \set{y \mid q}$.  Definirajmo
    formulo
    \[ \mu(z,x,y) \equiv \varphi(z,x) \land \gamma(z,y). \]
    Dokažimo, da $\mu$ inducira enolični morfizem
    $\set{z \mid r} \to \set{x,y \mid \varphi \land \gamma}$, za
    katerega velja $\set{\pi_1} \circ \set{\mu} = \set{\varphi}$ in
    $\set{\pi_2} \circ \set{\mu} = \set{\gamma}$.  Kot prvo pokažimo,
    da $\mu$ res inducira morfizem.  Ker
    $\mu \vdash \varphi \land \gamma$ in $\varphi \vdash r \land p$
    ter $\gamma \vdash q$, sledi
    $\mu(z,x,y) \vdash_{z,x,y}r(z) \land p(x) \land q(y)$.  Da je to
    totalna relacija, sledi iz dejstva, da sta taki že $\varphi$ in
    $\gamma$, oziroma eksplicitno imamo
    $r(z) \vdash_z \exists x \varphi(z,x)$ in
    $r(z) \vdash_z \exists y \gamma(z,y)$, kar nam da
    \[ r(z) \vdash_z \exists x \varphi(z,x) \land \exists y
      \gamma(z,y) \vdash_z \exists x \exists y (\varphi(z,x) \land
      \gamma(z,y)), \] z uporabo leme \ref{lema:uporabne-izpeljave}.
    Pokazati moramo še, da je to funkcijska relacija.  Ker je
    $\varphi$ funkcijska, velja
    \[ \mu(z,x,y_1) \land \mu(z,x,y_2) \vdash_{z,x,y_1,y_2}
      \varphi(z,y_1) \land \varphi(z,y_2) \vdash_{z,x,y_1,y_2} y_1 =
      y_2 \] in podobno za $\gamma$.  Kompozitum
    $\set{\pi_1} \circ \set{\mu}$ je po definiciji podan s formulo
    \[ \exists x,y(\mu(z,x,y) \land p(x) \land q(y) \land x=x').\] Ker
    je $x = x'$ je to ekvivalentno formuli
    \begin{align*}
      \exists y(\mu(z,x',y) \land p(x') \land q(y)) \equiv&\ 
                                                            \exists y(\varphi(z,x') \land \gamma(z,y) \land p(x') \land q(y)) \\
      \iff& \varphi(z,x') \land p(x') \land \exists y(\gamma(z,y) \land q(y)) \\
      \iff& \varphi(z,x') \land \exists y \gamma(z,y) \\
      \iff& \varphi(z,x').
    \end{align*}
    Podoben argument velja za $\set{\pi_2} \circ \set{\gamma}$.

    Za enoličnost bomo pokazali, da iz
    \[\set{\pi_1} \circ \set{\mu} = \set{\varphi}\quad \text{in}\quad
      \set{\pi_2} \circ \set{\mu} = \set{\gamma}\] sledi
    $\T \vdash_{z,x',y'} \mu \iff \varphi \land \gamma$.  Po
    predpostavki je $\varphi(z,x')$ ekvivalentna
    \[ \exists x,y(\mu(z,x,y) \land p(x) \land q(y) \land x=x'),\] kar
    je ekvivalentno $\exists y(\mu(z,x',y))$.  Podobno za $\gamma$
    velja, da je ekvivalentna $\exists x(\mu(z,x,y'))$. Torej je
    \begin{align*}
      \varphi(z,x') \land \gamma(z,z') &\iff \exists y(\mu(z,x',y)) \land \exists x(\mu(z,x,y')) \\
                                       &\iff \mu(z,x',y'),
    \end{align*}
    kjer smo uporabili dejstvo, da je $\mu$ funkcijska slika $z$.

  \item Najprej pokažimo, da je
    $\set{x,x' \mid \epsilon(x') \land x = x'}$ morfizem.  Iz
    definicije~$\epsilon$ sledi, da
    $\epsilon(x') \land x = x' \vdash_{x,x'} p(x)$.  Za totalnost
    moramo pokazati
    \[ \epsilon(x') \vdash_{x'} \exists x (\epsilon(x') \land x=x'),\]
    za kar uporabimo izpeljavo iz primera
    \ref{primer:vpeljava-eksist-kvantifikatorja}.  Zadnjo točko dobimo
    ker
    \[\epsilon(x') \land x'=x_1 \land \epsilon(x') \land x'= x_2
      \vdash x_1 = x_2.\] Pokazati moramo, da velja
    $\set{\gamma} \circ \set{\epsilon(x') \land x = x'} = \set{\gamma}
    \circ \set{\epsilon(x') \land x = x'}$.  Prvi kompozitum je po
    definiciji enak
    \[\exists x( \exists y(\gamma(x',y) \land \gamma'(x',y)) \land x =
      x' \land \gamma(x,y')). \] Ta formula je
    ekvivalentna $\exists y(\gamma(x',y) \land \gamma(x',y))$ in enako
    velja za drugi kompozitum.  Denimo, da imamo morfizem
    \[\set{\theta} : \set{z \mid r} \to \set{x \mid p}\]
    za katerega velja
    $\set{\gamma} \circ \set{\theta} = \set{\gamma'} \circ
    \set{\theta}$.  Potem iščemo formulo $\mu$, ki bo definirala
    morfizem $\set{z \mid r} \to \set{x' \mid \epsilon}$, ki bo
    faktoriziral $\{ \theta\}$.  Definirajmo kar
    $\mu(z,x') \equiv \theta(z,x')$. Ni se težko prepričati, da s tako
    definicijo res dobimo morfizem
    $\set{z \mid r} \to \set{x' \mid \epsilon}$.  Kompozitum z
    $\set{\epsilon(x') \land x = x'}$ je po definiciji enak
    \[ \exists x' (\theta(z,x') \land \epsilon(x') \land x=x'),\] kar
    je ekvivalentno kar $\theta(z,x)$.  Iz
    zgornje izpeljave je tudi očitno, da za vsak morfizem $\set{\mu}$,
    ki nam da tako faktorizacijo, velja
    \[\T \vdash \mu \iff \theta.\]

  \item Opis povlekov v interni logiki sledi iz konstrukcije povlekov
    iz produktov in zožkov, ki velja v vsaki kategoriji s končnimi
    limitami.
  \end{enumerate}
\end{dokaz}
\begin{lema}\label{lema:morfizmi-v-sintakticni-kategoriji}
  Morfizem
  $\set{\varphi} : \set{\bar{x}\,\middle|\, p} \to
  \set{\bar{y}\,\middle|\, q}$ je:
  \begin{enumerate}[label=(\roman*)]
  \item monomorfizem natanko takrat, ko
      $$\T \vdash_{\bar{x}_1,\bar{x}_2}  \exists \bar{y} \left( \varphi(\bar{x}_1,\bar{y}) \wedge \varphi(\bar{x}_2,\bar{y}) \right) \implies \bar{x}_1 = \bar{x}_2 ,$$
    \item regularen epimorfizem natanko takrat, ko
      $$\T \vdash_{\bar{y}} q(\bar{y}) \implies \exists \bar{x} \varphi(\bar{x},\bar{y}).$$
    \item Morfizem
      $\set{p(\bar{x}) \wedge \bar{x} = \bar{x}'} :
      \set{\bar{x}\,\middle|\, p } \to \set{\bar{x}\,\middle|\, q}$ je
      monomorfizem natanko takrat, ko
      $$\T \vdash_{\bar{x}} p \implies q.$$
    \end{enumerate}
  \end{lema}
  \begin{dokaz}
    \begin{enumerate}[label=(\roman*)]
    \item Najprej uporabimo opazko, da je $f : X \to Y$
      monomorfizem natanko takrat, ko je
      \begin{equation*}
        \begin{tikzcd}
          X \ar[d, " \mathrm{id}_X"'] \ar[r, " \mathrm{id}_X"] & X \ar[d, "f"] \\
          X \ar[r, "f"'] & Y
        \end{tikzcd}
      \end{equation*}
      povlek. Po prejšnji lemi lahko povlek $\{\varphi\}$ izrazimo kot
      \[ \{ x_1,x_2 \mid \exists y(\varphi(x_1,y) \land
        \varphi(x_2,y)) \}.\] Da je $\{\varphi\}$ monomorfizem je
      torej ekvivalentno temu, da v $\mathcal{R}(\mathbb{T})$ velja
      \[ \{ x_1,x_2 \mid \exists y(\varphi(x_1,y) \land
        \varphi(x_2,y)) \} = \{x \mid p\} \] Če predpostavimo, da sta
      ta dva objekta enaka v $\mathcal{R}(\mathbb{T})$, mora jasno
      veljati
    $$\T \vdash_{x_1,x_2}  \exists y \left( \varphi(x_1,y) \wedge \varphi(x_2,y) \right) \implies x_1 = x_2 .$$
    Obratno, če zgornja enačba velja, potem iz
    $\exists y(\varphi(x_1,y) \land \varphi(x_2,y))$ sledi
    \[p(x_1) \land p(x_2) \land x_1 = x_2,\] kar je ekvivalentno
    $p(x_1)$.

  \item Najprej bomo pokazali, da je
    \begin{equation}\label{diag:jedrni-par-v-sintakticni-kategoriji}
      \begin{tikzcd}[column sep=normal]
        \{ x_1,x_2 \mid \exists y(\varphi(x_1,y) \land \varphi(x_2,y))
        \} \ar[r, shift left, "\{\pi_1\}"] \ar[r, shift right,
        "\{\pi_2\}"'] & \{x \mid p\} \ar[r, "\{\varphi\}"] & \{y \mid
        q\}
      \end{tikzcd}
    \end{equation}
    kozožek natanko takrat, ko velja
    $$\T \vdash_{\bar{y}} q(\bar{y}) \implies \exists \bar{x} \varphi(\bar{x},\bar{y}).$$
    Nato bomo pokazali, da je $\{ \varphi\}$ epimorfizem natanko
    takrat, ko je \eqref{diag:jedrni-par-v-sintakticni-kategoriji}
    kozožek.


    Projekcija~$\{\pi_1\}$ je inducirana s formulo
    \[\exists y(\varphi(x_1,y) \land \varphi(x_2,y) \land x_1 = x )\]
    in $\{\pi_2\}$ s podobno.  Enakost
    $\{\varphi\} \circ \{\pi_1\} = \{ \varphi\} \circ \{\pi_2\}$
    velja, ker $\{\pi_1\}$ in $\{\pi_2\}$ dobimo iz povleka. Denimo
    sedaj, da imamo objekt $\{ z \mid r\}$ in morfizem
    $\{\psi\} : \{ x \mid p\} \to \{ z \mid r\}$, za katerega velja
    \[\{\psi\} \circ \{\pi_1\} = \{\psi\} \circ \{\pi_2\}.\] To po
    definiciji pomeni
    \[ \T \vdash_{x_1,x_2,z} \left(\exists x (\pi_1(x_1,x_2,x) \land
        \psi(x,z)) \iff \exists x (\pi_2(x_1,x_2,x) \land
        \psi(x,z))\right).
    \]
    Definirajmo formulo
    \[ \eta(x,z) \equiv \exists x(\varphi(x,y) \land \psi(x,z)).\]
    Pokazati želimo, da ta formula inducira enoličen morfizem
    ${\{y \mid q\} \to \{z \mid r\}}$, za katerega velja
    $\{\psi\} = \{ \eta\} \circ \{\varphi\}$.  Najprej pokažimo, da
    res definira morfizem v $\mathcal{R}(\mathbb{T})$. Računamo:
    \begin{align*}
      \eta(y,z) &\vdash \exists x(p(x) \land q(y) \land p(x) \land r(z)) \\
                &\vdash q(y) \land r(z).
    \end{align*}
    Za totalnost imamo izpeljavo:
    \begin{align*}
      q(y) &\vdash\exists x(\varphi(x,y)) \\
           &\vdash \exists x(p(x) \land q(y)) \\
           &\vdash \exists x (\exists z \psi(x,z)) \\
           &\vdash \exists x(\varphi(x,y) \land \exists z \psi(x,z)) \\
           &\vdash \exists z ( \exists x(\varphi(x,y) \land \psi(x,z))) \\
           &\equiv \eta(y,z).
    \end{align*}
    Za zadnji pogoj imamo:
    \[ \exists x(\varphi(x,y) \land \psi(x,z_1)) \land \exists x'
      (\varphi(x',y) \land \psi(x',z_2)) \vdash \exists
      x,x'(\varphi(x,y) \land \varphi(x',y)).\] Iz predpostavke potem
    sledi $z_1 = z_2$, kar smo želeli pokazati.  Za kompozitum moramo
    pogledati formulo
    \[ \exists y(\varphi(x,y) \land \eta(y,z)) \equiv \exists
      y(\varphi(x,y) \land \exists x'(\varphi(x',y) \land \psi(x',z))
      ), \] iz česar želimo izpeljati $\psi(x,z)$. Najprej izpostavimo
    $\exists x'$ in dobimo
    \[\exists y \exists x'( \varphi(x,y) \land \varphi(x',y) \land
      \psi(x',z)). \] Po predpostavki potem iz
    $\varphi(x,y) \land \varphi(x',y)$ sledi
    $\psi(x',z) \land \psi(x,z)$, iz česar seveda sledi $\psi(x,z)$.
    Za enoličnost privzamemo, da obstaja drug morfizem
    $\{\eta\} : \{y \mid q\} \to \{z \mid r\}$, za katerega velja
    $\{\psi\} = \{\tilde{\eta}\} \circ \{\varphi\}$. Potem imamo
    \begin{align*}
      \exists y (\varphi(x,y) \land \tilde{\eta}&(y,z)) \land \psi(x,z) \\
                                                &\vdash \exists y (\exists x(\varphi(x,y) \land \psi(x,z)) \land \tilde{\eta}(y,z)) \\
                                                &\vdash (\eta(y,z) \land \tilde{\eta}(y,z))
    \end{align*}
    Torej sta $\{\eta\}$ in $\{\tilde{\eta}\}$ isti morfizem v
    $\mathcal{R}(\mathbb{T})$. Za dokaz v drugo smer definiramo objekt
    $\{ y \mid \exists x \varphi(x,y)\}$, ki si ga lahko predstavljamo
    kot sliko morfizma $\{\varphi\}$ in morfizem iz $\{x \mid p\}$ v
    to sliko, ki ga predstavlja kar formula $\varphi$. Jasno bo ta
    formula res določala morfizem z zgornjo kodomeno. Po definiciji
    $\varphi$ bo ta morfizem tudi komutiral s projekcijama $\{\pi_1\}$
    in $\{\pi_2\}$. To bo porodilo enolični morfizem
    \[\{\mu\} : \{y \mid q\} \to \{ y \mid \exists x \varphi(x,y)\},\]
    ki bo faktoriziral morfizem v to sliko, določen s formulo
    $\varphi$. Ampak to pa pomeni, da mora biti $\{\mu\}$ izomorfizem,
    kajti v drugo smer imamo očitno inkluzijo slike. Torej, če
    predpostavimo $q(y)$, velja $\exists x \varphi(x,y)$.


    Pokažimo sedaj še drugi del trditve.  Če je
    \eqref{diag:jedrni-par-v-sintakticni-kategoriji} kozožek, je
    $\{ \varphi\}$ jasno kozožek, saj je to njegov jedrni
    par. Obratno, naj obstajajo $r, \eta, \theta$ taki, da je
    \begin{equation*}
      \begin{tikzcd}[column sep=normal]
        \lbrace z \mid r \rbrace \ar[r, shift left, "\lbrace \eta
        \rbrace"] \ar[r, shift right, "\lbrace \theta \rbrace"'] &
        \lbrace x \mid p \rbrace \ar[r, "\lbrace \varphi \rbrace"] &
        \lbrace y \mid q \rbrace
      \end{tikzcd}
    \end{equation*}
    kozožek. Pokazati želimo, da to potem velja tudi za
    \eqref{diag:jedrni-par-v-sintakticni-kategoriji}. V ta namen
    recimo, da imamo morfizem
    $\{ \sigma \} : \{ x \mid p\} \to \{ w \mid s\}$ za katerega velja
    $\{ \sigma\} \circ \{\pi_1\} = \{ \sigma\} \circ \{\pi_2\}$. V tem
    primeru zadostuje najti morfizem $\{\chi\}$, ki faktorizira
    $\{\eta\}$ in $\{\theta\}$ skozi $\{\pi_1\}$ in $\{\pi_2\}$. Potem
    dobimo
    \[ \{\sigma\} \circ \{\pi_1\} \circ \{\chi\} = \{\sigma\} \circ
      \{\pi_2\} \circ \{\chi\}\] iz česar sledi
    \[ \{ \sigma\} \circ \{\eta\} = \{\sigma\} \circ \{\theta\}\] kar
    nam da enoličen morfizem $\{\mu\}$, ki faktorizira $\{\sigma\}$,
    torej $\{\varphi\}$ res določa kozožek $\{\pi_1\}$ in
    $\{\pi_2\}$. Definiramo
    \[ \chi(z, x_1, x_2) \equiv \eta(z,x_1) \land \theta(z,x_2).\] Iz
    predpostavke $\chi(z,x_1,x_2)$ jasno sledi $r(z)$.  Da pokažemo,
    da določa pravilno kodomeno moramo najti tak $y$, da velja
    $\varphi(x_1,y) \land \varphi(x_2,y)$.  Po definiciji $\chi$
    dobimo neka $y_1, y_2$, za katera velja $\eta(x_1,y_1)$ in
    $\theta(x_2,y_2)$, ki sta v principu lahko različna. Ker pa
    $\{\varphi\}$ izenači $\{\eta\}$ in $\{\theta\}$, sledi
    $y_1 = y_2$. Torej smo našli ustrezen $y$, ki pokaže, da $\chi$
    res določa ustrezno domeno in kodomeno.  Da res definira morfizem
    sledi iz tega, da $\eta$ in $\theta$ vsak posebej definirata
    morfizem.  Če razpišemo definicijo projekcij $\pi_1$ in $\pi_2$ in
    kompozitumov $\{\pi_1\} \circ \{\chi\}$ in
    $\{\pi_2\} \circ \{\chi\}$ se ni težko prepričati, da res
    izpolnjujejo želeni enakosti.

  \item Pogoj
    $$\T \vdash_{\bar{x}} p \implies q$$
    je potreben, da formula res določa morfizem
    v~$\mathcal{R}(\mathbb{T})$.  To, da res določa monomorfizem,
    potem sledi iz točke $(\mathrm{i})$. Obratno, vsak monomorfizem je
    tudi morfizem, torej iz $p(x)$ sledi $\exists x' p(x) \land
    x=x'$. Ker gre za monomorfizem je ta $x'$ določen enolično, kar
    nam posledično da $q(x')$. Iz $x = x'$ nato sledi
    $q(x)$. Uporabili smo lemo \ref{lema:uporabne-izpeljave}.
  \end{enumerate}
\end{dokaz}
\begin{trditev}
  $\mathcal{R}(\T)$ je regularna kategorija.
\end{trditev}
\begin{dokaz}
  Po lemi \ref{lema:limite-v-interni-logiki} vemo, da ima
  $\mathcal{R}(\mathbb{T})$ končne limite in po lemi
  \ref{lema:morfizmi-v-sintakticni-kategoriji} vemo, da kozožki
  jedrnih parov obstajajo. Pokazati moramo še, da povleki ohranjajo
  regularne epimorfizme.  Torej naj bo
  \begin{equation*}
    \begin{tikzcd}[column sep=small]
      \{y,x \mid \exists z(\gamma(y,z) \land \varphi(x,z))\} \ar[d,
      "\{\pi\}"'] \ar[r] &
      \{ x \mid p\} \ar[d, "\{\varphi\}"] \\
      \{y \mid q\} \ar[r, "\{\gamma\}"'] & \{z \mid r\}
    \end{tikzcd}
  \end{equation*}
  diagram povleka. Projekcija $\{\pi\}$ je inducirana s formulo
  \[ \pi(y,x,y') \equiv \exists z(\gamma(y,z) \land \varphi(x,z))
    \land y=y'.\] Recimo, da je $\{\varphi\}$ regularen
  epimorfizem. Po prejšnji lemi vemo, da potem velja
  $\mathbb{T} \vdash r(z) \Rightarrow \exists x \varphi(x,z)$.  Iz
  $q(y)$ sledi $\exists z(\gamma(y,z))$, iz česar pa lahko izpeljemo
  $\exists z r(z)$, kar nam da $\exists z \exists x
  \varphi(x,z)$. Skupaj dobimo
  \[ \exists z(\gamma(y,z) \land \exists x \varphi(x,z)),\] kar je po
  lemi \ref{lema:uporabne-izpeljave} ekvivalentno
  $\exists x \exists z(\gamma(y,z) \land \varphi(x,z))$.  Iz primera
  \ref{primer:vpeljava-eksist-kvantifikatorja} nato sledi, da je
  $\{\pi\}$ regularen epimorfizem.
\end{dokaz}
\begin{definicija}
  Kategorija $\mathcal{R}(\T)$ vsebuje kanonično interpretacijo
  $\mathcal{U}$ jezika $\mathcal{L}(\Sigma)$:
  \begin{itemize}
  \item $\interp[\mathcal{U}]{X} = \set{x\,\middle|\, \top}$, kjer je
    $x$ spremenljivka sorte $X$.
  \item
    $\interp[\mathcal{U}]{c} = \set{x\,\middle|\, x = c} :
    \set{\cdot\,\middle|\, \top} \to \interp[\mathcal{U}]{X}$, kjer je
    $c:X$ konstanta.
  \item
    $\interp[\mathcal{U}]{f} = \set{(\bar{x},y)\,\middle|\, f(\bar{x})
      = y}$, kjer je $f : \bar{X}\to Y$ funkcijski simbol. Uporabimo
    dejstvo, da je
    $\bar{X}^{(\mathcal{U})} = \interp[\mathcal{U}]{\bar{X}_1} \times
    \ldots \times \interp[\mathcal{U}]{\bar{X}_n}$.
  \item
    $\interp[\mathcal{U}]{R} = \set{\bar{x}\,\middle|\, R(\bar{x})}$,
    kjer je $R$ relacijski simbol.
  \end{itemize}
\end{definicija}
\begin{izrek}\label{izrek:logicno-genericen-model-regularne-logike}
  Kanonična interpretacija $\mathcal{U}$ v regularni kategoriji
  $\mathcal{R}(\T)$ je logično generičen model teorije $\T$. Pravila
  sklepanja v regularni logiki so polna glede na interpretacije v
  majhnih regularnih kategorijah.
\end{izrek}
\begin{dokaz}
  Enostavna indukcija pokaže, da je za term $t(\bar{z})$ sorte $Y$
$$\caninterp{t(\bar{z})} = \set{(\bar{z},y)\,\middle|\, t(\bar{z}) = y}$$
morfizem $\caninterp{\bar{Z}} \to \caninterp{Y}$.
Podobno za regularno
formulo $\varphi(\bar{z})$ velja
$$\caninterp{\varphi(\bar{z})} = \set{\bar{z}\,\middle|\, \varphi}.$$
Sledi, da je $\mathcal{U}$ model teorije $\T$, kajti če je
$\varphi \Rightarrow \psi$ sekventa v $\T$, potem formula
$p(\bar{x}) \wedge \bar{x}= \bar{x}'$ določa monomorfizem iz
$\set{\bar{x}\,\middle|\, p}$ v $\set{\bar{x}'\,\middle|\, q}$, torej
je $\set{\bar{x}\,\middle|\, p} \leq \set{\bar{x}'\,\middle|\, q}$.
Za model $\mathcal{U}$ velja tudi, da za vse sekvente
$p \Rightarrow q$ velja
$$\text{če} \ \mathcal{U} \models p \implies q \text{, potem} \ \T \vdash_{\bar{x}} p \implies q$$
kar sledi iz zgornjih argumentov in leme o klasifikaciji monomorfizmov
v~$\mathcal{R}(\T)$.
\end{dokaz}
Sedaj lahko podamo funktorje, ki nastopajo v ekvivalenci
$$\Mod(\T, \cat{C}) \simeq \cat{RegCat}(\mathcal{R}(\T), \cat{C}).$$
\begin{definicija}\label{def:ekvivalenca-reg-logike-reg-kategorij}
  Funktor
$$\mathfrak{M}_{\cat{C}} : \cat{RegCat}(\mathcal{R}(\T), \cat{C}) \to \Mod(\T, \cat{C})$$
je funktor $\mathfrak{M}_{\mathcal{U}, \cat{C}}$, kot smo ga definiral
v \eqref{equation:def-funktor-iz-regcat-v-mod} in slika funktor
$F : \mathcal{R}(\T) \to \cat{C}$ v model $F_T(\mathcal{U})$ v
$\cat{C}$ ter naravno transformacijo $\alpha : F \to F$ v družino
$$F_T(\alpha) = \set{\alpha_{\caninterp{X}} : F(\caninterp{X}) \to G(\caninterp{X})}_{X \in \underline{\mathrm{sort}}_{\Sigma}}$$
V drugo smer, funktor
$$\mathfrak{F}_{\cat{C}} : \Mod(\T, \cat{C}) \to \cat{RegCat}(\mathcal{R}(\T), \cat{C})$$
slika model $M$ teorije $\T$ v kategoriji $\cat{C}$ v funktor
\begin{align*}
  \mathfrak{F}_{\cat{C}}(M) : \mathcal{R}(\T) &\to \cat{C}\\
  \set{\bar{x}\,\middle|\, p} &\mapsto \set{\bar{x}\,\middle|\, p}^{(M)} \\
  \set{\gamma} : \set{\bar{x}\,\middle|\, p} \to \set{\bar{y}\,\middle|\, q} &\mapsto \text{`enolični morfizem} \ f : \set{\bar{x}\,\middle|\, p}^{(M)} \to \set{\bar{y}\,\middle|\, q}^{(M)} \\ 
                                              &\qquad \text{da je} \ \graf(f) = \set{(\bar{x}, \bar{y}) \,\middle|\, \gamma}^{(M)} \text{'}
\end{align*}
\end{definicija}
Morfizem $f$ obstaja po lemi \ref{lema:funkcijska-relacija-ima-graf}.
Veljavnost pravil sklepanja in dejstvo, da model lahko definiramo z
interno logiko $\cat{C}$ pomeni, da je $\mathfrak{F}_{\cat{C}}(M)$
regularen funktor.  Morfizem med modeloma $h : M \to N$ porodi družino
morfizmov
$$h_{\set{\bar{x}\,\middle|\, p}} : \set{\bar{x}\,\middle|\, p}^{(M)} \to \set{\bar{x}\,\middle|\, p}^{(N)},$$
naravnih v $\{\bar{x} \mid p\}$ kajti, če imamo morfizem
$\set{\gamma} : \set{\bar{x}\,\middle|\, p} \to
\set{\bar{y}\,\middle|\, q}$ potem, ker oba kvadrata v naslednjem
diagramu komutirata, to pomeni, da komutira tudi zunanji, ki
predstavlja ravno diagram naravne transformacije:
\begin{equation*}
  \begin{tikzcd}[column sep=3cm]
    \set{\bar{x}\,\middle|\, p}^{(M)} \ar[d, two heads] \ar[r, "h_{\set{\bar{x}\,\middle|\, p}}"] & \set{\bar{x}\,\middle|\, p}^{(N)} \ar[d, two heads] \\
    \set{\bar{y}\,\middle|\, \exists \bar{x} \gamma(\bar{x}, \bar{y})}^{(M)}  \ar[d, hook] \ar[r, "h_{\set{\bar{y}\,\middle|\, \exists \bar{x} \gamma(\bar{x}, \bar{y})}}"] & \set{\bar{y}\,\middle|\, \exists \bar{x} \gamma(\bar{x}, \bar{y})}^{(N)} \ar[d, hook] \\
    \set{\bar{y}\,\middle|\, q}^{(M)} \ar[r,
    "h_{\set{\bar{y}\,\middle|\, q}}"] & \set{\bar{y}\,\middle|\,
      q}^{(N)}
  \end{tikzcd}
\end{equation*}
Ta funktor je naraven v $\cat{C}$: Če je $F : \cat{D} \to \cat{C}$
regularen funktor, potem diagram
\begin{equation}\label{diag:naravnost-F}
  \begin{tikzcd}
    \Mod(\T, \cat{D}) \ar[d, "F_T"'] \ar[r, "\mathfrak{F}_{\cat{D}}"] & \cat{RegCat}(\mathcal{R}(\T), \cat{D}) \ar[d, "F \circ (\_)"] \\
    \Mod(\T, \cat{C}) \ar[r, "\mathfrak{F}_{\cat{C}}"'] &
    \cat{RegCat}(\mathcal{R}(\T), \cat{C})
  \end{tikzcd}
\end{equation}
komutira.
\begin{izrek}
  Naj bosta funktorja $\mathfrak{M}_{\cat{C}}$ in
  $\mathfrak{F}_{\cat{C}}$ definirana kot v definiciji
  \ref{def:ekvivalenca-reg-logike-reg-kategorij}. Potem imamo za vsako
  regularno kategorij $\cat{C}$ ekvivalenco kategorij
  $$\Mod(\T, \cat{C}) \simeq \cat{RegCat}(\mathcal{R}(\T), \cat{C}),$$
  ki je naravna v $\cat{C}$. Do ekvivalence natančno, vsaka majhna
  regularna kategorija $\cat{C}$ nastane na tak način, kot
  ">klasifikacijska kategorija"< regularne teorije, kajti
  $\cat{C} \simeq \mathcal{R}(\T_{\cat{C}})$.
\end{izrek}
\begin{dokaz}
  Preverimo, da je kompozitum v obe smeri izomorfen identiteti.
  Najprej recimo, da je $M$ model $\T$ v $\cat{C}$. Potem je
  $\mathfrak{F}(M)$ regularen funktor iz $\mathcal{R}(\T)$ v
  $\cat{C}$. Po lemi \ref{lema:regularen-funktor-v-interni-logiki} vse
  sekvente interne logike $\mathcal{R}(\T)$ veljajo za
  interpretacijo inducirano s funktorjem $\mathfrak{F}(M)$. Interna
  logika $\mathcal{R}(\T)$ je pa ravno $\T$.  Če na tem funktorju nato
  uporabimo $\mathfrak{M}_{\cat{C}}$ dobimo sliko univerzalnega modela
  $\mathcal{U}$:
  \[ \mathfrak{F}_{\cat{C}}(M)(\mathcal{U}).\] To je model $\T$ v
  $\cat{C}$ podan z interpretacijami
  \begin{itemize}
  \item Za vsako sorto $X$ v $\T$ dobimo interpretacijo
    $\{ x \mid x = x\}^{(M)}$.
  \item Za konstanto $c : X$ dobimo morfizem
    $\{ x \mid x = c\}^{(M)} : 1 \to X^{(M)}$.
  \item Za funkcijski simbol $f : \bar{X} \to Y$ dobimo zožek
    \begin{equation*}
      \begin{tikzcd}
        \{ \bar{x},y \mid f(\bar{x}) = y\}^{(M)} \ar[r] &
        \bar{X}^{(M)} \times Y^{(M)} \ar[r, shift left,
        "f(\bar{x})^{(M)}"] \ar[r, shift right, "y^{(M)}"'] & Y^{(M)}
      \end{tikzcd}
    \end{equation*}
  \item Za relacijski simbol $R : \bar{X}$ dobimo podobjekt
    \[ \{ \bar{x} \mid R(\bar{x})\}^{(M)} \hookrightarrow X^{(M)}.\]
  \end{itemize}
  Z indukcijo to sedaj razširimo na vse terme in formule teorije $\T$.
  Iz konstrukcije sledi, da dobimo isti (do izomorfizma natančno)
  model $M$ s katerim smo začeli.  Obratno, če začnemo z regularnim
  funktorjem $G : \mathcal{R}(\T) \to \cat{C}$, najprej z aplikacijo
  $\mathfrak{M}_{\cat{C}}$ dobimo model $G(\mathcal{U})$ v
  $\cat{C}$. Nato z uporabo $\mathfrak{F}_{\cat{C}}$ dobimo regularen
  funktor, ki na objektih $\mathcal{R}(\T)$ deluje kot
  $\{ x \mid p\} \mapsto \{ x \mid p\}^{G(\mathcal{U})}$. Z uporabo
  leme \ref{lema:regularen-funktor-v-interni-logiki} in izreka
  \ref{izrek:logicno-genericen-model-regularne-logike} lahko vidimo,
  da je ta funktor izomorfen $G$. Naravnost v $\cat{C}$ dobimo z
  združitvijo diagramov (\ref{diag:naravnost-M}) in
  (\ref{diag:naravnost-F}).

  Drugi del sledi iz leme
  \ref{lema:regularen-funktor-v-interni-logiki}.
\end{dokaz}
\end{document}
%%% Local Variables:
%%% mode: latex
%%% TeX-master: t
%%% End:
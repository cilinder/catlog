\documentclass[../kategoricna_logika.tex]{subfiles}
\begin{document}
\section{Osnovne Definicije}
%
Najprej bomo uvedli nekaj definicij, ki jih bomo uporabljali v
celotni nalogi. Ta del služi tudi za to, da fiksiramo nekaj
notacije, ki jo bomo vseskozi uporabljali.
\begin{definicija}
  \emph{Kategorija} je sestavljena iz:
  \begin{itemize}
  \item Razreda \emph{objektov}, ki jih bomo običajno označevali z
    velikimi tiskanimi črkami, na primer $X, Y, A, B, \ldots$.
  \item Razreda \emph{morfizmov}, ki jih bomo običajno označevali z
    malimi tiskanimi črkami, na primer $f,g,\alpha,\beta, \ldots$.
    Vsak morfizem $f$ je asociiran z dvema objektoma:
    \[ \mathrm{dom}(f) \text{,} \qquad \mathrm{cod}(f), \]
    ki jima pravimo \emph{domena} in \emph{kodomena}. Če je $\mathrm{dom}(f)=X$
    in ${\mathrm{cod}(f)=Y}$, potem pišemo $f : X \to Y$.
  \end{itemize}
  Kategorija mora zadoščati naslednjima aksiomoma:
  \begin{itemize}
  \item   Za vsaka dva morfizma $f,g$ za katera velja $\mathrm{cod}(f) = \mathrm{dom}(g)$,
  mora obstajati njun \emph{kompozitum}, ki ga označujemo z $g \circ f$, in
  za katerega velja $\mathrm{dom}(g \circ f) = \mathrm{dom}(f)$ in
  $\mathrm{cod}(g \circ f) = \mathrm{cod}(g)$.
  \item Za vsak objekt $X$ mora obstajati t.i.\ \emph{identitetni morfizem}
  $\mathrm{id}_X : X \to X$, za katerega velja $\mathrm{id}_X \circ f = f$,
  za vsak $f : Y \to X$ in $g \circ \mathrm{id}_X = g$, za vsak $g : X \to Z$.
\end{itemize}
Kategorije običajno označujemo z velikimi tiskanimi črkami, na primer $\cat{C}, \cat{D}, \ldots$.
\end{definicija}
\begin{definicija}
  Razred vseh morfizmov v kategoriji $\cat{C}$, z domeno $X$ in kodomeno $Y$ označimo s
  \[ \Hom_{\cat{C}}(X,Y).\]
  Če je $\Hom_{\cat{C}}(X,Y)$ množica za vsaka $X$ in $Y$, pravimo, da je $\cat{C}$
  \emph{lokalno majhna}.
\end{definicija}
\begin{definicija}
  Naj bosta $\cat{C}$ in $\cat{D}$ kategoriji. \emph{Funktor} od $\cat{C}$ do $\cat{D}$
  je preslikava $F$, ki slika objekte iz $\cat{C}$ v objekte v $\cat{D}$ in morfizme iz
  $\cat{C}$ v morfizme v~$\cat{D}$ in izpolnjuje naslednje pogoje:
  \begin{itemize}
  \item $F(f : X \to Y) = F(f) : F(X) \to F(Y)$,
\item $F(\mathrm{id}_X) = \mathrm{id}_{F(X)}$,
\item Za kompozitum $g \circ f$ velja $F(g \circ f) = F(g) \circ F(f)$.
  \end{itemize}
\end{definicija}
\begin{definicija}
  Naj bosta $\cat{C}$ in $\cat{D}$ kategoriji in $F,G : \cat{C} \to \cat{D}$ par funktorjev
  med njima. \emph{Naravna transformacija} $\vartheta : F \to G$, iz $F$ v $G$ je
  družina morfizmov
  \[ (\vartheta_X : F(X) \to G(X))_{X \in \cat{C}}\]
  tako, da za vsak morfizem $f : X \to Y$ v $\cat{C}$ naslednji diagram
  \begin{equation*}
    \begin{tikzcd}
      F(X) \ar[d, "F(f)"'] \ar[r, "\vartheta_X"] & G(X) \ar[d, "G(f)"] \\
      F(Y) \ar[r, "\vartheta_Y"'] & G(Y)
    \end{tikzcd}
  \end{equation*}
  komutira. To pomeni, da velja enačba
  \[ G(f) \circ \vartheta_X = \vartheta_Y \circ F(f).\]
\end{definicija}
\begin{definicija}
  Naj bosta $\cat{C}$ in $\cat{D}$ kategoriji in $F : \cat{C} \to \cat{D}$, $G : \cat{D} \to \cat{C}$
  funktorja. Potem sta $F$ in $G$ \emph{adjungirana}, oziroma tvorita
  \emph{par adjungiranih funktorjev}, če za vsaka objekta $X \in \cat{C}$ in $Y \in \cat{D}$
  obstaja bijekcija
  \[ \Hom_{\cat{D}}(FX, Y) \cong \Hom_{\cat{C}}(X, GY),\]
  naravna v $X$ in $Y$.
  V tem primeru pravimo, da je $F$ \emph{levi adjunkt} $G$ in simetrično, da je $G$
  \emph{desni adjunkt} $F$. To označimo kot $F \dashv G$.
\end{definicija}
Več osnovnih pojmov iz teorije kategorij lahko najdemo v ??(Tukaj citiram svojo diplomsko).
\end{document}

%%% Local Variables:
%%% mode: latex
%%% TeX-master: t
%%% End:

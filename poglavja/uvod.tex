\documentclass[../kategoricna_logika.tex]{subfiles}
\begin{document}
\section{Osnovne Definicije}
%
\section{Splošni izreki}
\begin{lema}
Naj bo $\cat{C}$ poljubna kategorija in $f : X \to Y$ morfizem. 
Denimo, da v $\cat{C}$ obstaja produkt $X \times Y$.
Potem je projekcija $X \times Y \to X$ epimorfizem.
\end{lema}
\begin{dokaz}
Naj bo $p : X \times Y \to X$ projekcija iz produkta in $u = \langle id_X, f \rangle : X \to X \times Y$. Recimo, da imamo dva vzporedna morfizma $g,h : X \to Z$ za katera velja $g \circ p = h \circ p$. To prikažemo v diagramu
\begin{center}
\begin{tikzcd}
& X \ar[d, "u", dashed] \ar[dr, "id_X"] \ar[dl, "f"'] \\
Y & X\times Y \ar[l] \ar[r, "p"] & X \ar[r, shift left, "g"] \ar[r, shift right, "h"'] & Z
\end{tikzcd}
\end{center}
Sedaj velja
\begin{align*}
g \circ p &= h \circ p \\
g \circ p \circ u &= h \circ p \circ u \\
g \circ id_X &= h \circ id_X \\
g &= h
\end{align*}
\end{dokaz}
\begin{lema}
  Če je morfizem $f : X \to Y$ epimorfizem in sekcija, potem je izomorfizem.
\end{lema}
\begin{dokaz}
To lahko razberemo iz diagrama
\begin{equation*}
\begin{tikzcd}
  X \ar[r, "f"] & Y \ar[dr, "id_Y"'] \ar[r, "g"] & X \ar[d, "f"] \\
  & & Y 
\end{tikzcd}
\end{equation*}
\end{dokaz}
\begin{lema}
  Skladanje kvadratov povleka.
\end{lema}
\begin{izrek} [Adjoint functor theorem]
Naj bo $\cat{C}$ lokalno majhna in polna. Za vsako kategorijo $\cat{X}$ in funktor 
$$U : \cat{X} \to \cat{C}$$
ki ohranja limite, so naslednje trditve ekvivalentne:
\begin{enumerate}
\item $U$ ima levi adjunkt
\item Za vsak objekt $X \in \cat{X}$ funktor $U$ zadošča kriteriju množice rešitev:
Obstaja množica $(C_i)_{i \in I}$ objektov v $\cat{C}$ tako, da za vsak objekt $C \in \cat{C}$ in vsak morfizem $f : X \to UC$ obstaja indeks $i \in I$ in morfizma $\phi : X \to UC_i$ in $\overline{f} : C_i \to C$, da velja
$$f = U(\overline{f}) \circ \phi$$
\[ \begin{tikzcd}
X \arrow[r, "\phi"] \arrow[rd, "f"'] & UC_i \arrow[d, "U(\overline{f})"] \\
& UC
\end{tikzcd} \]
\end{enumerate}
\end{izrek}
\begin{lema} \label{lema2}
Funktor $U : \cat{C} \to \cat{X}$ ima levi adjunkt natanko takrat, ko ima comma-kategorija $(X \downarrow U)$ začetni objekt.
\end{lema}
\begin{proof}
Naj bo $F : \cat{X} \to \cat{C}$ levi adjunkt za $U$ in $\eta : 1_{\cat{X}} \to UF$ enota adjunkcije. Potem je $(FX, \eta_X : X \to UFX)$ začetni objekt v $(X \downarrow U)$. To velja, saj če je $(C, f : X \to UC)$ objekt v $(X \downarrow U)$, potem po UMP enote obstaja natanko en $g : FX \to C$, da diagram
\[ \begin{tikzcd}
X \arrow[d, "\eta_X"'] \arrow[dr, "f"] & \\
UFX \arrow[r, "Ug"'] & UC
\end{tikzcd} \]
komutira. To je ravno enolični morfizem $(FX,\eta_X) \to (C,f)$. \\
Obratno, denimo da imamo začetni objekt v $(X \downarrow U)$, ki ga pomenljivo označimo kar z $(FX, \eta_X : X \to U(FX))$. Zaradi obstoja in enoličnosti takega objekta nam to določi funkcijo objektov 
$$F : \obj(\cat{X}) \to \obj(\cat{C})$$
Eksplicitno $F(X) = \text{prva komponenta začetnega objekta v } (X \downarrow U)$. Ta $F$ želimo razširito do funktorja, zato naj bo $f : X \to X'$ morfizem v $X$. Morfizem $F(f)$ naj bo enolični morfizem, tako da kvadrat
\[ \begin{tikzcd}
X \arrow[d, "\eta_X"'] \arrow[r, "f"] & X' \arrow[d, "\eta_{X'}"] \\
UFX \arrow[r, "UFf"', dashed] & UFX'
\end{tikzcd} \]
komutira. Jasno je, da $F$ slika domeno morfizma v domeno ter kodomeno v kodomeno. Za asociativnost denimo, da imamo morfizma 
\begin{center}
\begin{tikzcd}
X \arrow[r, "f"] & X' \arrow[r, "f'"] & X''
\end{tikzcd}
\end{center}
To nam da enolična morfizma 
\begin{center}
\begin{tikzcd}
FX \arrow[r, "Ff", dashed] & FX' \arrow[r, "Ff'", dashed] & X''
\end{tikzcd}
\end{center}
zaradi katerih diagram
\begin{center}
\begin{tikzcd}
X \ar[d, "\eta_X"'] \ar[r, "f"] & X' \ar[d, "\eta_{X'}"'] \ar[r, "f"] & X'' \ar[d, "\eta_{X''}"] \\
UFX \ar[r, "UFf"', dashed] & UFX' \ar[r, "UFf'"', dashed] & UFX''
\end{tikzcd}
\end{center}
komutira, kar določi kompozitum kot
$$F(f' \circ f) = F(f') \circ F(f).$$
Od tod tudi sledi, da je $\eta$ naravna transformacija, določena s komponentami $\eta_X$, ki zadošča UMP enote. Torej je $F$ res levi adjunkt $U$.
\end{proof}
\begin{definicija}
\mbox{}
\begin{enumerate}[label=(\roman*)]
\item Objekt $C \in \cat{C}$ je \emph{šibko začeten}, če za vsak $X \in \cat{C}$ obstaja morfizem $C \to X$ (ki ni nujno enoličen).
\item Zbirka objektov $\Phi = (C_i)_{i \in I}$ je \emph{skupno šibko začetna}, če za vsak $X \in C$ obstaja nek $C_i \in \Phi$, za katerega obstaja morfizem $C_i \to X$.
\end{enumerate}
\end{definicija}
\begin{lema}
Naj bo $\cat{C}$ kategorija za katero ima identitetni funktor $1_\cat{C} : \cat{C} \to \cat{C}$ limito. Potem ima $\cat{C}$ začetni objekt.
\end{lema}
\begin{proof}
Naj bo $l = \lim (1_\cat{C})$ z zbirko morfizmov $(\lambda_C)_{C \in \cat{C}}$. Iz definicije limite sledi, da je $l$ šibko začeten objekt. Recimo, da imamo še en morfizem $f : l \to C$. Ker je $l$ stožec sledi
$$ \lambda_C  = f \circ \lambda_l$$
Če pa vzamemo $\lambda_C$ kot morfizem diagrama $1_\cat{C}$, dobimo enačbo
$$\lambda_C = \lambda_C \circ \lambda_l$$
Iz diagrama
\begin{center}
\begin{tikzcd}
& l \ar[ddl, "1_l"', bend right] \ar[d, "u", dashed] \ar[ddr, "\lambda_C", bend left] & \\
& l \ar[dl, "\lambda_l"'] \ar[dr, "\lambda_C"]  & \\
l \ar[rr, "\lambda_C"'] & & C
\end{tikzcd}
\end{center}
dobimo enačbi $1_l = \lambda_l \circ u$ in $\lambda_C = \lambda_C \circ u$. Iz enoličnosti morfizma $u$ sledi $u = 1_l$, kar pomeni $\lambda_C = f$.
\end{proof}
\begin{lema} \label{lema1}
Naj bo $\cat{C}$ lokalno majhna, polna kategorija, ki ima skupno šibko začetno množico objektov $\Phi = (C_i)$. Potem ima $\cat{C}$ začetni objekt.
\end{lema}
\begin{proof}
Naj bo $\iota : \bm{\Phi} \to \cat{C}$ inkluzija polne podkategorije $\cat{C}$ generirane z objektov iz $\Phi$. Ker je $\bm{\Phi}$ majhna in $\cat{C}$ polna, obstaja limita
$$\ell = \lim \iota$$
Ker je množica $\Phi$ skupno šibko začetna imamo za vsak objekt $X \in \cat{C}$ morfizem $C_i \to X$ za nek $C_i \in \Phi$. Če komponiramo z ustreznim $l_i : \ell \to C_i$ dobimo morfizem
$$\lambda_X : \ell \to X$$
kar pomeni, da je $\ell$ šibko začetni objekt. Sedaj želimo pokazati, da morfizmi $\lambda_X$ določajo stožec nad $1_\cat{C}$ z vrhom $\ell$ in lastnostjo, da je $\lambda_l$ identiteta. Naj bo torej $f : X \to Y$ morfizem v $\cat{C}$. Ker je $\Phi$ šibko začetna, imamo morfizma $c_1 : C_1 \to X$ in $c_2 : C_2 \to Y$. Tvorimo povlek $P$ morfizmov $f \circ c_1$ in $c_2$
\begin{center}
\begin{tikzcd}
\ell \ar[dddrr, bend right, "l_1"'] \ar[ddrrrr, bend left, "l_2"] \ar[dr, dashed, "u"] & & & & \\
& C_3 \ar[ddr, bend right, dotted] \ar[drrr, bend left, dotted] \ar[dr, "h"] & & & \\
& & P \ar[d, "p_1"'] \ar[rr, "p_2"] & & C_2 \ar[d, "c_2"] \\
& & C_1 \ar[r, "c_1"'] & X \ar[r, "f"'] & Y
\end{tikzcd}
\end{center}
Ker je $\Phi$ šibko začetna, obstaja nek morfizem $h : C \to P$. Črtkana kompozituma $p_1 \circ h$ in $p_2 \circ h$ živita v polni podkategoriji generirani s $\Phi$, kar nam pove, da zgornja trikotnika komutirata. Iz komutativnosti ostalega dela diagrama vidimo, da je $\ell$ res vrh stožca nad $1_\cat{C}$. Iz tega sledi komutitativnost diagrama
\begin{center}
\begin{tikzcd}
& \ell \ar[dl, "\lambda_\ell"'] \ar[dr, "l_C = \lambda_C"] & \\
\ell \ar[rr, "\lambda_C"'] & & C
\end{tikzcd}
\end{center}
kar pomeni, da je $\lambda_\ell$ faktorizacija limitnega stožca skozi samega sebe, iz česar sledi $\lambda_\ell = 1_\ell$. Na dolgo, imamo komutativni diagram
\begin{center}
\begin{tikzcd}
& \ell \ar[ddl, bend right, "\lambda_\ell"'] \ar[ddr, bend left, "\lambda_C"] \ar[d, dashed, "u"] & \\
& \ell \ar[dl, "\lambda_\ell"'] \ar[dr, "\lambda_C"] & \\
\ell \ar[rr, "\lambda_C"'] & & C
\end{tikzcd}
\end{center}
iz spodnjega trikotnika imamo enačbo
$$\lambda_C = \lambda_C \circ \lambda_\ell$$
Ker pa je $\ell$ limita, dobimo enoličen $u$ za katerega velja
$$\lambda_\ell = \lambda_\ell \circ u \quad \& \quad \lambda_C = \lambda_C \circ u$$
Ker je $u$ enoličen tak, ki zadošča tema enačbama in je $1_\ell$ tudi tak, sledi $\lambda_\ell = 1_\ell$. Iz leme \ref{lema1} sledi, da ima $\cat{C}$ začetni objekt.
\end{proof}
Z uporabo teh lem lahko sedaj dokažemo izrek.
\begin{proof}
Če ima $U$ levi adjunkt, potem je množica $\set{FX}$ zadošča kriteriju množice rešitev.\\
Obratno, po lemi \ref{lema2} ima $U$ levi adjunkt natanko takrat, ko ima za vsak $X$ kategorija $(X \downarrow U)$ začetni objekt. Preveriti moramo torej
\begin{enumerate}
\item $(X \downarrow U)$ je lokalno majhna
\item $(X \downarrow U)$ je polna 
\item $(X \downarrow U)$ ima skupno šibko začetni objekt
\end{enumerate}
Ker je $\cat{C}$ lokalno majhna, je tudi $(X \downarrow U)$. Iz predpostavke sledi, da je
$$\set{(C_i, \phi : X \to UC_i) \ \vert \ i \in I}$$
skupno šibko začetni objekt. Da je $(X \downarrow U)$ polna pokažemo tako, da konstruiramo produkte in zožke, iz česar sledi obstoj limit. 
\end{proof}
\end{document}
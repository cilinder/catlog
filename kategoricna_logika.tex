\documentclass[12pt,a4paper]{book}
\usepackage{import}
\usepackage{magnastavitve}
\usepackage{subfiles}

\begin{document}
%
\frontmatter%

\subfile{poglavja/preduvod.tex}
\addcontentsline{toc}{chapter}{Kazalo}

%------------------------------------------------------------------------
%         IZVLECEK
%-----------------------------------------------------------------------

\cleardoublepage
% povzetek
\begin{center}
\addcontentsline{toc}{chapter}{Povzetek}  
{\bf Kategorična logika}\\[3mm]
{\sc  Izvleček}
\end{center}
\vspace{10mm}
Kratek izvleček v slovenskem jeziku, do 300 besed.\\[10mm]
{\bf Ključne besede:}\\[3mm]


\cleardoublepage
% abstract
\foreignlanguage{english}{  %  angleski delilni vzorci
  \begin{center}
    \addcontentsline{toc}{chapter}{Abstract}

{\bf Categorical logic}\\[3mm]
{\sc  Abstract}
\end{center}
\vspace{10mm}
Kratek izvleček v angleškem jeziku, do 300 besed.\\[10mm]
{\bf Keywords:}\\[3mm]
}

%-------------------------------------------------------------------
%
\noindent\textbf{Keywords:} \tkeywordsEn.
% prazna stran
\clearemptydoublepage
%
\selectlanguage{slovene}
\mainmatter%
\setcounter{page}{1}
\pagestyle{fancy}
%
\chapter{Uvod}
\subfile{poglavja/uvod.tex}
%
\chapter{Algebrajske teorije}
\subfile{poglavja/algebrajske_teorije.tex}
%
\chapter{Regularna logika in Regularne kategorije}
\subfile{poglavja/regularna_logika.tex}
%
\chapter{Zaključek}
\subfile{poglavja/zakljucek.tex}
%

\bibliographystyle{plain}
\bibliography{reference}
\end{document}
%%% Local Variables:
%%% mode: latex
%%% TeX-master: t
%%% End:

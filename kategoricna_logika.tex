\documentclass[12pt,a4paper]{book}
\usepackage{import}
\usepackage{magnastavitve}
\usepackage{subfiles}
%
\begin{document}
%
\frontmatter%
\subfile{poglavja/preduvod.tex}
\addcontentsline{toc}{chapter}{Kazalo}
%------------------------------------------------------------------------
%         IZVLECEK
%-----------------------------------------------------------------------
\cleardoublepage
% povzetek
\begin{center}
\addcontentsline{toc}{chapter}{Povzetek}  
{\bf Algebrajska in regularna kategorna logika}\\[3mm]
{\sc  Izvleček}
\end{center}
\vspace{10mm}
V nalogi je razvita funktorialna semantika za algebrajsko in regularno logiko.
V prvem delu je najprej na kratko predstavljena teorija kategorij, nato se uvede
pojem algebrajske teorije, ki je poseben primer logične teorije prvega reda, v kateri
nastopajo samo enačbe in operacije. Razširi se klasična interpretacija modela teorije
na vse kategorije v katerih je mogoče tako teorijo izraziti.
Za vsako algebrajsko teorijo lahko definiramo posebno sintaktično kategorijo,
ki to teorijo predstavlja. Izkaže se, da lahko vsak model algebrajske teorije
enolično identificiramo s funktorjem, ki ohranja strukturo sintaktične kategorije.
To je izraženo v obliki ekvivalence kategorij. S pomočjo te ekvivalence je raziskana
dualnost med sintakso in semantiko algebrajske teorije.
Drugi del se začne z opisom razreda kategorij imenovanih regularne in motivacijo
za njihovo upeljavo v obliki primerov in lepih lastnosti s katerimi se ponašajo.
Nato se razvije razširitev enostavne algebrajske logike iz prvega dela na tako imenovano
regularno logiko, v kateri poleg enačb in operacij nastopajo še relacijski simboli,
resničnostna konstanta, konjunkcija in kvantifikator obstoja. To naredi logiko
bolj bogato in v njej je mogoče izraziti koncepte kot je slika morfizma.
Analogno kot v prvem delu se za regularno teorijo definira njeno sintaktično kategorijo,
s pomočjo katere se pokaže ekvivalenco med modeli regularne logike in funktorji, ki
ohranjajo regularno strukturo.\\[10mm]
{\bf Ključne besede:} \tkeywords \\[3mm]
\cleardoublepage
% abstract
\foreignlanguage{english}{  %  angleški delilni vzorci
  \begin{center}
\addcontentsline{toc}{chapter}{Abstract}
{\bf Algebraic and regular kategorical logic}\\[3mm]
{\sc  Abstract}
\end{center}
\vspace{10mm}
The thesis develops functorial semantics for algebraic and regular logic.
The first part starts by briefly presenting category theory, then the concept of
an algebraic theory is introduced as a special case of a first order logic theory,
in which you only have equations and operations. The classical notion of a model
is expanded to categories in which such a theory can be expressed.
For each algebraic theory we may define a special sintactic category, which represents
it. It turns out that you can uniquely identify each model of such a theory with a
functor that preserves the structure of the sintactic category. This is expressed
in the form of an equivalence of categories. With the help of this equivalence a
duality between sintax and semantics is explored.
The second part begins with the description of a class of categories called regular
caregories and the motivation for their definition in terms of examples and nice
properties these categories posses. An extension of the simple algebraic logic is
then developed into the so called regular logic which besides equations and
operations includes relation symbols, the truth constant, conjunction and the
exsistential quantifier. This makes the logic more rich and makes it possible
to express concepts like the image of a morphism. Analogous with the first part
we define the syntactic category of a regular theory, with the help of which
you can show an equivalence between models of a regular theory and functors that
perserve regular structure.\\[10mm]
{\bf Keywords:}\tkeywordsEn \\[3mm]
}
%-------------------------------------------------------------------
%
% prazna stran
\clearemptydoublepage
%
\selectlanguage{slovene}
\mainmatter%
\setcounter{page}{1}
\pagestyle{fancy}
%
%\chapter{Uvod}
%\subfile{poglavja/uvod.tex}
%
\chapter{Algebrajske teorije}
\subfile{poglavja/algebrajske_teorije.tex}
%
\chapter{Regularna logika in regularne kategorije}
\subfile{poglavja/regularna_logika.tex}
%
\chapter{Zaključek}
\subfile{poglavja/zakljucek.tex}
%
\bibliographystyle{plain}
\bibliography{reference}
\end{document}
%%% Local Variables:
%%% mode: latex
%%% TeX-master: t
%%% End:
